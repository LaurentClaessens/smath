% This is part of Un soupçon de mathématique sans être agressif pour autant
% Copyright (c) 2012
%   Laurent Claessens
% See the file fdl-1.3.txt for copying conditions.

\begin{corrige}{Seconde-0058}

    \begin{enumerate}
        \item
            En écrivant étape par étape, nous trouvons la formule
            \begin{equation}
                \Big( 3(x+5)-2 \Big)\times\frac{ 3 }{2}.
            \end{equation}
            Pour la simplifier, nous effectuons les distributivités :
            \begin{subequations}
                \begin{align}
                    \Big( 3(x+5)-2 \Big)\times\frac{ 3 }{2}&=\big( 3x+15-2 \big)\times\frac{ 3 }{2}\\
                    &=\frac{ 3 }{2}(3x-13)=\frac{ 9 }{2}x-\frac{ 39 }{2}.
                \end{align}
            \end{subequations}
        \item
            Pour trouver les images, il suffit de remplacer \( x\) par les valeurs demandées. L'image de \( 0\) par le programme est \( -\frac{ 39 }{2}\). L'image de \( 1\) est 
            \begin{equation}
                \frac{ 9 }{ 2 }-\frac{ 39 }{ 2 }=\frac{ 9-39 }{2}=-\frac{ 30 }{ 2 }.
            \end{equation}
            L'image de \( \frac{ 2 }{ 9 }\) est
            \begin{equation}
                \frac{ 9 }{2}\times \frac{ 2 }{ 9 }-\frac{ 39 }{ 2 }=1-\frac{ 39 }{2}=\frac{ 2-39 }{2}=-\frac{ 37 }{2}.
            \end{equation}
        \item
            Il s'agit de résoudre l'équation
            \begin{equation}
                \frac{ 9 }{2}x-\frac{ 39 }{ 2 }=\frac{ 9 }{2}.
            \end{equation}
            qui se transforme successivement en
            \begin{subequations}
                \begin{align}
                    \frac{ 9 }{2}x-\frac{ 39 }{ 2 }&=\frac{ 9 }{2}\\
                    \frac{ 9 }{2}x&=\frac{ 9+39 }{2}=24\\
                    \frac{ 9 }{2}x&=24\\
                    x&=24\times \frac{ 2 }{ 9 }=\frac{ 48 }{ 9 }.
                \end{align}
            \end{subequations}
            Un antécédent de \( \frac{ 9 }{2}\) est \( x=\frac{ 48 }{ 9 }\).
        \item
            Un antécédent de \( \frac{ 1 }{2}\) est donné par la solution à l'équation
            \begin{equation}
                \frac{ 9 }{2}x-\frac{ 39 }{2}=\frac{ 1 }{2}.
            \end{equation}
            En multipliant les deux membres par \( 2\) nous trouvons à résoudre l'équation
            \begin{equation}
                9x-39=1,
            \end{equation}
            ou encore \( 9x=40\), c'est à dire \( x=\frac{ 40 }{ 9 }\).
    \end{enumerate}

\end{corrige}
