% This is part of Un soupçon de mathématique sans être agressif pour autant
% Copyright (c) 2015
%   Laurent Claessens
% See the file fdl-1.3.txt for copying conditions.

% This is part of Un soupçon de mathématique sans être agressif pour autant
% Copyright (c) 2015
%   Laurent Claessens
% See the file fdl-1.3.txt for copying conditions.

%--------------------------------------------------------------------------------------------------------------------------- 
\subsection*{Activité : pliage}
%---------------------------------------------------------------------------------------------------------------------------

\begin{enumerate}
    \item
        
Prendre une feuille de papier et réaliser des pliages suivant les lignes indiquées :

    \begin{center}
        \input{Fig_DWHNooSCJJNE.pstricks}                                          
    \end{center}  

    \item

        Refermer le tout en faisant coïncider les deux bords en gras. On obtient ainsi un solide sans «fond» ni «couvercle».

    \item

        Comment s'appelle le polygone formant les bases ?

    \item

         Combien de faces comporte le solide (y compris les bases) ?
    \item
          Quelles sont les formes des autres faces appelées « faces latérales » ?
    \item
           Combien de sommets comporte ton solide ?

\end{enumerate}

De \cite{NRHooXFvgpp5}

%+++++++++++++++++++++++++++++++++++++++++++++++++++++++++++++++++++++++++++++++++++++++++++++++++++++++++++++++++++++++++++ 
\section{Prisme}
%+++++++++++++++++++++++++++++++++++++++++++++++++++++++++++++++++++++++++++++++++++++++++++++++++++++++++++++++++++++++++++

\begin{definition}
    Un \defe{prisme droit}{prisme droit} est un solide composé de deux bases qui sont superposables et parallèles et de faces latérales qui sont des rectangles.
\end{definition}


\begin{example}
    Dessiner le patron d'un prisme droit dont la base est un triangle de côtés \SI{5}{\centi\meter}, \SI{4}{\centi\meter} et \SI{3}{\centi\meter}, et dont la hauteur est égale à \SI{2}{\centi\meter} (aide : la base est un triangle rectangle).

    \begin{enumerate}
        \item
            D'abord tracer une base :

JYAKooXjBUpt

    \end{enumerate}
    <++>


\end{example}
<++>

