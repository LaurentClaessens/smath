% This is part of Un soupçon de mathématique sans être agressif pour autant
% Copyright (c) 2012-2013
%   Laurent Claessens
% See the file fdl-1.3.txt for copying conditions.

\begin{exercice}\label{exoPremiere-0059}

    \begin{minipage}{0.5\textwidth}
    À partir du graphique ci-contre,
    \begin{enumerate}
        \item
            Donner les solutions de \( f(x)=2\), \( f(x)=-2\) et \( f(x)=0\).
\item
    Quelle est l'image de \( -2\) par la fonction tracée ?
\item
    Dire laquelle est la bonne parmi les trois possibilités suivantes : \( f(x)=x^2+2\), \( f(x)=(x+1)^2\), \( f(x)=2x^2+4x+2\).
\item
    Quelles sont les coordonnées du sommet ?
\item
    Dresser le tableau de variations de \( f\).
    \end{enumerate}
    \end{minipage}
    \hspace{1mm}
    \begin{minipage}{0.5\textwidth}
   \centering     
   \input{Fig_ParaboleoytUKk.pstricks}
    \end{minipage}

\corrref{Premiere-0059}
\end{exercice}
