% This is part of Un soupçon de mathématique sans être agressif pour autant
% Copyright (c) 2014
%   Laurent Claessens
% See the file fdl-1.3.txt for copying conditions.

%--------------------------------------------------------------------------------------------------------------------------- 
\subsection*{Petits et grands carrés}
%---------------------------------------------------------------------------------------------------------------------------

Nous découpons le bord d'un grand carré en petits carrés comme indiqué sur les dessins :

% Les figures sont de phystricksHVXooRtjPkd.py

\begin{center}
   \input{Fig_XYWooOPFwaca.pstricks} \input{Fig_BPZooBCyuyK.pstricks}
\end{center}

\begin{enumerate}
    \item
Réaliser une figure avec cinq petits carrés sur un côté et indiquer le nombre total de carrés coloriés. Recommencer avec une figure de six petits carrés de côté.

\item
S'il y a $100$ petits carrés sur le côté, combien y-a-t-il de carrés coloriés au total ?
\item
    Nous appelons \( n\) le nombre de petits carrés d'un côté du grand carré, et nous voulons trouver une formule donnant le nombre total de carrés coloriés. Valérian dit :
    \begin{quote}
        « Il y a \( 4\) côtés et \( n\) carrés par côtés, donc \( 4n\) petits carrés.» 
    \end{quote}
    Laureline n'est pas d'accord :
    \begin{quote}
       « Tu en as trop !»
    \end{quote}
    Qui a raison ? Pourquoi ? Donner une formule correcte.
\end{enumerate}
