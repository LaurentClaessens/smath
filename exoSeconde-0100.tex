% This is part of Un soupçon de mathématique sans être agressif pour autant
% Copyright (c) 2012
%   Laurent Claessens
% See the file fdl-1.3.txt for copying conditions.

\begin{exercice}\label{exoSeconde-0100}

    Soient deux cercles \( C\) et \( C'\) de centre \( O\) et \( O'\) et de rayons différents. Nous supposons qu'ils se coupent en deux points que nous nommons \( A\) et \( B\). Nous nommons \( K\) le point d'intersection entre les droites \( (AB)\) et \( (OO')\).
    \begin{multicols}{2}
        \begin{enumerate}
        \item
            Dessiner la situation.
        \item
            Le triangle \( AO'B\) est isocèle en \( O'\). Pourquoi ?
        \item
            En déduire que \( O'\) est sur la médiane du segment \( [AB]\).
        \item
            Prouver que \( O\) est également sur la médiane de \( [AB]\).
        \item
            En déduire que les droites \( (AB)\) et \( (OO')\) sont perpendiculaires.
        \item
            Si \( C\) et \( C'\) ont même rayon, quelle est la nature du quadrilatère \( OAO'B\) ?
    \end{enumerate}
    \end{multicols}

\corrref{Seconde-0100}
\end{exercice}
