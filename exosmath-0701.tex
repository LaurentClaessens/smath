% This is part of Un soupçon de mathématique sans être agressif pour autant
% Copyright (c) 2014
%   Laurent Claessens
% See the file fdl-1.3.txt for copying conditions.

\begin{exercice}\label{exosmath-0701}

\begin{wrapfigure}{r}{8.0cm}
   \vspace{-0.5cm}        % à adapter.
   \centering
   \input{Fig_IERooEvNjp.pstricks}
\end{wrapfigure}

    Soient les points \( A(2;2)\), \( B(0;0)\) et \( C(3;0)\). Dessiner et construire sur la feuille les points \( P\) et \( M\) définis par
    \begin{enumerate}
        \item
            $\vect{ AM }=\vect{ AB }+\vect{ CA }$
        \item
            $\vect{ AP }=\vect{ AC }-\vect{ AB }$
    \end{enumerate}
    Montrer que
    \begin{enumerate}
        \item
            \( BAPC\) est un parallélogramme;
        \item
            \( A\) est le milieu du segment \( [PM]\).
    \end{enumerate}


\corrref{smath-0701}
\end{exercice}
