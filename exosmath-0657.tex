% This is part of Un soupçon de mathématique sans être agressif pour autant
% Copyright (c) 2014
%   Laurent Claessens
% See the file fdl-1.3.txt for copying conditions.

\begin{exercice}\label{exosmath-0657}


Le triangle \( ABC\) est quelconque, le point \( M\) est variable sur le segment \( [AC]\). Nous notons \( x\) la distance \( AM\). Les points \( N\) et \( G\) sont placés sur les côtés \( [AB]\) et \( [BC]\) de telle sorte que \( CGNM\) soit un parallélogramme.

Nous notons \( f(x)\) l'aire de \( AMN\) et \( g(x)\) celle de \( NBG\).

    \begin{center}
   \input{Fig_QFpJtQc.pstricks}
    \end{center}

\begin{enumerate}
    \item
        Voici un tableau de variation :
        \(
            \begin{array}[]{c|ccc}
                x&0&&6\\
                \hline
                &15&&\\
                &&\searrow&\\
                &&&0\\
            \end{array}
 \)
        Est-ce celui de \( f\) ou de \( g\) ?
    \item
        Dresser le tableau de variations de l'autre fonction.
    \item
        On note \( h\) la fonction qui donne l'aire du parallélogramme \( CGNM\) en fonction de \( x\). Roger affirme que la fonction \( h\) est croissante sur \( \mathopen[ 0 , 6 \mathclose]\). Qu'en penser ?
\end{enumerate}

\corrref{smath-0657}
\end{exercice}
