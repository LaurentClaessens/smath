% This is part of Un soupçon de mathématique sans être agressif pour autant
% Copyright (c) 2015
%   Laurent Claessens
% See the file fdl-1.3.txt for copying conditions.

\begin{exercice}\label{exo2smath-0070}

    Dans chacune des situations suivantes, ajouter les codages utiles et déterminer les angles manquants.
    \begin{enumerate}
        \item
   \input{Fig_SHSRooHgvofoZ.pstricks}sachant
   \( \hat \widehat{QCM}=\SI{120}{\degree}\) et \( \widehat{CMQ}=\SI{40}{\degree}\)
\item
    \input{Fig_SHSRooHgvofoO.pstricks}sachant \( \hat{K}=\SI{50}{\degree}\)
\item
    \input{Fig_SHSRooHgvofoT.pstricks}sachant \( \hat{R}=\SI{100}{\degree}\) et \( \hat{T}=\SI{20}{\degree}\).
\item
    \input{Fig_SHSRooHgvofoTh.pstricks}sachant \( \widehat{BOD}=\SI{10}{\degree}\) et \( \widehat{OBD}=\SI{95}{\degree}\).
\item
   \input{Fig_SHSRooHgvofoF.pstricks}sachant \( FE=EH\).
\item
   \input{Fig_SHSRooHgvofoFi.pstricks}sachant que \( ABC\) est équilatéral.
            
    \end{enumerate}

\corrref{2smath-0070}
\end{exercice}
