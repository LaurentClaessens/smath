% This is part of Un soupçon de mathématique sans être agressif pour autant
% Copyright (c) 2012
%   Laurent Claessens
% See the file fdl-1.3.txt for copying conditions.

\begin{corrige}{smath-0027}

    Pour qu'un quadrilatère soit un parallélogramme, il suffit que ses diagonales se coupent en leur milieu. Les diagonales sont \( [BD]\) et \( [AC]\). Le milieu de \( [AC] \) est 
    \begin{equation}
        \left( \frac{ 1+5 }{2};\frac{ 1+3 }{2} \right)=(3;8).
    \end{equation}
    Nous devons donc fixer le point \( D\) de telle façon à ce que \( (3;8)\) soit le milieu de \( [BD]\). C'est le point \( D=(4;2)\) qui fonctionne.

    Pour vérifier que le quadrilatère ainsi construit soit ou non un rectangle, nous pouvons vérifier si le triangle \( ABC\) est rectangle en utilisant le théorème de Pythagore. Nous avons
    \begin{subequations}
        \begin{align}
            AC^2=4^2+2^2=20\\
            AB^2=1^2+2^2=5\\
            BC^2=3^2+1^1=10,
        \end{align}
    \end{subequations}
    donc nous n'avons pas \( AC^2=AB^2+BC^2\) et le parallélogramme n'est pas un rectangle.

\end{corrige}
