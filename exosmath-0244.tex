% This is part of Un soupçon de mathématique sans être agressif pour autant
% Copyright (c) 2013
%   Laurent Claessens
% See the file fdl-1.3.txt for copying conditions.

\begin{exercice}\label{exosmath-0244}

    On compare les salaires de deux entreprises de \( 6\) personnes. Dans la première les salaires sont :
    \begin{center}
    \begin{multicols}{3}
        1150€\\
        1300€\\ 
        1300€\\
        1400€\\
        1450€\\
        1500€
    \end{multicols}
    \end{center}
    Dans la seconde entreprise, les salaires sont :
    \begin{center}
    \begin{multicols}{3}
        700€\\
        900€\\
        1300€\\
        1400€\\
        1800€\\
        2000€
    \end{multicols}
    \end{center}
    \begin{enumerate}
        \item
            Calculer les moyennes des salaires dans les deux entreprises.
        \item
            Sachant cette moyenne, peut-on dire que ces entreprises donnent des salaires «justes» ?
        \item
            Les deux entreprises distribuent-elles leurs salaires de la même façon ? Préciser en utilisant l'écart-type.
    \end{enumerate}

\corrref{smath-0244}
\end{exercice}
