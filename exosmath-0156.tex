% This is part of Un soupçon de mathématique sans être agressif pour autant
% Copyright (c) 2012
%   Laurent Claessens
% See the file fdl-1.3.txt for copying conditions.

\begin{exercice}\label{exosmath-0156}

    Soit le rectangle \( OABC\) avec \( O=(0;0)\), \( A=(4;0)\) et \( C=(0;3)\). Soit aussi \( K\) le milieu de \( [CB]\). Nous nommons \( E\) le point d'intersection \( (OK)\cap (CA)\) et \( F=(KA)\cap(OB)\).
    \begin{enumerate}
        \item
            Déterminer les coordonnées de \( B\) et de \( K\).
        \item
            Prouver que \( \vect{ EF }=\frac{1}{ 3 }\vect{ BC }\).
    \end{enumerate}

\corrref{smath-0156}
\end{exercice}
