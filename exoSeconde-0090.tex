% This is part of Un soupçon de mathématique sans être agressif pour autant
% Copyright (c) 2012
%   Laurent Claessens
% See the file fdl-1.3.txt for copying conditions.

\begin{exercice}\label{exoSeconde-0090}

    \begin{multicols}{2}
        \begin{enumerate}
            \item
                Quel est le volume d'un cube de \unit{3}{\centi\meter} d'arête ? 
            \item
                Quelle est l'aire d'un triangle de base \unit{4}{\meter} et de hauteur \unit{50}{\centi\meter} ?
            \item
                Quel est le volume d'une sphère de rayon \( \unit{7}{\meter}\) ?
            \item
                Quel est le volume de un litre d'eau ?
            \item
                Quel est le volume d'un cube dont les côtés font \unit{5}{\meter} ? Par combien multiplie-t-on ce volume en considérant un cube de \unit{10}{\meter} ?
            \item
                Quel est le volume d'un parallélépipède rectangle d'arêtes de longueurs \( 4\), \( 7\), \( 3\) ? Par combien on multiplie ce volume si on double la largeur ? Et si on double tous les côtés ?
            \item
                Quel est le volume d'une citerne cylindrique de diamètre \unit{1.5}{\meter} et de longueur \unit{3}{\meter} ?
        \end{enumerate}
    \end{multicols}

\corrref{Seconde-0090}
\end{exercice}
