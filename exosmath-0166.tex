% This is part of Un soupçon de mathématique sans être agressif pour autant
% Copyright (c) 2012
%   Laurent Claessens
% See the file fdl-1.3.txt for copying conditions.

\begin{exercice}\label{exosmath-0166}

    Un sportif participe à \( 4\) compétitions par an. Nous notons \( u_1\) le nombre de compétitions auxquelles il aura participé après une année de carrière; \( u_2\) le nombre de compétitions à la fin de sa seconde année, etc. Nous notons plus généralement \( u_n\) le nombre de compétitions auxquelles il aura participé après sa \( n\)ième année de carrière.

    Montrer que \( u_n\) est une suite arithmétique, donner sa raison et son terme initial. À combien de compétitions aura-t-il participé au bout de \( 10\) ans de carrière ?

\corrref{smath-0166}
\end{exercice}
