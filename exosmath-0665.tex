% This is part of Un soupçon de mathématique sans être agressif pour autant
% Copyright (c) 2014
%   Laurent Claessens
% See the file fdl-1.3.txt for copying conditions.

\begin{exercice}\label{exosmath-0665}

    Soit une parallélogramme \( ABCD\); nous définissons les points \( M\) et \( N\) de telle sorte que \( \vect{ BM }=\frac{ 1 }{2}\vect{ AB }\) et \( \vect{ AN }=3\vect{ AD }\).

    \begin{enumerate}
        \item
            Montrer que \( \vect{ CM }=\frac{ 1 }{2}\vect{ AB }-\vect{ AD }\).
        \item
            Est-il possible de trouver un réel \( k\) tel que \( \vect{ CN }=k\vect{ CM }\) ?
    \end{enumerate}

\corrref{smath-0665}
\end{exercice}
