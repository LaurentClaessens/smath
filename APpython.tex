% This is part of Un soupçon de mathématique sans être agressif pour autant
% Copyright (c) 2012
%   Laurent Claessens
% See the file fdl-1.3.txt for copying conditions.

Ce cours est à propos de la version \( 3\) de python.

\chapter{Structures de base}

%+++++++++++++++++++++++++++++++++++++++++++++++++++++++++++++++++++++++++++++++++++++++++++++++++++++++++++++++++++++++++++
\section{Listes}
%+++++++++++++++++++++++++++++++++++++++++++++++++++++++++++++++++++++++++++++++++++++++++++++++++++++++++++++++++++++++++++

Une liste est une collection ordonnée d'éléments. Elle se définit avec des crochets.

\lstinputlisting{ex_listes.py}

donne

\lstinputlisting[title=Résultat]{res_ex_listes.txt}

Nous pouvons ajouter un élément à une liste en utilisant la \emph{méthode} \info{append}, et retrouver un élément d'une liste par son numéro (attention : python numérote à partie de zéro). Cas spécial : le dernier élément de la liste est le numéro \( -1\).

\lstinputlisting{ex_listes2.py}

donne

\lstinputlisting[title=Résultat]{res_ex_listes2.txt}

