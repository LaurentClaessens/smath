% This is part of Un soupçon de mathématique sans être agressif pour autant
% Copyright (c) 2012
%   Laurent Claessens
% See the file fdl-1.3.txt for copying conditions.

\begin{corrige}{smath-0080}

    Entre le 8 novembre et le 9 novembre, le nombre d'auditeurs a été multiplié par \( 1.15\). Le nombre d'auditeurs a donc été \emph{divisé} par \( 1.15\) entre le 9 et le 10 novembre.

    La question est donc de savoir à quel taux correspond une division par \( 1.15\). Le coefficient multiplicateur est \( C=\frac{1}{ 1.15 }\simeq 0.8695\) que nous arrondirons à \( 0.87\). Nous avons déjà vu qu'un coefficient multiplicateur de \( 0.87\) correspond à un taux de variation de \( -13\%\).

\end{corrige}
