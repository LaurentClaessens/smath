% This is part of Un soupçon de mathématique sans être agressif pour autant
% Copyright (c) 2015
%   Laurent Claessens
% See the file fdl-1.3.txt for copying conditions.

kmlkmlkml

\begin{definition}[\cite{NRHooXFvgpp4}]
    Une \defe{pyramide}{pyramide} est un solide dont :
    \begin{itemize}
        \item 
            une face est un polygone appelée la \defe{base}{base!d'une pyramide} de la pyramide ;
\item
les autres faces, appelées faces latérales, sont des triangles qui ont un sommet commun, appelé le sommet de la pyramide.
    \end{itemize}
\end{definition}


\begin{definition}[\cite{NRHooXFvgpp4}]
    \begin{itemize}
        \item 
La hauteur d'une pyramide est le segment issu de son sommet et perpendiculaire à la base.
\item
Une arête latérale est un segment joignant les sommets de la base au sommet de la pyramide.
    \end{itemize}
    <++>
\end{definition}

sdfgdfg



The result is on figure \ref{LabelFigCylindresxKDOdy}. % From file CylindresxKDOdy
\newcommand{\CaptionFigCylindresxKDOdy}{<+Type your caption here+>}
\input{Fig_CylindresxKDOdy.pstricks}
See also the subfigure \ref{LabelFigCylindresxKDOdyssLabelSubFigCylindresxKDOdy0}
See also the subfigure \ref{LabelFigCylindresxKDOdyssLabelSubFigCylindresxKDOdy1}

<++>

\end{document}
% Les exercices de quatrième sont dans  6_exercices_feuilles_quatrieme.tex
% et les autres exercices dans 19_

dfgdfg


