% This is part of Un soupçon de mathématique sans être agressif pour autant
% Copyright (c) 2013
%   Laurent Claessens
% See the file fdl-1.3.txt for copying conditions.

\begin{exercice}\label{exosmath-0349}

    Nous souhaitons estimer la proportion de personnes dans une ville \( V\) souffrant de surpoids. Pour cela, les enquêteurs ont sélectionné \( 460\) personnes dans manière aléatoire à partir de la liste des logements connus par la municipalité.
    \begin{enumerate}
        \item
            Dans un premier temps, les enquêteurs veulent savoir si l'échantillon sélectionné est représentatif de la population de la ville. Ils savent que la ville comprend \( 46\%\) d'hommes et \( 20\%\) de personnes de plus de \( 60\) ans.
            \begin{enumerate}
                \item
                    Déterminer l'intervalle de confiance au seuil de \( 0.95\) de la proportion de femmes dans un échantillon de \( 460\) personnes de la ville.
                \item
                    Déterminer l'intervalle de confiance au seuil de \( 0.95\) de la proportion de personnes de plus de \( 60\) pour un échantillon de \( 460\) personnes de la ville.
                \item
                    Le tableau suivant montre comment est composé l'échantillon prélevé par les enquêteurs :
                    \begin{equation*}
                        \begin{array}[]{|c||c|c|c|}
                            \hline
                            &\text{hommes}&\text{femmes}&\text{total}\\
                            \hline
                            \text{plus de \( 60\)} ans&&&108\\
                            \hline
                            \text{moins de \( 60\)} ans&&&352\\
                            \hline
                            \text{total}&200&260&460\\
                            \hline
                        \end{array}
                    \end{equation*}
                    <++>
            \end{enumerate}
            <++>
    \end{enumerate}
    <++>

    %AFAIRE : terminer.

\corrref{smath-0349}
\end{exercice}
