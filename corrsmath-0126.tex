% This is part of Un soupçon de mathématique sans être agressif pour autant
% Copyright (c) 2012
%   Laurent Claessens
% See the file fdl-1.3.txt for copying conditions.

\begin{corrige}{smath-0126}

    Les \( 25\) euros dont il est question dans l'énoncé est un prix TTC. Nous avons
    \begin{equation}
        25=HT+TVA=\frac{ 105.5 }{ 100 }\times HT.
    \end{equation}
    Donc le prix HT est
    \begin{equation}
        HT=\frac{ 100\times 25 }{ 105.5 }\simeq 23.6966.
    \end{equation}
    La TVA (ce que le gouvernement récupère) est \( 5.5\%\) du prix HT, c'est à dire
    \begin{equation}
        TVA=23\times \frac{ 5.5 }{ 100 }=1.3033.
    \end{equation}
    Ces \( 1.3033\) euros représentent quel pourcentage du prix TTC payé par le client ? Pour le savoir, il suffit de faire
    \begin{equation}
        \frac{ 1.3033 }{ 25 }\simeq 0.052,
    \end{equation}
    donc \( 5.2\%\).

    Donc dans le cas de cette viande, \( 5.2\%\) du prix payé par le client est intercepté par l'état sous forme de TVA.

\end{corrige}
