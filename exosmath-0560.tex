% This is part of Un soupçon de mathématique sans être agressif pour autant
% Copyright (c) 2013
%   Laurent Claessens
% See the file fdl-1.3.txt for copying conditions.

\begin{exercice}\label{exosmath-0560}

Vrai ou faux (expliquer) ?
\begin{enumerate}
    \item
        Pour tout \( x\) dans \( \eR\) nous avons \( (x+1)^2=x^2+1\).
    \item
        Il est possible de trouver un \( x\) dans \( \eR\) tel que \( (x+1)^2=x^2+1\).
    \item
        Pour tout réels \( a\) et \( b\) nous avons la formule \( (a+b)^2=a^2+3ab+b^2\).
    \item
        Il existe un réel tel que \( x^2+2x-3\geq 0\).
\end{enumerate}

\corrref{smath-0560}
\end{exercice}
