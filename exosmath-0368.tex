% This is part of Un soupçon de mathématique sans être agressif pour autant
% Copyright (c) 2013
%   Laurent Claessens
% See the file fdl-1.3.txt for copying conditions.

%TODO : remettre le cite
\begin{exercice}%[\cite{FGMWzeK}]
    \label{exosmath-0368}

    Nous considérons la fonction
    \begin{equation}
        f(x)=\frac{ x-2 }{ x+1 }.
    \end{equation}
    \begin{enumerate}
        \item
            Compléter le tableau suivant :
            \begin{equation*}
                \begin{array}[]{|c||c|c|c|c|c|}
                    \hline
                     x&4&2&1&0&-2\\
                      \hline
                      f(x)&&&&&\\ 
                      \hline 
                   \end{array}
               \end{equation*}
           \item
               Calculer l'image de \( 0\) par la fonction \( f\).
           \item
               Donner un antécédent de \( 0\) par \( f\).
           \item
               Calculer l'image de \( -\frac{ 1 }{2}\) par \( f\).
           \item
               Donner un antécédent de \( -\frac{ 1 }{2}\) par \( f\).

    \end{enumerate}
    Donner les réponses sous forme de fractions simplifiées.

\corrref{smath-0368}
\end{exercice}
