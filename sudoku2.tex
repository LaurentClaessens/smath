% This is part of Un soupçon de mathématique sans être agressif pour autant
% Copyright (c) 2013
%   Laurent Claessens
% See the file fdl-1.3.txt for copying conditions.


% Ceci est pour le début de la seconde.

\begin{multicols}{2}

    \begin{minipage}{7cm}
        \input{Fig_FPIkJJx.pstricks}
    \end{minipage}

    \vspace{1cm}

    \begin{enumerate}
        \item
            Les solutions de l'équation \( (x-1)(x+2)\) dans les cases I7 et H9.
        \item
            Soit la fonction \( f\colon x\mapsto x^2-4\). Mettre \( f(0)\) dans la case F7 et \( f(-2)\) dans la case G3.
        \item
            Soit \( f(x)=mx\) la fonction linéaire dont le graphe passe par le point \( (5;10)\). Mettre \( m\) dans la case C3.
        \item
            Soit \( x\mapsto 5x+p\) la fonction affine dont le graphe passe par le point \( (4;23)\). Mettre \( p\) dans la case B7.
        \item
            Les solutions de \( x^2-4=0\) dans les cases E8 et A7.
        \item
            Les cases D5 et H6 et C2 sont les mêmes.
        \item
            Soit la fonction \( f(x)=3x+8\). L'image de \( -10/3\) par \( f\) dans la case G6. L'antécédent de \( -1\) dans la case I8.
        \item
            Soit le milieu du segment entre \( A(3;5)\) et \( B(-7;1)\). Mettre les abscisses en D8 et les ordonnées en E9.
        \item
            Soit le point d'abscisse \( \frac{ 1 }{2}\) sur le graphe de le fonction \( x\mapsto \frac{4}{ x }-9\). Mettre les ordonnées dans la case F8.
        \item
            La distance entre le point \( (3;7)\) et \( -1;7\) dans la case G4.        
    \end{enumerate}
\end{multicols}

