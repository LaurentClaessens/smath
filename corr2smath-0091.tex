% This is part of Un soupçon de mathématique sans être agressif pour autant
% Copyright (c) 2015
%   Laurent Claessens
% See the file fdl-1.3.txt for copying conditions.

\begin{corrige}{2smath-0091}

    \begin{enumerate}
        \item
            Diviser par une fraction revient à multiplier par l'inverse (voir la section \ref{SecDivision}). Il faut donc faire
            \begin{subequations}
                \begin{align}
                    \frac{ 10 }{ 7 }\div\frac{ 4 }{ 5 }&=\frac{ 10 }{ 7 }\times \frac{ 5 }{ 4 }\\
                    &=\frac{ 10\times 5 }{ 7\times 4 }\\
                    &=\frac{ 50 }{ 28 }\\
                    &=\frac{ 25 }{ 14 }.
                \end{align}
            \end{subequations}
            La dernière ligne est une simplification par \( 2\) : \( \dfrac{ 50\div2 }{ 28\div 2 }\).
        \item
            Pour additionner ou soustraire des fractions, il faut passer par un dénominateur commun. Les dénominateurs étant \( 3\) et \( 6\), nous choisissons de convertir
            \begin{equation}
                \frac{ 6 }{ 3 }=\frac{ 6\times 2 }{ 3\times 2 }=\frac{ 12 }{ 6 }.
            \end{equation}
            Alors
            \begin{subequations}
                \begin{align}
                    \frac{ 6 }{ 3 }-\frac{ 3 }{ 6 }&=\frac{ 12 }{ 6 }-\frac{ 3 }{ 6 }\\
                    &=\frac{ 12-3 }{ 6 }\\
                    &=\frac{ 9 }{ 6 }\\
                    &=\frac{ 3 }{ 2 }.
                \end{align}
            \end{subequations}
    \end{enumerate}

\end{corrige}
