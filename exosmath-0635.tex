% This is part of Un soupçon de mathématique sans être agressif pour autant
% Copyright (c) 2014
%   Laurent Claessens
% See the file fdl-1.3.txt for copying conditions.

\begin{exercice}[\ldots/4]\label{exosmath-0635}

    Pour chacune des questions suivantes, une seule réponse est correcte; dire laquelle (et expliquer le choix)
    \begin{multicols}{2}
    \begin{enumerate}
        \item
           L'algorithme

    \begin{fmpage}{0.9\linewidth}

    Demander \( x\)

    Si  \( x< 10\) , alors :

    \hspace{0.5cm} Écrire «plus petit» 

    Si \( x>15\), alors :

    \hspace{0.5cm} Écrire «plus grand» 

\end{fmpage}

    \begin{enumerate}
        \item
            Ne fait rien si l'utilisateur rentre \( 12\).
        \item
            Écrit «plus petit»  si l'utilisateur rentre \( 10\).
        \item
            Écrit «plus grand» si l'utilisateur rentre \( 10\).
    \end{enumerate}
    
\item

    L'expression \( (x-7)^2\)
    \begin{enumerate}
        \item
            est égale à \( -36\) si \( x=1\).
        \item   \label{ItemPMxYdYQ}
            est égale à \( 36\) si \( x=1\).
        \item
            est égale à \( x^2-49\) pour tout réel \( x\).
    \end{enumerate}
    
\item

    La fraction
    \begin{equation*}
        \frac{ 6a^2 }{ 8a^2+4a }
    \end{equation*}
    
    \begin{enumerate}
        \item
            peut seulement être simplifiée par \( a\)
        \item
            peut être simplifiée par \( 2\) et par \( a\).
        \item
            peut seulement être simplifiée par \( 2\)
    \end{enumerate}

\item

    Les représentations graphiques des fonctions \( f(x)=4x^2+7\) et \( g(x)=4x^2-2\)
    \begin{enumerate}
        \item
            ne sont pas des droites
        \item
            sont des droites parallèles
        \item
            sont des droites sécantes
    \end{enumerate}

    \end{enumerate}
    \end{multicols}

\corrref{smath-0635}
\end{exercice}
