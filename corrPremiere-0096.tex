% This is part of Un soupçon de mathématique sans être agressif pour autant
% Copyright (c) 2012
%   Laurent Claessens
% See the file fdl-1.3.txt for copying conditions.

\begin{corrige}{Premiere-0096}

    \begin{enumerate}
        \item
            Les racines du polynôme \( f(x)=ax^2+bx+c\) sont données par les formules
            \begin{equation}
                \begin{aligned}[]
                    x_1&=\frac{ -b+\sqrt{b^2-4ac} }{ 2a }&x_2&=\frac{ -b-\sqrt{b^2-4ac} }{ 2a }.
                \end{aligned}
            \end{equation}
        \item
            Pour calculer les racines du polynôme nous commençons par calculer le discriminant :
            \begin{equation}
                \Delta=1-4\times1\times (-6)=1+24=25.
            \end{equation}
            Les racines sont donc
            \begin{subequations}
                \begin{align}
                    x_1&=\frac{ -1+5 }{2}=2\\
                    x_2&=\frac{ -1-5 }{2}=-3.
                \end{align}
            \end{subequations}
    \end{enumerate}

\end{corrige}
