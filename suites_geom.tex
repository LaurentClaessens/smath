% This is part of Un soupçon de mathématique sans être agressif pour autant
% Copyright (c) 2013
%   Laurent Claessens
% See the file fdl-1.3.txt for copying conditions.

\begin{definition}
    Soit \( q\in \eR\). Une suite \( (u_n)\) est une \defe{suite géométrique}{suite!géométrique} de raison \( q\) si pour tout entier \( n\) nous avons \( u_{n+1}=qu_n\).
\end{definition}

\begin{example}
    Une colonie de fourmis arrive dans une foret dans laquelle les fourmis trouvent une nourriture abondante, mais pas de prédateurs naturels. Le nombre de fourmis va donc doubler chaque année. La première année, elles sont un million.

    La seconde année, elles seront deux millions, la troisième année, quatre millions, etc.

    Combien de fourmis avons-nous au bout de dix ans ?

    %AFAIRE : vérifier le nombre de fourmis dans une fourmilière.
\end{example}

\begin{Aretenir}
    Sens de variation.
    \begin{enumerate}
%AFAIRE : finir ce bout de théorie.
    \end{enumerate}
\end{Aretenir}

%+++++++++++++++++++++++++++++++++++++++++++++++++++++++++++++++++++++++++++++++++++++++++++++++++++++++++++++++++++++++++++ 
\section{Exercices}
%+++++++++++++++++++++++++++++++++++++++++++++++++++++++++++++++++++++++++++++++++++++++++++++++++++++++++++++++++++++++++++

\Exo{smath-0263}
