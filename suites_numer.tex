% This is part of Un soupçon de mathématique sans être agressif pour autant
% Copyright (c) 2012-2013
%   Laurent Claessens
% See the file fdl-1.3.txt for copying conditions.

%+++++++++++++++++++++++++++++++++++++++++++++++++++++++++++++++++++++++++++++++++++++++++++++++++++++++++++++++++++++++++++ 
\section{Suites numériques}
%+++++++++++++++++++++++++++++++++++++++++++++++++++++++++++++++++++++++++++++++++++++++++++++++++++++++++++++++++++++++++++

\begin{example}
    Une population de \( 500\) renards est introduite en 2013 dans une foret renfermant une nourriture abondante et peu de prédateurs. L'évolution de la population de renards est modélisée comme suit : chaque année le nombre de renards double, mais 30 renards meurent mangés par des prédateurs.

    Combien de renards y aura-t-il en 2014, 2015 et 2016 ?
\end{example}

\begin{definition}
    Une \defe{suite numérique}{suite numérique} est une liste de nombres réels numérotés par les nombres entiers soit à partir de zéro soit à partir de \( 1\).

    La suite est \defe{croissante}{croissante!suite} si pour tout \( n\) on a \( u_n<u_{n+1}\) et \defe{décroissante}{décroissante!suite} si pour tout \( n\) on a \( u_n>u_{n+1}\).
\end{definition}

%+++++++++++++++++++++++++++++++++++++++++++++++++++++++++++++++++++++++++++++++++++++++++++++++++++++++++++++++++++++++++++ 
\section{Suites définies par récurrence}
%+++++++++++++++++++++++++++++++++++++++++++++++++++++++++++++++++++++++++++++++++++++++++++++++++++++++++++++++++++++++++++

\begin{example}
    Chaque mois un locataire reçoit \( 1700\) euros de salaire, en paye \( 700\) de loyer, \( 400\) de nourriture et \( 200\) de frais divers. Au premier janvier, il avait \( 5200\) euros sur son compte. Donner la suite qui donne l'argent qu'il lui reste en fonction du nombre de mois écoulés.

    Première étape : on commence à \( u_0=5200\). Chaque mois il gagne \( 1700\) et perd \( 1300\), donc il met \( 400\) euros sur son compte, donc
    \begin{equation}
        u_{n+1}=u_n+400.
    \end{equation}
\end{example}

%+++++++++++++++++++++++++++++++++++++++++++++++++++++++++++++++++++++++++++++++++++++++++++++++++++++++++++++++++++++++++++ 
\section{Exercices}
%+++++++++++++++++++++++++++++++++++++++++++++++++++++++++++++++++++++++++++++++++++++++++++++++++++++++++++++++++++++++++++

%--------------------------------------------------------------------------------------------------------------------------- 
\subsection{Suites numériques}
%---------------------------------------------------------------------------------------------------------------------------

\Exo{smath-0158}
\Exo{smath-0291}
\Exo{smath-0160}
\Exo{smath-0168}

%--------------------------------------------------------------------------------------------------------------------------- 
\subsection{Suite définie par récurrence}
%---------------------------------------------------------------------------------------------------------------------------

\Exo{smath-0161}
\Exo{smath-0162}
\Exo{smath-0159}

%+++++++++++++++++++++++++++++++++++++++++++++++++++++++++++++++++++++++++++++++++++++++++++++++++++++++++++++++++++++++++++ 
\section{À l'ordinateur}
%+++++++++++++++++++++++++++++++++++++++++++++++++++++++++++++++++++++++++++++++++++++++++++++++++++++++++++++++++++++++++++
% Je mets ici des trucs à faire à l'ordinateur, de tout types de suites confondues.

\VerbatimInput{TD_suites_stmg.txt}

\Exo{smath-0314}
\Exo{smath-0312}
\Exo{smath-0313}
\Exo{smath-0304}     
