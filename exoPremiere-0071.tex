% This is part of Un soupçon de mathématique sans être agressif pour autant
% Copyright (c) 2012
%   Laurent Claessens
% See the file fdl-1.3.txt for copying conditions.

\begin{exercice}\label{exoPremiere-0071}

    Dire si les expériences suivantes sont des épreuves de Bernoulli. Donnez si possible la probabilité de succès dans chaque cas de Bernoulli.
    \begin{multicols}{2}
        \begin{enumerate}
            \item
                Tirer une carte dans un paquet et considérer l'événement «c'est une carte noire». 
            \item
                Tirer une carte dans un paquet et considérer sa valeur (as, roi, \( 10\), \( 7\), etc.).
            \item
                Lancer un dé et considérer le nombre indiqué.
            \item
                Dans un jeu de rôle, un nécromancien vous jette un sort. Lancez un dé à \( 12\) faces; si vous obtenez un nombre plus grand que votre capacité «résistance à la magie», vous résistez et pouvez attaquer; sinon vous perdez \( 6\) points de vie.
            \item
                Un elfe décoche une flèche sur un Gobelin. Il lance en dé à \( 6\) faces; si il fait \( 1\) ou \( 2\), le gobelin est touché à la tête et meurt. Si il faut \( 3\) ou \( 4\), le gobelin est touché au bras et perd \( 3\) points d habilité. Si il fait \( 5\) la flèche rate sa cible. Si il fait \( 6\), l'elfe perd l'équilibre, tombe de son arbre et casse son arc (rayez cet objet de votre équipement).
            \item
                À une question «vrai ou faux», un élève tape complètement au hasard.
            \item
                En classant vos amis sur Facebook dans l'ordre alphabétique, vous considérez l'événement «mon cinquième ami est dans le groupe des fans de Michael Jackson».
        \end{enumerate}
    \end{multicols}

\corrref{Premiere-0071}
\end{exercice}
