% This is part of Un soupçon de mathématique sans être agressif pour autant
% Copyright (c) 2014
%   Laurent Claessens
% See the file fdl-1.3.txt for copying conditions.

% This is part of Un soupçon de mathématique sans être agressif pour autant
% Copyright (c) 2014
%   Laurent Claessens
% See the file fdl-1.3.txt for copying conditions.

%--------------------------------------------------------------------------------------------------------------------------- 
\subsection*{Petits et grands carrés}
%---------------------------------------------------------------------------------------------------------------------------

Nous découpons le bord d'un grand carré en petits carrés comme indiqué sur les dessins :

% Les figures sont de phystricksHVXooRtjPkd.py

\begin{center}
   \input{Fig_XYWooOPFwaca.pstricks} \input{Fig_BPZooBCyuyK.pstricks}
\end{center}

\begin{enumerate}
    \item
Réaliser une figure avec cinq petits carrés sur un côté et indiquer le nombre total de carrés coloriés. Recommencer avec une figure de six petits carrés de côté.

\item
S'il y a $100$ petits carrés sur le côté, combien y-a-t-il de carrés coloriés au total ?
\item
    Nous appelons \( n\) le nombre de petits carrés d'un côté du grand carré, et nous voulons trouver une formule donnant le nombre total de carrés coloriés. Valérian dit :
    \begin{quote}
        « Il y a \( 4\) côtés et \( n\) carrés par côtés, donc \( 4n\) petits carrés.» 
    \end{quote}
    Laureline n'est pas d'accord :
    \begin{quote}
       « Tu en as trop !»
    \end{quote}
    Qui a raison ? Pourquoi ? Donner une formule correcte.
\end{enumerate}

De \cite{NRHooXFvgpp4}

%+++++++++++++++++++++++++++++++++++++++++++++++++++++++++++++++++++++++++++++++++++++++++++++++++++++++++++++++++++++++++++ 
\section{Substitution}
%+++++++++++++++++++++++++++++++++++++++++++++++++++++++++++++++++++++++++++++++++++++++++++++++++++++++++++++++++++++++++++

\begin{Aretenir}
    Pour calculer une valeur numérique d'une expression contenant une lettre dont on connaît la valeur, on lui substitue sa valeur numérique.
\end{Aretenir}

\begin{example}
    Dans l'activité, la formule donnant le nombre de petits carrés était
    \begin{equation}
        4\times n-4.
    \end{equation}
    Pour savoir le nombre de petits carrés dessinés lorsqu'on en a \( 12\) par côté, on écrit «\( 12\)» à la place de \( n\) :
    \begin{equation}
        4\times 12-4=48-4=44.
    \end{equation}
\end{example}

\begin{example}
    Lorsque \( x=7\), l'expression \( A=4\times x-3\) vaut \( 4\times 7-3=25\). Lorsque \( x=-5\) l'expression \( A\) vaut
    \begin{equation}
        4\times (-5)-3=-20-3=-23.
    \end{equation}
\end{example}

%+++++++++++++++++++++++++++++++++++++++++++++++++++++++++++++++++++++++++++++++++++++++++++++++++++++++++++++++++++++++++++ 
\section{Développer}
%+++++++++++++++++++++++++++++++++++++++++++++++++++++++++++++++++++++++++++++++++++++++++++++++++++++++++++++++++++++++++++

\begin{Aretenir}
    Si \( a\), \( b\) et \( k\) sont des nombres quelconques, alors la formule
    \begin{equation}
        k\times (a+b)= k\times a+k\times b
    \end{equation}
    est encore valable en quatrième.

    Cette opération s'appelle \defe{développer}{développer} ou \defe{distribuer}{distribution}.
\end{Aretenir}

\begin{example}
    Pour calculer mentalement \( 43\times 102\), on fait \( 43\times 100+43\times 2\). Plus généralement
    \begin{equation}
        43\times (100+x)=43\times 100+43\times x.
    \end{equation}
\end{example}

\begin{example}
    Pour calculer le périmètre d'une rectangle de largeur \( \ell\) et de longueur \( L\) on peut faire soit
    \begin{equation}
        2\ell+2L
    \end{equation}
    soit
    \begin{equation}
        2(\ell+L).
    \end{equation}
\end{example}

%+++++++++++++++++++++++++++++++++++++++++++++++++++++++++++++++++++++++++++++++++++++++++++++++++++++++++++++++++++++++++++ 
\section{Factoriser}
%+++++++++++++++++++++++++++++++++++++++++++++++++++++++++++++++++++++++++++++++++++++++++++++++++++++++++++++++++++++++++++

\begin{Aretenir}
    La \defe{factorisation}{factorisation} est la formule
    \begin{equation}
        k\times a+k\times b=k\times (a+b).
    \end{equation}
    Une expression est \defe{factorisée}{expression!factorisée} lorsqu'elle est écrite sous forme d'un produit.
\end{Aretenir}

\begin{example}
    \begin{enumerate}
        \item
            \( 2\ell+2L\) n'est pas factorisé
        \item
            \( 2(\ell+L)\) est factorisée.
        \item
            \( 3x+2\) n'est pas factorisé.
        \item
            \( (2x-5)\times 3\) est factorisé.
        \item
            \( (-x+9)\times (a+8)\) est factorisé.
    \end{enumerate}
\end{example}


%+++++++++++++++++++++++++++++++++++++++++++++++++++++++++++++++++++++++++++++++++++++++++++++++++++++++++++++++++++++++++++ 
\section{Simplification et réduction}
%+++++++++++++++++++++++++++++++++++++++++++++++++++++++++++++++++++++++++++++++++++++++++++++++++++++++++++++++++++++++++++

\begin{definition}
    \begin{itemize}
        \item 
    Nous notons \( x\times x\) par \( x^2\), qui se prononce «\( x\) au carré».
\item
    Nous notons \( x\times x\times x\) par \( x^3\), qui se prononce «\( x\) au cube».
    \end{itemize}
\end{definition}

\begin{Aretenir}
    \defe{Réduire}{réduire} une expression, c'est l'écrire avec le moins de termes possibles.
\end{Aretenir}

\begin{example}
    \begin{enumerate}
        \item
            \( 3x+5x=8x\) parce que
            \begin{subequations}
                \begin{align}
                    x+1+3x&=x+3x+1&\text{regrouper}\\
                    &=x(1+3)+1&\text{factoriser}\\
                    &=4x+1&\text{effectuer}.
                \end{align}
            \end{subequations}
        \item
            \( 3(x+4)-6x+4=3x+12-6x+4=-3x+16\).
    \end{enumerate}
\end{example}

