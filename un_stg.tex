% This is part of Un soupçon de mathématique sans être agressif pour autant
% Copyright (c) 2012
%   Laurent Claessens
% See the file fdl-1.3.txt for copying conditions.

\chapter{Proportions, pourcentages}

Nous disons qu'un gaz est en concentration de une \defe{partie par million}{partie par million} si un million de grammes d'air contient un gramme du gaz. Voici quelque chiffre concernant l'évolution de la concentration de \( CO_2\) dans l'atmosphère; les chiffres sont en \( \unit{}{ppm}\) :
\begin{center}
\begin{tabular}{|c|c|c|}
    \hline
    1750    &   2005    &   1012\\
    \hline
    280&380&395\\
    \hline
\end{tabular}
\end{center}
Pour information, cette concentration n'a pas dépassé les \unit{300}{ppm} depuis au moins \( 600.000\) ans.

\begin{enumerate}
    \item
        De combien de pourcent la concentration de \( CO_2\) a augmenté entre 1750 et 2005 ?
    \item
        Sur un kilo d'air, combien de grammes de \( CO_2\) ?
    \item 
        Quelle est la vitesse (en ppm par an) d'augmentation de la concentration entre 1750 et 2005 ? Même question entre 2005 et 2012.
\end{enumerate}

\begin{definition}
    Une \defe{population}{population} est un ensemble fini. Si \( E\) est une population, une \defe{sous-population}{sous-population} est un sous-ensemble \( A\subset E\). L'\defe{effectif}{effectif} d'une population est le nombre de ses éléments.

    La \defe{proportion}{proportion} de \( A\) dans \( E\) est le rapport
    \begin{equation}
        p_A=\frac{ n_A }{ n_E }
    \end{equation}
    où \( n_A\) et \( n_E\) sont les effectifs de \( A\) et \( E\).
\end{definition}
Note : une proportion est un nombre compris entre zéro et un.

\Exo{Premiere-0001}
\Exo{Premiere-0002}
\Exo{Premiere-0003}


