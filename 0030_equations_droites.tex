% This is part of Un soupçon de mathématique sans être agressif pour autant
% Copyright (c) 2012-2013
%   Laurent Claessens
% See the file fdl-1.3.txt for copying conditions.

%+++++++++++++++++++++++++++++++++++++++++++++++++++++++++++++++++++++++++++++++++++++++++++++++++++++++++++++++++++++++++++ 
\section{Introduction}
%+++++++++++++++++++++++++++++++++++++++++++++++++++++++++++++++++++++++++++++++++++++++++++++++++++++++++++++++++++++++++++

\begin{Aprojeter}
    Le taxi Besacdanslesac divise son prix en deux paries : $0.2$ euros de frais de prise en charge plus un euro par km parcouru. Le taxi Ledoubstoudoux par contre divise son prix en $1$ euro de frais de prise en charge plus $0.8$ euros par kilomètre parcouru.

    \begin{enumerate}
        \item
            Combien coûte un trajet de \unit{5}{\kilo\meter} avec Besacdanslesac ?
        \item
            Donner une expression algébrique du prix d'une course en fonction du nombre de kilomètres parcourus.
        \item
            Combien de kilomètres peut-t-on effectuer dans Ledoubstoudoux avec \( 10\) euros ?
        \item
            Exprimer les prix en fonction du nombre de kilomètres parcourus sur un graphique (les deux taxis sur le même graphique).
        \item
            À partir de combien de kilomètres parcourus vaut-il mieux prendre Ledoubstoudoux ?
    \end{enumerate}
\end{Aprojeter}

%+++++++++++++++++++++++++++++++++++++++++++++++++++++++++++++++++++++++++++++++++++++++++++++++++++++++++++++++++++++++++++ 
\section{Définitions}
%+++++++++++++++++++++++++++++++++++++++++++++++++++++++++++++++++++++++++++++++++++++++++++++++++++++++++++++++++++++++++++

\begin{definition}
    Une \defe{fonction affine}{affine}\index{fonction!affine} est une fonction définie sur \( \eR\) par
    \begin{equation}
        f(x)=mx+p
    \end{equation}
    où \( m\) et \( p\) sont deux nombres réels fixés.
\end{definition}

\begin{Aretenir}
    Le graphe de la fonction \( x\mapsto mx+p\) passe par le point de coordonnées \( (0;p)\).
    \begin{enumerate}
        \item
            \( p\) est l'\defe{ordonnées à l'origine}{ordonnées!à l'origine} de la fonction affine \( mx+p\).
        \item
            \( m\) est le \defe{coefficient directeur}{coefficient directeur} de la fonction affine \( mx+p\).
    \end{enumerate}
\end{Aretenir}

Cas particuliers :
\begin{enumerate}
    \item
        Si \( p=0\) alors \( f(x)=mx\) et nous disons que \( f\) est une fonction \defe{linéaire}{fonction!linéaire}. Son graphe passe par l'origine \( (0;0)\).
    \item
        Si \( m=0\) alors \( f(x)=p\) et nous disons que \( f\) est une fonction \defe{constante}{fonction!constante}.
\end{enumerate}
