% This is part of Un soupçon de mathématique sans être agressif pour autant
% Copyright (c) 2014
%   Laurent Claessens
% See the file fdl-1.3.txt for copying conditions.

\begin{exercice}[\ldots/6]\label{exosmath-0633}

    Nous considérons un rectangle \( ABCD\) avec \( AB=6\) et \( BC=4\). Le point \( M\) est mobile sur \( [CB]\) et nous notons \( x\) la distance \( CM\). Le point \( N\) est placé sur \( [AB]\) de telle sorte que \( AN=CM=x\). Enfin le point \( P\) est placé à l'intérieur du rectangle de telle sorte que \( NPMB\) soit un rectangle

\begin{wrapfigure}{r}{7.0cm}
   \vspace{-0.5cm}        % à adapter.
   \centering
   \input{Fig_DDRbyQk.pstricks}
\end{wrapfigure}

    Nous nommons \( f\), \( g\) et \( h\) les fonctions qui à \( x\) font correspondre les aires de \( DAN\), \( PNBM\) et \( DCM\).

    \begin{enumerate}
        \item
            Dans quel intervalle varie \( x\) ?
        \item
            Reporter les différentes mesures sur le dessin.
        \item
            Expliquer pourquoi \( f(x)=2x\).
        \item   \label{ItemSBlWgNviv}
            Calculer les fonction \( g\) et \( h\).
        \item\label{ItemSBlWgNvv}
            Montrer que l'aire de la zone grisée en fonction de \( x\) est donnée par la fonction \( A\colon x\mapsto -x^2+5\).
        \item\label{ItemSBlWgNvvi}
            Bob prétend que l'aire de la zone grisée s'annule pour \( x=0\) et \( x=5\). Quel est son raisonnement ? Qu'en penser ?
    \end{enumerate}
    Note : il y a moyen de répondre à la question \ref{ItemSBlWgNvvi} même sans répondre aux questions \ref{ItemSBlWgNviv} et \ref{ItemSBlWgNvv}.

\corrref{smath-0633}
\end{exercice}
