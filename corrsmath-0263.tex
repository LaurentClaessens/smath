% This is part of Un soupçon de mathématique sans être agressif pour autant
% Copyright (c) 2012
%   Laurent Claessens
% See the file fdl-1.3.txt for copying conditions.

\begin{corrige}{smath-0263}

    \begin{enumerate}
        \item
            \( 4000\), \( 3200\), \( 2560\).
        \item
            C'est une suite géométrique.
        \item
            Le premier terme est \( u_{0}=5000\) et sa raison est \( 0.8\). Ici nous prenons l'année \( 1990\) comme année zéro.
        \item
            Le nombre de dieux restants est donné par la formule
            \begin{equation}
                u_n=500\times (0.8)^n.
            \end{equation}
            L'année \( 2013\) correspond à \( u_{23}\) et donc
            \begin{equation}
                u_{23}=5000\times (0.8)^{23}\simeq 29.5
            \end{equation}
    \end{enumerate}

\end{corrige}
