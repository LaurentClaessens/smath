% This is part of Un soupçon de mathématique sans être agressif pour autant
% Copyright (c) 2013
%   Laurent Claessens
% See the file fdl-1.3.txt for copying conditions.

\begin{exercice}\label{exosmath-0579}

    À propos de lancers de dés.
    \begin{enumerate}
        \item
            Écrire un programme qui simule le lancé d'un dé (i.e. qui choisit un nombre au hasard entre \( 1\) et \( 6\)) et qui dit «oui» si le nombre choisit est \( 3\).
        \item
            Écrire un programme qui simule \( 50\) lancers de dés et qui donne la fréquence d'apparition du \( 3\) (en comptant le nombre de \( 3\) qui sortent). Vous devriez obtenir un nombre pas loin de \( 1/6\), c'est à dire environ \( 0.166\).

            Essayer avec plus que \( 50\) lancers : \( 100\), \( 1000\), \ldots
        \item
            Écrire un programme qui évalue la probabilité d'obtenir une somme de \( 10\) en lançant \( 4\) dés.
    \end{enumerate}

\corrref{smath-0579}
\end{exercice}
