% This is part of Un soupçon de mathématique sans être agressif pour autant
% Copyright (c) 2014
%   Laurent Claessens
% See the file fdl-1.3.txt for copying conditions.

\begin{corrige}{smath-0631}

    \begin{enumerate}
        \item
            Étant donné que \( g \) est croissante sur \( \mathopen[ -5 ; 0 \mathclose]\), que \( g(-5)=0\) et que \( g(0)=3\), le nombre \( g(5)\) est compris entre \( 0\) et \( 3\) (voir tableau de variations).
        \item
            Par ailleurs \( g(6)=5\) et entre \( x=5\) et \( x=6\) la fonction est décroissante; donc \( g(5)>g(6)=5\), c'est à dire \( g(5)>5\). Nous avons donc \( g(5)>g(-1)\).

        \item
            Il y a de nombreuses possibilités. En voici une :
            \begin{center}
                \input{Fig_ZMGMLvNBa.pstricks}
            \end{center}
        \item
            Pour ce graphique le tableau de variations est :
            \begin{equation*}
                \begin{array}[]{c|ccccc}
                    x&-5&&2&&7\\
                    \hline
                    &&&7&&\\
                    f(x)&&\nearrow&&\searrow&\\
                    &0&&&&4\\
                \end{array}
            \end{equation*}
            Notez qu'il est inutile de mettre les points intermédiaires dans le tableau de variations.
        \item
            Pour le tableau de signe, il est facile parce que la fonction est toujours positive :
            \begin{equation*}
                \begin{array}[]{c|ccc}
                     x&-5&&7\\
                 \hline
                      f(x)&0&+&+\\
                       \end{array}
                   \end{equation*}
    \end{enumerate}

\end{corrige}
