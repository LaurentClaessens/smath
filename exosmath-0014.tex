% This is part of Un soupçon de mathématique sans être agressif pour autant
% Copyright (c) 2012
%   Laurent Claessens
% See the file fdl-1.3.txt for copying conditions.

\begin{exercice}\label{exosmath-0014}

    Les réponses à cet exercice peuvent dépendre de \( a\) et de \( x\).
    \begin{multicols}{2}
        \begin{enumerate}
            \item
                Simplifier
                \begin{equation*}
                    \frac{ 6+12a }{ 3 }.
                \end{equation*}
            \item
                Résoudre
                \begin{equation*}
                    \frac{ 2x+1 }{ 3 }=7
                \end{equation*}

            \item
                Mettre \( x\) en évidence dans \( ax^2-x\).
            \item
                Mettre au même dénominateur et effectuer la somme
                \begin{equation*}
                    \frac{ a }{ 4 }+\frac{ 2 }{ x }.
                \end{equation*}
        \end{enumerate}
    \end{multicols}

\corrref{smath-0014}
\end{exercice}
