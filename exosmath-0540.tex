% This is part of Un soupçon de mathématique sans être agressif pour autant
% Copyright (c) 2013
%   Laurent Claessens
% See the file fdl-1.3.txt for copying conditions.

\begin{exercice}\label{exosmath-0540}

    L'entreprise Valaba loue des bus de \( 50\) places pour transporter des supporters de foot. Louer un bus coûte \( 800\) euros.
    \begin{enumerate}
        \item
            Un groupe de 40 personnes veut louer un bus. Combien devra payer chaque participant ?
        \item
            Même question pour un groupe de \( 67\) participants.
        \item
            Esquisser le graphe de la fonction qui à \( x\) fait correspondre le prix à payer par chaque supporter pour un groupe de \( x\) personnes (pour \( x\) entre \( 0\) et \( 150\)).
        \item
            Afin d'aider le secrétariat de Valaba, écrire un programme qui demande le nombre de supporters à transporter et affiche le prix à payer par chacun. Se limiter au cas du nombre de personnes inférieur à \( 50\).
    \end{enumerate}
    
    Extensions possibles.
    \begin{enumerate}
        \item
            Améliorer le programme pour qu'il puisse répondre à la question pour un nombre de voyageurs allant jusqu'à \( 100\), \( 150\) ou plus.
        \item
            Faire tracer le graphe donnant le prix par personne en fonction du nombre de personnes.
        \item
            Si les supporters ne veulent pas payer plus de \( 20\) euros par personne, combien doivent s'inscrire ? (attention : question plus difficile qu'il paraît)
    \end{enumerate}

\corrref{smath-0540}
\end{exercice}
