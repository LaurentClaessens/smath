% This is part of Un soupçon de mathématique sans être agressif pour autant
% Copyright (c) 2013
%   Laurent Claessens
% See the file fdl-1.3.txt for copying conditions.

\begin{exercice}\label{exosmath-0546}

    Soit une fonction \( f\) définie sur \( \mathopen[ -2; 10 \mathclose]\) dont le tableau de variations est
    \begin{equation*}
        \begin{array}[]{|c||ccccccccc|}
            \hline
            x&-2&&-1&&2&&4&&10\\
            \hline\hline
            &1&&&&5&&&&6\\
            f(x)&&\searrow&&\nearrow&&\searrow&&\nearrow&\\
            &&&-2&&&&2&&\\
            \hline
        \end{array}
    \end{equation*}
    \begin{enumerate}
        \item
            Comparer l'image de \( 5\) par \( f\) et l'image de \( -\frac{ 3 }{2}\) par \( f\). 
        \item
            Donner un encadrement de \( f(3)\).
        \item
            Est-ce que le point \( (-1;5)\) appartient à la courbe représentative de \( f\) ?
        \item
            Tracer un graphique de ce que pourrait être la fonction \( f\).
        \item
            Dresser le tableau de signe de la fonction que vous avez dessinée.
    \end{enumerate}

\corrref{smath-0546}
\end{exercice}
