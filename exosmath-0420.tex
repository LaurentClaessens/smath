% This is part of Un soupçon de mathématique sans être agressif pour autant
% Copyright (c) 2013
%   Laurent Claessens
% See the file fdl-1.3.txt for copying conditions.

\begin{exercice}\label{exosmath-0420}

    Expliquer brièvement pourquoi dans le phrase «Que valent les nombres \( f(1)\), \( f(0.1)\), etc.», le mot «que» n'introduit pas une subordonnée. À quel temps est alors conjugué le verbe «valoir» sous la forme «valent» ?

\corrref{smath-0420}
\end{exercice}
