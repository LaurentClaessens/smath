% This is part of Un soupçon de mathématique sans être agressif pour autant
% Copyright (c) 2014
%   Laurent Claessens
% See the file fdl-1.3.txt for copying conditions.

\begin{exercice}\label{exosmath-0782}

    Exprimer les phrases suivantes sous la forme «si \ldots alors \ldots» 
    \begin{enumerate}
        \item
            Un bon élève a toujours un stylo à la main.
        \item
            Un bon élève ne ferme jamais son cahier avant la sonnerie.
    \end{enumerate}
    En s'appuyant sur les affirmations précédentes, répondre par «vrai», «faux» ou «on ne sait pas» :
    \begin{enumerate}
        \item
            Julien a un stylo à la main. Est-il un bon élève ?
        \item
            Marc est un mauvais élèves. Est-il certain qu'il va fermer son cahier avant la sonnerie ?
    \end{enumerate}

\corrref{smath-0782}
\end{exercice}
