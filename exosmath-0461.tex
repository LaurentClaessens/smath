% This is part of Un soupçon de mathématique sans être agressif pour autant
% Copyright (c) 2013
%   Laurent Claessens
% See the file fdl-1.3.txt for copying conditions.

\begin{exercice}\label{exosmath-0461}

    Le but de cet exercice est de donner une méthode pour trouver une solution approchée à une équation du type \( f(x)=0\) en utilisant la dérivée de \( f\).

    Montrer que la tangente au graphe de \( f\) au point d'abscisse \( a\) coupe l'axe des abscisses en 
    \begin{equation}
        x=a-\frac{ f(a) }{ f'(a) }.
    \end{equation}
    Pour rappel, l'équation de la tangente est donnée par \( y=f'(a)(x-a)+f(a)\).

    Soit \( f(x)=(x+3)(x-4)\) et \( u_0=7\).
    \begin{enumerate}
        \item
            Calculer \( f(u_0)\) et \( f'(u_0)\). Donner l'équation de la tangente au graphe de \( f\) au point d'abscisse \( u_0\).
        \item 
            Soit \( u_1\) le point d'abscisse de l'intersection de cette tangente avec l'axe horizontal.
        \item
            Faire un dessin
        \item
            Continuer, sur le même modèle à calculer \( u_2\) et \( u_3\), en valeurs approchées.
        \item
            En regardant sur le dessin, vers quelle valeur converge la suite \( (u_n)\) ?
    \end{enumerate}

    Après avoir fait tout ça, qu'est-ce que vous pensez de la suite donnée par récurrence par
    \begin{equation}
        u_{n+1}=u_n-\frac{ f(u_n) }{ f'(u_n) }.
    \end{equation}

    Donner une solution approchée de l'équation $\ln(x)+x=0$.

\corrref{smath-0461}
\end{exercice}
