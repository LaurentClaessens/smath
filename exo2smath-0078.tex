% This is part of Un soupçon de mathématique sans être agressif pour autant
% Copyright (c) 2014
%   Laurent Claessens
% See the file fdl-1.3.txt for copying conditions.

\begin{exercice}\label{exo2smath-0078}

    Donner la valeur des expressions suivantes lorsque \( z=4\) :
    \begin{multicols}{4}
        \begin{enumerate}
            \item
                \( z+3\)
            \item
                \( 5z\)
            \item
                \( 6\times(z+3)\)
            \item
                \( 6\times z+3\)
        \end{enumerate}
    \end{multicols}
    Quelle valeur faut-il donner à \( a\) pour avoir \( 4a=12\) ?


\corrref{2smath-0078}
\end{exercice}
