% This is part of Un soupçon de mathématique sans être agressif pour autant
% Copyright (c) 2012
%   Laurent Claessens
% See the file fdl-1.3.txt for copying conditions.

\begin{corrige}{Seconde-0074}

    En ce qui concerne la moyenne de la série statistique :
    \begin{equation}
        \frac{ 5+8+6+5+7+6+8+5+7+7 }{ 10 }=\frac{ 64 }{ 10 }=6.4.
    \end{equation}
    La moyenne est donc \( 6.4\).

    Il y a \( 10\) valeurs dans la série, ce qui est un nombre pair. La médiane est donc la moyenne des \emph{deux} valeurs centrales, c'est à dire de \( 6\) et \( 7\). La médiane est donc \( \frac{ 6+7 }{2}=6.5\).

    Pour les quartiles, nous avons dix valeurs. Étant donné que \( \frac{ 10 }{ 4 }=2.5\), le premier quartile est la troisième valeur, c'est à dire \( 5\). Étant donné que \( \frac{ 3*10 }{ 4 }=7.5\), le troisième quartile est la septième valeur, à savoir \( 7\).

\end{corrige}
