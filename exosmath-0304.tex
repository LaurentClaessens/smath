% This is part of Un soupçon de mathématique sans être agressif pour autant
% Copyright (c) 2013
%   Laurent Claessens
% See the file fdl-1.3.txt for copying conditions.

\begin{exercice}\label{exosmath-0304}

    Afin d'éviter ce que l'ONU qualifie de «conséquences ingérables» du réchauffement climatique, il est impératif de diviser par trois les émissions planétaires de \( CO_2\) d'ici l'an 2050\cite{KXZPUlP}. Le but de cet exercice est de calculer quel est le taux annuel de diminution qu'il faut obtenir durant les 37 prochaines années pour réaliser cet objectif\footnote{Pour votre culture générale, sachez qu'il n'est \emph{jamais} arrivé que cette planète diminue ses émissions, à part durant les deux années 2008-2009, grâce à la crise.}.

    \begin{enumerate}
        \item
            Nous supposons que les émissions 2012 valent \( u_0=\)\unit{30}{\giga\tonne}\cite{XZZqclR} et qu'elles soient multipliées par \( q\) chaque année. Afin que nous ayons affaire à une diminution, est-ce que \( q\) devrait être plus grand ou plus petit que \( 1\) ? Expliquer.
        \item   \label{ItemUFcQLwv}
            Prendre un nombre \( q\) un peu au hasard comme point de départ (par exemple \( q=0.5\)) et calculer les émissions que nous aurions en 2050 si nous suivions une diminution de \( q\) chaque année.
        \item
            Est-ce que le \( q\) choisi à la question \ref{ItemUFcQLwv} est trop grand ou trop petit ?
        \item
            Recommencer en choisissant une autre valeur de \( q\) et tenter de trouver une ou deux décimales de \( q\) en procédant ainsi par essais erreurs.
        \item
            Donner en pourcentage la décroissance en \( CO_2\) qu'il faut atteindre chaque année.
        \item
            Illustrer la décroissance des émissions par un graphique.
    \end{enumerate}

\corrref{smath-0304}
\end{exercice}
