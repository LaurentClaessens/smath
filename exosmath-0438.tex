% This is part of Un soupçon de mathématique sans être agressif pour autant
% Copyright (c) 2013
%   Laurent Claessens
% See the file fdl-1.3.txt for copying conditions.

\begin{exercice}\label{exosmath-0438}

    \begin{multicols}{2}
        \begin{enumerate}
            \item
                Lesquels des points suivants sont sur la droite d'équation \( y=1-x\) ?
                \begin{enumerate}
                    \item
                        \( (1;1)\)
                    \item
                        \( (10;-9)\)
                    \item
                        \( (0;0)\)
                    \item
                        \( (-10;11)\)
                \end{enumerate}
                \columnbreak
            \item
                Un minerais coûtait \( 14\)€ le kilo avant une augmentation de \( 12\%\). Quel est son nouveau prix au kilo ?
                \begin{enumerate}
                    \item
                        \( 26\)€
                    \item
                        \( 1.68\)€
                    \item
                        \( 15.68\)€
                    \item
                        \( 12.32\)€
                \end{enumerate}
        \end{enumerate}
    \end{multicols}

\corrref{smath-0438}
\end{exercice}
