% This is part of Un soupçon de mathématique sans être agressif pour autant
% Copyright (c) 2014
%   Laurent Claessens
% See the file fdl-1.3.txt for copying conditions.

% This is part of Un soupçon de mathématique sans être agressif pour autant
% Copyright (c) 2014
%   Laurent Claessens
% See the file fdl-1.3.txt for copying conditions.

%--------------------------------------------------------------------------------------------------------------------------- 
\subsection*{Activité : somme de fractions}
%---------------------------------------------------------------------------------------------------------------------------

Compléter les phrases
\begin{center}
   \input{Fig_UKRHooEvocBg.pstricks}
\end{center}
\begin{enumerate}
    \item
        La partie grisée représente \( \dfrac{ 2 }{ \ldots }\) de l'aire totale.
    \item
        La partie hachurée représente \( \dfrac{ 1 }{ \ldots }\) de l'aire totale.
    \item
        En s'aidant du dessin, calculer la somme
        \begin{equation}
            \frac{ 2 }{ 3 }+\frac{1}{ 4 }=\ldots
        \end{equation}
\end{enumerate}

De \cite{NRHooXFvgpp4}

%+++++++++++++++++++++++++++++++++++++++++++++++++++++++++++++++++++++++++++++++++++++++++++++++++++++++++++++++++++++++++++ 
\section{Addition et soustraction}
%+++++++++++++++++++++++++++++++++++++++++++++++++++++++++++++++++++++++++++++++++++++++++++++++++++++++++++++++++++++++++++

\begin{Aretenir}
    Pour additionner (ou soustraire) des nombres en écriture fractionnaire :
    \begin{itemize}
        \item 
            on écrit les nombres avec le même dénominateur ;
        \item 
            on additionne (ou on soustrait) les numérateurs et on garde le dénominateur commun.
    \end{itemize}
\end{Aretenir}

\begin{example}
    Parfois le dénominateur commun n'est aucun des deux dénominateurs :
    \begin{equation}
        B=\frac{5}{ 6 }-\frac{ 3 }{ 14 }.
    \end{equation}
    Le plus petit dénominateur commun est \( 42\) : c'est le plus petit nombre à être multiple en même temps de \( 6\) et \( 14\). En choisissant cela,
    \begin{equation}
        B=\frac{ 5\times 7 }{ 6\times 7 }-\frac{ 3\times 3 }{ 14\times 3 }=\frac{ 35 }{ 42 }-\frac{ 9 }{ 42 }=\frac{ 26 }{ 42 }=\frac{ 13 }{ 21 }.
    \end{equation}
    
    Dans ce cas, il est peut-être plus simple de choisir le produit \( 14\times 6\) comme dénominateur commun (ça marche toujours) :
    \begin{equation}
        B=\frac{ 5\times 14 }{ 6\times 14 }-\frac{ 3\times 6 }{ 14\times 6 }=\frac{ 70 }{ 84 }-\frac{ 18 }{ 84 }=\frac{ 52 }{ 84 }=\frac{ 13 }{ 21 }.
    \end{equation}
    Prendre le produit des dénominateurs fonctionne toujours, mais mène à des calculs sur de plus grands nombres.    
\end{example}
