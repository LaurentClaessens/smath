% This is part of Un soupçon de mathématique sans être agressif pour autant
% Copyright (c) 2014
%   Laurent Claessens
% See the file fdl-1.3.txt for copying conditions.

\begin{exercice}\label{exosmath-0740}

    Un biscuit coûte \( 1\) euro et une bouteille d'eau coûte \( 2.5\) euros. Sarah souhaite constituer \( 12\) sacs contenant chacun un biscuit et une bouteille d'eau. 

    \begin{enumerate}
        \item
            Donner une expression donnant le prix total.
        \item
            Lesquelles parmi les expressions suivantes donnent le bon résultat ?
            \begin{enumerate}
                \item
                    \( A=12+12\times 2.5\)
                \item
                    \( B=12\times 2.5+1\)
                \item
                    \( C=12\times 3.5\).
            \end{enumerate}
    \end{enumerate}

\corrref{smath-0740}
\end{exercice}
