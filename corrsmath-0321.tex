% This is part of Un soupçon de mathématique sans être agressif pour autant
% Copyright (c) 2013
%   Laurent Claessens
% See the file fdl-1.3.txt for copying conditions.

\begin{corrige}{smath-0321}

    \begin{multicols}{2}
    \begin{equation*}
        \begin{array}[]{|c||ccccc|}
            \hline
             x&-\infty&&1/3&&+\infty\\
              \hline
              -3x+1&&+&0&-&\\ 
              \hline 
               \end{array}
           \end{equation*}


    \begin{equation*}
        \begin{array}[]{|c||ccccc|}
            \hline
             x&-\infty&&-2&&+\infty\\
              \hline
              2x+4&&-&0&+&\\ 
              \hline 
               \end{array}
           \end{equation*}

    \end{multicols}
    Pour l'inéquation, nous voyons dans le tableau de signe que les solutions sont \( x<\frac{1}{ 3 }\). Sous forme d'intervalle, ça donne
    \begin{equation}
        x\in\mathopen] -\infty ; \frac{1}{ 3 } \mathclose[.
    \end{equation}

\end{corrige}
