% This is part of Un soupçon de mathématique sans être agressif pour autant
% Copyright (c) 2012
%   Laurent Claessens
% See the file fdl-1.3.txt for copying conditions.

\begin{corrige}{Seconde-0082}


    \begin{multicols}{2}

        L'angle \( A\) du triangle \( ABC\) est coupé par les deux droites parallèles \( LK\) et \( BC\); nous sommes donc dans la situation du théorème de Thalès. Nous avons donc les égalités
        \begin{equation}
            \frac{ AB }{ AL }=\frac{ AC }{ AK }=\frac{ BC }{ LK }.
        \end{equation}
        Étant donné que \( K\) est au milieu de \( [AC]\), nous avons
        \begin{equation}
            \frac{ AC }{ AK }=2
        \end{equation}
        et donc aussi \( \frac{ AB }{ AL }=2\) et donc \( L\) est au milieu de \( [AB]\).

        Toujours pour la même raison de proportionnalité, \( [LK]\) a une longueur moitié de \( [BC]\) et donc a une longueur \( 8\).

        \columnbreak

        \begin{center}
%The result is on figure \ref{LabelFigThaleszlOKVq}. % From file ThaleszlOKVq
%\newcommand{\CaptionFigThaleszlOKVq}{<+Type your caption here+>}
\input{Fig_ThaleszlOKVq.pstricks}
        \end{center}

    \end{multicols}

\end{corrige}
