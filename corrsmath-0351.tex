% This is part of Un soupçon de mathématique sans être agressif pour autant
% Copyright (c) 2013
%   Laurent Claessens
% See the file fdl-1.3.txt for copying conditions.

\begin{corrige}{smath-0351}

Nous partons de l'équation générale \( f(x)=ax^2+bx+c\). Le fait que la droite d'équation \( y=2x-5\) soit tangente en abscisse \( x=3\) impose que la parabole passe par le point \( (3,1)\). Nous avons donc
\begin{equation}
    9a+3b+c=1.
\end{equation}
En ce qui concerne le coefficient directeur de la tangente, c'est \( 2\). Donc il faut que la dérivée de \( f\) en \( x=2\) soit égale à \( 2\), ce qui donne
\begin{equation}
    f'(3)=6a+b=2.
\end{equation}
Nous devons donc trouver \( a\), \( b\) et \( c\) vérifiant les équations
\begin{subequations}
    \begin{numcases}{}
        9a+3b+c=1\\
        6a+b=2.
    \end{numcases}
\end{subequations}
Il y a beaucoup de solutions possibles. Une façon d'en trouver une est de poser d'emblée \( b=0\) et voir ce qu'il se passe. Nous obtenons alors \( a=3\) et \( c=-26\). Donc la fonction suivante répond à la question :
\begin{equation}
    f(x)=3x^2-26.
\end{equation}

\end{corrige}
