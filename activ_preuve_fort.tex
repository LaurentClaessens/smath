% This is part of Un soupçon de mathématique sans être agressif pour autant
% Copyright (c) 2015
%   Laurent Claessens
% See the file fdl-1.3.txt for copying conditions.

%--------------------------------------------------------------------------------------------------------------------------- 
\subsection*{Activité : la démonstration pour les plus forts}
%---------------------------------------------------------------------------------------------------------------------------

Voici un triangle rectangle en \( B\) :

\begin{center}
\input{Fig_NHGWooBOAkGS.pstricks}
\end{center}
Le point \( K\) est le milieu de l'hypoténuse \( [AC]\). Nous donnons les longueurs \( AC=2\), \( LK=x\) et \( LB=y\).

Le but est de montrer que \( m\) est le moitié de l'hypoténuse, c'est à dire \( m=1\).

\begin{enumerate}
    \item
        Exprimer les longueurs \( AL\) et \( KC\) en fonction de \( x\) et \( y\).
    \item
        Quel segment a pour longueur au carré le nombre \( x^2+y^2\) ?
    \item
        Exprimer les longueurs \( AB\) et \( BC\) en termes de \( x\) et \( y\). (écrire les égalités de Pythagore dans \( ALB\) et \( LCB\))
    \item
        Écrire l'égalité de Pythagore dans le triangle \( ABC\).
    \item
        Remplacer dans l'égalité de Pythagore toutes les mesures en termes de \( x\) et \( y\).
    \item
        Calculer beaucoup et déduire que \( m=1\).
\end{enumerate}
Aide : on peut utiliser l'égalité \( (1-x)^2+(1+x)^2=2+2x^2\).
