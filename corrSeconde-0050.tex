% This is part of Un soupçon de mathématique sans être agressif pour autant
% Copyright (c) 2012
%   Laurent Claessens
% See the file fdl-1.3.txt for copying conditions.

\begin{corrige}{Seconde-0050}

    Il s'agit de résoudre l'équation
    \begin{equation}
        x^2+1=3,
    \end{equation}
    c'est à dire \( x^2=2\). Cette équation \emph{a deux solutions} : \( x=\sqrt{2}\) et \( x=-\sqrt{2}\).

\end{corrige}
