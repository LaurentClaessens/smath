% This is part of Un soupçon de mathématique sans être agressif pour autant
% Copyright (c) 2013
%   Laurent Claessens
% See the file fdl-1.3.txt for copying conditions.

\begin{exercice}\label{exosmath-0489}

    À propos d'une fonction \( f\) définie sur \( \mathopen[ -2 , 4 \mathclose]\), nous avons les informations suivantes :
    \begin{equation*}
        \begin{array}[]{|c||c|c|c|c|c|c|c|}
            \hline
            x&-2&-1&0&1&2&3&4\\
            \hline\hline
            f(x)&0&1&2&1&3&4&5\\
            \hline
        \end{array}
    \end{equation*}
    Répondre par «vrai», «faux» ou «on ne sait pas» aux questions suivantes :
    \begin{enumerate}
        \item
            \( f(3)=2\).
        \item
            La fonction est croissante sur \( \mathopen[ -2 , 0 \mathclose]\).
        \item
            \( f(2)=3\).
        \item
            La fonction \( f\) a un maximum en \( x=4\) et ce maximum vaut \( 5\).
        \item
            L'image de \( 0\) par \( f\) est \( 2\).
        \item
            L'antécédent de \( 1\) par \( f\) est \( -1\).
    \end{enumerate}

\corrref{smath-0489}
\end{exercice}
