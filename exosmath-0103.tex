% This is part of Un soupçon de mathématique sans être agressif pour autant
% Copyright (c) 2012,2014
%   Laurent Claessens
% See the file fdl-1.3.txt for copying conditions.

\begin{exercice}\label{exosmath-0103}

    \begin{center}
   \input{Fig_figureYWEhCkv.pstricks}
    \end{center}

        \begin{enumerate}
            \item
                À partir du dessin ci-contre, donner les coordonnées des vecteurs \( \vect{ u }\), \( \vect{ v }\), \( \vect{ w }\), \( \vect{ AB }\) et \( \vect{ CA }\).
            \item
                Dessiner deux représentations du vecteur de coordonnées \( \begin{pmatrix}
                    1    \\ 
                    3    
                \end{pmatrix}\).

            \item
                Pour quel point \( E\) a-t-on \( \vect{ AE }=-\vect{ u }\) ? Quelles sont les coordonnées du vecteur \( \vect{ AE }\) ?
             
        \end{enumerate}

    Faire l'exercice 21 de la page 212.

\corrref{smath-0103}
\end{exercice}
