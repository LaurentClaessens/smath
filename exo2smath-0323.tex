% This is part of Un soupçon de mathématique sans être agressif pour autant
% Copyright (c) 2015
%   Laurent Claessens
% See the file fdl-1.3.txt for copying conditions.

\begin{exercice}\label{exo2smath-0323}

    Voici la liste des ingrédients nécessaires à la fabrication du mystérieux beignet syldave aux pommes.
    \begin{center}
    \begin{tabular}[]{|c||c|}
        \hline
        Ingrédients&quantité\\
        \hline\hline
        farine&\SI{50}{\gram}\\
        \hline
        sucre&\SI{20}{\gram}\\
        \hline
        pomme&\SI{150}{\gram}\\
        \hline
        ingrédient syldave secret&\SI{30}{\gram}\\
        \hline
    \end{tabular}
    \end{center}
    \begin{enumerate}
        \item
            Pour quelle quantité de beignet cette liste est-elle établie ?
        \item
            Quelle est la proportion de farine dans ce beignet ? Donner une approximation en pourcentage.
        \item
            Quelle est la proportion d'ingrédient mystère dans ce beignet ? Donner une approximation en pourcentage.
        \item
            Michel respecte toutes les quantités, sauf qu'il mets \SI{40}{\gram} de sucre au lieu de \SI{20}{\gram}. Quelle sera la proportion de sucre dans son beignet ?
    \end{enumerate}

\corrref{2smath-0323}
\end{exercice}
