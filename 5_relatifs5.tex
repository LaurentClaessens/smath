% This is part of Un soupçon de mathématique sans être agressif pour autant
% Copyright (c) 2014-2015
%   Laurent Claessens
% See the file fdl-1.3.txt for copying conditions.


% This is part of Un soupçon de mathématique sans être agressif pour autant
% Copyright (c) 2014
%   Laurent Claessens
% See the file fdl-1.3.txt for copying conditions.

%--------------------------------------------------------------------------------------------------------------------------- 
\subsection*{Activité : quelque situations plus petites que zéro}
%---------------------------------------------------------------------------------------------------------------------------

\begin{enumerate}
    \item
        Fantine a une dette de \( 15\) sous vis-à-vis des Thénardier. Elle parvient à leur envoyer \( 12\) sous, mais ensuite elle contracte une nouvelle dette de \( 6\) sous. Quelle est sa nouvelle situation ?
    \item
        Le premier Harry Potter a été écrit en 1997. Il y a combien d'année de cela ? Écrire l'opération effectuée. L'odyssée aurait été écrit environ \( 800\) ans avant notre ère. Il y a combien de temps de cela ? Écrire l'opération effectuée.
    \item
        Un plongeur saute d'un plongeoir situé à \SI{5}{\meter} et touche le fond de la piscine de \SI{4}{\meter} de profondeur. Quelle a été la distance verticale totale parcourue ? Écrire le calcul effectué.

\end{enumerate}



%+++++++++++++++++++++++++++++++++++++++++++++++++++++++++++++++++++++++++++++++++++++++++++++++++++++++++++++++++++++++++++ 
\section{Les opposés}
%+++++++++++++++++++++++++++++++++++++++++++++++++++++++++++++++++++++++++++++++++++++++++++++++++++++++++++++++++++++++++++

Nous connaissons déjà les nombres «positifs» $1$, $2$, $3$, \( 7.52\), \( \dfrac{ 7 }{ 3 }\), etc.
\begin{definition}
    Nous ajoutons les nombres \( -1, -2, -3\); \( -7.52\); \( -\dfrac{ 7 }{ 3 }\), etc. 

    Pour chaque nombre positif \( x\), on introduit un nouveau nombre noté «\( -x\)». Nous disons que \( x\) et \( -x\) sont \defe{opposés}{opposé} l'un de l'autre.
\end{definition}

\begin{example}
    \begin{enumerate}
        \item
            \( -4\) est l'opposé de $4$,
        \item
            \( -9.43\) est l'opposé de \( 9.43\),
        \item
            \( 10\) est l'opposé que \( -10\).
    \end{enumerate}
\end{example}

\begin{definition}
    L'ensemble de tous les nombres positifs et négatifs est l'ensemble des \defe{nombres relatifs}{nombre!relatifs}.
\end{definition}

\begin{example}
    \begin{itemize}
        \item les dettes,
        \item les étages au sous-sol,
        \item Les températures.
    \end{itemize}
\end{example}

% This is part of Un soupçon de mathématique sans être agressif pour autant
% Copyright (c) 2014
%   Laurent Claessens
% See the file fdl-1.3.txt for copying conditions.

\begin{enumerate}
    \item
        Tracer une droite graduée d'origine \( O\) plaçant points \( A(3)\), \( B(4)\) et \( D(7)\). 

CZFJooUDaKCj

    \item
        Construire ensuite le point \( C\) tel que \( A\) soit le milieu de \( [BC]\). Quelle est l'absisse du point \( \)<++>
        
\end{enumerate}

%+++++++++++++++++++++++++++++++++++++++++++++++++++++++++++++++++++++++++++++++++++++++++++++++++++++++++++++++++++++++++++ 
\section{Droite graduée}
%+++++++++++++++++++++++++++++++++++++++++++++++++++++++++++++++++++++++++++++++++++++++++++++++++++++++++++++++++++++++++++


Nous étendons la demi-droite graduée pour y ajouter les abscisses «négatives» :
\begin{center}
    \input{Fig_ZMSWooTAPggA.pstricks}
\end{center}

\begin{definition}
    La \defe{distance à zéro}{distance!à zéro} d'un nombre relatif \( x\) est la distance sur la droite graduée entre le point d'abscisse \( x\) et l'origine.
\end{definition}

\begin{example}
    Sur le dessin suivant est représenté la distance à zéro de \( -3\) et \( 4\).
    \begin{center}
        \input{Fig_JSRHooOdlPDT.pstricks}
    \end{center}
    L'abscisse de \( A\) est \( -3\) et celle de \( B\) est \( 4\).
\end{example}

\begin{example}
    \begin{enumerate}
        \item
            La distance à zéro de \( 4\) est \( 4\),
        \item
            La distance à zéro de \( -\dfrac{ 7 }{ 3 }\) est \( \dfrac{ 7 }{ 3 }\),
        \item
            La distance à zéro de \( 7.5\) est \( 7.5\),
        \item
            La distance à zéro de \( -7.5 \) est \( 7.5\),
    \end{enumerate}
\end{example}

% This is part of Un soupçon de mathématique sans être agressif pour autant
% Copyright (c) 2015
%   Laurent Claessens
% See the file fdl-1.3.txt for copying conditions.

%--------------------------------------------------------------------------------------------------------------------------- 
\subsection*{Activité : classer des dates}
%---------------------------------------------------------------------------------------------------------------------------

Classer les événements suivants par ordre chronologique.
\begin{itemize}
    \item Le premier pas sur la Lune (\( 1969\)).
    \item La conquête des Gaules (\( -51\))
    \item Décès de Catule (\( -54\))
    \item Couronnement de Charlemagne (\( 800\))
    \item Fondation du comté de Bourgogne (\( 986\))
    \item Arrivée d'Aristote à Mytilène (\( -345\))
    \item Disparition des dinosaures (il y a \( 65\) millions d'années)
    \item Formation des Pyrénées (il y a \( 40\) millions d'années)
\end{itemize}


%+++++++++++++++++++++++++++++++++++++++++++++++++++++++++++++++++++++++++++++++++++++++++++++++++++++++++++++++++++++++++++ 
\section{Comparaison de nombres relatifs}
%+++++++++++++++++++++++++++++++++++++++++++++++++++++++++++++++++++++++++++++++++++++++++++++++++++++++++++++++++++++++++++

\begin{Aretenir}
    Pour comparer des nombres relatifs,
    \begin{itemize}
\item
Un nombre relatif négatif est toujours inférieur à un nombre relatif positif. 
\item
    De deux nombres négatifs, le plus grand est le plus proche de zéro.
    \end{itemize}
\end{Aretenir}

\begin{example}
    Comparer les nombres \( -9.9\) et \( -7.7\).
       
    Ce sont deux nombres négatifs, donc il faut trouver le plus proche de zéro.
\end{example}

%+++++++++++++++++++++++++++++++++++++++++++++++++++++++++++++++++++++++++++++++++++++++++++++++++++++++++++++++++++++++++++ 
\section{Dans le plan}
%+++++++++++++++++++++++++++++++++++++++++++++++++++++++++++++++++++++++++++++++++++++++++++++++++++++++++++++++++++++++++++

% This is part of Un soupçon de mathématique sans être agressif pour autant
% Copyright (c) 2015
%   Laurent Claessens
% See the file fdl-1.3.txt for copying conditions.

{\Large Cette activité ne va pas. Il faut trouver un graphique plus simple.}

%--------------------------------------------------------------------------------------------------------------------------- 
\subsection*{Activité : placer dans un repère}
%---------------------------------------------------------------------------------------------------------------------------

Répondre aux questions en se servant du graphique :

\includegraphics[width=0.7\textwidth]{EcartsTempSurface2013.pdf}

De \url{http://fr.wikipedia.org/wiki/Réchauffement_climatique}


\begin{enumerate}
    \item
        Entre \( 1900\) et \( 1910\), on observe une belle dégringolade. De combien de degré environ ?
    \item
        Quel est l'écart de température entre l'année la plus chaude et la plus froide enregistrée ?
    \item
        Quelle est l'écart de température entre l'année \( 1980\) et \( 1920\) ?
    \item
        Donner quelques années pour lesquelles la température ont été (environ) dans la moyenne \( 1961-1990\).
\end{enumerate}


\begin{definition}
    Deux droites graduées perpendiculaires de même origine est un \defe{repère orthogonal}{repère!orthogonal}. 
\end{definition}

\begin{Aretenir}
    Pour repérer un point dans le plan, on donne deux nombre : l'abscisse et l'ordonnée.
\end{Aretenir}

\begin{example}
    \begin{center}
        \input{Fig_QBDMooGskfKN.pstricks}
    \end{center}
    Sur ce dessin,
    \begin{itemize}
        \item les coordonnées du point \( A\) sont \(  (3;2)  \)
        \item les coordonnées du point \( B\) sont \(  (-5;3)  \)
        \item les coordonnées du point \( C\) sont \(  (1;-4)  \)
        \item les coordonnées du point \( H\) sont \(  (-3.5;0)  \)
    \end{itemize}
\end{example}

\begin{definition}
    \begin{itemize}
        \item 
            L'axe horizontal est appelé l'\defe{axe des abscisses}{axe!abscisse}.
        \item
            L'axe vertical est appelé l'\defe{axe des ordonnées}{axe!ordonnées}.
    \end{itemize}
\end{definition}


