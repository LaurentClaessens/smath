% This is part of Un soupçon de mathématique sans être agressif pour autant
% Copyright (c) 2013
%   Laurent Claessens
% See the file fdl-1.3.txt for copying conditions.

\begin{exercice}\label{exosmath-0563}

    Voici le tableau de signe d'une fonction \( f\) :
    \begin{equation*}
        \begin{array}[]{c|ccccccc}
            x&-\infty&&-3&&5&&+\infty\\
            \hline
            f(x)&&+&0&-&0&+&\\
        \end{array}
    \end{equation*}
    À partir de là répondre aux affirmations suivantes par «vrai», «faux» ou «on ne peut pas savoir».
    \begin{enumerate}
        \item
            \( f(2)=6\).
        \item
            L'équation \( f(x)=1\) admet exactement deux solutions.
        \item
            La fonction \( f\) est affine.
        \item
            L'inéquation \( f(x)<0\) a pour ensemble de solutions \( \mathopen] -4 ;5 \mathclose[\).
        \item
            Le point \( A(0;5)\) appartient à la courbe représentative de la fonction \( f\).
        \item
            Si \( f(1)=-4\), alors le minimum de la fonction \( f\) sur \( \eR\) est \( -4\).
    \end{enumerate}
\corrref{smath-0563}
\end{exercice}
