% This is part of Un soupçon de mathématique sans être agressif pour autant
% Copyright (c) 2012
%   Laurent Claessens
% See the file fdl-1.3.txt for copying conditions.

\begin{exercice}\label{exoSeconde-0045}

    Soit \( ABCD\) un parallélogramme de centre \( O\), et \( I\) le milieu de \( [AD]\). Nous nommons \( J\) le point d'intersection des droites \( IC\) et \( BD\).
    \begin{enumerate}
        \item
            Dessiner la situation.
        \item
            Que représente le point \( J\) pour le triangle \( ADC\) ? 
        \item
            Déduire que la droite \( AJ\) coupe le segment \( [DC]\) en son milieu.
    \end{enumerate}
    Indice : dans un triangle, les médianes des côtés issus des sommets opposés sont concourantes.

\corrref{Seconde-0045}
\end{exercice}
