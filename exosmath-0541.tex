% This is part of Un soupçon de mathématique sans être agressif pour autant
% Copyright (c) 2013
%   Laurent Claessens
% See the file fdl-1.3.txt for copying conditions.

\begin{exercice}\label{exosmath-0541}


\begin{wrapfigure}{r}{2.0cm}
   \vspace{-0.5cm}        % à adapter.
   \centering
   \input{Fig_WGhpCVm.pstricks}
\end{wrapfigure}

    Le but de cet exercice est de calculer la somme \( 1+2+3+5+\ldots+1000\). Pour cela nous commençons par regarder \( S=1+2+3+4\). La figure ci-contre  montre comment on peut arranger \( 1+2+3+4\) petits carrés.

    Compléter le dessin pour placer deux fois \( 1+2+3+4\) petits carrés en un beau rectangle. Calculer l'aire du rectangle. Conclure.

\corrref{smath-0541}
\end{exercice}
