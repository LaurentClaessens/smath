% This is part of Un soupçon de mathématique sans être agressif pour autant
% Copyright (c) 2012
%   Laurent Claessens
% See the file fdl-1.3.txt for copying conditions.

\begin{corrige}{Seconde-0011}

    Le milieu de \( [AC]\) est le point
    \begin{equation}        \label{EqkQlsmw}
        \big( \frac{ 9+2 }{2},\frac{ 5+2 }{2} \big)=\big( \frac{ 11 }{2},\frac{ 7 }{2} \big).
    \end{equation}
    Nous donnons à \( D\) les coordonnées \( D=(x,y)\) et nous allons chercher à déterminer \( x\) et \( y\) de telle sorte que les segments \( [AC]\) et \( [DB]\) aient même milieu. Le milieu de \( [BD]\) est
    \begin{equation}
        \big( \frac{ x+6 }{2},\frac{ y }{2} \big).
    \end{equation}
    Pour que ce point soit le même que le point \eqref{EqkQlsmw}, il faut avoir en même temps
    \begin{equation}
        \frac{ x+6 }{ 2 }=\frac{ 11 }{2}
    \end{equation}
    et
    \begin{equation}
        \frac{ y }{ 2 }=\frac{ 7 }{2}.
    \end{equation}
    En multipliant par deux, la seconde équation donne tout de suite \( y=2\); la première donne \( x+6=11\) et donc \( x=5\). Le seul point \( D\) tel que \( ABCD\) soit un parallélogramme est le point \( D=(5,7)\).

\end{corrige}
