% This is part of Un soupçon de mathématique sans être agressif pour autant
% Copyright (c) 2012
%   Laurent Claessens
% See the file fdl-1.3.txt for copying conditions.

\begin{exercice}\label{exosmath-0105}

    \begin{multicols}{2}

    Dessiner sur le dessin ci-contre les vecteurs
    \begin{enumerate}
        \item
            \( \vect{ AB }+\vect{ BC }\)
        \item
            \( \vect{ EA }+\vect{ BC }\)
        \item
             \( \vect{ AC }+\vect{ DE }\)
        \item
            \( \vect{ BA }+\vect{ DC }+\vect{ CE }\)
         \item
             \( \vect{ CB }+\vect{ BE }+\vect{ EA }+\vect{ AD }+\vect{ DC }\)
         \item
             \( 2\vect{ EA }\)
         \item
             \( -\vect{ BA }\)
         \item
             \( \frac{ 1 }{2}\vect{ DA }\)
    \end{enumerate}

    \columnbreak   
%The result is on figure \ref{LabelFigfigureAEcZqpP}. % From file figureAEcZqpP
%\newcommand{\CaptionFigfigureAEcZqpP}{<+Type your caption here+>}
    \begin{center}
\input{Fig_figureAEcZqpP.pstricks}
    \end{center}

    \end{multicols}

\corrref{smath-0105}
\end{exercice}
