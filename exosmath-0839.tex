% This is part of Un soupçon de mathématique sans être agressif pour autant
% Copyright (c) 2014
%   Laurent Claessens
% See the file fdl-1.3.txt for copying conditions.

\begin{exercice}\label{exosmath-0839}

    L'énoncé suivant est faux : « un nombre divisible en même temps par \( 3\) et par \( 7\) est un multiple de \( 42\)». Pour trouver un contre-exemple, que devez-vous donner ?
    \begin{enumerate}
        \item
            Un nombre divisible par \( 3\) mais pas par \( 42\).
        \item
            Un nombre multiple de \( 42\) mais divisible ni par \( 3\) ni par \( 7\).
        \item
            Un nombre divisible par \( 3\) et par \( 7\) mais qui n'est pas multiple de \( 42\).
    \end{enumerate}
    Donner un tel contre-exemple.

\corrref{smath-0839}
\end{exercice}
