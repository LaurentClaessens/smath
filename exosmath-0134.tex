% This is part of Un soupçon de mathématique sans être agressif pour autant
% Copyright (c) 2012
%   Laurent Claessens
% See the file fdl-1.3.txt for copying conditions.

\begin{exercice}\label{exosmath-0134}

    Marc veut choisir un abonnement pour son nouveau téléphone portable. Les formules suivantes sont proposées :
    \begin{center}
    \begin{tabular}[]{|c||c|c|}
        \hline
        &Forfait pour \( 2\) heures de communication& Supplément par minute de dépassement\\
        \hline\hline
        Formule 1&\( 30\)€&\( 0.25\)€\\
        \hline
        Formule 2&\( 15\)€&\( 0.75\)€\\
        \hline
        Formule 3&\( 20\)€&\( 0.5\)€\\
        \hline
    \end{tabular}
    \end{center}

    Marc sait qu'il va dépasser les deux heures de communications, et voudrait savoir quelle est la formule la plus avantageuse en fonction du nombre \( x\) de minutes de dépassement. Nous notons \( f_1\), \( f_2\) et \( f_3\) les fonctions qui donnent le coût total à payer en fonction du nombre de minutes de dépassement.
    \begin{enumerate}
        \item
            Écrire les fonctions \( f_1\), \( f_2\) et \( f_3\).
        \item   \label{ItemUDAhHnx}
            Résoudre par le calcul les équations \( f_1(x)=f_2(x)\), \( f_1(x)=f_3(x)\) et \( f_2(x)=f_3(x)\).
        \item
            Dessiner dans un même repère les graphes de ces trois fonctions pour \( x\in \mathopen[ 0 , 60 \mathclose]\). 
        \item
            Vérifier graphiquement les solutions obtenues au point \ref{ItemUDAhHnx}.
        \item
            Sur le même dessin, tracer en couleur le graphique de la fonction qui à chaque \( x\) fait correspondre le tarif le plus avantageux pour une personne qui dépasse son forfait de \( x\) minutes.
        \item
            Pour le mois de décembre, Marc estime qu'il va téléphoner \( 153\) minutes. Quelle formule devrait-il choisir ?
    \end{enumerate}

\corrref{smath-0134}
\end{exercice}

% TODO : calculer le milieu du segment --> Calculer les coordonnées du milieu du segment.
