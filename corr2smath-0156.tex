% This is part of Un soupçon de mathématique sans être agressif pour autant
% Copyright (c) 2015
%   Laurent Claessens
% See the file fdl-1.3.txt for copying conditions.

\begin{corrige}{2smath-0156}

    \begin{enumerate}
        \item
            Le diamètre du cercle \( \mC_2\) est \( 10-x\).
        \item
            La circonférence d'un cercle s'obtient en multipliant le diamètre par \( \pi\). Les trois circonférences sont donc :
            \begin{itemize}
                \item \( \mC_1\) : \( \pi x\)
                \item \( \mC_2\) : \( \pi\times(10-x)\)
                \item \( \mC_3\) : \( 10\pi\).
            \end{itemize}
            La somme du premier et du deuxième est :
            \begin{subequations}
                \begin{align}
                    \pi x+\pi(10-x)&=\pi x+10 \pi-\pi x&&\text{développement de \( \pi(10-x)\)}\\
                    &=10\pi&&\text{\( \pi x-\pi x\) disparait.}
                \end{align}
            \end{subequations}
    \end{enumerate}

\end{corrige}
