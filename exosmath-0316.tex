% This is part of Un soupçon de mathématique sans être agressif pour autant
% Copyright (c) 2013
%   Laurent Claessens
% See the file fdl-1.3.txt for copying conditions.

%TODO: remettre le cite
\begin{exercice}\label{exosmath-0316}
    %[\cite{ZNloWAM}]

Dans la mesure de la précision de la figure \ref{LabelFigYAaXJqQ}, répondre aux questions suivantes.
\begin{enumerate}
    \item
        Déterminer l'image de \( 0\) et de \( -3\) par la fonction \( f\).
    \item
        Résoudre graphiquement l'équation \( f(x)=0\).
    \item
        Résoudre graphiquement l'inéquation \( f(x)<0\).
    \item
        Déterminer graphiquement l'antécédent de \( 1\)
    \item
        Déterminer graphiquement l'ensemble des réels \( x\) tels que \( 1\leq f(x)\leq 3\).
    \item
        Tracer sur le même graphique la droite d'équation \( y=x+5\).
\end{enumerate}
\newcommand{\CaptionFigYAaXJqQ}{Le graphe de la fonction \( f\) étudiée dans l'exercice \ref{exosmath-0316}.}
\input{Fig_YAaXJqQ.pstricks}

Nous savons que le graphique est celui de la fonction \( f(x)=\frac{1}{ x-a }+2\). En regardant le graphique, déterminer la valeur interdite et donc la valeur de \( a\).

Les questions suivantes sont à résoudre par le calcul, en utilisant la valeur de \( a\) que nous venons de trouver.
\begin{enumerate}
    \item
        Résoudre par le calcul \( f(x)=0\).
    \item
        Résoudre par le calcul l'inéquation \( f(x)<0\).
    \item
        Résoudre par le calcul l'inéquation \( f(x)\geq 3\).
    \item
        Résoudre par le calcul l'équation \( f(x)=x+5\).
    \item
        Résoudre par le calcul l'inéquation \( f(x)<x+5\).
\end{enumerate}

\corrref{smath-0316}
\end{exercice}
