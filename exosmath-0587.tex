% This is part of Un soupçon de mathématique sans être agressif pour autant
% Copyright (c) 2013
%   Laurent Claessens
% See the file fdl-1.3.txt for copying conditions.

\begin{exercice}\label{exosmath-0587}

    Quelles sont les règles du jeu décrit par le programme suivant ?

    \begin{fmpage}{0.9\linewidth}

    \( x\) est un nombre entier aléatoire entre \( 0\) et \( 100\)

    \( n=0\)

    Tant que \( n\) est différent de \( x\), faire :

    \hspace{0.5cm} demander la valeur de \( n\)

    \hspace{0.5cm} si \ldots\ldots\ldots\ldots :

    \hspace{1cm} Écrire «plus grand» 

    \hspace{0.5cm} si \ldots\ldots\ldots\ldots :

    \hspace{1cm} Écrire «plus petit» 

    Écrire «gagné !» 

\end{fmpage}

\corrref{smath-0587}
\end{exercice}
