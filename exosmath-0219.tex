% This is part of Un soupçon de mathématique sans être agressif pour autant
% Copyright (c) 2012
%   Laurent Claessens
% See the file fdl-1.3.txt for copying conditions.

\begin{exercice}\label{exosmath-0219}

    Une classe de terminale compte \( 33\) élèves dont \( 12\) filles. La classe compte \( 4\) élèves majeurs, dont trois garçons.
    \begin{enumerate}
        \item
            Compléter le tableau suivant :
            \begin{center}
            \begin{tabular}[]{|c||c|c|c|}
                \hline
                &Filles&Garçons&Total\\
                \hline\hline
                Majeur(e)&&3&4\\\hline
                Mineur(e)&&&\\
                \hline
                Total&12&&33\\
                \hline
            \end{tabular}
            \end{center}
        \item
            Quel est le pourcentage de filles dans la classe ?
        \item
            Quelle est la fréquence des garçons majeurs dans la classe ?
        \item
            Représenter les données sous formes de diagrammes d'ensemble, et sous forme d'arbre.
        \item
            On tire un élève au hasard. Quelle est la probabilité que ce soit
            \begin{enumerate}
                \item
                    une fille mineure ?
                \item
                    un(e) mineur(e) ?
                \item
                    une fille ?
                \item
                    une fille ou un(e) mineur(e) ?
            \end{enumerate}
            
    \end{enumerate}
    <++>

\corrref{smath-0219}
\end{exercice}
