% This is part of Un soupçon de mathématique sans être agressif pour autant
% Copyright (c) 2012
%   Laurent Claessens
% See the file fdl-1.3.txt for copying conditions.

\documentclass[a4paper,12pt,draft]{article}


\usepackage{ifthen}
\usepackage{calc}

\usepackage{latexsym}
\usepackage{amsfonts}
\usepackage{amsmath}
\usepackage{amsthm}
\usepackage{amssymb}
\usepackage{bbm}
\usepackage{mathrsfs}           
\usepackage{mathabx}           % Pour \divides

\usepackage[fr]{exocorr}

\usepackage[utf8]{inputenc}
\usepackage[T1]{fontenc}
\usepackage{textcomp}
\usepackage{lmodern}
\usepackage[a4paper,margin=2cm]{geometry} 
\usepackage[english,frenchb]{babel}



\begin{document}

\corrPosition{0}

\thispagestyle{empty}

\large
\begin{center}
    Feuille d'exercices numéro 1
\end{center}

%\vspace{0.5cm}

\tiny
\begin{center}
    Toutes les réponses doivent être justifiées soit par un calcul soit par un raisonnement clairement rédigé.
\end{center}
\normalsize

\Exo{Seconde-0023}
\Exo{Seconde-0024}
\Exo{Seconde-0030}
\Exo{Seconde-0031}
\Exo{Seconde-0026}
\Exo{Seconde-0025}
\Exo{Seconde-0027}
\Exo{Seconde-0029}
\Exo{Seconde-0028}

\end{document}

