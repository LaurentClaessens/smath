% This is part of Un soupçon de mathématique sans être agressif pour autant
% Copyright (c) 2013
%   Laurent Claessens
% See the file fdl-1.3.txt for copying conditions.

\begin{exercice}\label{exosmath-0570}

\begin{wrapfigure}{r}{7cm}
   \vspace{-0.5cm}        % à adapter.
   \centering
   \input{Fig_FFyFBoe.pstricks}
\end{wrapfigure}

    Nous considérons la fonction \( f\) dont la représentation graphique est donnée ci-contre :

    \begin{enumerate}
        \item
            Résoudre \( f(x)=1\) et \( f(x)=2\).
        \item
            Est-ce que la fonction \( f\) est décroissante sur l'intervalle \( \mathopen[ -2 ;-1 \mathclose]\) ?
        \item
            Donner les antécédents de zéro par la fonction \( f\).
        \item
            Quel est le maximum de \( f\) sur l'intervalle \( \mathopen[ -1 ;1 \mathclose]\) ?
        \item 
            Combien vaut \( f(0)\) ?
        \item
            Dresser le tableau de variation de \( f\).
        \item
            Albert prétend que c'est le graphe de la fonction \( x\mapsto 2x\). Il argumente : « le graphe passe par l'origine, donc c'est linéaire; de plus elle passe par le point \( (\frac{ 1 }{2};1)\) dont c'est \( 2x\)». Qu'en penser ?
    \end{enumerate}

\corrref{smath-0570}
\end{exercice}
