% This is part of Un soupçon de mathématique sans être agressif pour autant
% Copyright (c) 2012
%   Laurent Claessens
% See the file fdl-1.3.txt for copying conditions.

\begin{exercice}\label{exoPremiere-0032}

À propos de la parabole de la figure \ref{LabelFigParabolesfTKFw}. 
\newcommand{\CaptionFigParabolesfTKFw}{La parabole de l'exercice \ref{exoPremiere-0032}.}
\input{Fig_ParabolesfTKFw.pstricks}
\begin{enumerate}
    \item
        Quel est l'antécédent de \( 1\) par la fonction ? Autrement dit : pour quelle(s) valeur(s) de \( x\) la courbe vaut-elle \( 1\) ?
    \item
        En quel \( x\) se trouve le sommet de la courbe ?
    \item
        Remarquer que le sommet se trouve «au milieu» des deux antécédents de \( 1\).
    \item
        Tracer les deux antécédents de \( 2\) (avec une règle). Est-ce que le sommet est encore au milieu des deux ?
\end{enumerate}


\corrref{Premiere-0032}
\end{exercice}
