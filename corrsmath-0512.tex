% This is part of Un soupçon de mathématique sans être agressif pour autant
% Copyright (c) 2013
%   Laurent Claessens
% See the file fdl-1.3.txt for copying conditions.

\begin{corrige}{smath-0512}

    La règle d'or des graphiques dit que le point \( (a,b)\) est sur le graphe de \( f\) si et seulement si \( f(a)=b\). Il suffit donc de calculer un peu.
    \begin{enumerate}
        \item
            \( f(0)=0\), donc le point \( A(0;1)\) n'est pas sur le graphe (c'est le point \( (0,0)\) qui est dessus).
        \item
            \( f(2)=4-6=-2\), donc le point \( B(2;-2)\) est sur le graphe.
        \item
            \( f(-3)=9-(-9)=9+9=18\), donc le point \( C(-3;0)\) n'est pas sur le graphe (par contre le point \( (-3;18)\) est dessus)
        \item
            \( f(\frac{ 1 }{2})=\frac{1}{ 4 }-\frac{ 3 }{ 2 }=-\frac{ 5 }{ 4 }\). Donc le point \( D\) n'est pas sur le graphe.
    \end{enumerate}


\end{corrige}
