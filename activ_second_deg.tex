%This is part of Un soupçon de mathématique sans être agressif pour autant
% Copyright (c) 2014
%   Laurent Claessens
% See the file fdl-1.3.txt for copying conditions.

Bertrand l'artisan vend des pots de terre cuite sur le marché. Chaque pot lui coûte \( 2\)€ de matériel. Au début de sa carrière il avait fixé le prix à \( 10\)€ et il vendait \( 8\) pots par semaine. Avec le temps Bertrand a remarqué qu'à chaque baisse de \( 0.5\)€, il vendait \( 4\) pots supplémentaires.

\begin{enumerate}
    \item
        Donner le nombre de pots vendus ainsi que son bénéfice au début de sa carrière ainsi qu'après \( 1\), \( 2\) et \( 3\) baisses de prix (de \( 0.5\)€ chaque).
    \item
        Exprimer le nombre de pots vendus ainsi que le bénéfice après \( x\) baisses.
    \item
        Tracer un graphique.
    \item
        Après combien de baisses de prix Bertrand aura-t-il intérêt à cesser de baisser le prix ?
\end{enumerate}
