% This is part of Un soupçon de mathématique sans être agressif pour autant
% Copyright (c) 2014
%   Laurent Claessens
% See the file fdl-1.3.txt for copying conditions.

\begin{exercice}[\cite{JGWooXAVokw}]\label{exosmath-0894}

    Antoine a une collection de $126$ petites voitures. Les \( \frac{ 2 }{ 9 }\) des voitures sont vertes, les \( \frac{ 5 }{ 7 }\) des voitures sont rouges et les autres sont bleues.
    \begin{enumerate}
        \item
 Quelle fraction de sa collection représente les petites voitures bleues ?
 \item
 Combien a-t-il de voitures rouges, vertes et bleues ?
    \end{enumerate}

\corrref{smath-0894}
\end{exercice}
