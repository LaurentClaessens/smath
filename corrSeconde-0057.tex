% This is part of Un soupçon de mathématique sans être agressif pour autant
% Copyright (c) 2012
%   Laurent Claessens
% See the file fdl-1.3.txt for copying conditions.

\begin{corrige}{Seconde-0057}

    La hauteur d'un rectangle de \unit{12}{\centi\meter\squared} et de base \( x\) est \( \frac{ 12 }{ x }\)\unit{}{\centi\meter}. Exprimé sous forme de fonction, nous avons
    \begin{equation}
        h(x)=\frac{ 12 }{ x }.
    \end{equation}
    Le domaine de cette fonction est l'ensemble des nombres \emph{strictement} positifs : \( x>0\).

    Note : bien entendu, il est possible de calculer la valeur de \( h(x)\) pour \( x=-1\), mais la fonction représentant des longueurs, elle n'accepte pas les négatifs.

\end{corrige}
