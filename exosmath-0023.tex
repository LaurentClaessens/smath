% This is part of Un soupçon de mathématique sans être agressif pour autant
% Copyright (c) 2012
%   Laurent Claessens
% See the file fdl-1.3.txt for copying conditions.

\begin{exercice}\label{exosmath-0023}

    Une usine construit dix mille condensateurs par jour, et chacun a une probabilité \( 0.002\) d'être défectueux. Très à cheval sur la qualité, l'usine les teste tous.
    \begin{enumerate}
        \item
            Tester un condensateur coûte \( 0.1\) euro. Si il est défectueux, il est détruit et cette destruction coûte un euro. Quel est le coût moyen journalier des tests de qualité (contrôle plus destruction)? 
        \item
            Changement de technologie. Maintenant l'usine regroupe ses condensateurs par lots de \( 10\) et les teste le lot en une seule fois. Le test coûte \( 0.1\) euros et détermine si au moins une pièce du lot est défectueuse, sans être capable de déterminer \emph{laquelle}. Si le tests montre que le lot est défectueux, le lot complet est détruit en bloc (ce qui coûte \( 1\) euros).

            Combien ce système coûte par jour ?
    \end{enumerate}

\corrref{smath-0023}
\end{exercice}
