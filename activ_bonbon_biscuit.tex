% This is part of Un soupçon de mathématique sans être agressif pour autant
% Copyright (c) 2015
%   Laurent Claessens
% See the file fdl-1.3.txt for copying conditions.

%--------------------------------------------------------------------------------------------------------------------------- 
\subsection*{Activité : boîtes de bonbons et de biscuits}
%---------------------------------------------------------------------------------------------------------------------------

\begin{enumerate}
    \item   \label{ItemILDXooQXINUca}
        
On possède \( 36\) bonbons et \( 18\) biscuits, et on veut les emballer dans des boîtes identiques. Quel serait le contenu d'une boîte, et combien de telles boîtes pourra-t-on faire ?

\item
    Parmi les expressions suivantes, lesquelles sont égales à \( 18a+36b\) ?
    \begin{enumerate}
        \item
            \( 6(6a+3b)\). Note : par rapport à la question \ref{ItemILDXooQXINUca}, cette expression correspondrait à confectionner \( 6\) boîtes contenant chacune \( 6\) biscuits et \( 3\) bonbons.
        \item
            \( 18(a+b)\)
        \item
            \( 18a\times (a+2b)\)
        \item
            \( 18(a+2b)\)
    \end{enumerate}
\end{enumerate}
