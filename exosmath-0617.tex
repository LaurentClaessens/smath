% This is part of Un soupçon de mathématique sans être agressif pour autant
% Copyright (c) 2014
%   Laurent Claessens
% See the file fdl-1.3.txt for copying conditions.

\begin{exercice}\label{exosmath-0617}

\begin{wrapfigure}{r}{5.cm}
   \vspace{-0.5cm}        % à adapter.
   \centering
   \input{Fig_CFFyezr.pstricks}
\end{wrapfigure}

    Deux sous-marins veulent déterminer la position d'un porte-avion ennemi. Le sonar d'un sous-marin ne peut pas déterminer la position d'un objet, mais seulement la direction dans laquelle il se trouve. Dans un repère dont l'unité est la dizaine de kilomètres, la situation est dessinée ci-contre. Déterminer la position du porte-avion par le calcul.

    Est-il à portée de tir si les torpilles ont une portée de \unit{50}{\kilo\meter} ?


\corrref{smath-0617}
\end{exercice}
