% This is part of Un soupçon de mathématique sans être agressif pour autant
% Copyright (c) 2012-2015
%   Laurent Claessens
% See the file fdl-1.3.txt for copying conditions.

\documentclass[a4paper,10pt]{article}
% This is part of Un soupçon de mathématique sans être agressif pour autant
% Copyright (c) 2012-2014
%   Laurent Claessens
% See the file fdl-1.3.txt for copying conditions.


\usepackage{etex}
\usepackage{ifthen}
%\usepackage{pdfsync}       % This package is obsolete : compile with pdflatex -synctex=1 instead.

\usepackage{latexsym}
\usepackage{amsfonts}
\usepackage{amsmath}
\usepackage{amsthm}
\usepackage{amssymb}
\usepackage{bbm}
\usepackage{mathrsfs}           
\usepackage{mathabx}           % Pour \divides

\usepackage{framed} % pour oframed
\usepackage{wrapfig}

\usepackage{calc}   % Les dépendances de phystricks si on n'utilise que le pdf.
%\usepackage{pstricks,pst-eucl,pstricks-add,calc,pst-math}   % Les dépendances de phystricks. Peut être qu'il faut ajouter catchfile


% Les dépendances de phystricks en mode Tikz
\usepackage{tikz}
\usepackage{calc}
\usetikzlibrary{calc}
\usetikzlibrary{patterns}

\usetikzlibrary{external}
\tikzexternalize
%\newcommand{\tikzsetnextfilename}[1]{}
\newcounter{defHatch}
\newcounter{defPattern}
\setcounter{defHatch}{0}
\setcounter{defPattern}{0}


\usepackage{graphicx}                   % Pour l'inclusion d'image en pfd.

%\newcommand{\EpsOrPdfincludegraphics}[2][]{%
%        \ifpdf
%            \includegraphics[#1]{#2.png}
%        \else
%            \includegraphics[#1]{#2.eps}
%        \fi
%        }

\usepackage{subfigure}

\usepackage{fancyvrb}
\usepackage{stmaryrd}       % Pour le \obslash
\usepackage{xstring}        % Utilisé pour les références vers wikipédia
\usepackage{cases}
\usepackage{lscape}         % pour l'environnement landscape, utilisé dans la correction corr0076.tex
\usepackage{multicol}
\setlength{\columnseprule}{0.5pt}
\usepackage{import}         % Pour le hack qui sert à inclure GeomAnal

% TODO : n'en utiliser qu'un
\usepackage[normalem]{ulem}     % Pour le barré, commande \sout
\usepackage{soul}       % Pour le barré, commande \st

\usepackage[all]{xy}

\let\second\undefined      % le paquet amthabx définit \second
\let\degree\undefined       % le paquet amthabx définit \degree
%\usepackage[cdot,thinqspace,amssymb]{SIunits} 
\usepackage[parse-numbers=false,binary-units=true,detect-all=true,per-mode=symbol]{siunitx} 
\newcommand{\unit}[2]{\SI{#1}{#2}}
 % L'option amssymb sert à éviter un conflit avec la commande \square de amssymb. Note qu'elle n'est plus accessible. Si tu en as besoin, faudra RTFM
%ftp://ftp.belnet.be/packages/ctan/macros/latex/contrib/SIunits/SIunits.pdf

\usepackage[nottoc]{tocbibind}
\usepackage[numbers]{natbib}

%%%%%%%%%%%%%%%%%%%%%%%%%%
%
%   Trucs mathématiques
%
%%%%%%%%%%%%%%%%%%%%%%%%

% ENSEMBLES DE NOMBRES
\newcommand{\eA}{\mathbbm{A}}
\newcommand{\eC}{\mathbbm{C}}
\newcommand{\eD}{\mathbbm{D}}
\newcommand{\eE}{\mathbbm{E}}
\newcommand{\eF}{\mathbbm{F}}
\newcommand{\eG}{\mathbbm{G}}
\newcommand{\eH}{\mathbbm{H}}
\newcommand{\eK}{\mathbbm{K}}
\newcommand{\eL}{\mathbbm{L}}
\newcommand{\eM}{\mathbbm{M}}
\newcommand{\eN}{\mathbbm{N}}
\newcommand{\eP}{\mathbbm{P}}
\newcommand{\eQ}{\mathbbm{Q}}
\newcommand{\eR}{\mathbbm{R}}
\newcommand{\eZ}{\mathbbm{Z}}

% ENSEMBLES de fonctions
\newcommand{\aL}{\mathcal{L}}       % Les applications linéaires
\newcommand{\aC}{\mathcal{C}}       % Les fonctions C^1, C^2 etc



\newcommand{\mF}{\mathcal{F}}
\newcommand{\mC}{\mathcal{C}}
\newcommand{\mG}{\mathcal{G}}
\newcommand{\mI}{\mathcal{I}}
\newcommand{\mL}{\mathcal{L}}
\newcommand{\mS}{\mathcal{S}}   % Utilisé pour l'espace des fonctions Schwartz
\newcommand{\mZ}{\mathcal{Z}}


\newcommand{\mtu}{\mathbbm{1}}              % La matrice unité

\DeclareMathOperator{\val}{val}     % valuation d'un polynôme
%\DeclareMathOperator{\opp}{opp}    % Les nombres négatifs


%\newcommand{\efrac}[2]{\frac{ \displaystyle #1 }{\displaystyle #2 }}
%%%%%%%%%%%%%%%%%%%%%%%%%%
%
%   Numérotations en tout genre
%
%%%%%%%%%%%%%%%%%%%%%%%%

\setcounter{tocdepth}{2}        % Profondeur de la table des matières
\setcounter{secnumdepth}{2}     % Profondeur dans le texte

\renewcommand{\thesubsection}{\thesection.\alph{subsection}}

%%%%%%%%%%%%%%%%%%%%%%%%%%
%
%   Les lignes magiques pour le texte en français.
%
%%%%%%%%%%%%%%%%%%%%%%%%

\usepackage[utf8]{inputenc}
\usepackage[T1]{fontenc}

\usepackage{listingsutf8}
%\lstset{language=python,basicstyle=\footnotesize,tabsize=3,numbers=left,numberstyle=\tiny,frame=single,commentstyle=\ttfamily\color[rgb]{0,0,0.5},stringstyle=\color[rgb]{0,0.5,0},title=\lstname,inputencoding=utf8/latin1}
\lstset{language=python,basicstyle=\footnotesize,tabsize=3,frame=single,commentstyle=\ttfamily\color[rgb]{0,0,0.5},stringstyle=\color[rgb]{0,0.5,0},title=\lstname,inputencoding=utf8/latin1}

\usepackage[fr]{exocorr}
\usepackage{textcomp}
\usepackage{lmodern}
\usepackage[a4paper,margin=2cm]{geometry} 
\usepackage[english,frenchb]{babel}


\usepackage{hyperref}                           %Doit être appelé en dernier.
\hypersetup{
colorlinks=true,
linkcolor=blue,
urlcolor=magenta,     % couleur des url
filecolor=magenta   % couleur des textes qui sont des liens
}

% Il me semble que cette commande doit être définie après l'appel à Babel.
\newcommand{\Ieme}{\up{\lowercase{ième}}\xspace}

%%%%%%%%%%%%%%%%%%%%%%%%%%
%
%   Les théorèmes et choses attenantes
%
%%%%%%%%%%%%%%%%%%%%%%%%


\newcounter{numtho}
\newcounter{numprob}

\makeatletter
\@addtoreset{numtho}{chapter}
%\@addtoreset{CountExercice}{chapter}
\@addtoreset{chapter}{part}
\makeatother

\newlength{\EnvSpace}
\setlength{\EnvSpace}{9pt}      % C'est la distance que je veux mettre avant et après les théorèmes, remarques, etc.

\newtheoremstyle{MyTheorems}%
        {\EnvSpace}{\EnvSpace}%
        {\itshape}%
        {}%
        {\bfseries}{.}%
        {\newline}%
        {}%
\newtheoremstyle{MyExamples}%
        {\EnvSpace}{\EnvSpace}%
        {}%
        {}%
        {\bfseries}{.}%
        {\newline}%
        {}%
\newtheoremstyle{MyRemarks}%
        {\EnvSpace}{\EnvSpace}%
        {}%
        {}%
        {\bfseries}{.}%
        {\newline}%
        {}%

%\theoremstyle{MyExamples}   %\newtheorem{exemple}[numtho]{Exemple}      % Pour unification, ne plus utiliser
%                            \newtheorem{example}[numtho]{Exemple}
\newcounter{CounterExample}
\renewcommand{\theCounterExample}{\thechapter.\arabic{CounterExample}}

% J'ai décidé de ne plus numéroter les choses encadrées. 8 avril 2014
\newenvironment{example}{\vspace{\EnvSpace}\refstepcounter{numtho}\noindent{\bf Exemple}\\\nopagebreak}{\phantom{a}\hfill $\triangle$\vspace{\EnvSpace}}
\newenvironment{Aretenir}{\refstepcounter{numtho}\begin{oframed}\noindent{\bf À retenir}\newline}{\end{oframed}\vspace{\EnvSpace}}
\newenvironment{Aprojeter}{\clearpage\phantom{a}\vfill}{\vfill\newpage}
\newenvironment{definition}[1][]{\refstepcounter{numtho}\begin{oframed}\noindent{\bf Définition}#1\newline}{\end{oframed}\vspace{\EnvSpace}}
\newenvironment{propriete}{\refstepcounter{numtho}\begin{oframed}\noindent{\bf Propriété}\newline}{\end{oframed}\vspace{\EnvSpace}}

\newenvironment{Enmini}{\begin{oframed}\noindent{\bf Mini résumé}\newline}{\end{oframed}\vspace{\EnvSpace}}
% Ce bout de code provient de BrunoJ
% https://brunoj.wordpress.com/2009/10/08/latex-the-framed-minipage/
\newsavebox{\fmbox}
 \newenvironment{fmpage}[1]
 {\begin{lrbox}{\fmbox}\begin{minipage}{#1}}
     {\end{minipage}\end{lrbox}
     \fbox{\usebox{\fmbox}}
 }

\theoremstyle{MyRemarks}    \newtheorem{remark}[numtho]{Remarque}

                \newtheorem{amusement}[numtho]{Amusement}
                \newtheorem{erreur}[numtho]{Error}
                \newtheorem{probleme}[numprob]{\fbox{\bf Problèmes et choses à faire}}


\theoremstyle{MyTheorems}
\newtheorem{lemma}[numtho]{Lemme}
\newtheorem{corollary}[numtho]{Corollaire}
\newtheorem{theorem}[numtho]{Théorème}      
\newtheorem{proposition}[numtho]{Proposition}      


\renewcommand{\thenumtho}{\thechapter.\arabic{numtho}}
% La numérotation des équations change dans les corrigés
\renewcommand{\theequation}{\thechapter.\arabic{equation}}
\renewcommand{\theCountExercice}{\arabic{CountExercice}}       % Ce compteur est défini dans SystemeCorr.sty
\newcommand{\defe}[2]{\textbf{#1}\index{#2}}

\renewcommand{\theenumi}{(\alph{enumi})}
\renewcommand{\theenumii}{(\alph{enumi}\arabic{enumii})}

\renewcommand{\labelenumi}{\theenumi}
\renewcommand{\labelenumii}{\theenumii}

\newcommand{\justification}{ {\small \begin{center}    Attention : vous devez laisser sur votre feuille les traces de vos recherches et les étapes intermédiaires de vos calculs !    \end{center}}}
    

    % L'une des deux est avec le nom et l'autre sans.
    \newenvironment{feuilleDS}[1]{\noindent Nom, Prénom : \begin{center}\large #1\\\justification\end{center}\setcounter{CountExercice}{0}  }{\clearpage}
    %\newenvironment{feuilleDS}[1]{\begin{center}\large #1\\\justification\end{center}\setcounter{CountExercice}{0}  }{\clearpage}


    \newenvironment{feuilleExo}[1]{\newpage\begin{center}\large #1\end{center}\setcounter{CountExercice}{0}  }{\clearpage}
    %\newenvironment{feuilleExo}[1]{\begin{center}\large #1\end{center}\setcounter{CountExercice}{0}  }{}

        
        \newcounter{numactivmentale}
        \setcounter{numactivmentale}{0}
        \newcounter{numExoMental}
        \setcounter{numExoMental}{0}
        \newenvironment{MentalActivity}{\setcounter{numExoMental}{0}\newpage\refstepcounter{numactivmentale}\section{Activité mentale \arabic{numactivmentale}}}{\newpage\hphantom{jj}\vfill\large C'est tout pour aujourd'hui\vfill\newpage}
        \newenvironment{mental}{\refstepcounter{numExoMental}\newpage\begin{center}\fbox{\huge Activité mentale \arabic{numactivmentale}}\\\huge Question \arabic{numExoMental}\end{center}\huge\vfill}{\vfill}


\newcommand{\enteteInterro}[3]{
    \begin{center}
        #1\\
        Intérogation #2, sujet #3
    \end{center}
    Nom, prénom, classe : \ldots\\
    \setcounter{CountExercice}{0}
}


%%%%%%%%%%%%%%%%%%%%%%%%%%
%
%   Les macros qui font des choses
%
%%%%%%%%%%%%%%%%%%%%%%%%

\newcommand{\mA}{\mathcal{A}}
\newcommand{\mO}{\mathcal{O}}
\newcommand{\mR}{\mathcal{R}}
\newcommand{\mT}{\mathcal{T}}
\newcommand{\mU}{\mathcal{U}}

\newcommand{\scal}[2]{ \langle #1,#2\rangle }

\newcommand{\tq}{\text{ tel que }}
\newcommand{\tqs}{\text{ tels que }}
\newcommand{\quext}[1]{ \footnote{\textsf{#1}}  }
\newcommand{\info}[1]{\texttt{#1}}
\newcommand{\vect}[1]{\overrightarrow{#1}}    % Cette macro est codée en dur dans phystricksDefVecteurAXDDGP et dans d'autres

\newcommand{\VarAbs}{\text{Var}_{\text{abs}}}
\newcommand{\VarRel}{\text{Var}_{\text{rel}}}

\newcommand{\normal}{\lhd}
\newcommand{\swS}{\mathscr{S}}          % L'ensemble des fonctions Schwartz

%\newcommand{\defD}{\mathscr{D}}     % Ensemble de définition d'une fonction
\newcommand{\defD}{D}                % Le D avec des croles était impossible à comprendre pour les élèves.

\newcommand{\Borelien}{\mathcal{B}\text{or}}       % Les boréliens
\newcommand{\tribA}{\mathcal{A}}            % Une tribu A
\newcommand{\tribB}{\mathcal{B}}            
\newcommand{\tribF}{\mathcal{F}}            % Une tribu F

\newcommand{\affE}{\mathcal{E}}            % Un espace affine E
\newcommand{\affF}{\mathcal{F}}            
\newcommand{\affG}{\mathcal{G}}            

\newcommand{\statS}{\mathcal{S}}            % Un modèle statistique
\newcommand{\partP}{\mathcal{P}}            % L'ensemble des parties d'un ensemble

\newcommand{\polyP}{\mathcal{P}}            % L'ensemble des polynômes

\newcommand{\dB}{\mathscr{B}}       % la distribution de Bernoulli
\newcommand{\dE}{\mathscr{E}}       % la distribution exponentielle
\newcommand{\dG}{\mathscr{G}}       % la distribution géométrique.
\newcommand{\dM}{\mathscr{M}}       % la distribution multinomiale
\newcommand{\dN}{\mathscr{N}}       % la distribution normale.
\newcommand{\dP}{\mathscr{P}}       % la distribution de Poisson.
\newcommand{\dT}{\mathscr{T}}       % la distribution de Student
\newcommand{\dU}{\mathscr{U}}       % la distribution uniforme

\newcommand{\hL}{\mathscr{L}}       
\newcommand{\cL}{\hL}           % Pour la partie Chafai

\newcommand{\modE}{\mathcal{E}}         % Le E des modules
\newcommand{\modF}{\mathcal{F}}         % Le F des modules
\newcommand{\hH}{\mathscr{H}}           % Le H des espaces de Hilbert

%%%%%%%%%%%%%%%%%%%%%%%%%%
%
%   Bibliographie, index et liste des notations
%
%%%%%%%%%%%%%%%%%%%%%%%%

\usepackage{makeidx}
\usepackage[nottoc]{tocbibind}      % Le paquetage qui fait en sorte que la biblio soit inclue correctement dans la table des matières.
\usepackage[refpage]{nomencl}
\renewcommand{\nomname}{Liste des notations}
%
%   Comment introduire des éléments dans l'index des notations.
%
% La liste des tags à mettre pour bien classer mes notations est :
% T     pour la topologie et théorie des ensembles
%
% La syntaxe est facile, par exemple 
%       $\SL(2,\eR)$\nomenclature[G]{$\SL(2,\eR)$}{Le groupe de matrices deux par deux de déterminant 1.}
%\renewcommand{\nomgroup}[1]{%
%    \ifthenelse{\equal{#1}{A}}{\item[\textbf{Algèbre}]}{}%
%    \ifthenelse{\equal{#1}{G}}{\item[\textbf{Géométrie}]}{}%
%    \ifthenelse{\equal{#1}{R}}{\item[\textbf{Théorie des groupes}]}{}%
%    \ifthenelse{\equal{#1}{P}}{\item[\textbf{Probabilités et statistique}]}{}%
%    \ifthenelse{\equal{#1}{Y}}{\item[\textbf{Analyse}]}{}%
%    \ifthenelse{\equal{#1}{M}}{\item[\textbf{Chaînes de Markov}]}{}%
%}

%%%%%%%%%%%%%%%%%%%%%%%%%%
%
%   DeclareMathOperator
%
%%%%%%%%%%%%%%%%%%%%%%%%

\DeclareMathOperator{\signe}{sgn}
\DeclareMathOperator{\Vol}{Vol}
\DeclareMathOperator{\Int}{Int}     % Intérieur d'un ensemble.
\DeclareMathOperator{\Ind}{Ind}     % l'indice d'un chemin en analyse complexe
\DeclareMathOperator{\Diam}{Diam}   
\DeclareMathOperator{\id}{Id}   
\DeclareMathOperator{\Graph}{Graph} 
\DeclareMathOperator{\pr}{\texttt{proj}}
\DeclareMathOperator{\dom}{dom}

\DeclareMathOperator{\Graphe}{Gr}
\DeclareMathOperator{\Spec}{Spec}   % spectre d'un opérateur
\DeclareMathOperator{\arctg}{arctg}
\DeclareMathOperator{\cotg}{cotg}
\DeclareMathOperator{\cosec}{cosec}
\DeclareMathOperator{\arcsinh}{arcsinh}

\DeclareMathOperator{\GL}{GL}   % le groupe linéaire
\DeclareMathOperator{\PGL}{PGL}   % le groupe projectif
\DeclareMathOperator{\SO}{SO}           
\DeclareMathOperator{\SL}{SL}           
\DeclareMathOperator{\PSL}{PSL}   % Le groupe modulaire SL(2,Z)/Z2
\DeclareMathOperator{\gO}{O}           
\DeclareMathOperator{\SU}{SU}           
\DeclareMathOperator{\gU}{U}           

\DeclareMathOperator{\Reel}{Re}        % La partie réelle d'un nombre complexe

\DeclareMathOperator{\Image}{Image}        % ... avec \Image qui donne l'image d'une fonction ou d'un opérateur.
\DeclareMathOperator{\rang}{rg}   
\DeclareMathOperator{\Kernel}{Ker}
\DeclareMathOperator{\Domaine}{Dom}
\DeclareMathOperator{\Span}{Span}
\DeclareMathOperator{\Hom}{Hom}
\DeclareMathOperator{\End}{End}     % L'ensemble des endomorphismes
\DeclareMathOperator{\tr}{Tr}       % la trace
\DeclareMathOperator{\Majorant}{Maj}
\DeclareMathOperator{\codim}{codim} % pour la codimension.
\DeclareMathOperator{\diam}{diam} % le diamètre d'un ensemble.

\DeclareMathOperator{\Var}{Var}     % Variance d'une variable aléatoire.
\DeclareMathOperator{\Fun}{\texttt{Fun}}     % Ensemble des applications d'un ensemble vers l'autre.
\DeclareMathOperator{\Cov}{Cov}     % la covariance.
\DeclareMathOperator{\gr}{gr}     % le groupe engendré
\DeclareMathOperator{\pgcd}{pgcd}     
\DeclareMathOperator{\ppcm}{ppcm}     
\DeclareMathOperator{\Frob}{Frob}     
\DeclareMathOperator{\Card}{Card}       % Le cardinal d'un ensemble.
\DeclareMathOperator{\Stab}{Stab}       % Le stabilisateur d'un point sous l'action d'un groupe.

\DeclareMathOperator{\Frac}{Frac}       % le corps des fractions d'un anneau
\DeclareMathOperator{\Aff}{Aff}         %  l'espace affine engendré

\newenvironment{subproof}{\begin{description}}{\end{description}}

\newcommand{\telque}{\vert\,}
\newcommand{\donc}{\Rightarrow}

\usepackage{mathtools}
\mathtoolsset{showonlyrefs}


\pagestyle{empty}   % Pour éviter les numéros de page.

\renewcommand{\cite}[1]{}           % Les citations ne servent à rien dans les feuilles distribuées.


% 1 on utilise external et 0 on ne l'utilise pas
\newcounter{useexternal}
\setcounter{useexternal}{1}
\ifthenelse{\value{useexternal}=1}{ \usetikzlibrary{external} \tikzexternalize }{ \newcommand{\tikzsetnextfilename}[1]{} }



\begin{document}

% Le fichier d'évaluation de cinquième : 15_evaluation
% Le fichier d'évaluation de quatrième : 16_evaluation

% Pour avoir les évaluations, c'est lst_eval.py

% Les exercices en réserve de quatrième: 19_autresquestions
% Les exercices en réserve de cinquième : 24_autrecinquieme




% CHOSES À VÉRIFIER

% Les pages imprimées plus grand sont correctes. En particulier il y a parfois des choses en multicolonne qui ne passent pas bien.
% Le barème 
% La calculatrice
% vrai/faux et QCM
% pour les 4A : graphique et proportionnalité




% DS 5A numéro 4 vendredi 24 avril 2015
% DS_5A7
% proportionnalité, nombres relatifs, triangles
\begin{feuilleDS}{Devoir surveillé numéro 7, 5A\\ \small Vendredi 24 avril 2015}
    \begin{center}
        Calculatrice autorisée
    \end{center}
\Exo{2smath-0250}
\Exo{2smath-0251}
\Exo{2smath-0252}
\Exo{2smath-0253}
\Exo{2smath-0254}
\vspace{1cm}
{\bf Compétences}
\small
\begin{enumerate}
    \item
        Rechercher, extraire et organiser l'information utile : \ref{exo2smath-0167}.
    \item
        Réaliser, manipuler, mesurer, calculer, appliquer des consignes : \ref{exo2smath-0156}, \ref{exo2smath-0162},  \ref{exo2smath-0161}.
    \item
        Raisonner, argumenter, pratiquer une démarche expérimentale ou technologique, démontrer : \ref{exo2smath-0160}, \ref{exo2smath-0161}\ref{ItemBRMGooQoNoou}.
    \item 
        Présenter la démarche suivie, les résultats obtenus, communiquer à l'aide d’un langage adapté : \ref{exo2smath-0156}\ref{ItemCUFLooIdJOVd}, \ref{exo2smath-0167}.
\end{enumerate}
\end{feuilleDS}


% DS 4A numéro 7 vendredi 24 avril 2015
% DS_4A7
\begin{feuilleDS}{Devoir surveillé numéro 7, 4A\\ \small Vendredi 24 avril 2015}
    % Pyramides, littéral et cosinus
    \begin{center}
        \small Calculatrice autorisée.
    \end{center}

%    Pour l'ensemble du devoir nous donnons les approximations suivantes : \( \cos(\SI{34}{\degree})\simeq 0.829\), \( \cos(\SI{56}{\degree})\simeq 0.559\), \( \cos(\SI{35}{\degree})\simeq 0.819\) et \( \cos(\SI{55}{\degree})\simeq 0.574\).

\Exo{2smath-0245}
\Exo{2smath-0246}
\Exo{2smath-0247}
\Exo{2smath-0248}
\Exo{2smath-0249}
\vspace{1cm}
{\bf Compétences}
\small
\begin{enumerate}
    \item
        Rechercher, extraire et organiser l'information utile : exercices \ref{exo2smath-0247},\ref{exo2smath-0246}.
    \item
        Réaliser, manipuler, mesurer, calculer, appliquer des consignes : exercices \ref{exo2smath-0248}, \ref{exo2smath-0249}.
    \item
        Raisonner, argumenter, pratiquer une démarche expérimentale ou technologique, démontrer : exercices \ref{exo2smath-0245}, \ref{exo2smath-0246}.
    \item 
        Présenter la démarche suivie, les résultats obtenus, communiquer à l'aide d’un langage adapté : exercices \ref{exo2smath-0245}, \ref{exo2smath-0247}.
\end{enumerate}
\end{feuilleDS}


% DS 4A numéro 6 vendredi 20 février 2015
% DS_4A6
\begin{feuilleDS}{Devoir surveillé numéro 6, 4A\\ \small Vendredi 20 février 2015}
\Exo{2smath-0155}
\Exo{2smath-0152}
\Exo{2smath-0153}
\Exo{2smath-0154}
\Exo{2smath-0159}
\vspace{1cm}
{\bf Compétences}
\small
\begin{enumerate}
    \item
        Rechercher, extraire et organiser l'information utile : exercices \ref{exo2smath-0154}\ref{ItemITNOooXrgyJt} et \ref{exo2smath-0152}.
    \item
        Réaliser, manipuler, mesurer, calculer, appliquer des consignes : exercices \ref{exo2smath-0155}, \ref{exo2smath-0152} et \ref{exo2smath-0153}.
    \item
        Raisonner, argumenter, pratiquer une démarche expérimentale ou technologique, démontrer : exercices \ref{exo2smath-0154} et \ref{exo2smath-0159}\ref{ItemVAFNooUXPciH}.
    \item 
        Présenter la démarche suivie, les résultats obtenus, communiquer à l'aide d’un langage adapté : \ref{exo2smath-0152} et \ref{exo2smath-0154}.
\end{enumerate}
\end{feuilleDS}

% DS 5A numéro 6 vendredi 20 février 2015
% DS_5A6
% angles et paralléllisme, expressions littérales, nombres relatifs
\begin{feuilleDS}{Devoir surveillé numéro 6, 5A\\ \small Vendredi 20 février 2015}
\Exo{2smath-0161}
\Exo{2smath-0162}
\Exo{2smath-0160}
\Exo{2smath-0156}
\Exo{2smath-0167}
\vspace{1cm}
{\bf Compétences}
\begin{enumerate}
    \item
        Rechercher, extraire et organiser l'information utile : \ref{exo2smath-0167}.
        % Les angles qui sont les mêmes (vu l'équerre) est une donné. Les parallèles sont à démontrer.
    \item
        Réaliser, manipuler, mesurer, calculer, appliquer des consignes : \ref{exo2smath-0156}, \ref{exo2smath-0162},  \ref{exo2smath-0161}.
        % Appliquer la formule de circonférence
        % calculer correctement
        % Placer les points sur un axe correctement gradué. parce que 'construire en appliquant des consignes etc.'
    \item
        Raisonner, argumenter, pratiquer une démarche expérimentale ou technologique, démontrer : \ref{exo2smath-0160}, \ref{exo2smath-0161}\ref{ItemBRMGooQoNoou}.
        % repérer les angles 'qui vont ensemble'
    \item 
        Présenter la démarche suivie, les résultats obtenus, communiquer à l'aide d’un langage adapté : \ref{exo2smath-0156}\ref{ItemCUFLooIdJOVd}, \ref{exo2smath-0167}.
\end{enumerate}
\end{feuilleDS}

% DS 5B numéro 6 vendredi 20 février 2015
% DS_5B6
% Nombres relatifs, expressions littérales, symétrie centrale.
\begin{feuilleDS}{Devoir surveillé numéro 6, 5B\\ \small Vendredi 20 février 2015}
\Exo{2smath-0161}
\Exo{2smath-0162}
\Exo{2smath-0156}
\Exo{2smath-0164}
\Exo{2smath-0165}
\vspace{1cm}
{\bf Compétences}
\small
\begin{enumerate}
    \item
        Rechercher, extraire et organiser l'information utile : \ref{exo2smath-0156}, \ref{exo2smath-0165}.
        % Pour trouver les rayons
        % Pour lire correctement les coordonnées
    \item
        Réaliser, manipuler, mesurer, calculer, appliquer des consignes : \ref{exo2smath-0162}, \ref{exo2smath-0165}, \ref{exo2smath-0156}\ref{ItemCUFLooIdJOVd}.
    \item
        Raisonner, argumenter, pratiquer une démarche expérimentale ou technologique, démontrer : \ref{exo2smath-0164}, \ref{exo2smath-0165}
        % Dans la symétrie, pour utiliser les propriétés afin de ne pas tout construire.
    \item 
        Présenter la démarche suivie, les résultats obtenus, communiquer à l'aide d’un langage adapté : \ref{exo2smath-0164}, \ref{exo2smath-0156}\ref{ItemCUFLooIdJOVd}.
\end{enumerate}
\end{feuilleDS}


% DS 4A VENDREDI 23 JANVIER 2015
% DS_4A5
% droite du milieu, opérations sur les fractions et proportionnalité
\begin{feuilleDS}{Devoir surveillé numéro 5, 4A\\ \small Vendredi 23 janvier 2015}
\Exo{2smath-0093}
\Exo{smath-0964}
\Exo{2smath-0095}
\Exo{2smath-0096}
\Exo{2smath-0097}

\vspace{1cm}
{\bf Compétences}
\begin{enumerate}
    \item
        Rechercher, extraire et organiser l'information utile : 3,4
    \item
        Réaliser, manipuler, mesurer, calculer, appliquer des consignes : 4,5
    \item
        Raisonner, argumenter, pratiquer une démarche expérimentale ou technologique, démontrer : 1,3,4,2
    \item 
        Présenter la démarche suivie, les résultats obtenus, communiquer à l'aide d’un langage adapté : 1,3,2
\end{enumerate}
\end{feuilleDS}


% DS 5A VENDREDI 23 JANVIER 2015
% DS_5A5
% expressions littérales, nombres relatifs, droites remarquables dans un triangle
\begin{feuilleDS}{Devoir surveillé numéro 5, 5A\\ \small Vendredi 23 janvier 2015}
\Exo{2smath-0082}
\Exo{2smath-0083}
\Exo{2smath-0084}
\Exo{2smath-0086}
\Exo{2smath-0087}

\vspace{1cm}
{\bf Compétences}
\begin{enumerate}
    \item
        Rechercher, extraire et organiser l'information utile : 4,5
    \item
        Réaliser, manipuler, mesurer, calculer, appliquer des consignes : 1,2,3,5
        % 1 pour la réalisation pas à pas de la figure
        % 2 le passage du zéro ou de l'unité 'à l'envers'
        % 3 placer les points dans le repère
        % 5 le calcul littéral
    \item
        Raisonner, argumenter, pratiquer une démarche expérimentale ou technologique, démontrer : 1,5
        % 1 les codages
        % 5 la question (c)
    \item 
        Présenter la démarche suivie, les résultats obtenus, communiquer à l'aide d’un langage adapté : 2,4,5
        % 2 Dire quelle est la suite logique
        % 4 expliquer les différentes étapes des échanges d'argent
        % 5 : confronter un résultat attendu à un résultat obtenu
\end{enumerate}


\end{feuilleDS}

% DS 5B VENDREDI 23 JANVIER 2015
% DS_5B5
% expressions littérales,symétrie centrale, opérations sur les fractions
\begin{feuilleDS}{Devoir surveillé numéro 5, 5B\\ \small Vendredi 23 janvier 2015}
\Exo{2smath-0081}
\Exo{2smath-0087}
\Exo{2smath-0088}
\Exo{2smath-0099}
\Exo{2smath-0100}

\vspace{1cm}
{\bf Compétences}
\begin{enumerate}
    \item
        Rechercher, extraire et organiser l'information utile : 2 et 5.
    \item
Réaliser, manipuler, mesurer, calculer, appliquer des consignes : 1,2,3,5
% 5 pour la simplification parce qu'on tombe sur 2/6
\item
Raisonner, argumenter, pratiquer une démarche expérimentale ou technologique, démontrer : 1,3,4
% 1 pour continuer la figure sans faire tous les points
% 3 pour exclure/accepter les possibilités
% 4 pour faire des essais
% 2(c) 
\item 
Présenter la démarche suivie, les résultats obtenus, communiquer à l'aide d’un langage adapté : 1,5
% 1 pour les traits de construction et les codages
% 5 la question (c)
\end{enumerate}

\end{feuilleDS}




% DS 5B MARDI 13 janvier 2015
% DS_5B4rattrap
\begin{feuilleDS}{Rattrapage : devoir surveillé numéro 4, 5B\\ \small mardi 13 janvier 2015}
\Exo{2smath-0063}
\Exo{2smath-0064}
\Exo{2smath-0065}
\Exo{2smath-0005}  
\Exo{2smath-0066}
\end{feuilleDS}


% DS 4A VENDREDI 19 DÉCEMBRE 2014
% DS_4A4
\begin{feuilleDS}{Devoir surveillé numéro 4, 4A\\ \small Vendredi 19 décembre 2014}
\Exo{2smath-0006} 
\Exo{2smath-0018}
\Exo{2smath-0008}  
\Exo{2smath-0020}

\Exo{2smath-0021}
\Exo{2smath-0036}
\end{feuilleDS}


% DS 5A MARDI 16 décembre 2014
% DS_5A4
\begin{feuilleDS}{Devoir surveillé numéro 4, 5A\\ \small Mardi 16 décembre 2014}
\Exo{smath-0973}    % Ce serait bien de changer les nombres pour que la part totale ne soit pas un demi.
\Exo{2smath-0009}  
\Exo{2smath-0005}  
\Exo{2smath-0006} 
\Exo{2smath-0007}  
\end{feuilleDS}

% DS 5B MARDI 16 décembre 2014
% DS_5B4
\begin{feuilleDS}{Devoir surveillé numéro 4, 5B\\ \small Mardi 16 décembre 2014}
\Exo{2smath-0003}  
\Exo{2smath-0009}  
\Exo{smath-0973}
\Exo{2smath-0005}  
\Exo{2smath-0017}
\end{feuilleDS}


% DS 5A MARDI 25 NOVEMBRE 2014
% DS_5A3
\begin{feuilleDS}{Devoir surveillé numéro 3, 5A\\ \small Mardi 25 novembre 2014}
\Exo{smath-0971}
\Exo{smath-0972}
\Exo{smath-0976}
\Exo{smath-0974}
\Exo{smath-0975}
\end{feuilleDS}

% DS 5B MARDI 25 NOVEMBRE 2014
% DS_5B3
\begin{feuilleDS}{Devoir surveillé numéro 3, 5B\\ \small Mardi 25 novembre 2014}
\Exo{smath-0971}
\Exo{smath-0972}
\Exo{smath-0976}
\Exo{smath-0978}
\Exo{smath-0977}
\end{feuilleDS}

% DS 4A VENDREDI 21 NOVEMBRE 2014
% DS_4A3
\begin{feuilleDS}{Devoir surveillé numéro 3, 4A\\ \small Vendredi 21 novembre 2014}
\Exo{smath-0966}
\Exo{smath-0969}
\Exo{smath-0970}
\Exo{smath-0963}
\Exo{smath-0965}
\end{feuilleDS}

% DS2_5AB
\begin{feuilleDS}{Devoir surveillé numéro 2, 5AB\\ \small Vendredi 17 octobre 2014}  % réciproques
\Exo{smath-0886}
\Exo{smath-0885}
\Exo{smath-0893}    % gâteau au chocolat
\Exo{smath-0907}
\Exo{smath-0889}
\Exo{smath-0903}
\end{feuilleDS}

% DS2_4A
\begin{feuilleDS}{Devoir surveillé numéro 2, 4A\\ \small Vendredi 17 octobre 2014}
\Exo{smath-0898}
\Exo{smath-0896}
\Exo{smath-0897}
\Exo{smath-0899}
\Exo{smath-0895}
\end{feuilleDS}


% RATTRAPAGE DS1 POUR CINQUIÈME.
\begin{feuilleDS}{Devoir surveillé numéro 1 (rattrapage), 5A,5B\\ \small Mardi 8 octobre 2014}
\Exo{smath-0847}
\Exo{smath-0824}
\Exo{smath-0846}
\Exo{smath-0845}
\Exo{smath-0823}
\end{feuilleDS}


% DS 5A et 5B VENDREDI 3 NOVEMBRE 2014
% DS1_5A
% DS1_5B
\begin{feuilleDS}{Devoir surveillé numéro 1, 5A,5B\\ \small Vendredi 3 octobre 2014}
\Exo{smath-0820}
\Exo{smath-0821}
\Exo{smath-0822}
\Exo{smath-0823}
\Exo{smath-0824}
\end{feuilleDS}


% DS 4A LUNDI 29 SEPTEMBRE 2014

\begin{feuilleDS}{Devoir surveillé numéro 1, 4A\\ \small Lundi 29 septembre 2014}
\Exo{smath-0772}
\Exo{smath-0778}

\Exo{smath-0827}
\Exo{smath-0828}
\Exo{smath-0831}
\Exo{smath-0834}
\end{feuilleDS}

\end{document}
