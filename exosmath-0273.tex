% This is part of Un soupçon de mathématique sans être agressif pour autant
% Copyright (c) 2013
%   Laurent Claessens
% See the file fdl-1.3.txt for copying conditions.

\begin{exercice}\label{exosmath-0273}

    Donner l'image par la fonction inverse des nombres suivants :
    \begin{multicols}{3}
        \begin{enumerate}
            \item
                \( 3\)
            \item
                \( \frac{ 1 }{2}\)
            \item
                \( \frac{ 2 }{ 3 }\)
            \item
                \( \sqrt{2}\)
            \item
                \( \pi+1\)
            \item
                \( 10^4\)
            \item
                \( 3\times 10^8\)
            \item
                \( 2\times 10^{-7}\)
            \item
                \( -10^{14}\)
        \end{enumerate}
    \end{multicols}

\corrref{smath-0273}
\end{exercice}
