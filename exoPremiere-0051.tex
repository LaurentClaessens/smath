% This is part of Un soupçon de mathématique sans être agressif pour autant
% Copyright (c) 2012
%   Laurent Claessens
% See the file fdl-1.3.txt for copying conditions.

\begin{exercice}\label{exoPremiere-0051}

    Dans cet exercices, faites attention aux espaces qui python n'ajoute pas automatiquement (encore heureux).

    Nous avons une liste de noms d'élèves : \info{[``Charles'',``Louis'',``Camille'']} et nous voulons écrire
    \begin{quote}
        \info{Charles, Louis, Camille} 
    \end{quote}
    Écrire un programme qui le fait. Notez que ici il y a un espace après la virgule mais pas avant.

    Soit la liste \info{[``Mozilla'',``IE'',``Konqueror'',``Opera'']}. Écrire un programme qui affiche
    \begin{quote}
        \info{Mozilla Vs IE Vs Konqueror Vs Opera}
    \end{quote}
    Ici il y a un espace avant et après le  «Vs».

\corrref{Premiere-0051}
\end{exercice}
