% This is part of Un soupçon de mathématique sans être agressif pour autant
% Copyright (c) 2012
%   Laurent Claessens
% See the file fdl-1.3.txt for copying conditions.

\begin{corrige}{smath-0038}

<+Corrsmath-0038+>

    \begin{multicols}{2}
        \begin{tabular}[]{|c||c|}
            \hline
            Prix de départ HT&120\\
            \hline\hline
            TVA (\( 19.6\%\))&\( 23.52\)\\
            \hline
            Prix TTC avant augmentation&143.52\\
            \hline
            Augmentation de \( 20\%\)&\( 28.74\)\\
            \hline
            Prix TTC avec augmentation&\( 172.224\)\\
            \hline
        \end{tabular}

    \columnbreak

        \begin{tabular}[]{|c||c|}
            \hline
            Prix de départ HT&120\\
            \hline\hline
            Augmentation de \( 20\%\)&24\\
            \hline
            Prix HT avant augmentation&144\\
            \hline
            TVA (\( 19.6\%\))&28.224\\
            \hline
            Prix TTC avec augmentation&172.224\\
            \hline
        \end{tabular}
    \end{multicols}

    Dans les deux cas de figure, le client paye \( 172.224\) euros. Dans le premier cas, l'état prend \( 23.52\) et le commerçant prend \( 148.703\); alors que dans le second l'état prend $28.24$ et le commerçant \( 143.98\)€.

\end{corrige}
