% This is part of Un soupçon de mathématique sans être agressif pour autant
% Copyright (c) 2014
%   Laurent Claessens
% See the file fdl-1.3.txt for copying conditions.

\begin{exercice}\label{exosmath-0690}

    Nous considérons les points \( A(-1;1)\), \( B(5;3)\) et \( C(5;5)\). Faire un dessin qui sera complété au fur et à mesure.
    \begin{enumerate}
        \item
            Calculer les coordonnées de \( M\) tel que \( \vect{ MC }=\frac{1}{ 3 }\vect{ AC }\).
        \item
            Calculer les coordonnées du point \( D\) tel que \( ABCD\) soit un parallélogramme.
        \item
            Quelles sont les coordonnées du milieu \( I\) de \( [CD]\) ?
        \item
            Prouver que les points \( I\), \( M\) et \( B\) sont alignés.
    \end{enumerate}

\corrref{smath-0690}
\end{exercice}
