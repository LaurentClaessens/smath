% This is part of Un soupçon de mathématique sans être agressif pour autant
% Copyright (c) 2013
%   Laurent Claessens
% See the file fdl-1.3.txt for copying conditions.

\begin{corrige}{smath-0215}

    Les ensembles en question sont :
    \begin{equation}
        A=\{15,16,17,18,19,20\}
    \end{equation}
    et 
    \begin{equation}
        B=\{0,3,6,9,12,15,18\}.
    \end{equation}
    En comptant nous avons \( p(A)=\frac{ 6 }{ 21 }=\frac{ 2 }{ 7 }\) et \( p(B)=\frac{1}{ 3 }\). En ce qui concerne l'intersection,
    \begin{equation}
        A\cap B=\{ 15,18 \},
    \end{equation}
    donc \( p(A\cap B)=\frac{ 2 }{ 21 }\).

    Pour l'union,
    \begin{equation}
        p(A\cup B)=p(A)+p(B)-p(A\cap B)=\frac{ 11 }{ 21 }.
    \end{equation}

\end{corrige}
