% This is part of Un soupçon de mathématique sans être agressif pour autant
% Copyright (c) 2012
%   Laurent Claessens
% See the file fdl-1.3.txt for copying conditions.

\begin{exercice}\label{exosmath-0115}

    Sur le dessin ci-contre, nous nommons \( c\) la longueur des côtés du cube. Le point \( I\) est le milieu de \( [AF]\). Les réponses aux questions suivantes dépendent de \( c\).
    \begin{multicols}{2}

        \begin{enumerate}
            \item
                Quelle est la longueur de \( [AI]\) ?
            \item
                Quel est le périmètre du triangle \( AFC\) ?
            \item
                Quelle est la longueur de la hauteur \( [IC]\) ?
            \item
                Calculer la surface du triangle \( AFC\).
            \item
                Quelle est la nature du triangle \( AEF\) ?
            \item
                Calculer la surface du triangle \( AEF\). Comparer à la surface du triangle \( ADC\).
        \end{enumerate}

        \columnbreak

%The result is on figure \ref{LabelFigfigureXQZwoWu}. % From file figureXQZwoWu
%\newcommand{\CaptionFigfigureXQZwoWu}{<+Type your caption here+>}


        \begin{center}
\input{Fig_figureXQZwoWu.pstricks}
        \end{center}

    \end{multicols}
    

\corrref{smath-0115}
\end{exercice}
