% This is part of Un soupçon de mathématique sans être agressif pour autant
% Copyright (c) 2012
%   Laurent Claessens
% See the file fdl-1.3.txt for copying conditions.

\begin{exercice}\label{exosmath-0110}

    \begin{enumerate}
        \item
            Soit \( F=(4;1)\). Donner un point \( E\) tel que \( (EF)\) soit parallèle à la droite représentative de la fonction \( f(x)=3x-1\).
        \item
            Pour quelle valeur de \( a\), la droite représentative de \( f(x)=3x+1\) est-elle parallèle à la droite représentative de \( g(x)=ax-9\) ?
    \end{enumerate}

\corrref{smath-0110}
\end{exercice}
