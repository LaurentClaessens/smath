% This is part of Un soupçon de mathématique sans être agressif pour autant
% Copyright (c) 2015
%   Laurent Claessens
% See the file fdl-1.3.txt for copying conditions.


% This is part of Un soupçon de mathématique sans être agressif pour autant
% Copyright (c) 2015
%   Laurent Claessens
% See the file fdl-1.3.txt for copying conditions.

Tracer un parallélogramme et le découper.
\begin{enumerate}
    \item
Découper le parallélogramme de manière à pouvoir reconstituer un rectangle en un seul coup de ciseaux.
\item
 Calculer l'aire du rectangle ainsi obtenu.
\item
 En déduire l'aire du parallélogramme.
\end{enumerate}
<++>




De \cite{SIUXooOJshFN}

%+++++++++++++++++++++++++++++++++++++++++++++++++++++++++++++++++++++++++++++++++++++++++++++++++++++++++++++++++++++++++++ 
\section{Aire du parallélogramme}
%+++++++++++++++++++++++++++++++++++++++++++++++++++++++++++++++++++++++++++++++++++++++++++++++++++++++++++++++++++++++++++


\begin{propriete}
    L'aire du parallélogramme est donnée par la formule 
    \begin{equation}
        \text{aire}=\text{base}\times \text{hauteur}
    \end{equation}
\end{propriete}

\begin{proof}
    Voir activité.
\end{proof}


\begin{center}
   \input{Fig_IKZEooAwkZSy.pstricks}
\end{center}



% This is part of Un soupçon de mathématique sans être agressif pour autant
% Copyright (c) 2015
%   Laurent Claessens
% See the file fdl-1.3.txt for copying conditions.

%--------------------------------------------------------------------------------------------------------------------------- 
\subsection*{Activité : aire du triangle}
%---------------------------------------------------------------------------------------------------------------------------


\begin{center}
   \input{Fig_VFJVooUXwMvM.pstricks}
\end{center}
\begin{enumerate}
    \item
        Sachant que \( ABCD\) est un rectangle, ajouter tous les codages (angles et longueurs) possibles.
    \item
        Comparer les triangles \( AED\) et \( DEF\).
    \item
        Quelle est l'aire du triangle \( DEC\) ?
\end{enumerate}


%+++++++++++++++++++++++++++++++++++++++++++++++++++++++++++++++++++++++++++++++++++++++++++++++++++++++++++++++++++++++++++ 
\section{Aire du triangle}
%+++++++++++++++++++++++++++++++++++++++++++++++++++++++++++++++++++++++++++++++++++++++++++++++++++++++++++++++++++++++++++

\begin{propriete}
    L'aire du triangle est donnée par la formule 
    \begin{equation}
        \text{aire}=\frac{ \text{base}\times \text{hauteur}}{2}
    \end{equation}
\end{propriete}

\begin{proof}
    Voir activité. Attention : l'activité ne traite pas le cas

\begin{center}
   \input{Fig_ZQRZooGsxVIr.pstricks}
\end{center}

\end{proof}


%+++++++++++++++++++++++++++++++++++++++++++++++++++++++++++++++++++++++++++++++++++++++++++++++++++++++++++++++++++++++++++ 
\section{Aire du disque}
%+++++++++++++++++++++++++++++++++++++++++++++++++++++++++++++++++++++++++++++++++++++++++++++++++++++++++++++++++++++++++++


Pour l'aire du disque, il y a une annimation\cite{RHJUooRIQIuv} :
\url{interact/airedisque.ggb}

\begin{propriete}
    L'aire d'un disque de rayon \( r\) est donnée par la formule
    \begin{equation}
        \text{aire}=\pi\times r\times r.
    \end{equation}
\end{propriete}

\begin{example}

    L'aire de ce cercle
    \begin{center}
        \input{Fig_UTDOooJFkZWu.pstricks}
    \end{center}
    est 
    \begin{equation}
        \text{aire}=\pi\times \SI{4}{\centi\meter}\times \SI{4}{\centi\meter}=\pi\times \SI{16}{\centi\meter\squared}\simeq \SI{50.26}{\centi\meter\squared}.
    \end{equation}

\end{example}


