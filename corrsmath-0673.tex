% This is part of Un soupçon de mathématique sans être agressif pour autant
% Copyright (c) 2014
%   Laurent Claessens
% See the file fdl-1.3.txt for copying conditions.

\begin{corrige}{smath-0673}

    \begin{enumerate}
        \item
            Il n'est possible de simplifier que ce que l'on peut mettre en facteur :
            \begin{equation}
                \frac{ a^2+2a }{ 8a }=\frac{ a(a+2) }{ 8a }=\frac{ a+2 }{ 8 }.
            \end{equation}
        \item
            Elle est décroissant parce que c'est une fonction affine dont le coefficient directeur est négatif (\( m=-3\)).
        \item
            {\bf Le seul cas dans la vie où le «produit en croix» est utilisable}. Lorsqu'on a égalité entre deux fractions, on a égalité entre les deux termes du produit en croix : 
            \begin{equation}
                \frac{1}{ x }=\frac{ 4 }{ 7 }
            \end{equation}
            donne
            \begin{equation}
                1\times 7=x\times 4,
            \end{equation}
            et donc \( x=\frac{ 7 }{ 4 }\).
        \item
            C'est un produit remarquable \( (a-b)^2=a^2-2ab+b^2\) :
            \begin{equation}
                (x-4)^2=x^2-8x+16.
            \end{equation}
        \item
            Le nombre \( -3\) étant entre \( -4\) et \( 0\), le nombre \( f(-3)\) sera entre \( 3\) et \( 7\). La réponse la plus précise est
            \begin{equation}
                x\in\mathopen] 3 , 7 \mathclose[.
            \end{equation}
    \end{enumerate}

\end{corrige}
