% This is part of Un soupçon de mathématique sans être agressif pour autant
% Copyright (c) 2013
%   Laurent Claessens
% See the file fdl-1.3.txt for copying conditions.

\begin{exercice}\label{exosmath-0452}

    Un randonneur très consciencieux met \( 30\) minutes pour se préparer à sortir et marche ensuite à une vitesse de \unit{5}{\kilo\meter\per\hour} (si bien qu'il met par exemple 2h30 pour parcourir \unit{10}{\kilo\meter}).
    \begin{enumerate}
        \item
            Tracer sur un graphique la fonction qui donne le nombre de kilomètres parcourus en fonction du temps depuis le début de sa préparation.
        \item
            Un autre promeneur sort de chez lui au moment où le premier commence tout juste à se préparer et marche tranquillement à la vitesse de \unit{2}{\kilo\meter\per\hour}. Tracer son graphique sur le même dessin que le premier.
        \item
            À partir de combien de temps le premier randonneur a-t-il parcouru plus de distance que le second ? Quelle est la distance parcourue à ce moment ?
    \end{enumerate}

\corrref{smath-0452}
\end{exercice}
