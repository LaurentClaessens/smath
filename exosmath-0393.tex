% This is part of Un soupçon de mathématique sans être agressif pour autant
% Copyright (c) 2013
%   Laurent Claessens
% See the file fdl-1.3.txt for copying conditions.

\begin{exercice}\label{exosmath-0393}

    \begin{minipage}{0.485\textwidth}

        \begin{enumerate}
            \item
    Quel est le coefficient directeur de la droite ci-contre ?

        \begin{enumerate}
            \item
                \( -\frac{ 3 }{ 2 }\)
            \item
                \( 1\)
            \item
                \( \frac{ 3 }{ 2 }\)
            \item
                \( \frac{ 1 }{2}\)
        \end{enumerate}

    \item
        Quelle est l'équation de cette droite ?

        \begin{enumerate}
            \item
                \( y=\frac{ 3 }{2}x+1\)
            \item
                \( y=\frac{ 2 }{ 3 }x+1\)
            \item
                \( y=2x-1\)
            \item
                \( y=\frac{ 2 }{ 3 }x+2\)
        \end{enumerate}

        \end{enumerate}

    \end{minipage}
    \hspace{1mm}    
    \begin{minipage}{0.485\textwidth}
%The result is on figure \ref{LabelFigIQnEPpt}. % From file IQnEPpt
%\newcommand{\CaptionFigIQnEPpt}{<+Type your caption here+>}
\begin{center}
\input{Fig_IQnEPpt.pstricks}
\end{center}
    \end{minipage}

\corrref{smath-0393}
\end{exercice}
