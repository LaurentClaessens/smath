% This is part of Un soupçon de mathématique sans être agressif pour autant
% Copyright (c) 2012
%   Laurent Claessens
% See the file fdl-1.3.txt for copying conditions.

%+++++++++++++++++++++++++++++++++++++++++++++++++++++++++++++++++++++++++++++++++++++++++++++++++++++++++++++++++++++++++++
\section{Translation}
%+++++++++++++++++++++++++++++++++++++++++++++++++++++++++++++++++++++++++++++++++++++++++++++++++++++++++++++++++++++++++++

\begin{multicols}{2}
    \begin{definition}  \label{DefAAJEuS}
    La \defe{translation}{translation} \( t_{A,B}\) est la transformation du plan qui à un point \( C\) fait correspondre l'unique point \( D\) tel que les segments \( [AD]\) et \( [BC]\) aient même milieu.
\end{definition}

\columnbreak

%The result is on figure \ref{LabelFigDefVecteurAXDDGP}. % From file DefVecteurAXDDGP
%\newcommand{\CaptionFigDefVecteurAXDDGP}{<+Type your caption here+>}
\input{Fig_DefVecteurAXDDGP.pstricks}
\end{multicols}
Nous notons \( \vect{ AB }\) le vecteur associé à cette translation.

Étant donné qu'un quadrilatère dont les diagonales se coupent en leur milieu est un parallélogramme, nous avons immédiatement le règle suivante :
\begin{Aretenir}
    Le quadrilatère \( ABDC\) est un parallélogramme si et seulement si \( D=t_{A,B}(C)\).
\end{Aretenir}
Attention : il s'agit bien de \( ABDC\) et non de \( ABCD\).

Le dessin à côté de la définition \ref{DefAAJEuS}, aplati, donne immédiatement aussi
\begin{equation}
    t_{A,B}(A)=B.
\end{equation}

\begin{definition}
    Nous disons que \( \vect{ AB }=\vect{ CD }\) si et seulement si \( t_{A,B}(C)=D\).
\end{definition}

\begin{propriete}
    Si \( A\), \( B\), \( C\) et \( D\) sont des points tels que \( \vect{ AB }=\vect{ CD }\) alors \( ABDC\) est un parallélogramme.

    Attention : il s'agit bien de \( ABDC\) et non de \( ABCD\).
\end{propriete}

\begin{proof}
    Nous supposons que \( \vect{ AB }=\vect{ CD }\) et nous prouvons qu'alors \( ABCD\) est un parallélogramme.  Pour tout point \( K\) du plan, nous avons \( t_{A,D}(K)=t_{C,D}(K)\). En particulier, avec \( K=C\) nous avons
    \begin{equation}
        t_{A,B}(C)=t_{C,D}(C)=D.
    \end{equation}
    Le fait d'avoir \( t_{A,B}(C)=D\) nous indique que \( ABDC\) est un parallélogramme.
\end{proof}

La réciproque est également vraie.
\begin{propriete}
    Si \( ABCD\) est un parallélogramme, alors \( \vect{ AB }=\vect{ CD }\).
\end{propriete}

\begin{proof}
    Nous devons prouver que pour tout point \( K\) du plan, \( t_{A,B}(K)=t_{C,D}(K)\).
\end{proof}
<++>

Au final, l'égalité de vecteurs est caractérisée par la règle du parallélogramme résumée ci-dessous.
\begin{propriete}
    Soient quatre points du plan \( A\), \( B\), \( C\) et \( D\). Le quadrilatère \( ABCD\) est un parallélogramme si et seulement si \( \vect{ AB }=\vect{ CD }\).
\end{propriete}

%+++++++++++++++++++++++++++++++++++++++++++++++++++++++++++++++++++++++++++++++++++++++++++++++++++++++++++++++++++++++++++
\section{Exercices}
%+++++++++++++++++++++++++++++++++++++++++++++++++++++++++++++++++++++++++++++++++++++++++++++++++++++++++++++++++++++++++++

\Exo{smath-0053}
\Exo{smath-0052}
\Exo{smath-0055}
\Exo{smath-0063}
\Exo{smath-0064}

%---------------------------------------------------------------------------------------------------------------------------
\subsection{Colinéarité}
%---------------------------------------------------------------------------------------------------------------------------

\Exo{smath-0061}
\Exo{smath-0062}
\Exo{smath-0060}
\Exo{smath-0054}

%---------------------------------------------------------------------------------------------------------------------------
\subsection{Opérations sur les vecteurs}
%---------------------------------------------------------------------------------------------------------------------------

\Exo{smath-0059}

%---------------------------------------------------------------------------------------------------------------------------
\subsection{Théorie}
%---------------------------------------------------------------------------------------------------------------------------

\Exo{smath-0056}
\Exo{smath-0057}
\Exo{smath-0058}
