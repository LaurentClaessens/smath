% This is part of Un soupçon de mathématique sans être agressif pour autant
% Copyright (c) 2012
%   Laurent Claessens
% See the file fdl-1.3.txt for copying conditions.

%+++++++++++++++++++++++++++++++++++++++++++++++++++++++++++++++++++++++++++++++++++++++++++++++++++++++++++++++++++++++++++
\section{Translation}
%+++++++++++++++++++++++++++++++++++++++++++++++++++++++++++++++++++++++++++++++++++++++++++++++++++++++++++++++++++++++++++

\begin{multicols}{2}
\begin{definition}
    La \defe{translation}{translation} \( t_{A,B}\) est la transformation du plan qui à un point \( C\) fait correspondre l'unique point \( D\) tel que les segments \( [AD]\) et \( [BC]\) aient même milieu.
\end{definition}

\columnbreak

%The result is on figure \ref{LabelFigDefVecteurAXDDGP}. % From file DefVecteurAXDDGP
%\newcommand{\CaptionFigDefVecteurAXDDGP}{<+Type your caption here+>}
\input{Fig_DefVecteurAXDDGP.pstricks}
\end{multicols}
Nous notons \( \vect{ AB }\) le vecteur associé à cette translation.

\begin{definition}
    Nous disons que \( \vect{ AB }=\vect{ CD }\) si et seulement si \( t_{A,B}=t_{C,D}\), c'est à dire si pour tout point \( K\) dans le plan nous avons \( t_{A,B}(K)=t_{C,d}(K)\).
\end{definition}

\begin{propriete}
    Soient quatre points du plan \( A\), \( B\), \( C\) et \( D\). Le quadrilatère \( ABCD\) est un parallélogramme si et seulement si \( \vect{ AB }=\vect{ CD }\).
\end{propriete}
<++>

%+++++++++++++++++++++++++++++++++++++++++++++++++++++++++++++++++++++++++++++++++++++++++++++++++++++++++++++++++++++++++++
\section{Exercices}
%+++++++++++++++++++++++++++++++++++++++++++++++++++++++++++++++++++++++++++++++++++++++++++++++++++++++++++++++++++++++++++

\Exo{smath-0053}
\Exo{smath-0052}
\Exo{smath-0055}

%---------------------------------------------------------------------------------------------------------------------------
\subsection{Colinéarité}
%---------------------------------------------------------------------------------------------------------------------------

\Exo{smath-0054}
