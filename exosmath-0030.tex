% This is part of Un soupçon de mathématique sans être agressif pour autant
% Copyright (c) 2012
%   Laurent Claessens
% See the file fdl-1.3.txt for copying conditions.

\begin{exercice}\label{exosmath-0030}

    Dire si les fonctions suivantes sont des polynômes du second degré, et identifier la cas échéant les coefficients \( a\), \( b\) et \( c\).
    \begin{multicols}{2}
    \begin{enumerate}
        \item
            \( f(x)=2x^2-3x+5\)
        \item
            \( g(x)=3x^2+x\sqrt{2}-\frac{1}{ 3 }\)
        \item
            \( h(x)=x^2-\frac{ 2 }{ x }+4\)
        \item
            \( i(x)=x+2\)
        \item
            \( j(x)=x^2+3\sqrt{x}-1\)
        \item
            \( k(x)=-x^2+\frac{ x }{ 3 }+7\)
    \end{enumerate}
    \end{multicols}

\corrref{smath-0030}
\end{exercice}
