% This is part of Un soupçon de mathématique sans être agressif pour autant
% Copyright (c) 2012
%   Laurent Claessens
% See the file fdl-1.3.txt for copying conditions.

\begin{exercice}\label{exosmath-0190}

    Une urne contient \( 9\) boules jaunes, \( 7\) rouges, \( 5\) vertes et \( 3\) bleues. De plus ces boules sont numérotées de \( 1\) à \( 9\) pour les boules jaunes,  de \( 1\) à \( 7\) pour les rouges, de \( 1\) à \( 5\) pour les vertes et de \( 1\) à \( 3\) pour les bleues.

    \begin{enumerate}
        \item
            Combien de boules portant le numéro \( 2\) y a-t-il ?
        \item
            Combien de boules portant le numéro \( 7\) y a-t-il ?
        \item
            Décrire l'univers de cette nouvelle expérience aléatoire qui consiste à regarder à la fois le numéro et la couleur de la boule tirée. Combien d'élément contient cet univers ?
        \item
            Donner la probabilité des événements suivants :
            \begin{itemize}
                \item «La boule tirée porte le numéro 4».
                \item
                    «La boule tirée porte un numéro pair». 
                \item
                    «La boule tirée est verte ou porte un numéro impair».
                \item
                    «La boule tirée porte un numéro plus grand ou égal à \( 3\), mais n'est pas jaune».
            \end{itemize}
    \end{enumerate}

\corrref{smath-0190}
\end{exercice}
