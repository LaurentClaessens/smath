% This is part of Un soupçon de mathématique sans être agressif pour autant
% Copyright (c) 2014
%   Laurent Claessens
% See the file fdl-1.3.txt for copying conditions.

%--------------------------------------------------------------------------------------------------------------------------- 
\subsection*{Mesurer sur un dessin}
%---------------------------------------------------------------------------------------------------------------------------

Rechercher dans votre cahier d'exercices l'activité «mesure astronomique». Quelle distance avez-vous trouvé entre la Terre et la comète ? Mettons tous les résultats en commun et comparons.

Est-ce que nous pouvons nous fier à un dessin ?

%--------------------------------------------------------------------------------------------------------------------------- 
\subsection*{Recherche d'exemples}
%---------------------------------------------------------------------------------------------------------------------------

Si \( x\) est un nombre positif, est-il vrai que \( x\times x\) est plus grand que \( x\) ? (exemple : si \( x=5\) alors \( x\times x=25\), et c'est effectivement plus grand que \( 5\))
