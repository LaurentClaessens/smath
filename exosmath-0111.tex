% This is part of Un soupçon de mathématique sans être agressif pour autant
% Copyright (c) 2012
%   Laurent Claessens
% See the file fdl-1.3.txt for copying conditions.

\begin{exercice}\label{exosmath-0111}

    \begin{enumerate}
        \item
            Un tapis roulant roule à la vitesse de \unit{5}{\kilo\meter\per\hour}. Un enfant cours à la vitesse de \unit{7}{\kilo\meter\per\hour} en sens inverse. À quelle vitesse avance-t-il par rapport au sol ? Dans quelle direction ?
        \item
            Un poisson nage perpendiculairement d'une rive à l'autre d'un fleuve à la vitesse de \unit{15}{\kilo\meter\per\hour}. Le fleuve a une largeur de \unit{25}{\meter} et le poisson est emporté par un courant de \unit{5}{\kilo\meter\per\hour}. De combien de mètres sera-t-il dévié dans le sens du courant durant sa traversée ?
    \end{enumerate}

\corrref{smath-0111}
\end{exercice}
