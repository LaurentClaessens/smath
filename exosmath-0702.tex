% This is part of Un soupçon de mathématique sans être agressif pour autant
% Copyright (c) 2014
%   Laurent Claessens
% See the file fdl-1.3.txt for copying conditions.

\begin{exercice}\label{exosmath-0702}

    Un sondage portant sur \( 100\) personnes montre que le candidat \( K\) à une élection municipale obtiendrait \( 44\%\) des votes au premier tour. La règle électorale indique qu'un candidat obtenant \( 50\%\) des suffrages gagne dès le premier tour. Monsieur \( K\) affirme qu'il compte gagner sans second tour. Est-il trop optimiste ?

    Le lendemain, un autre sondage, basé sur \( 1000\) personnes lui donnerait \( 46\% \). Étant monté de \( 1\%\), il estime que ses chances de gagner ont encore augmenté. Qu'estimez-vous ?

\corrref{smath-0702}
\end{exercice}
