% This is part of Un soupçon de mathématique sans être agressif pour autant
% Copyright (c) 2012
%   Laurent Claessens
% See the file fdl-1.3.txt for copying conditions.

\begin{exercice}\label{exoSeconde-0082}

    Soient \( A\), \( B\) et \( C\), trois points du plan. Soit \( K\) le milieu du segment \( [AC]\), et \( L\) le point d'intersection entre la droite \( (AB)\) et la parallèle à \( (BC)\) passant par \( K\).
    \begin{enumerate}
        \item
            Faire un dessin de la situation.
        \item
            Déterminer où se trouve le point \( L\) par rapport au segment \( [AB]\).
        \item
            Si la distance entre \( B\) et \( C\) est de \unit{16}{\centi\meter}, quelle est la distance entre \( K\) et \( L\) ?
    \end{enumerate}

\corrref{Seconde-0082}
\end{exercice}
