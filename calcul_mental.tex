% This is part of Un soupçon de mathématique sans être agressif pour autant
% Copyright (c) 2013
%   Laurent Claessens
% See the file fdl-1.3.txt for copying conditions.


\documentclass{beamer}

\usepackage[utf8]{inputenc}
\usepackage[T1]{fontenc}

\usepackage{textcomp}
\usepackage{lmodern}
\usepackage[english,frenchb]{babel}



\usetheme{default}
\begin{document}

\begin{frame}{Calcul mental 2}
    \pause
    \begin{enumerate}
        \item
            Simplifier \( \sqrt{72}\)
            \pause
        \item
            Calculer la distance entre \( (0;10)\) et \( (2;0)\).
            \pause
        \item
            Résoudre \( \frac{1}{ x }=\frac{ 2 }{ 3 }\).
            \pause
        \item
            Simplifier 
            \begin{equation*}
                \frac{ 12x+3b }{ 21 }.
            \end{equation*}
            \pause
        \item
            Calculer le milieu du segment entre \( (1;3)\) et \( (-3;8)\).
            \pause
        \item
            Résoudre
            \begin{equation*}
                (x+2)(x-1)=0.
            \end{equation*}
            \pause
    \end{enumerate}
    \begin{center}
        FIN pour aujourd'hui.
    \end{center}
\end{frame}

\begin{frame}{Calcul mental 1}

    \pause
    \begin{enumerate}
        \item
            Simplifier : \( \sqrt{12}\)

            \phantom{\( 2\sqrt{3}\)}

            \pause
        \item
            Les coordonnées du milieu entre \( (1;-2)\) et \( (6;4)\)
            \pause
        \item
            Simplifier 
            \begin{equation*}
                \frac{ 2xy }{ 4 }
            \end{equation*}
            \pause
        \item
            Simplifier
            \begin{equation*}
                \frac{ 8x+4a }{ 2 }
            \end{equation*}
            \pause
        \item
            Résoudre : \( 3x=12\).
            \pause
        \item
            Résoudre : \( 2x-6=4\).
    \end{enumerate}

\end{frame}

\end{document}
