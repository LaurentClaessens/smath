% This is part of Un soupçon de mathématique sans être agressif pour autant
% Copyright (c) 2014
%   Laurent Claessens
% See the file fdl-1.3.txt for copying conditions.

% This is part of Un soupçon de mathématique sans être agressif pour autant
% Copyright (c) 2014
%   Laurent Claessens
% See the file fdl-1.3.txt for copying conditions.



\begin{wrapfigure}{r}{2.5cm}
   \vspace{-0.5cm}        % à adapter.
   \centering
   \input{Fig_HHOooQUedri.pstricks}
\end{wrapfigure}

Tentatives de construire des triangles de longueurs imposées.

\begin{enumerate}
    \item
        
Choisir trois nombres compris entre $2$ et $15$ et tenter de tracer un triangle dont les côtés ont ces mesures (règle, rapporteur, compas, équerre).

\item

    Voyant le triangle ci-contre, Louise s'est exclamée «il est complètement faux !». Pourquoi ? Essayer de dessiner un triangle correct ayant ces mesures.

\end{enumerate}


%+++++++++++++++++++++++++++++++++++++++++++++++++++++++++++++++++++++++++++++++++++++++++++++++++++++++++++++++++++++++++++ 
\section{Constructions de triangles}
%+++++++++++++++++++++++++++++++++++++++++++++++++++++++++++++++++++++++++++++++++++++++++++++++++++++++++++++++++++++++++++

%--------------------------------------------------------------------------------------------------------------------------- 
\subsection{Connaissant les trois longueurs}
%---------------------------------------------------------------------------------------------------------------------------

Nous voulons construire un triangle \(ABC\) dont les côtés sont de longueurs sont \( AB=\unit{3}{\centi\meter}\), \( BC=\unit{4}{\centi\meter}\) et \( AC=\unit{2}{\centi\meter}\).


IPAooQliVZD

\begin{enumerate}
    \item
        Tracer un segment \( [AB]\) de longueur \unit{3}{\centi\meter}.
    \item
        Vu que le point \( C\) est à distance \unit{2}{\centi\meter} de \( A\), tracer un cercle de centre \( A\) et de rayon \unit{2}{\centi\meter}.
    \item
        Vu que le point \( C\) est à distance \unit{4}{\centi\meter} de \( B\), tracer un cercle de centre \( B\) et de rayon \unit{4}{\centi\meter}.
    \item
        Les intersections des deux cercles est en même temps à distance \unit{2}{\centi\meter} de \( A\) et \unit{4}{\centi\meter} de \( B\); le point \( C\) peut y être placé.
\end{enumerate}


