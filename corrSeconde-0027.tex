% This is part of Un soupçon de mathématique sans être agressif pour autant
% Copyright (c) 2012
%   Laurent Claessens
% See the file fdl-1.3.txt for copying conditions.

\begin{corrige}{Seconde-0027}

    La première chose à faire est d'ordonner les valeurs : 0, 0, 2, 2, 5, 5, 5, 6, 8, 8, 8, 9, 9, 9, 10, 10, 10, 10, 11, 11, 12, 12, 13, 14, 16, 17, 18.
    \begin{enumerate}
        \item
            En ce qui concerne la moyenne, on fait la somme (240) et on divise par le nombre de données (27) : la moyenne est \( \frac{ 240 }{ 27 }\approx 8.88\).
        \item
            Étant donné que le nombre de valeurs est impair (27), la médiane est simplement la valeur centrale, c'est à dire la \( 14\)\up{e}, qui vaut \( 9\).
        \item
            Pour le premier quartile, nous divisons les effectifs par \( 4\) et nous prenons le premier entier \emph{supérieur}; cela donne le rang du quartile :
            \begin{equation}
                \frac{ 24 }{ 4 }=6.75,
            \end{equation}
            donc le premier quartile est la septième valeur, c'est à dire \( 5\).
        \item
            Pour le troisième quartile, nous faisons la même chose, mais en utilisant \( 3/4\) au lieu de \( 1/4\) :
            \begin{equation}
                \frac{ 3 }{ 4 }\times 27=20.25,
            \end{equation}
            donc le troisième quartile est la \( 21\)\up{e} valeur, c'est à dire \( 12\).
        \item
            Le graphique des effectifs cumulés est donné à figure \ref{LabelFigEffectifsCumulwfqAhj}.
            \newcommand{\CaptionFigEffectifsCumulwfqAhj}{Le graphique des effectifs cumulés de l'exercice \ref{exoSeconde-0027}.}
\input{Fig_EffectifsCumulwfqAhj.pstricks}

    \end{enumerate}

            Les fautes courantes :
            \begin{enumerate}
                \item
                    Il faut trier les notes par ordre croissant avant de chercher les médianes et quartiles.
                \item
                    Le graphe des effectifs cumulés est \emph{croissante} : elle ne montre pas combien d'élèves ont obtenu une certaine note, mais bien combien d'élèves ont obtenu \emph{au moins} la note.
                \item
                    Il ne faut pas relier les points par des segments de droites (et encore moins par des courbes !). Le graphe doit être en escalier.
            \end{enumerate}


    Si vous voulez vous aider de python pour faire les calculs, voici quelque lignes.

    \lstinputlisting{exo_mediane.py}

    Des fonctions toutes faites pour calculer médiane et quartiles sont données à la section \ref{SecZYftar}.



\end{corrige}
