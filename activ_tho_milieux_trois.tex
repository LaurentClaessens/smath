% This is part of Un soupçon de mathématique sans être agressif pour autant
% Copyright (c) 2014
%   Laurent Claessens
% See the file fdl-1.3.txt for copying conditions.

%--------------------------------------------------------------------------------------------------------------------------- 
\subsection*{Activité : troisième théorème du milieu}
%---------------------------------------------------------------------------------------------------------------------------

Soit un triangle \( ABC\) et \( I\) le milieu de \( [AC]\).
\begin{enumerate}
    \item
        Combien de droites passant par \( I\) sont parallèles à \( (AB)\) ?
    \item
        Si nous notons \( K\) le milieu de \( [BC]\), que dire de la droite \( (IK)\) par rapport à \( (AB)\) ?
    \item
        Si une droite parallèle à \( (AB)\) passe par le point \( I\), est-ce qu'elle doit obligatoirement passer par \( K\) ?
\end{enumerate}
