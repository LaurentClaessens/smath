% This is part of Un soupçon de mathématique sans être agressif pour autant
% Copyright (c) 2014-2015
%   Laurent Claessens
% See the file fdl-1.3.txt for copying conditions.

%--------------------------------------------------------------------------------------------------------------------------- 
\subsection*{Activité : moyenne pondérée}
%---------------------------------------------------------------------------------------------------------------------------

On a demandé, à un groupe de $50$ étudiants, le montant mensuel (en euros) de leur abonnement de téléphone portable. En voici le détail :
\begin{equation*}
    \begin{array}[]{cccccccccc}
        23&14&14&36&36&41&18&36&18&36\\
        23&32&23&41&18&18&36&27&36&27\\
        23&32&18&32&27&36&36&36&36&32\\
        41&14&41&23&14&41&18&27&36&41\\
        14&14&36&32&27&14&36&27&27&27
    \end{array}
\end{equation*}
\begin{enumerate}
    \item
 Calculer le montant mensuel moyen, en euros, de l'abonnement téléphonique de ces $50$ étudiants.
 \item
 Construis et remplis un tableau pour lire plus facilement ces données.
 \item
     Comment calculer le montant mensuel moyen de l'abonnement téléphonique de ce groupe à partir de ce tableau ? Justifie.
\end{enumerate}
