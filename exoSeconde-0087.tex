% This is part of Un soupçon de mathématique sans être agressif pour autant
% Copyright (c) 2012
%   Laurent Claessens
% See the file fdl-1.3.txt for copying conditions.

\begin{exercice}\label{exoSeconde-0087}

    \begin{minipage}{0.485\textwidth}
            Le solide ci-contre est un cube. 
        \begin{enumerate}
            \item
                Pourquoi peut-on affirmer que \( H\) est dans le plan \( (EBC)\) ?
            \item
                Est-ce que \( (BG)\) et \( (CF)\) sont perpendiculaires.
    \item
                Est-ce que \( (BH)\) et \( (CE)\) sont sécantes ? Perpendiculaires ?
    \item
        Est-ce que les droites \( (CH)\) et \( (EH)\) sont perpendiculaires ?
        \end{enumerate}
    \end{minipage}
    \vspace{1mm}
    \begin{minipage}{0.485\textwidth}
\begin{center}
%    The result is on figure \ref{LabelFigLignesCubeshBfjxk}. % From file LignesCubeshBfjxk
%\newcommand{\CaptionFigLignesCubeshBfjxk}{<+Type your caption here+>}
\input{Fig_LignesCubeshBfjxk.pstricks}
\end{center}
    \end{minipage}


\corrref{Seconde-0087}
\end{exercice}
