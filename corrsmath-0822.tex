% This is part of Un soupçon de mathématique sans être agressif pour autant
% Copyright (c) 2014
%   Laurent Claessens
% See the file fdl-1.3.txt for copying conditions.

\begin{corrige}{smath-0822}

    \begin{enumerate}
        \item
            \( 6\times 3+5=23\)
        \item
            Il faut que \( 4\times \ldots\) soit égal à \( 20-4=16\). Donc il faut compléter par \( 4\) :
            \begin{equation}
                4\times 4+4=20.
            \end{equation}
        \item
            Il faut calculer le numérateur et le dénominateur séparément :
            \begin{equation}
                \frac{ 9\times 4 }{ 2\times 3 }=\frac{ 36 }{ 6 }=6.
            \end{equation}
            Ou alors on voit les simplification : \( \dfrac{ 9 }{ 3 }=3\) et \( \dfrac{ 4 }{ 2 }=2\).
        \item
            Vu qu'au numérateur on multiplie \( 125\) par \( 39\) et que l'on veut obtenir \( 125\) après division, le numérateur doit être \( 39 \) :
            \begin{equation}
                \frac{ 39\times 125 }{ 39 }=125.
            \end{equation}
    \end{enumerate}

\end{corrige}
