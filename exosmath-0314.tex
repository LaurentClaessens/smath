% This is part of Un soupçon de mathématique sans être agressif pour autant
% Copyright (c) 2013
%   Laurent Claessens
% See the file fdl-1.3.txt for copying conditions.

\begin{exercice}\label{exosmath-0314}

    Une population de \( 300\) renards est introduite en 2013 dans une foret renfermant une nourriture abondante et peu de prédateurs. L'évolution de la population de renards est modélisée comme suit : chaque année le nombre de renards double, mais 200 renards meurent mangés par des prédateurs.  pour le taux d'évolution trouvé.Rendez-vous dans le fichier \info{TD\_suites\_stmg.ods}, onglet «doublement de population». 

    \begin{enumerate}
        \item
            Calculer sur papier la population de renards en 2014 et en 2015.
        \item
            En utilisant un tableur, calculer la population en 2020.
        \item
            Après combien d'années la population aura été multipliée par quatre ?
        \item
            En quelle année la population dépasse les dix mille renards ?
        \item
            Ce modèle très simple mène vite à des résultats invraisemblables. Qu'en pensez-vous ? Répondre en s'appuyant sur les chiffres obtenue en poussant plus loin la simulation.
    \end{enumerate}

\corrref{smath-0314}
\end{exercice}
