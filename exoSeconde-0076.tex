% This is part of Un soupçon de mathématique sans être agressif pour autant
% Copyright (c) 2012-2013
%   Laurent Claessens
% See the file fdl-1.3.txt for copying conditions.

\begin{exercice}\label{exoSeconde-0076}

Soit un rectangle de base \( 4\) et de hauteur \( x\).
\begin{multicols}{2}
    \begin{enumerate}
        \item
            Donner la longueur de la diagonale en fonction de \( x\). Écrire la réponse sous la forme \( d(x)=\ldots\)
        \item
            Quel est l'ensemble de définition de cette fonction ?
        \item
            Est-elle croissante ? Décroissante ?
        \item
            Pour quelle valeur de \( x\) la diagonale est de longueur \( 5\) ?
    \end{enumerate}
\end{multicols}


\corrref{Seconde-0076}
\end{exercice}
