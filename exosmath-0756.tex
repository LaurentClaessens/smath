% This is part of Un soupçon de mathématique sans être agressif pour autant
% Copyright (c) 2014
%   Laurent Claessens
% See the file fdl-1.3.txt for copying conditions.

\begin{exercice}[\cite{NRHooXFvgpp4}]\label{exosmath-0756}

Compléter les pyramides suivantes sachant que le nombre contenu dans une case est le produit des nombres contenus dans les deux cases situées en dessous de lui :

% Le fichier phystricksSWHooJzpmlM.py est spécial : la fonction SWHooJzpmlM lance les deux fonctions dont on a besoin ici.

\begin{multicols}{2}

\begin{center}
   \input{Fig_SCVooUjmESN.pstricks}
\end{center}

\columnbreak

\begin{center}
   \input{Fig_LYJooVYRkMy.pstricks}
\end{center}

\end{multicols}

\corrref{smath-0756}
\end{exercice}
