% This is part of Un soupçon de mathématique sans être agressif pour autant
% Copyright (c) 2014
%   Laurent Claessens
% See the file fdl-1.3.txt for copying conditions.

\begin{exercice}[\ldots/4]\label{exosmath-0695}

\begin{wrapfigure}{r}{5.0cm}
   \vspace{-1cm}        % à adapter.
   \centering
   \input{Fig_NUWooJepuh.pstricks}
\end{wrapfigure}

    À partir du dessin ci-contre.
    \begin{enumerate}
        \item
            Déterminer graphiquement l'équation de la droite \( d_1\).
        \item
            Déterminer par le calcul l'équation de la droite \( d_2\) passant par les points \( A\) et \( B\).
        \item
            Calculer le point d'intersection \( I=d_1\cap d_2\).
        \item
            Donner les coordonnées du point d'abscisse \( 10\) se trouvant sur la droite \( d_1\).
        \item
            Donner les coordonnées du point d'ordonnée \( 4\) se trouvant sur la droite \( d_2\).
    \end{enumerate}

\corrref{smath-0695}
\end{exercice}
