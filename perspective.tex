% This is part of Un soupçon de mathématique sans être agressif pour autant
% Copyright (c) 2012
%   Laurent Claessens
% See the file fdl-1.3.txt for copying conditions.

%+++++++++++++++++++++++++++++++++++++++++++++++++++++++++++++++++++++++++++++++++++++++++++++++++++++++++++++++++++++++++++
\section{Introduction}
%+++++++++++++++++++++++++++++++++++++++++++++++++++++++++++++++++++++++++++++++++++++++++++++++++++++++++++++++++++++++++++

La figure \ref{LabelFigDesSections} montre un cube. Êtes-vous capables de donner la nature des surfaces coloriées ?
\newcommand{\CaptionFigDesSections}{Exercice de vision dans l'espace.}
\input{Fig_DesSections.pstricks}
Le triangle vert est isocèle et rectangle parce que deux de ses côtés sont des arrêtes du cube. Le triangle rouge est plus troublant, mais il est équilatéral : ses trois côtés sont des diagonales des faces du cube. Notez que \emph{sur le dessin}, les trois côtés ont des longueur différentes.

%+++++++++++++++++++++++++++++++++++++++++++++++++++++++++++++++++++++++++++++++++++++++++++++++++++++++++++++++++++++++++++
\section{Les règles de la perspective cavalière}
%+++++++++++++++++++++++++++++++++++++++++++++++++++++++++++++++++++++++++++++++++++++++++++++++++++++++++++++++++++++++++++

Lorsque nous dessinons en trois dimension, nous prenons les conventions suivantes qui définissent la \defe{\wikipedia{fr}{Perspective_cavalière}{perspective cavalière}}{perspective!cavalière}.

\begin{Aretenir}
    D'abord nous choisissons
    \begin{enumerate}
        \item
            Nous choisissons un \defe{angle de fuite}{angle!de fuite} \( \alpha\) qui sera entre \unit{30}{\degree} et \( \unit{45}{\degree}\) avec l'horizontale\footnote{Sur une feuille à carreaux, le plus simple est de prendre \unit{45}{\degree}.}.
        \item
            Un \defe{coefficient de réduction}{coefficient de réduction} que nous noterons \( k\) et qui sera compris entre \( 0\) et \( 1\).
    \end{enumerate}

    Ensuite nous prenons les correspondances suivantes entre la réalité et le dessin :
    \begin{center}
        \begin{tabular}{|p{7.5cm}|p{7.5cm}|}
            \hline
            {\bf dans la réalité}&{\bf sur le dessin}\\
            \hline\hline
            Segment caché  & Segment pointillé\\
            \hline
            Segment parallèle à la feuille de dessin & Segment représenté en vraie grandeur\\
            \hline
            Segment perpendiculaire à la feuille de dessin & Segment faisant un angle \( \alpha\) avec l'horizontale.\\
            \hline
            Une arrête de longueur \( l\) perpendiculaire à la feuille de dessin & Une arrête de longueur \( k\times l\) faisant un angle \( \alpha\) avec l'horizontale.\\
            \hline
        \end{tabular}
        % Note : c'est ce genre de tableaux qui ne fonctionnent pas avec le paquet pdfsync.
        % Il faut oublier pdfsync et compiler avec 'pdflatex -synctex=1'
    \end{center}
\end{Aretenir}

\begin{propriete}
    La perspective cavalière respecte les conditions suivantes.
    \begin{enumerate}
        \item
             Deux segments parallèles dans la réalité sont représentés par deux segments parallèles sur le dessin.
         \item
             Trois points alignés dans la réalité sont représentés par trois points alignés sur le dessin.
         \item
             Si \( M\) est le milieu du segment \( [AB]\) dans la réalité, alors \( M\) est le milieu du segment \( [AB]\) dans la réalité.
         \item
             La perspective cavalière respecte les proportions. C'est à dire que si le segment \( [AB]\) est \( p\) fois plus grand que le segment \( [CD]\) dans la réalité, alors il sera \( p\) fois plus grand sur le dessin.
    \end{enumerate}
\end{propriete}

Le cube de la figure \ref{LabelFigCubeLFZuiW} est dessiné avec \( \alpha=\unit{45}{\degree}\) et \( k=0.5\). Notez que les côtés parallèles restent parallèles.
\newcommand{\CaptionFigCubeLFZuiW}{Les segments perpendiculaires à la feuille sont de longueur moitié des autres.}
\input{Fig_CubeLFZuiW.pstricks}

La perspective cavalière n'est pas parfaite; il est aisé de créer des illusions d'optique comme celle de la figure \ref{LabelFigIllusionNHwEtp}. % From file IllusionNHwEtp
\newcommand{\CaptionFigIllusionNHwEtp}{Une petite illusion d'optique facile.}
\input{Fig_IllusionNHwEtp.pstricks}


%+++++++++++++++++++++++++++++++++++++++++++++++++++++++++++++++++++++++++++++++++++++++++++++++++++++++++++++++++++++++++++
\section{Exercices}
%+++++++++++++++++++++++++++++++++++++++++++++++++++++++++++++++++++++++++++++++++++++++++++++++++++++++++++++++++++++++++++


\Exo{Seconde-0088}
\Exo{Seconde-0087}
