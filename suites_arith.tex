% This is part of Un soupçon de mathématique sans être agressif pour autant
% Copyright (c) 2013
%   Laurent Claessens
% See the file fdl-1.3.txt for copying conditions.

%+++++++++++++++++++++++++++++++++++++++++++++++++++++++++++++++++++++++++++++++++++++++++++++++++++++++++++++++++++++++++++ 
\section{Suites arithmétiques}
%+++++++++++++++++++++++++++++++++++++++++++++++++++++++++++++++++++++++++++++++++++++++++++++++++++++++++++++++++++++++++++

\begin{definition}
    Soit \( a\in \eR\). Une suite \( (u_n)\) est une \defe{suite arithmétique}{suite!arithmétique} de raison \( a\) si pour tout entier \( n\) nous avons \( u_{n+1}=u_n+a\).
\end{definition}

\begin{example}
    Une chaîne de supermarché a déjà \( 200\) points de vente. Chaque année elle en ouvre \( 6\) de plus.
\end{example}

\begin{Aretenir}
    Sens de variation.
    \begin{enumerate}
        \item
            Si \( a>0\) alors la suite est croissante.
        \item
            Si \( a<0\) alors la suite est décroissante.
        \item
            Si \( a=0\) alors la suite est constante.
    \end{enumerate}
\end{Aretenir}
Placés sur un graphique, les éléments d'une suite arithmétique se mettent sur une droite.

%+++++++++++++++++++++++++++++++++++++++++++++++++++++++++++++++++++++++++++++++++++++++++++++++++++++++++++++++++++++++++++ 
\section{Exercices}
%+++++++++++++++++++++++++++++++++++++++++++++++++++++++++++++++++++++++++++++++++++++++++++++++++++++++++++++++++++++++++++

%--------------------------------------------------------------------------------------------------------------------------- 
\subsection{Suites arithmétiques}
%---------------------------------------------------------------------------------------------------------------------------

\Exo{smath-0163}
\Exo{smath-0164}
\Exo{smath-0165}
\Exo{smath-0166}
\Exo{smath-0211}
\Exo{smath-0167}
\Exo{smath-0169}
\Exo{smath-0170}

