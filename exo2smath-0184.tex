% This is part of Un soupçon de mathématique sans être agressif pour autant
% Copyright (c) 2015
%   Laurent Claessens
% See the file fdl-1.3.txt for copying conditions.

\begin{exercice}[\cite{NRHooXFvgpp4}]\label{exo2smath-0184}

\begin{wrapfigure}{r}{3.0cm}
   \vspace{-0.5cm}        % à adapter.
   \centering
   \input{Fig_LOPLooZJmTuk.pstricks}
\end{wrapfigure}

Sur cette pyramide, les trois angles en \( G\) sont droits et nous donnons les mesures \( GE=\SI{5}{\centi\meter}\), \( GF=\SI{4}{\centi\meter}\) et \( GS=\SI{6}{\centi\meter}\). Dessiner un patron pour cette pyramide.


    \begin{multicols}{2}
    Sur le dessin ci-contre, le point \( I\) est le centre du cube. Réaliser le patron de la pyramide \( BFGCI\).

 \columnbreak

\begin{center}
   \input{Fig_NFCHooAFFYPx.pstricks}
\end{center}
    \end{multicols}



\corrref{2smath-0184}
\end{exercice}
