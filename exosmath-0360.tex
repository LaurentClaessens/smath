% This is part of Un soupçon de mathématique sans être agressif pour autant
% Copyright (c) 2013
%   Laurent Claessens
% See the file fdl-1.3.txt for copying conditions.

\begin{exercice}\label{exosmath-0360}

Nous savons que la circonférence du cercle est de \( 2\pi\) pour un tour de \unit{360}{\degree}. Pour cette question, nous considérons un cercle trigonométrique de centre \( O\), le point \( I\) de coordonnées \( (1,0)\) et un point \( M\) variable sur le cercle.

\begin{enumerate}
    \item
Compléter le tableau de proportionnalité suivant :
\begin{center}
\begin{tabular}[]{c||c|c|c|c|c|c}
    Mesure de l'angle \( \widehat{IOM}\)&\unit{180}{\degree}&\unit{90}{\degree}&\unit{60}{\degree}&\unit{45}{\degree}&&\unit{0}{\degree}\\
    Longueur de l'arc \( \wideparen{IM}\)&&&&&$\frac{ \displaystyle\pi }{ \displaystyle 6 }$&\\
\end{tabular}
\end{center}

\item
    Quelle est la longueur de l'arc \( \wideparen{IM}\) lorsque l'angle \( \widehat{IOM}\) vaut \unit{57}{\degree} ?
\item
    Pour quelle valeur de l'angle \( \widehat{IOM}\) la longueur de \( \wideparen{IM}\) vaut-elle \( \frac{ \pi }{ 5 }\) ?
        
\end{enumerate}

% Wohaaaw. Vise un peu le numéro de ce fichier d'exercice :) 

\corrref{smath-0360}
\end{exercice}
