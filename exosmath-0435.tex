% This is part of Un soupçon de mathématique sans être agressif pour autant
% Copyright (c) 2013
%   Laurent Claessens
% See the file fdl-1.3.txt for copying conditions.

\begin{exercice}\label{exosmath-0435}

    Soit la fonction \( f(x)=2x^2+7x-1\).
    \begin{enumerate}
        \item
            Calculer les racines de $f$.
        \item
            Calculer le coefficient directeur de la tangente au graphe de \( f\) au point d'abscisse \( \ell=-2\).
    \end{enumerate}

    Soit la fonction \( g(x)=\frac{ x^2 }{ 5 }-x-30\).
    \begin{enumerate}
        \item
            Donner les coordonnées du sommet de la parabole représentative de \( g\).
        \item
            Donner l'équation de la tangente au graphe de \( g\) au point d'abscisse \( \ell=15\).
    \end{enumerate}

\corrref{smath-0435}
\end{exercice}
