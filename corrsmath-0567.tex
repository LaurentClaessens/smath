% This is part of Un soupçon de mathématique sans être agressif pour autant
% Copyright (c) 2013
%   Laurent Claessens
% See the file fdl-1.3.txt for copying conditions.

\begin{corrige}{smath-0567}

    L'intersection de deux plans non confondus et non parallèles est une droite. Pour désigner une droite, il faut donner deux points.
    \begin{enumerate}
        \item

            Pour donner l'intersection entre \( (BCD)\) et \( (DIJ)\), il faut trouver deux points de l'intersection. 

            Le plan \( (BCD)\) est celui de la base. Le point \( D\) fait partie de l'intersection parce qu'il est explicitement cité dans la définition des deux plans \( (BCD)\) et \( (DIJ)\). Le point \( I\) fait partie de la droite \( (BC)\) et est donc dans le plan \( (BCD)\) -- si deux points sont dans un plan alors toute la droite passant par ces deux points est dans le plan.

            Donc les points \( D\) et \( I\) sont tous deux à la fois dans \( (DIJ)\) et dans \( (BCD)\). La droite d'intersection est donc la droite \( (DI)\).

            Notez bien qu'il y a quatre justifications à donner : \( D\in (BCD)\), \( D\in(DIJ)\), \( I\in (BCD)\) et \( I\in(DIJ)\) parce que \emph{chacun} des deux points doit être dans les deux plans. Des quatre, seule la justification de \( I\in (BCD)\) n'est pas immédiate.

        \item
            
            Ici encore il faut trouver deux points et donner quatre justifications. Le point \( D\) fait encore partie de l'intersection parce qu'il est par définition dans les deux plans.

            Il faut trouver un deuxième point à être à la fois dans le plan \( (ABD)\) (qui est la face arrière) et dans le plan \( (DIJ)\). Ce point est \( J\). Il est effectivement dans \( (DIJ)\) par définition et il est également dans \( (ABD)\) parce qu'il est sur la droite \( (AD)\).

    \end{enumerate}

\end{corrige}
