% This is part of Un soupçon de mathématique sans être agressif pour autant
% Copyright (c) 2014
%   Laurent Claessens
% See the file fdl-1.3.txt for copying conditions.

\begin{exercice}\label{exosmath-0985}

    \begin{enumerate}
        \item
            Est-ce que le triangle \( FGH\) de mesures \( FG=41\), \( GH=9\) et \( FH=40\) est rectangle ? Si oui, en quel point est l'angle droit ?
        \item
            Pour le triangle suivant, écrire l'égalité de Pythagore (exemple : \( ST^2+SU^2=TU^2\)) et calculer la mesure manquante.
            \begin{center}
   \input{Fig_JQFHooEDhWVO.pstricks}
            \end{center}
    \end{enumerate}


\corrref{smath-0985}
\end{exercice}
