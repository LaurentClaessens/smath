% This is part of Un soupçon de mathématique sans être agressif pour autant
% Copyright (c) 2012
%   Laurent Claessens
% See the file fdl-1.3.txt for copying conditions.

\begin{corrige}{Seconde-0086}
    Nous allons passer en revue tous les nombres de zéro à \( 1000\) et vérifier si il contient \( 3\) chiffres. La difficulté est que si \info{a} est un \emph{nombre}\footnote{\info{int} comme \emph{integer} en anglais.} alors \info{len(a)} ne fonctionne pas : la fonction \info{len} ne fonctionne que sur des chaînes de caractères. D'où le fait que le second programme ne fonctionnait pas.

    Voici un programme qui résout l'exercice.

    \lstinputlisting{ex_comptage.py}

    Notez que le test d'égalité se note avec un double égal : \info{==}.

\end{corrige}
