% This is part of Un soupçon de mathématique sans être agressif pour autant
% Copyright (c) 2014
%   Laurent Claessens
% See the file fdl-1.3.txt for copying conditions.

%--------------------------------------------------------------------------------------------------------------------------- 
\subsection*{Sans multiplications}
%---------------------------------------------------------------------------------------------------------------------------

\begin{enumerate}
    \item
Calculer \( A=12+5-4\), \( B=12-4+5\) et \( C=5-4+12\).
\item
    Qu'observe-t-on ?
\end{enumerate}

%--------------------------------------------------------------------------------------------------------------------------- 
\subsection*{Avec multiplications}
%---------------------------------------------------------------------------------------------------------------------------

\begin{enumerate}
    \item
        Calculer mentalement \( D=4\times 5+2\), \( E=5\times 4+2\) et \( F=2\times 4+5\).
    \item
        Recalculer ces expressions en les tapant à la calculatrice exactement comme elles sont écrites.
    \item
        Qu'observe-t-on ?
\end{enumerate}

%--------------------------------------------------------------------------------------------------------------------------- 
\subsection*{Avec des parenthèses}
%---------------------------------------------------------------------------------------------------------------------------

\begin{enumerate}
    \item
        Calculer \( G=(7+3)\times 3\), \( H=4\times (30-21)\) et \( K=(3\times 4)\times (7-2)\).
    \item
        Dans quel ordre faut-il faire les calculs ?
\end{enumerate}
