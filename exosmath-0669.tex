% This is part of Un soupçon de mathématique sans être agressif pour autant
% Copyright (c) 2014
%   Laurent Claessens
% See the file fdl-1.3.txt for copying conditions.

\begin{exercice}\label{exosmath-0669}

    Soient les points \( A(-2;1)\), \( B(7;1)\), \( C(25;3)\) et \( D(15;3)\).
    \begin{enumerate}
        \item
            Est-ce que \( ABCD\) est un parallélogramme (justifier par un calcul) ? 
        \item
            Si \( ABCD\) est un parallélogramme, est-ce un rectangle ? Si \( ABCD\) n'est pas un parallélogramme, où aurait-t-on dû placer \( D\) pour qu'il en fût un ?
        \item
            Calculer les coordonnées du point \( E\) tel que \( \vect{ AE }=\vect{ AB }+\vect{ CB }\).
        \item
            Démontrer que \( \vect{ BC }=\vect{ EB }\).
        \item
            Où se trouve le point \( B\) par rapport au segment \( [CE]\) ? (justifier)
    \end{enumerate}

\corrref{smath-0669}
\end{exercice}
