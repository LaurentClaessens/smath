% This is part of Un soupçon de mathématique sans être agressif pour autant
% Copyright (c) 2013
%   Laurent Claessens
% See the file fdl-1.3.txt for copying conditions.

\begin{corrige}{smath-0290}

    \begin{enumerate}
        \item
            L'espérance est le nombre d'essais multiplié par la probabilité :
            \begin{equation}
                1000\times\frac{1}{ 50 }=20.
            \end{equation}
        \item
            Nous sommes dans le cas d'une loi binomiale avec nombre d'essais égal à \( 1000\) et probabilité de succès égale à \( 1/50\). Nous avons (calculatrice) :
            \begin{equation}
                P(X\geq 40)=1-P(X\leq 39)=4.33\times 10^{-5},
            \end{equation}
            autant dire très proche de zéro.
        \item
            Ici le calcul est
            \begin{equation}
                P(x\leq 10)\simeq 0.0102\simeq 1\%.
            \end{equation}
    \end{enumerate}

\end{corrige}
