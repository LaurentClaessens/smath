% This is part of Un soupçon de mathématique sans être agressif pour autant
% Copyright (c) 2012
%   Laurent Claessens
% See the file fdl-1.3.txt for copying conditions.

\begin{exercice}\label{exoSeconde-0098}

    \( ABCD\) est un parallélogramme de centre \( O\). Nous notons \( I\) le milieu du segment \( [AB]\) et \( K\) celui de \( [CD]\). La droite \( (AK)\) coupe \( (BD)\) au point \( M\) et nous notons \( N\) l'intersection entre \( (CI)\) et \( BD\).
    \begin{enumerate}
        \item
            Dessiner la situation.
        \item
            Démontrer que \( BN=NM=MD\) (penser à Thalès).
        \item
            Quel est le point \( N\) par rapport au triangle \( ABC\) ?
    \end{enumerate}

\corrref{Seconde-0098}
\end{exercice}
