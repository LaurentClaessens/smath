% This is part of Un soupçon de mathématique sans être agressif pour autant
% Copyright (c) 2013
%   Laurent Claessens
% See the file fdl-1.3.txt for copying conditions.

\begin{exercice}\label{exosmath-0522}

    Vrai ou faux.
    \begin{enumerate}
        \item
            Le nombre \( x=-5\) est solution de l'inéquation \( -x\leq 2\).
        \item
            La représentation graphique de la fonction \( g\colon x\mapsto 5x+2\) passe par les points \( A(2;0)\) et \( B(1;7)\).
        \item
            Un triangle dont les côtés mesurent \unit{5}{\kilo\meter}, \unit{6}{\kilo\meter} et \unit{7}{\kilo\meter} est rectangle.
        \item
            La fonction \( k(x)=\frac{ 1-3x }{2}\) est croissante.
        \item
            Les droites représentatives des fonctions affines \( f(x)=25x+12\) et \( g(x)=25x-12\) sont parallèles.
    \end{enumerate}

\corrref{smath-0522}
\end{exercice}
