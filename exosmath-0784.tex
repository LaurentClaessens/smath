% This is part of Un soupçon de mathématique sans être agressif pour autant
% Copyright (c) 2014
%   Laurent Claessens
% See the file fdl-1.3.txt for copying conditions.

\begin{exercice}[\cite{NRHooXFvgpp}]\label{exosmath-0784}

Pour chacun des énoncés suivants, dire s'il est vrai ou faux; énoncer ensuite sa réciproque et dire si elle est vraie ou fausse.
\begin{enumerate}
    \item
        {\bf Si} un nombre se termine par $3$ {\bf alors} il est divisible par $3$.
\item
    {\bf Si} $x = 3$ {\bf alors} $x^2 = 9$.
\item
    {\bf Si} un nombre est divisible par $3$ {\bf alors} il est divisible par $9$.
\item
    {\bf Si} un nombre est pair {\bf alors} il se termine par $2$.
\item
    {\bf Si} un quadrilatère a ses diagonales qui se coupent en leur milieu {\bf alors} c'est un parallélogramme.
\item
    {\bf Si} un quadrilatère est un carré {\bf alors} il a ses quatre côtés de même longueur.
\end{enumerate}

\corrref{smath-0784}
\end{exercice}
