% This is part of Un soupçon de mathématique sans être agressif pour autant
% Copyright (c) 2014
%   Laurent Claessens
% See the file fdl-1.3.txt for copying conditions.

\begin{exercice}\label{exosmath-0943}

    Un train roule à la vitesse \( v\) durant un temps \( t\) et parcours une distance \( d\). Rappeler une formule liant \( d\), \( t\) et \( v\).

    Dans chacun des cas suivants, déterminer la quantité manquante :
    \begin{enumerate}
        \item
            \( v=\SI{100}{\kilo\meter\per\hour}\), \( t=\SI{3}{\hour}\), \( d=\ldots\).
        \item
            \( v=\SI{130}{\kilo\meter\per\hour}\), \( t=\SI{30}{\minute}\), \( d=\ldots\).
        \item
            \( v=\ldots\), \( t=\SI{2}{\hour}\), \( d=\SI{600}{\kilo\meter}\).
        \item
            \( v=\SI{200}{\kilo\meter\per\hour}\), \( t=\ldots\), \( d=\SI{500}{\kilo\meter}\).
        \item
            \( v=\SI{50}{\kilo\meter\per\hour}\), \( t=\ldots\), \( d=\SI{200}{\kilo\meter}\).
        \item
            \( v=\SI{15}{\meter\per\second}\), \( t=\SI{12}{\second}\), \( d=\ldots\).
        \item
            \( v=\ldots\), \( t=\SI{45}{\minute}\), \( d=\SI{100}{\kilo\meter}\).
    \end{enumerate}

\corrref{smath-0943}
\end{exercice}
