% This is part of Un soupçon de mathématique sans être agressif pour autant
% Copyright (c) 2012-2013
%   Laurent Claessens
% See the file fdl-1.3.txt for copying conditions.

\begin{exercice}\label{exosmath-0148}

    Soient les fonctions \( f(x)=3x+4\) et \( g(x)=-4x+3\). Pour chacune de ces deux fonctions,
    \begin{enumerate}
        \item
            Étudier le signe.
        \item
            Dessiner le graphe.
        \item
            Donner sous forme d'intervalles l'ensemble des abscisses sur lesquelles la fonction est négative.
    \end{enumerate}

\corrref{smath-0148}
\end{exercice}
