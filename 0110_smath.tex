% This is part of Un soupçon de mathématique sans être agressif pour autant
% Copyright (c) 2012-2014
%   Laurent Claessens
% See the file fdl-1.3.txt for copying conditions.

%+++++++++++++++++++++++++++++++++++++++++++++++++++++++++++++++++++++++++++++++++++++++++++++++++++++++++++++++++++++++++++ 
\section{Produit d'un vecteur par un réel}
%+++++++++++++++++++++++++++++++++++++++++++++++++++++++++++++++++++++++++++++++++++++++++++++++++++++++++++++++++++++++++++

\begin{definition}
    Soit \( \lambda\in \eR\) et \( \vect{ u }\), un vecteur. Si dans un repère \( \vect{ u }=\begin{pmatrix}
        x    \\ 
        y    
    \end{pmatrix}\), alors le vecteur \( \lambda\vect{ u }\) est le vecteur de coordonnées \( \begin{pmatrix}
        \lambda x    \\ 
        \lambda y    
    \end{pmatrix}\) dans ce repère.
\end{definition}

%\begin{Aretenir}
%    Les règles de calcul vectoriel vis-à-vis de la multiplication et de l'addition sont essentiellement les mêmes que celles qui fonctionnent avec les réels :
%    \begin{enumerate}
%        \item
%            \( (\lambda+\mu)\vect{ u }=\lambda\vect{ u }+\mu\vect{ u }\)
%        \item
%            \( \lambda(\mu\vect{ u })=(\lambda\mu)\vect{ u }\)
%        \item
%            \( \lambda(\vect{ u }+\vect{ v })=\lambda\vect{ u }+\lambda\vect{ v }\).
%    \end{enumerate}
%\end{Aretenir}

%+++++++++++++++++++++++++++++++++++++++++++++++++++++++++++++++++++++++++++++++++++++++++++++++++++++++++++++++++++++++++++ 
\section{Parallélisme et colinéarité}
%+++++++++++++++++++++++++++++++++++++++++++++++++++++++++++++++++++++++++++++++++++++++++++++++++++++++++++++++++++++++++++

\begin{definition}
    Deux vecteurs \( \vect{ u }\) et \( \vect{ v }\) sont \defe{colinéaires}{colinéaire (vecteurs)} si il existe \( \lambda\in \eR\) tel que \( \vect{ u }=\lambda\vect{ v }\).
\end{definition}

\begin{propriete}
    \begin{enumerate}
        \item
            Les droites \( (AB)\) et \( (CD)\) sont parallèles si et seulement si les vecteurs \( \vect{ AB }\) et \( \vect{ CD }\) sont colinéaires.
        \item
            Les points \( A\), \( B\) et \( C\) sont alignés si et seulement si les vecteurs \( \vect{ AB }\) et \( \vect{ AC }\) sont colinéaires.
    \end{enumerate}
\end{propriete}

%+++++++++++++++++++++++++++++++++++++++++++++++++++++++++++++++++++++++++++++++++++++++++++++++++++++++++++++++++++++++++++ 
\section{Milieu d'un segment}
%+++++++++++++++++++++++++++++++++++++++++++++++++++++++++++++++++++++++++++++++++++++++++++++++++++++++++++++++++++++++++++

\begin{propriete}\label{PropFHznUfJ}

        Le point \( M\) est milieu du segment \( [AB]\) si et seulement si \( \vect{ AM }=\vect{ MB }\).

        \begin{center}
   \input{Fig_VectoMilieuNuWgHW.pstricks}
        \end{center}

\end{propriete}
