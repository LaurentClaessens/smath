% This is part of Un soupçon de mathématique sans être agressif pour autant
% Copyright (c) 2012-2013
%   Laurent Claessens
% See the file fdl-1.3.txt for copying conditions.

%+++++++++++++++++++++++++++++++++++++++++++++++++++++++++++++++++++++++++++++++++++++++++++++++++++++++++++++++++++++++++++ 
\section{Parallélisme et colinéarité}
%+++++++++++++++++++++++++++++++++++++++++++++++++++++++++++++++++++++++++++++++++++++++++++++++++++++++++++++++++++++++++++

\begin{definition}
    Deux vecteurs \( \vect{ u }\) et \( \vect{ v }\) sont \defe{colinéaires}{colinéaire (vecteurs)} si il existe \( \lambda\in \eR\) tel que \( \vect{ u }=\lambda\vect{ v }\).
\end{definition}

\begin{propriete}
    \begin{enumerate}
        \item
            Les droites \( (AB)\) et \( (CD)\) sont parallèles si et seulement si les vecteurs \( \vect{ AB }\) et \( \vect{ CD }\) sont colinéaires.
        \item
            Les points \( A\), \( B\) et \( C\) sont alignés si et seulement si les vecteurs \( \vect{ AB }\) et \( \vect{ AC }\) sont colinéaires.
    \end{enumerate}
\end{propriete}

Nous savons qu'un quadrilatère ayant deux côtés parallèles de même longueur est un parallélogramme. Donc nous avons le critère suivant pour savoir si \( ABCD\) est un parallélogramme :
\begin{equation}
    \vect{ AB }=\vect{ CD }.
\end{equation}
Bien entendu les autres côtés fonctionnent aussi :
\begin{equation}
    \vect{ AC }=\vect{ BD }.
\end{equation}
Si une de ces deux égalités vectorielle est satisfaite, alors \( ABCD\) est un parallélogramme.

\begin{example}
    Les points \( A=(-2;-3)\), \( B=(-1,-1)\), \( C=(2;-2)\) et \( D=(1;-4)\) forment un parallélogramme.
\end{example}

%--------------------------------------------------------------------------------------------------------------------------- 
\subsection{Le milieu revisité}
%---------------------------------------------------------------------------------------------------------------------------

Notre travail sur les coordonnées de vecteurs nous permet de donner une preuve alternative à la propriété \ref{PropFHznUfJ}.

\begin{propriete}
    Soient les points \( A\), \( B\) et \( K\) tels que \( K\) soit le milieu du segment \( [AB]\). Alors nous avons \( \vect{ AK }=\vect{ BK }\).
\end{propriete}

\begin{proof}
    Nous divisons la preuve en petits pas.
    \begin{subproof}
        \item[Création du repère]
            Nous considérons un repère orthonormé \footnote{En réalité il n'est pas obligatoire qu'il soit orthonormé, mais ça ne coûte rien qu'il le soit.} dont \( A\) est l'origine.
        \item[Coordonnées des points]
            Dans le repère choisi, nous considérons les coordonnées des points \( K=(x_K;y_K)\) et \( B=(x_B,y_B)\). Nous allons aussi nommer \( I\) et \( J\) les points \( I=(x_K;0)\) et \( J=(x_B;0)\). Le dessin est maintenant comme suit :

            \begin{center}
%The result is on figure \ref{LabelFigfigureSZyxsvp}. % From file figureSZyxsvp
%\newcommand{\CaptionFigfigureSZyxsvp}{<+Type your caption here+>}
\input{Fig_figureSZyxsvp.pstricks}
            \end{center}
        \item[Utilisation du théorème de Thalès]
            Vu que les droites \( (KI)\) et \( (BJ)\) sont parallèles, nous pouvons utiliser le théorème de Thalès :
            \begin{equation}
                \frac{ AB }{ AK }=\frac{ AJ }{ AI }=\frac{ BJ }{ KI }.
            \end{equation}
            Nous remplaçons dans ces égalités les longueurs par ce qu'on connait. Vu que \( K\) est le milieu de \( [AB] \), nous avons \( \frac{ AB }{ AK }=2\). D'autre part, $AJ=x_B$, \( AI=x_K\), \( BJ=y_B\) et \( KI=y_K\), donc
            \begin{equation}
                2=\frac{ x_B }{ x_K }
            \end{equation}
            et
            \begin{equation}
                2=\frac{ y_B }{ y_K }.
            \end{equation}
            Autrement dit,
            \begin{equation}
                x_B=2x_K
            \end{equation}
            et
            \begin{equation}
                y_B=2y_K.
            \end{equation}
        \item[Les vecteurs en présence]
            Les vecteurs \( \vect{ AB }\) et \( \vect{ AK }\) ont pour coordonnées
            \begin{subequations}
                \begin{align}
                    \vect{ AB }=\begin{pmatrix}
                        x_B    \\ 
                        y_B    
                    \end{pmatrix}\\
                    \vect{ AK }=\begin{pmatrix}
                        x_K    \\ 
                        y_K    
                    \end{pmatrix},
                \end{align}
            \end{subequations}
            Donc, étant donné que \( x_B=2x_K\) et \( y_B=2y_K\) nous avons
            \begin{equation}    \label{EqNGxKxaY}
                \vect{ AB }=2\vect{ AK }.
            \end{equation}
        \item[Utilisation de la loi de Chasles et conclusion]
            Nous savons que \( \vect{ AB }=\vect{ AK }+\vect{ KB }\), donc en replaçant \( \vect{ AB }\) par \( 2\vect{ AK }\) nous avons
            \begin{equation}
                2\vect{ AK }=\vect{ AK }+\vect{ KB },
            \end{equation}
            ce qui implique que
            \begin{equation}
                \vect{ AK }=\vect{ KB },
            \end{equation}
            ce qu'il fallait.
    \end{subproof}
\end{proof}
Nous notons aussi au passage l'intéressante formule \eqref{EqNGxKxaY} :
\begin{equation}
    \vect{ AB }=2\vect{ AK }.
\end{equation}


%+++++++++++++++++++++++++++++++++++++++++++++++++++++++++++++++++++++++++++++++++++++++++++++++++++++++++++++++++++++++++++ 
\section{Milieu d'un segment}
%+++++++++++++++++++++++++++++++++++++++++++++++++++++++++++++++++++++++++++++++++++++++++++++++++++++++++++++++++++++++++++

\begin{propriete}\label{PropFHznUfJ}
    \begin{multicols}{2}

        Le point \( M\) est milieu du segment \( [AB]\) si et seulement si \( \vect{ AM }=\vect{ MB }\).

        \columnbreak

%The result is on figure \ref{LabelFigVectoMilieuNuWgHW}. % From file VectoMilieuNuWgHW
%\newcommand{\CaptionFigVectoMilieuNuWgHW}{<+Type your caption here+>}
\input{Fig_VectoMilieuNuWgHW.pstricks}

    \end{multicols}
\end{propriete}

\begin{proof}
    Supposons pour commencer que \( M\) soit le milieu du segment \( [AB]\). Afin de prouver que \( \vect{ AM }=\vect{ MB }\), nous prouvons que \( AMBM\) est un parallélogramme. Vu que \( M\) est sur la droite \( (AB)\) nous avons évidemment le parallélisme des côtés : \( (AM)\parallel (MB)\).

    En ce qui concerne l'égalité des longueurs, \( AM=BM\) parce que \( M\) est milieu de \( [AB]\).

    Pour la réciproque, nous supposons que \( \vect{ AM }=\vect{ MB }\) et nous devons prouver que \( M\) est le milieu de \( [AB]\).  Par hypothèse, \( AMBM\) est un parallélogramme, et donc les diagonales se coupent en leur milieu. Ces diagonales sont \( [AB]\) et \( [MM]\). Le milieu de \( [MM]\) est évidemment \( M\), qui est alors le milieu de \( [AB]\), ce qu'il fallait prouver.
\end{proof}

\begin{theorem}[Théorème de milieux]

    \begin{multicols}{2}
    La droite joignant les milieux de deux côtés d'un triangle est parallèle au troisième côté.

    Sur la figure ci-contre \( I\) et \( J\) sont les milieux de \( [AB]\) et \( [AC]\) et nous avons l'égalité vectorielle
    \begin{equation}
        \vect{ BC }=2\vect{ IJ }.
    \end{equation}

    \columnbreak

%The result is on figure \ref{LabelFigfigureVNaHvXi}. % From file figureVNaHvXi
%\newcommand{\CaptionFigfigureVNaHvXi}{<+Type your caption here+>}
    \begin{center}
\input{Fig_figureVNaHvXi.pstricks}
    \end{center}

    \end{multicols}
\end{theorem}

\begin{proof}
    En utilisant les relations de Chasles : \( \vect{ IJ }=\vect{ IB }+\vect{ BC }+\vect{ CJ }\). Mais \( \vect{ IB }=\vect{ AI }\) et \( \vect{ CJ }=\vect{ JA }\), donc
    \begin{equation}
        \vect{ IJ }=\vect{ AI }+\vect{ BC }+\vect{ JA }=\vect{ JI }+\vect{ BC }.
    \end{equation}
    Étant donné que \( \vect{ IJ }=-\vect{ JI }\) nous trouvons \( 2\vect{ IJ }=\vect{ BC }\).
\end{proof}

On démontre de la même façon que si \( \vect{ AI }=k\vect{ AC }\) et \( \vect{ BJ }=k\vect{ BC }\), alors
\begin{equation}
    \vect{ IJ }=(1-k)\vect{ AB }.
\end{equation}

\begin{propriete}
    Si \( ABCD\) est un quadrilatère quelconque et si \( I\), \( J\), \( K\) et \( L\) sont les milieux des segments consécutifs de \( ABCD\), alors \( IJKL\) est un parallélogramme.
\end{propriete}

\begin{proof}

    \begin{multicols}{2}

        Nous considérons la diagonale \( [AC]\) et le triangle \( ABC\). Les points \( I\) et \( J\) étant les milieux des côtés \( [AB]\) et \( [BC]\), nous avons l'égalité vectorielle \( \vect{ AC }=2\vect{ IJ }\).

    De la même façon, dans le triangle \( ACD\), nous avons \( \vect{ AC }=2\vect{ LK }\). Par conséquent \( \vect{ LK }=\vect{ IJ }\), ce qui prouve que \( IJKL\) est un parallélogramme.

        \columnbreak

%The result is on figure \ref{LabelFigfigureBOuQJyj}. % From file figureBOuQJyj
%\newcommand{\CaptionFigfigureBOuQJyj}{<+Type your caption here+>}
   \begin{center}
\input{Fig_figureBOuQJyj.pstricks}
   \end{center}

    \end{multicols}


\end{proof}
