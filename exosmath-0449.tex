% This is part of Un soupçon de mathématique sans être agressif pour autant
% Copyright (c) 2013
%   Laurent Claessens
% See the file fdl-1.3.txt for copying conditions.

\begin{exercice}\label{exosmath-0449}

    \begin{wrapfigure}{r}{8.0cm}
        \vspace{-1.5cm}
        \centering
        \input{Fig_ITlywMb.pstricks}
    \end{wrapfigure}

    Les questions suivantes se rapportent à la figure ci-contre représentant le graphe d'une fonction \( f\). Les réponses peuvent être des valeurs approximatives, dans la mesure de la précision du dessin.
    \begin{enumerate}
        \item
            Quel est le domaine de définition de \( f\) ?
        \item
            Donner l'image de \( -1\) par la fonction \( f\).
        \item
            Résoudre \( f(x)=3\) et \( f(x)=-2\).
        \item
            Vrai ou faux ?
            \begin{enumerate}
                \item
                    La fonction \( f\) est croissante sur \( \mathopen[ -3 , -2 \mathclose]\).
                \item
                    \( f(2)=-1\)
                \item
                    La fonction \( f\) est décroissante sur \( \mathopen[ 2 , 4 \mathclose]\).
                \item
                    La fonction \( f\) est décroissante sur \( \mathopen[ -2 , 0 \mathclose]\).
            \end{enumerate}
        \item
            Donner les tableau de variation de \( f\).
        \item
            Résoudre \( f(x)\geq 2.5\).
    \end{enumerate}

\corrref{smath-0449}
\end{exercice}
