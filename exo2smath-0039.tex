% This is part of Un soupçon de mathématique sans être agressif pour autant
% Copyright (c) 2014
%   Laurent Claessens
% See the file fdl-1.3.txt for copying conditions.

\begin{exercice}\label{exo2smath-0039}

    Effectuer les soustractions suivantes :
    \begin{multicols}{3}
        \begin{enumerate}
            \item
                \( \dfrac{  12  }{ 13 }-\dfrac{  7  }{ 13 }\) 
            \item
                \( \dfrac{  1  }{ 3 }-\dfrac{  1  }{ 6 }\) 
            \item
                \( \dfrac{  1  }{ 2 }-\dfrac{  1  }{ 4 }\) 
            \item
                \( \dfrac{  9  }{5 }-\dfrac{  5  }{ 15 }\) 
            \item
                \( \dfrac{  5  }{ 6 }-\dfrac{  3  }{ 48 }\) 
            \item
                \( \dfrac{  9  }{ 7 }-\dfrac{  64  }{ 63 }\) 
            \item
                \( \dfrac{  19  }{ 99 }-\dfrac{  1  }{ 11 }\) 
        \end{enumerate}
    \end{multicols}


\corrref{2smath-0039}
\end{exercice}
