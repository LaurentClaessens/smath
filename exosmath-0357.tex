% This is part of Un soupçon de mathématique sans être agressif pour autant
% Copyright (c) 2013
%   Laurent Claessens
% See the file fdl-1.3.txt for copying conditions.

\begin{exercice}\label{exosmath-0357}

    Nous lançons un dé et puis un autre, et nous notons les résultats.
    \begin{enumerate}
        \item
            Quel est le cardinal de l'univers de cette expérience ?
        \item
            Nous considérons la différence entre le plus grand des deux et le plus petit. Dessiner un tableau résumant les différentes possibilités.
        \item
            Donner les probabilités de
            \begin{enumerate}
                \item
                    «Obtenir une différence égale à \( 1\)».
                \item
                    «Obtenir une différence plus petite ou égale à 3».
            \end{enumerate}
    \end{enumerate}

% AFAIRE : ceci provient d'un corrigé d'un DS du jeudi 31 mai 2012 de M.Lenzen. Il faut le retrouver et le citer.

\corrref{smath-0357}
\end{exercice}
