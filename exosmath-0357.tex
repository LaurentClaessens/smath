% This is part of Un soupçon de mathématique sans être agressif pour autant
% Copyright (c) 2013
%   Laurent Claessens
% See the file fdl-1.3.txt for copying conditions.

\begin{exercice}[\cite{MPQfqzH}]\label{exosmath-0357}

    Nous lançons un dé et puis un autre, et nous notons les résultats.
    \begin{enumerate}
        \item
            Combien d'éléments contient l'univers (l'ensemble de toutes les issues possibles) de cette expérience ? 
        \item
            Nous considérons maintenant la différence entre le plus grand des deux et le plus petit. Par exemple si nous lançons un \( 2\) puis un \( 5\), nous considérons avoir obtenu trois. 
            \begin{enumerate}
                \item 
            Dessiner un tableau résumant les différentes possibilités.
                \item
                   Quelle est la probabilité d'obtenir \( 1\) ? Quelle est la probabilité d'obtenir \( 6\) ?
                \item
                    Quelle est la probabilité d'obtenir un nombre plus petit ou égal à \( 3\) ?
            \end{enumerate}
    \end{enumerate}

\corrref{smath-0357}
\end{exercice}
