% This is part of Un soupçon de mathématique sans être agressif pour autant
% Copyright (c) 2014
%   Laurent Claessens
% See the file fdl-1.3.txt for copying conditions.

\begin{corrige}{smath-0599}


    Une façon de compléter le programme est comme ceci :

    \begin{fmpage}{0.9\linewidth}

    Écrire « Première note :» ; demander \( a\)

    Écrire « Seconde note :»;   demander \( b\)

    Écrire « Troisième note :»; demander \( c\)

    Si \info{(a+b+c)/3>10} , alors :

    \hspace{0.5cm} Écrire « Moyenne plus grande que \( 10/20\)» 

    Sinon :

    \hspace{0.5cm} Écrire « Moyenne plus petite que \( 10/20\)» 

\end{fmpage}

Dans ce cas nous avons écrit une condition \emph{stricte}, de telle sorte que lorsque la moyenne est exactement \( 10\), le programme tombe dans le «Sinon» et écrit «Moyenne plus petite que \( 10/20\)».

    En ce qui concerne le deuxième, le programme doit demander le rayon de la base et la hauteur :

    \begin{fmpage}{0.9\linewidth}

        Écrire «Combien vaut \info{la rayon de la base} ?»; Demander \info{r}

        Écrire «Combien vaut \info{la hauteur} ?»; Demander \info{h}

    \( V=\pi r^2h\)

    Écrire \( V\)

\end{fmpage}
\end{corrige}
