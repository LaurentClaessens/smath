% This is part of Un soupçon de mathématique sans être agressif pour autant
% Copyright (c) 2014
%   Laurent Claessens
% See the file fdl-1.3.txt for copying conditions.

\begin{exercice}\label{exosmath-0681}

    Un (vrai) fan de Michael Jackson possède une liste de lecture de \( 300\) titres dont \( 100\) de son idole. En mode «lecture aléatoire» il remarque que sont lecteur lui sort du Micheal Jackson trois fois sur quatre. Est-ce qu'il peut soupçonner son mode aléatoire de ne pas être tellement aléatoire ?

\corrref{smath-0681}
\end{exercice}
