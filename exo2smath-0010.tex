% This is part of Un soupçon de mathématique sans être agressif pour autant
% Copyright (c) 2015
%   Laurent Claessens
% See the file fdl-1.3.txt for copying conditions.

\begin{exercice}[Mesurer des hauteurs inaccessibles\cite{NRHooXFvgpp4}]\label{exo2smath-0010}

    Sur la figure suivante, les droites $(OL)$ et $(TE)$ sont parallèles. Les points $O$ et $L$ appartiennent respectivement aux demi-droites $[HT)$ et $[HL)$. On donne $HE =\SI{5}{\centi\meter}$, $HL = \SI{2}{\centi\meter}$, $TE=\SI{7}{\centi\meter}$ et $HO=\SI{3}{\centi\meter}$. Calculer les longueurs $HT$ et $OL$.


\begin{center}
   \input{Fig_LZKGooYGQxGy.pstricks}
\end{center}


\corrref{2smath-0010}
\end{exercice}
