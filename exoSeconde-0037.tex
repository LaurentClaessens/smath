% This is part of Un soupçon de mathématique sans être agressif pour autant
% Copyright (c) 2012,2014
%   Laurent Claessens
% See the file fdl-1.3.txt for copying conditions.

\begin{exercice}[\ldots/6]\label{exoSeconde-0037}

Le tableau suivant donne le salaire brut mensuel dans une entreprise : 
\bigskip

\begin{tabular}{|c||c|c|c|c|c|c|c|c|c|c|c|}
  \hline 
  \textbf{Salaire} & \ 900&1100&1300&1500&1700&1900&2100&2500&3100&4500&\text{total}\\
  \hline 
  \textbf{Effectif} & 12&10&20&18&8&8&5&5&2&1&89\\
  \hline
   & &&&&&&&&&&\\
  \hline
   & &&&&&&&&&&\\
  \hline
\end{tabular}


\begin{multicols}{2}
\begin{enumerate}
\item Dresser le tableau des effectifs cumulés croissants de cette série. 
\item Calculer le salaire moyen brut.

\item Est-il vrai que la moitié des salariés gagnent moins que la moyenne ?
\item Calculer le pourcentage des salariés dont le salaire est compris dans l'intervalle interquartile. 

%\item Représenter le diagramme en bâtons correspondant à la répartition des salaires dans cette entreprise.

\item Dans la série initiale, une erreur a été commise : il y a en fait $8$ salaires de $1500$ euros et $18$ salaires de $1700$ euros. Dire, sans calculatrice, si cette erreur augmente ou diminue ou n'influence pas le salaire moyen et le salaire médian. Expliquer.

\end{enumerate}
\end{multicols}
Note : des lignes vides ont été laissées dans le tableau; utilisez-les comme bon vous semble.

\corrref{Seconde-0037}
\end{exercice}
