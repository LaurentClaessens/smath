\vbox{1
\begin{wrapfigure}{r}{5.0cm}
   \vspace{-0.5cm}        % à adapter.
   \centering
   \input{Fig_OKTXHoc.pstricks}
\end{wrapfigure}

    La figure ci-contre est un cube de \unit{3}{\centi\meter}.

    \begin{enumerate}
    \item
    Quelle est la nature du triangle \(CAB\) ? 
    \item
    Quel est son périmètre ?
    \item
    Quelle est son aire ? (pour les rapides)
    \end{enumerate}
    

}
\vspace{2cm}
\vbox{2
\begin{wrapfigure}{r}{5.0cm}
   \vspace{-0.5cm}        % à adapter.
   \centering
   \input{Fig_OKTXHoc.pstricks}
\end{wrapfigure}

    La figure ci-contre est un cube de \unit{5}{\centi\meter}.

    \begin{enumerate}
    \item
    Quelle est la nature du triangle \(HAE\) ? 
    \item
    Quel est son périmètre ?
    \item
    Quelle est son aire ? (pour les rapides)
    \end{enumerate}
    

}
\vspace{2cm}
\vbox{3
\begin{wrapfigure}{r}{5.0cm}
   \vspace{-0.5cm}        % à adapter.
   \centering
   \input{Fig_OKTXHoc.pstricks}
\end{wrapfigure}

    La figure ci-contre est un cube de \unit{3}{\centi\meter}.

    \begin{enumerate}
    \item
    Quelle est la nature du triangle \(BCA\) ? 
    \item
    Quel est son périmètre ?
    \item
    Quelle est son aire ? (pour les rapides)
    \end{enumerate}
    

}
\vspace{2cm}
\vbox{4
\begin{wrapfigure}{r}{5.0cm}
   \vspace{-0.5cm}        % à adapter.
   \centering
   \input{Fig_OKTXHoc.pstricks}
\end{wrapfigure}

    La figure ci-contre est un cube de \unit{6}{\centi\meter}.

    \begin{enumerate}
    \item
    Quelle est la nature du triangle \(ACF\) ? 
    \item
    Quel est son périmètre ?
    \item
    Quelle est son aire ? (pour les rapides)
    \end{enumerate}
    

}
\vspace{2cm}
\vbox{5
\begin{wrapfigure}{r}{5.0cm}
   \vspace{-0.5cm}        % à adapter.
   \centering
   \input{Fig_OKTXHoc.pstricks}
\end{wrapfigure}

    La figure ci-contre est un cube de \unit{2}{\centi\meter}.

    \begin{enumerate}
    \item
    Quelle est la nature du triangle \(CHF\) ? 
    \item
    Quel est son périmètre ?
    \item
    Quelle est son aire ? (pour les rapides)
    \end{enumerate}
    

}
\vspace{2cm}
\vbox{6
\begin{wrapfigure}{r}{5.0cm}
   \vspace{-0.5cm}        % à adapter.
   \centering
   \input{Fig_OKTXHoc.pstricks}
\end{wrapfigure}

    La figure ci-contre est un cube de \unit{3}{\centi\meter}.

    \begin{enumerate}
    \item
    Quelle est la nature du triangle \(BEC\) ? 
    \item
    Quel est son périmètre ?
    \item
    Quelle est son aire ? (pour les rapides)
    \end{enumerate}
    

}
\vspace{2cm}
\vbox{7
\begin{wrapfigure}{r}{5.0cm}
   \vspace{-0.5cm}        % à adapter.
   \centering
   \input{Fig_OKTXHoc.pstricks}
\end{wrapfigure}

    La figure ci-contre est un cube de \unit{3}{\centi\meter}.

    \begin{enumerate}
    \item
    Quelle est la nature du triangle \(CHE\) ? 
    \item
    Quel est son périmètre ?
    \item
    Quelle est son aire ? (pour les rapides)
    \end{enumerate}
    

}
\vspace{2cm}
\vbox{8
\begin{wrapfigure}{r}{5.0cm}
   \vspace{-0.5cm}        % à adapter.
   \centering
   \input{Fig_OKTXHoc.pstricks}
\end{wrapfigure}

    La figure ci-contre est un cube de \unit{2}{\centi\meter}.

    \begin{enumerate}
    \item
    Quelle est la nature du triangle \(GHE\) ? 
    \item
    Quel est son périmètre ?
    \item
    Quelle est son aire ? (pour les rapides)
    \end{enumerate}
    

}
\vspace{2cm}
\vbox{9
\begin{wrapfigure}{r}{5.0cm}
   \vspace{-0.5cm}        % à adapter.
   \centering
   \input{Fig_OKTXHoc.pstricks}
\end{wrapfigure}

    La figure ci-contre est un cube de \unit{3}{\centi\meter}.

    \begin{enumerate}
    \item
    Quelle est la nature du triangle \(GBD\) ? 
    \item
    Quel est son périmètre ?
    \item
    Quelle est son aire ? (pour les rapides)
    \end{enumerate}
    

}
\vspace{2cm}
\vbox{10
\begin{wrapfigure}{r}{5.0cm}
   \vspace{-0.5cm}        % à adapter.
   \centering
   \input{Fig_OKTXHoc.pstricks}
\end{wrapfigure}

    La figure ci-contre est un cube de \unit{7}{\centi\meter}.

    \begin{enumerate}
    \item
    Quelle est la nature du triangle \(EDA\) ? 
    \item
    Quel est son périmètre ?
    \item
    Quelle est son aire ? (pour les rapides)
    \end{enumerate}
    

}
\vspace{2cm}
\vbox{11
\begin{wrapfigure}{r}{5.0cm}
   \vspace{-0.5cm}        % à adapter.
   \centering
   \input{Fig_OKTXHoc.pstricks}
\end{wrapfigure}

    La figure ci-contre est un cube de \unit{6}{\centi\meter}.

    \begin{enumerate}
    \item
    Quelle est la nature du triangle \(ABC\) ? 
    \item
    Quel est son périmètre ?
    \item
    Quelle est son aire ? (pour les rapides)
    \end{enumerate}
    

}
\vspace{2cm}
\vbox{12
\begin{wrapfigure}{r}{5.0cm}
   \vspace{-0.5cm}        % à adapter.
   \centering
   \input{Fig_OKTXHoc.pstricks}
\end{wrapfigure}

    La figure ci-contre est un cube de \unit{5}{\centi\meter}.

    \begin{enumerate}
    \item
    Quelle est la nature du triangle \(GAC\) ? 
    \item
    Quel est son périmètre ?
    \item
    Quelle est son aire ? (pour les rapides)
    \end{enumerate}
    

}
\vspace{2cm}
\vbox{13
\begin{wrapfigure}{r}{5.0cm}
   \vspace{-0.5cm}        % à adapter.
   \centering
   \input{Fig_OKTXHoc.pstricks}
\end{wrapfigure}

    La figure ci-contre est un cube de \unit{5}{\centi\meter}.

    \begin{enumerate}
    \item
    Quelle est la nature du triangle \(ECF\) ? 
    \item
    Quel est son périmètre ?
    \item
    Quelle est son aire ? (pour les rapides)
    \end{enumerate}
    

}
\vspace{2cm}
\vbox{14
\begin{wrapfigure}{r}{5.0cm}
   \vspace{-0.5cm}        % à adapter.
   \centering
   \input{Fig_OKTXHoc.pstricks}
\end{wrapfigure}

    La figure ci-contre est un cube de \unit{5}{\centi\meter}.

    \begin{enumerate}
    \item
    Quelle est la nature du triangle \(ECF\) ? 
    \item
    Quel est son périmètre ?
    \item
    Quelle est son aire ? (pour les rapides)
    \end{enumerate}
    

}
\vspace{2cm}
\vbox{15
\begin{wrapfigure}{r}{5.0cm}
   \vspace{-0.5cm}        % à adapter.
   \centering
   \input{Fig_OKTXHoc.pstricks}
\end{wrapfigure}

    La figure ci-contre est un cube de \unit{3}{\centi\meter}.

    \begin{enumerate}
    \item
    Quelle est la nature du triangle \(DAH\) ? 
    \item
    Quel est son périmètre ?
    \item
    Quelle est son aire ? (pour les rapides)
    \end{enumerate}
    

}
\vspace{2cm}
\vbox{16
\begin{wrapfigure}{r}{5.0cm}
   \vspace{-0.5cm}        % à adapter.
   \centering
   \input{Fig_OKTXHoc.pstricks}
\end{wrapfigure}

    La figure ci-contre est un cube de \unit{2}{\centi\meter}.

    \begin{enumerate}
    \item
    Quelle est la nature du triangle \(FBG\) ? 
    \item
    Quel est son périmètre ?
    \item
    Quelle est son aire ? (pour les rapides)
    \end{enumerate}
    

}
\vspace{2cm}
\vbox{17
\begin{wrapfigure}{r}{5.0cm}
   \vspace{-0.5cm}        % à adapter.
   \centering
   \input{Fig_OKTXHoc.pstricks}
\end{wrapfigure}

    La figure ci-contre est un cube de \unit{7}{\centi\meter}.

    \begin{enumerate}
    \item
    Quelle est la nature du triangle \(EBD\) ? 
    \item
    Quel est son périmètre ?
    \item
    Quelle est son aire ? (pour les rapides)
    \end{enumerate}
    

}
\vspace{2cm}
\vbox{18
\begin{wrapfigure}{r}{5.0cm}
   \vspace{-0.5cm}        % à adapter.
   \centering
   \input{Fig_OKTXHoc.pstricks}
\end{wrapfigure}

    La figure ci-contre est un cube de \unit{4}{\centi\meter}.

    \begin{enumerate}
    \item
    Quelle est la nature du triangle \(HFG\) ? 
    \item
    Quel est son périmètre ?
    \item
    Quelle est son aire ? (pour les rapides)
    \end{enumerate}
    

}
\vspace{2cm}
\vbox{19
\begin{wrapfigure}{r}{5.0cm}
   \vspace{-0.5cm}        % à adapter.
   \centering
   \input{Fig_OKTXHoc.pstricks}
\end{wrapfigure}

    La figure ci-contre est un cube de \unit{5}{\centi\meter}.

    \begin{enumerate}
    \item
    Quelle est la nature du triangle \(EBF\) ? 
    \item
    Quel est son périmètre ?
    \item
    Quelle est son aire ? (pour les rapides)
    \end{enumerate}
    

}
\vspace{2cm}
\vbox{20
\begin{wrapfigure}{r}{5.0cm}
   \vspace{-0.5cm}        % à adapter.
   \centering
   \input{Fig_OKTXHoc.pstricks}
\end{wrapfigure}

    La figure ci-contre est un cube de \unit{3}{\centi\meter}.

    \begin{enumerate}
    \item
    Quelle est la nature du triangle \(BAD\) ? 
    \item
    Quel est son périmètre ?
    \item
    Quelle est son aire ? (pour les rapides)
    \end{enumerate}
    

}
\vspace{2cm}
\vbox{21
\begin{wrapfigure}{r}{5.0cm}
   \vspace{-0.5cm}        % à adapter.
   \centering
   \input{Fig_OKTXHoc.pstricks}
\end{wrapfigure}

    La figure ci-contre est un cube de \unit{5}{\centi\meter}.

    \begin{enumerate}
    \item
    Quelle est la nature du triangle \(HEF\) ? 
    \item
    Quel est son périmètre ?
    \item
    Quelle est son aire ? (pour les rapides)
    \end{enumerate}
    

}
\vspace{2cm}
\vbox{22
\begin{wrapfigure}{r}{5.0cm}
   \vspace{-0.5cm}        % à adapter.
   \centering
   \input{Fig_OKTXHoc.pstricks}
\end{wrapfigure}

    La figure ci-contre est un cube de \unit{3}{\centi\meter}.

    \begin{enumerate}
    \item
    Quelle est la nature du triangle \(FGE\) ? 
    \item
    Quel est son périmètre ?
    \item
    Quelle est son aire ? (pour les rapides)
    \end{enumerate}
    

}
\vspace{2cm}
\vbox{23
\begin{wrapfigure}{r}{5.0cm}
   \vspace{-0.5cm}        % à adapter.
   \centering
   \input{Fig_OKTXHoc.pstricks}
\end{wrapfigure}

    La figure ci-contre est un cube de \unit{3}{\centi\meter}.

    \begin{enumerate}
    \item
    Quelle est la nature du triangle \(DEF\) ? 
    \item
    Quel est son périmètre ?
    \item
    Quelle est son aire ? (pour les rapides)
    \end{enumerate}
    

}
\vspace{2cm}
\vbox{24
\begin{wrapfigure}{r}{5.0cm}
   \vspace{-0.5cm}        % à adapter.
   \centering
   \input{Fig_OKTXHoc.pstricks}
\end{wrapfigure}

    La figure ci-contre est un cube de \unit{6}{\centi\meter}.

    \begin{enumerate}
    \item
    Quelle est la nature du triangle \(AFH\) ? 
    \item
    Quel est son périmètre ?
    \item
    Quelle est son aire ? (pour les rapides)
    \end{enumerate}
    

}
\vspace{2cm}
\vbox{25
\begin{wrapfigure}{r}{5.0cm}
   \vspace{-0.5cm}        % à adapter.
   \centering
   \input{Fig_OKTXHoc.pstricks}
\end{wrapfigure}

    La figure ci-contre est un cube de \unit{5}{\centi\meter}.

    \begin{enumerate}
    \item
    Quelle est la nature du triangle \(HCG\) ? 
    \item
    Quel est son périmètre ?
    \item
    Quelle est son aire ? (pour les rapides)
    \end{enumerate}
    

}
\vspace{2cm}
\vbox{26
\begin{wrapfigure}{r}{5.0cm}
   \vspace{-0.5cm}        % à adapter.
   \centering
   \input{Fig_OKTXHoc.pstricks}
\end{wrapfigure}

    La figure ci-contre est un cube de \unit{4}{\centi\meter}.

    \begin{enumerate}
    \item
    Quelle est la nature du triangle \(GDE\) ? 
    \item
    Quel est son périmètre ?
    \item
    Quelle est son aire ? (pour les rapides)
    \end{enumerate}
    

}
\vspace{2cm}
\vbox{27
\begin{wrapfigure}{r}{5.0cm}
   \vspace{-0.5cm}        % à adapter.
   \centering
   \input{Fig_OKTXHoc.pstricks}
\end{wrapfigure}

    La figure ci-contre est un cube de \unit{6}{\centi\meter}.

    \begin{enumerate}
    \item
    Quelle est la nature du triangle \(BGC\) ? 
    \item
    Quel est son périmètre ?
    \item
    Quelle est son aire ? (pour les rapides)
    \end{enumerate}
    

}
\vspace{2cm}
\vbox{28
\begin{wrapfigure}{r}{5.0cm}
   \vspace{-0.5cm}        % à adapter.
   \centering
   \input{Fig_OKTXHoc.pstricks}
\end{wrapfigure}

    La figure ci-contre est un cube de \unit{2}{\centi\meter}.

    \begin{enumerate}
    \item
    Quelle est la nature du triangle \(EFA\) ? 
    \item
    Quel est son périmètre ?
    \item
    Quelle est son aire ? (pour les rapides)
    \end{enumerate}
    

}
\vspace{2cm}
\vbox{29
\begin{wrapfigure}{r}{5.0cm}
   \vspace{-0.5cm}        % à adapter.
   \centering
   \input{Fig_OKTXHoc.pstricks}
\end{wrapfigure}

    La figure ci-contre est un cube de \unit{7}{\centi\meter}.

    \begin{enumerate}
    \item
    Quelle est la nature du triangle \(BEH\) ? 
    \item
    Quel est son périmètre ?
    \item
    Quelle est son aire ? (pour les rapides)
    \end{enumerate}
    

}
\vspace{2cm}
\vbox{30
\begin{wrapfigure}{r}{5.0cm}
   \vspace{-0.5cm}        % à adapter.
   \centering
   \input{Fig_OKTXHoc.pstricks}
\end{wrapfigure}

    La figure ci-contre est un cube de \unit{2}{\centi\meter}.

    \begin{enumerate}
    \item
    Quelle est la nature du triangle \(HFC\) ? 
    \item
    Quel est son périmètre ?
    \item
    Quelle est son aire ? (pour les rapides)
    \end{enumerate}
    

}
\vspace{2cm}
\vbox{31
\begin{wrapfigure}{r}{5.0cm}
   \vspace{-0.5cm}        % à adapter.
   \centering
   \input{Fig_OKTXHoc.pstricks}
\end{wrapfigure}

    La figure ci-contre est un cube de \unit{5}{\centi\meter}.

    \begin{enumerate}
    \item
    Quelle est la nature du triangle \(ABD\) ? 
    \item
    Quel est son périmètre ?
    \item
    Quelle est son aire ? (pour les rapides)
    \end{enumerate}
    

}
\vspace{2cm}
\vbox{32
\begin{wrapfigure}{r}{5.0cm}
   \vspace{-0.5cm}        % à adapter.
   \centering
   \input{Fig_OKTXHoc.pstricks}
\end{wrapfigure}

    La figure ci-contre est un cube de \unit{3}{\centi\meter}.

    \begin{enumerate}
    \item
    Quelle est la nature du triangle \(CAG\) ? 
    \item
    Quel est son périmètre ?
    \item
    Quelle est son aire ? (pour les rapides)
    \end{enumerate}
    

}
\vspace{2cm}
\vbox{33
\begin{wrapfigure}{r}{5.0cm}
   \vspace{-0.5cm}        % à adapter.
   \centering
   \input{Fig_OKTXHoc.pstricks}
\end{wrapfigure}

    La figure ci-contre est un cube de \unit{5}{\centi\meter}.

    \begin{enumerate}
    \item
    Quelle est la nature du triangle \(GCF\) ? 
    \item
    Quel est son périmètre ?
    \item
    Quelle est son aire ? (pour les rapides)
    \end{enumerate}
    

}
\vspace{2cm}
\vbox{34
\begin{wrapfigure}{r}{5.0cm}
   \vspace{-0.5cm}        % à adapter.
   \centering
   \input{Fig_OKTXHoc.pstricks}
\end{wrapfigure}

    La figure ci-contre est un cube de \unit{4}{\centi\meter}.

    \begin{enumerate}
    \item
    Quelle est la nature du triangle \(GEA\) ? 
    \item
    Quel est son périmètre ?
    \item
    Quelle est son aire ? (pour les rapides)
    \end{enumerate}
    

}
\vspace{2cm}
\vbox{35
\begin{wrapfigure}{r}{5.0cm}
   \vspace{-0.5cm}        % à adapter.
   \centering
   \input{Fig_OKTXHoc.pstricks}
\end{wrapfigure}

    La figure ci-contre est un cube de \unit{2}{\centi\meter}.

    \begin{enumerate}
    \item
    Quelle est la nature du triangle \(GDC\) ? 
    \item
    Quel est son périmètre ?
    \item
    Quelle est son aire ? (pour les rapides)
    \end{enumerate}
    

}
\vspace{2cm}
\vbox{36
\begin{wrapfigure}{r}{5.0cm}
   \vspace{-0.5cm}        % à adapter.
   \centering
   \input{Fig_OKTXHoc.pstricks}
\end{wrapfigure}

    La figure ci-contre est un cube de \unit{6}{\centi\meter}.

    \begin{enumerate}
    \item
    Quelle est la nature du triangle \(GDC\) ? 
    \item
    Quel est son périmètre ?
    \item
    Quelle est son aire ? (pour les rapides)
    \end{enumerate}
    

}
\vspace{2cm}
\vbox{37
\begin{wrapfigure}{r}{5.0cm}
   \vspace{-0.5cm}        % à adapter.
   \centering
   \input{Fig_OKTXHoc.pstricks}
\end{wrapfigure}

    La figure ci-contre est un cube de \unit{3}{\centi\meter}.

    \begin{enumerate}
    \item
    Quelle est la nature du triangle \(BHF\) ? 
    \item
    Quel est son périmètre ?
    \item
    Quelle est son aire ? (pour les rapides)
    \end{enumerate}
    

}
\vspace{2cm}
\vbox{38
\begin{wrapfigure}{r}{5.0cm}
   \vspace{-0.5cm}        % à adapter.
   \centering
   \input{Fig_OKTXHoc.pstricks}
\end{wrapfigure}

    La figure ci-contre est un cube de \unit{2}{\centi\meter}.

    \begin{enumerate}
    \item
    Quelle est la nature du triangle \(FBD\) ? 
    \item
    Quel est son périmètre ?
    \item
    Quelle est son aire ? (pour les rapides)
    \end{enumerate}
    

}
\vspace{2cm}
\vbox{39
\begin{wrapfigure}{r}{5.0cm}
   \vspace{-0.5cm}        % à adapter.
   \centering
   \input{Fig_OKTXHoc.pstricks}
\end{wrapfigure}

    La figure ci-contre est un cube de \unit{5}{\centi\meter}.

    \begin{enumerate}
    \item
    Quelle est la nature du triangle \(DHE\) ? 
    \item
    Quel est son périmètre ?
    \item
    Quelle est son aire ? (pour les rapides)
    \end{enumerate}
    

}
\vspace{2cm}
\vbox{40
\begin{wrapfigure}{r}{5.0cm}
   \vspace{-0.5cm}        % à adapter.
   \centering
   \input{Fig_OKTXHoc.pstricks}
\end{wrapfigure}

    La figure ci-contre est un cube de \unit{6}{\centi\meter}.

    \begin{enumerate}
    \item
    Quelle est la nature du triangle \(FDC\) ? 
    \item
    Quel est son périmètre ?
    \item
    Quelle est son aire ? (pour les rapides)
    \end{enumerate}
    

}
\vspace{2cm}
