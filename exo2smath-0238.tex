% This is part of Un soupçon de mathématique sans être agressif pour autant
% Copyright (c) 2015
%   Laurent Claessens
% See the file fdl-1.3.txt for copying conditions.

\begin{exercice}[\ldots\ldots/4]\label{exo2smath-0238}

    \setlength{\columnseprule}{0pt}       % ligne verticale de séparation entre les deux colonnes pour multicol
    \begin{enumerate}
        \item
            Dessiner un axe gradué de \( -5\) à \( 5\).
        \item
            Placer les points suivants sur un axe gradué : \( A(3)\), \( B(-3)\), \( C(4)\), \( D(-\dfrac{ 3 }{ 4 })\), \( E(-4)\).
        \item
            Donner deux points dont la distance est donnée par chacun des calculs suivants :
            \begin{multicols}{2}
            \begin{enumerate}
                \item
                    \( 4-(-3)\)
                \item
                    \( \dfrac{ 3 }{ 4 }+3\)
                \item
                    \( 4-3\)
                \item
                    \( 4+3\)
            \end{enumerate}
            \end{multicols}
    \end{enumerate}

    Note : vous devez graduer chaque unité votre axe gradué, et il doit être à l'échelle !

\corrref{2smath-0238}
\end{exercice}
