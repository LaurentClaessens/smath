% This is part of Un soupçon de mathématique sans être agressif pour autant
% Copyright (c) 2015
%   Laurent Claessens
% See the file fdl-1.3.txt for copying conditions.

% This is part of Un soupçon de mathématique sans être agressif pour autant
% Copyright (c) 2015
%   Laurent Claessens
% See the file fdl-1.3.txt for copying conditions.

%--------------------------------------------------------------------------------------------------------------------------- 
\subsection*{Activité : mieux que Thalès ?}
%---------------------------------------------------------------------------------------------------------------------------

À propos du dessin suivant,
\begin{center}
   \input{Fig_GSPNooCOfCGS.pstricks}
\end{center}
Maylis a écrit :
\begin{equation}
    \frac{ A'A }{ OA' }=\frac{ B'B }{ OB' }=\frac{ C'C }{ OC' }
\end{equation}
Est-ce correct ?


%+++++++++++++++++++++++++++++++++++++++++++++++++++++++++++++++++++++++++++++++++++++++++++++++++++++++++++++++++++++++++++ 
\section{Cosinus}
%+++++++++++++++++++++++++++++++++++++++++++++++++++++++++++++++++++++++++++++++++++++++++++++++++++++++++++++++++++++++++++

D'abord un peu de vocabulaire

\begin{definition}[\cite{NRHooXFvgpp4}]
Dans un triangle rectangle,
\begin{enumerate}
    \item
        l'\defe{hypoténuse}{hypoténuse} est le côté opposé à l'angle droit (c'est le plus long des trois côtés) ;
\item
    le côté \defe{adjacent}{adjacent} à un angle aigu est le côté de cet angle qui n'est pas l'hypoténuse.
\end{enumerate}
\end{definition}


\begin{definition}[\cite{NRHooXFvgpp4}]
    Dans un triangle rectangle, le \defe{cosinus}{cosinus} d'un angle aigu est le quotient de la longueur du côté adjacent à cet angle par la longueur de l'hypoténuse.
\end{definition}
