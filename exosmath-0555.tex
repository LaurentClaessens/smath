% This is part of Un soupçon de mathématique sans être agressif pour autant
% Copyright (c) 2013
%   Laurent Claessens
% See the file fdl-1.3.txt for copying conditions.

\begin{exercice}\label{exosmath-0555}

Suite à un plantage d'ordinateur, Luc, l'informaticien de service parvient à récupérer ce bout de programme servant à calculer le pourcentage de réduction auquel à droit un client (sur un article qui coûte \( 250\) euros) :
\begin{fmpage}{0.9\linewidth}

    Prix prend la valeur \( 250\)

    Écrire « Quel est l'âge du client » 

    Demander \( n\)

    Si \( n < 25\) alors

    \hspace{1cm} \( P\) prend la valeur \( 250\times \frac{ 100-n }{ 100 }\)

    Si \( n > 25 \) alors

    \hspace{1cm} \( P\) prend la valeur \( 250\times l\)

    Afficher \( P\)

\end{fmpage}

Malheureusement Luc n'a pas pu récupérer la valeur de \( l\). En regardant dans les archives des clients, Luc trouve une cliente de \( 30\) ans qui a payé \( 125\) euros. Est-ce qu'on peut en déduire la valeur de \( l\) ?

Quel type de réduction est appliquée aux clients de moins de \( 25\) ans ?

\corrref{smath-0555}
\end{exercice}
