% This is part of Un soupçon de mathématique sans être agressif pour autant
% Copyright (c) 2013
%   Laurent Claessens
% See the file fdl-1.3.txt for copying conditions.

\begin{exercice}[\cite{FGMWzeK}]\label{exosmath-0267}

Chaque jour, une entreprise fabrique $x$ objets, avec $x\in[0;50]$.  Le coût de production des $x$ objets est donné en euros par : $ C(x) = 60 - 0,3 x$. Le revenu des $x$ objets est donné en euros par $ R(x) = 20,\!1\, x - 0,3 x^2$. 

Le bénéfice quotidien de cette entreprise est noté $B(x)$, il correspond à la différence entre le revenu et le coût de production.
\begin{enumerate}
\item Exprimer $B(x)$ en fonction de $x$.
\item Vérifier que le bénéfice peut également s'écrire $ B(x) = -0,3(x-34)^2+286,8 $.
\item Quel est le bénéfice maximal ? Quel nombre d'objets l'entreprise doit-elle produire pour l'atteindre ?
\end{enumerate}

\corrref{smath-0267}
\end{exercice}
