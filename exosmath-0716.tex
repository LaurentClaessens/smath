% This is part of Un soupçon de mathématique sans être agressif pour autant
% Copyright (c) 2014
%   Laurent Claessens
% See the file fdl-1.3.txt for copying conditions.

%TODO : citer sésamath de seconde

\begin{exercice}[\ldots/6]\label{exosmath-0716}

Dans un lycée de $1470$ élèves, \( 350\) élèves ont été vaccinés contre la grippe au début de l'hiver. \( 10\%\) des élèves ont contracté la grippe pendant l'épidémie annuelle, dont \( 4\%\) des élèves vaccinés.

    Nous choisissons au hasard un des élèves du lycée et nous considérons les événements 
    \begin{itemize}
        \item \( V\) : «il a été vacciné»; 
        \item \( G\) : «il a eu la grippe». 
    \end{itemize}
        

\begin{enumerate}
    \item

        Avec les informations données, Laura a rempli le tableau suivant :

\begin{center}
    \begin{tabular}[]{|c|c|c|c|}
        \hline
        &grippé&pas grippé&total\\
        \hline
        vaccinés&\( 14\)&&\( 350\)\\
        \hline
        non vacciné&&&\\
        \hline
        total&\( 147\)&&\( 1470\)\\
        \hline
    \end{tabular}
\end{center}

        Que représente le \( 14\) dans ce tableau ? Comment Laura l'a-t-elle calculé ?
    \item
        Calculer les probabilités \( P(V)\) et \( P(G)\).
    \item
        Calculer la probabilité des événements \( V\cap G\) et \( V\cup G\).
    \item
        Décrire par une phrase l'événement \( \bar V\).
    \item
        Nous choisissons un élève au hasard parmi ceux qui ont été malades; quelle est la probabilité qu'il ait été vacciné ?
    \item
        Nous choisissons un élève au hasard parmi ceux qui n'ont pas été vaccinés. Quelle est la probabilité qu'il ait eu la grippe ?
    \item
        Le vaccin est-il efficace ? Expliquer.
\end{enumerate}


\corrref{smath-0716}
\end{exercice}
