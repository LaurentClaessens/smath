% This is part of Un soupçon de mathématique sans être agressif pour autant
% Copyright (c) 2012
%   Laurent Claessens
% See the file fdl-1.3.txt for copying conditions.

\begin{exercice}\label{exosmath-0217}

    Un sac de graines de fleurs doit contenir \( 23\%\) de graines de tulipes. L'intervalle de fluctuation du pourcentage de graines de tulipes est environ \( \mathopen[ 15 , 31 \mathclose]\) au seuil \( 95\%\). Nous tirons \( 100\) graines au hasard et nous comptons \( 17\) graines de tulipe. Peut-on admettre que le proportion de graines de tulipes dans le sac est de \( 23\%\) au seuil \( 95\%\) ?

\corrref{smath-0217}
\end{exercice}
