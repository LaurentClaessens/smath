% This is part of Un soupçon de mathématique sans être agressif pour autant
% Copyright (c) 2014
%   Laurent Claessens
% See the file fdl-1.3.txt for copying conditions.

\begin{exercice}\label{exosmath-0603}

    Soient les points \( A=(2;4)\) et \( B=(-4;-5)\).
    \begin{enumerate}
        \item
            Prouver que la droite \( (AB)\) a pour équation \( y=1.5x+1\).
        \item
            Donner l'équation de la droite parallèle à \( (AB)\) passant par l'origine \( O\) du repère. Nous allons appeler \(d\) cette droite.
        \item
            Le point \( E=(6;9)\) appartient-il à la droite \( d\) ?
        \item
            Calculer les longueurs \( AB\) et \( OE\). Que pouvons-nous en déduire ?
    \end{enumerate}

\corrref{smath-0603}
\end{exercice}
