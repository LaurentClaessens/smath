% This is part of Un soupçon de mathématique sans être agressif pour autant
% Copyright (c) 2014
%   Laurent Claessens
% See the file fdl-1.3.txt for copying conditions.

\begin{exercice}\label{exosmath-0932}

    Les affirmations suivantes sont toutes fausses. Donner des contre-exemples sous forme de dessins (avec codage).
    \begin{enumerate}
        \item
            Le point d'intersection des trois hauteurs d'un triangle est toujours à l'intérieur du triangle.
        \item
            Chaque médiane d'un triangle est perpendiculaire à un côté.
        \item
            Chaque médiane d'un triangle isocèle est perpendiculaire à un côté.
        \item
            Les médiatrices d'un triangles passent par des sommets du triangle.
        \item
            Dans un triangle \( ABC\), la hauteur issue de \( A\) intersecte le segment \( [BC]\).
    \end{enumerate}

\corrref{smath-0932}
\end{exercice}
