% This is part of Un soupçon de mathématique sans être agressif pour autant
% Copyright (c) 2012
%   Laurent Claessens
% See the file fdl-1.3.txt for copying conditions.

%+++++++++++++++++++++++++++++++++++++++++++++++++++++++++++++++++++++++++++++++++++++++++++++++++++++++++++++++++++++++++++ 
\section{Moyenne, écart-type}
%+++++++++++++++++++++++++++++++++++++++++++++++++++++++++++++++++++++++++++++++++++++++++++++++++++++++++++++++++++++++++++

Activité de départ : visionner une partie du cours de \href{http://www.manicore.com}{Jean-Marc Jancovici} disponible sur \href{http://podcast.paristech.fr/groups/mines/wiki/8f866/Energie_et_changement_climatique_de_JeanMarc_Jancovici.html}{le site de l'ENSMP}; plus précisément nous allons voir la partie de la vidéo
\begin{quote}
    \url{http://podcast.paristech.fr:8171/podcastproducer/attachments/127C0FCD-3670-4EE3-8EC7-070FFD430FBF/636EFE64-E9A2-42E8-B4A8-1D199846ADCE.mp4}
\end{quote}
située entre 11m20 et 17m20.


%+++++++++++++++++++++++++++++++++++++++++++++++++++++++++++++++++++++++++++++++++++++++++++++++++++++++++++++++++++++++++++ 
\section{Exercices}
%+++++++++++++++++++++++++++++++++++++++++++++++++++++++++++++++++++++++++++++++++++++++++++++++++++++++++++++++++++++++++++

%--------------------------------------------------------------------------------------------------------------------------- 
\subsection{Probabilités}
%---------------------------------------------------------------------------------------------------------------------------

\Exo{smath-0219}

%--------------------------------------------------------------------------------------------------------------------------- 
\subsection{Fréquence, moyenne et écart-type}
%---------------------------------------------------------------------------------------------------------------------------

\Exo{smath-0216}

%--------------------------------------------------------------------------------------------------------------------------- 
\subsection{Intervalle de fluctuation}
%---------------------------------------------------------------------------------------------------------------------------

\Exo{smath-0217} % Attention : avant de mettre celui-ci sur les feuilles, il faut le faire : il me semble qu'il y a un piège.
\Exo{smath-0218}
