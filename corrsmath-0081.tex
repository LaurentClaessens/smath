% This is part of Un soupçon de mathématique sans être agressif pour autant
% Copyright (c) 2012
%   Laurent Claessens
% See the file fdl-1.3.txt for copying conditions.

\begin{corrige}{smath-0081}

    \begin{enumerate}
        \item
            Nous considérons le plan \( (ACS)\). Ce plan (vertical sur le dessin) contient évidemment les points \( A\), \( C\), mais aussi les points \( I\) et \( J\) parce qu'ils sont respectivement sur les droites \( (AS)\) et \( (CS)\).

            Donc les droites \( (IJ)\) et \( (AC)\) sont coplanaires. Comme elles ne sont pas parallèles, elles sont sécantes.
        \item
            Où que l'on mette \( I\) sur \( (AS)\), les droites \( (IJ)\) et \( (AC)\) seront toujours coplanaires. L'unique moyen pour qu'elles ne soient pas sécantes est donc qu'elles soient parallèles.       

            Nous regardons le triangle \( ACS\); pour qu'un segment soit parallèle à \( (AC)\), il faut qu'il intersecte les autres côtés aux mêmes proportions (voir théorème des milieux). Il faut donc mettre \( I\) au tiers de \( [AS]\).
    \end{enumerate}

\end{corrige}
