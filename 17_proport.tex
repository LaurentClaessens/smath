% This is part of Un soupçon de mathématique sans être agressif pour autant
% Copyright (c) 2015
%   Laurent Claessens
% See the file fdl-1.3.txt for copying conditions.

% This is part of Un soupçon de mathématique sans être agressif pour autant
% Copyright (c) 2015
%   Laurent Claessens
% See the file fdl-1.3.txt for copying conditions.

%--------------------------------------------------------------------------------------------------------------------------- 
\subsection*{Activité : qui a dit «proportionnel» ?}
%---------------------------------------------------------------------------------------------------------------------------

Les situations suivantes relèvent-elles d’une situation de proportionnalité ? Pourquoi ?
\begin{enumerate}
    \item

 Saïd achète 2 mètres de corde qui coûte 2,30 € le mètre.

\item
    Daniel a planté huit pieds de tomates dans son potager, et en a récolté \SI{14}{\kilo\gram}. L'an passé, il en avait planté 12 pieds et en avait récolté \SI{18}{\kilo\gram}. L'an prochain, il en plantera 10 pieds et espère en récolter \SI{16}{\kilo\gram}. 

\item
 À 6 ans, Armand chaussait du 30 et à 18 ans, il chausse du 42.
\item

 Abonnement à une revue : \( 6\) mois pour \( 18\)€, un an pour \( 32\)€ et \( 2\) ans pour \( 60\)€.
\item
    Un piéton se promène à allure régulière le long des quais de la Seine et parcourt \SI{3.5}{\kilo\meter} en 1 h 30.
\item

    On peut acheter de l'enduit de lissage par sac de 1 kg, 5 kg et 25 kg. Le mode d’emploi précise qu'il faut \SI{2.5}{\liter} d’eau pour \SI{10}{\kilo\gram}.
\item

 Un commerçant a décidé de faire une journée promotion en baissant tous les prix de $10$\%.
\item

 Un loueur de DVD propose la formule d'abonnement suivante : la carte d'adhésion coûte $10$€ et on paye $2$€ par DVD.

\end{enumerate}

De \cite{NRHooXFvgpp5}

\begin{Aretenir}
    Un tableau de nombres décrit une relation de \defe{proportionnalité}{proportionnalité} si un même coefficient (non nul) multiplicateur s’applique dans tout le tableau. On parle alors de \defe{coefficient de proportionnalité}{coefficient!de proportionnalité}.
\end{Aretenir}

\begin{example}
    Un train de marchandises fait la liaison entre Paris et Berlin; voici le relevé de son avancement en fonction du temps :
    \begin{equation*}
        \begin{array}[]{|c|c|c|c|c|}
            \hline
            \text{temps de parcours (heures)}&3&5&6&8\\
            \hline
            \text{distance parcourue (\si{\kilo\meter})}&270&450&540&720\\
            \hline
        \end{array}
    \end{equation*}
    Est-ce une situation de proportionnalité ?

    Le coefficient multiplicateur pour la première colonne est :
    \begin{equation}
        \frac{ 270 }{ 3 }=90.
    \end{equation}
    Vérifions si ce nombre fonctionne pour toutes les colonnes :
    \begin{enumerate}
        \item
            \( 5\times 90=450\)
        \item
            \( 6\times 90=540\)
        \item
            \( 8\times 90=720\).
    \end{enumerate}
    Le tableau proposé est bien un tableau de proportionnalité parce qu'un même coefficient multiplicateur s'applique pour tout le tableau.
\end{example}

% This is part of Un soupçon de mathématique sans être agressif pour autant
% Copyright (c) 2015
%   Laurent Claessens
% See the file fdl-1.3.txt for copying conditions.
%--------------------------------------------------------------------------------------------------------------------------- 
\subsection*{Activité : prix en solde}
%---------------------------------------------------------------------------------------------------------------------------

Un magasin fait des soldes «\( -20\%\)» et donne le tableau suivant pour aider le clients à savoir les prix soldés en fonction du prix non soldé :
\begin{equation*}
    \begin{array}[]{|c||c|c|c|c|}
        \hline
        \text{prix non soldé}&50&70&130&200\\
        \hline\hline
        \text{prix soldé}&40&56&104&160\\
        \hline
    \end{array}
\end{equation*}
\begin{enumerate}
    \item
        Le prix soldé est-il proportionnel au prix non soldé ?
    \item
        Combien coûterait un article dont le prix non soldé est de \( 120\)€, et \( 150\)€ ?
\end{enumerate}
 % il y a deux activités dans ce fichier

%+++++++++++++++++++++++++++++++++++++++++++++++++++++++++++++++++++++++++++++++++++++++++++++++++++++++++++++++++++++++++++ 
\section{Pourcentage}
%+++++++++++++++++++++++++++++++++++++++++++++++++++++++++++++++++++++++++++++++++++++++++++++++++++++++++++++++++++++++++++

% This is part of Un soupçon de mathématique sans être agressif pour autant
% Copyright (c) 2015
%   Laurent Claessens
% See the file fdl-1.3.txt for copying conditions.

%--------------------------------------------------------------------------------------------------------------------------- 
\subsection*{Activité : des mélanges}
%---------------------------------------------------------------------------------------------------------------------------

Un restaurateur prépare un mélange de jus de fruits : deux litres de jus d'orange pour trois litres de jus de pomme. 
\begin{enumerate}
    \item
        Quelle quantité de jus de pomme faut-il pour obtenir cent litres de mélange ?
    \item
        Compléter ce tableau :
        \begin{equation*}
            \begin{array}[]{|c|c|c|}
                \hline
                \text{Volume de jus d'orange}&2&\ldots\ldots\\
                \hline
                \text{Volume de mélange}&\ldots\ldots&100\\
                \hline
            \end{array}
        \end{equation*}
    \item
        Exprimer la proportion de jus de pomme dans le mélange sous forme d'une fraction ayant \( 100\) comme dénominateur.
\end{enumerate}





