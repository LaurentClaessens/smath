% This is part of Un soupçon de mathématique sans être agressif pour autant
% Copyright (c) 2012
%   Laurent Claessens
% See the file fdl-1.3.txt for copying conditions.

\begin{exercice}\label{exoPremiere-0092}

    Un professeur conçoit un QCM de à trois propositions chacune. Nous voulons savoir combien de questions il faut mettre pour qu'un élève répondant au hasard n'ait que \( 5\%\) de chances de réussir.

    \begin{enumerate}
        \item
            Si le QCM contient \( 5\) questions, montrer que le nombre de bonnes réponses obtenues en tapant au hasard est une loi binomiale de paramètres \( n=5\) et \( p=\frac{1}{ 4 }\).
        \item   \label{ItemtewUiK}
            Si le QCM contient \( 5\) questions, quel est le plus petit \( k\) pour lequel \( P(X<k)\geq 0.95\) ?
        \item
            Si le QCM contient \( N\) questions, alors nombre de réponses correctes suit alors une loi binomiale de paramètres \( N\) et \( 0.25\). Donner le \( N\) minimum pour avoir 
            \begin{equation}
                P(X<\frac{ N }{ 2 })\geq 0.95.
            \end{equation}
            Pour cela, refaire la question \ref{ItemtewUiK} en remplaçant \( 5\) par \( 1\), \( 2\), \( 3\), etc. jusqu'à obtenir le résultat.
    \end{enumerate}

\corrref{Premiere-0092}
\end{exercice}
