% This is part of Un soupçon de mathématique sans être agressif pour autant
% Copyright (c) 2014
%   Laurent Claessens
% See the file fdl-1.3.txt for copying conditions.

\begin{corrige}{smath-0672}

    En ce qui concerne la proportion de gauchers, elle est de \( p=0.12\). La méthode des intervalles de fluctuations vue au cours n'étant valide que si \( p\) est entre \( 0.2\) et \( 0.8\), nous ne pouvons rien dire.

    En ce qui concerne la proportion de femmes parmi les cadres, elle «devrait» être de \( p=0.5\) si il n'y avait aucun problèmes. La fréquence observée est \( \frac{ 18 }{ 60 }=0.3 \) et l'intervalle de fluctuation pour un échantillon de taille \( 60\) est
    \begin{equation}
        \mathopen[ 0.5-\frac{1}{ \sqrt{60} } ; 0.5+\frac{1}{ \sqrt{60} } \mathclose]\simeq\mathopen[ 0.37 ; 0.63 \mathclose].
    \end{equation}
    La proportion de femmes observée (\( 0.3\)) étant en dehors de cet intervalle nous concluons qu'il y a quelque chose d'anormal qui se passe.

\end{corrige}
