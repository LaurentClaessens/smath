% This is part of Un soupçon de mathématique sans être agressif pour autant
% Copyright (c) 2014
%   Laurent Claessens
% See the file fdl-1.3.txt for copying conditions.

\begin{exercice}[\ldots/4]\label{exosmath-0693}

    Questions type «calcul mental». Les réponses doivent être justifiées.
    \begin{multicols}{2}
        \begin{enumerate}
            \item
                Mettre au même dénominateur et effectuer la somme :
                \begin{equation}
                    2-\frac{ 2 }{ a }
                \end{equation}
            \item
                Donner les équations de deux droites différentes passant par le point \( A(4;10)\).
            \item
                Le point \( B(-4;15)\) est-il sur le graphe de la fonction \( f(x)=x^2-1\) ?
            \item
                Est-ce que \( x=-3\) est solution de l'inéquation \( x^2+1\geq -1\) ?
            \item
                Résoudre $4x+3=10-x$.
        \end{enumerate}
    \end{multicols}

\corrref{smath-0693}
\end{exercice}
