% This is part of Un soupçon de mathématique sans être agressif pour autant
% Copyright (c) 2012
%   Laurent Claessens
% See the file fdl-1.3.txt for copying conditions.

\begin{exercice}\label{exosmath-0078}
 
    Soit un carré \( ABCD\) de longueur \unit{8}{\centi\meter}, et un point mobile \( M\) sur le segment \( [AB]\). Nous notons \( x\) la longueur \( [AM]\). Nous dessinons à l'intérieur de \( ABCD\) les deux choses suivantes :
    \begin{itemize}
        \item un carré de côté \( [AM]\);
        \item
            un triangle isocèle de base \( [MB]\) et dont la hauteur a même mesure que le côté \( [AM]\).
    \end{itemize}
    Nous voudrions savoir si il est possible de choisir \( x\) de telle sorte que la surface du petit carré soit égale à celle du triangle.
    \begin{enumerate}
        \item
            Déterminer la fonction \( A(x)\) qui donne l'aire totale du petit carré et du triangle en fonction de \( x\).
        \item
            Quel est le domaine de la fonction \( A\) ?
        \item
            Quelle est l'équation à résoudre ?
        \item
            Quelles sont les solutions ?
    \end{enumerate}

\corrref{smath-0078}
\end{exercice}
