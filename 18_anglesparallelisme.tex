% This is part of Un soupçon de mathématique sans être agressif pour autant
% Copyright (c) 2014-2015
%   Laurent Claessens
% See the file fdl-1.3.txt for copying conditions.


%+++++++++++++++++++++++++++++++++++++++++++++++++++++++++++++++++++++++++++++++++++++++++++++++++++++++++++++++++++++++++++ 
\section{Sécantes et parallèles}
%+++++++++++++++++++++++++++++++++++++++++++++++++++++++++++++++++++++++++++++++++++++++++++++++++++++++++++++++++++++++++++

% This is part of Un soupçon de mathématique sans être agressif pour autant
% Copyright (c) 2015
%   Laurent Claessens
% See the file fdl-1.3.txt for copying conditions.

%--------------------------------------------------------------------------------------------------------------------------- 
\subsection*{Activité : graphisme d'un N}
%---------------------------------------------------------------------------------------------------------------------------

Un ébéniste voudrait dessiner un «X» et un «N» de la forme suivante :

\begin{center}
\input{Fig_MLZLooYDRsFl.pstricks}\quad
   \input{Fig_EZTHooBttIaQ.pstricks}
\end{center}

Si il voudrait que l'angle indiqué sur le «X»  soit de \SI{30}{\degree} et celui sur le «N»  soit de \SI{45}{\degree}, quelles sont les mesures des autres angles ?


\begin{definition}
    Des angles \defe{opposés par le sommet}{angle!opposé par le sommet} sont deux angles qui ont un sommet commun et qui ont leurs côtés dans le prolongement l'un de l'autre.
\end{definition}

\begin{propriete}
    Deux angles opposés par le sommet ont même mesure.
\end{propriete}

\begin{proof}
    jkljljlkkl
\end{proof}
<++>


%+++++++++++++++++++++++++++++++++++++++++++++++++++++++++++++++++++++++++++++++++++++++++++++++++++++++++++++++++++++++++++ 
\section{Angles internes d'un triangle}
%+++++++++++++++++++++++++++++++++++++++++++++++++++++++++++++++++++++++++++++++++++++++++++++++++++++++++++++++++++++++++++

% This is part of Un soupçon de mathématique sans être agressif pour autant
% Copyright (c) 2014
%   Laurent Claessens
% See the file fdl-1.3.txt for copying conditions.


%+++++++++++++++++++++++++++++++++++++++++++++++++++++++++++++++++++++++++++++++++++++++++++++++++++++++++++++++++++++++++++ 
\section*{Angles dans les triangles}
%+++++++++++++++++++++++++++++++++++++++++++++++++++++++++++++++++++++++++++++++++++++++++++++++++++++++++++++++++++++++++++

\begin{enumerate}
    \item
    Tracer un triangle $EFG$ tel que $\widehat{EFG}=\unit{48}{\degree}$, \( \widehat{FGE}=\unit{70}{\degree}\) et \( \widehat{GEF}=\unit{62}{\degree}\), et en mesurer le périmètre. Comparer avec les résultats des autres élèves.
\item
    Dessiner un triangle au hasard, en mesurer les angles et faire la somme. Comparer la réponse avec celle du voisin.
\end{enumerate}


\begin{theorem}[Angles internes d'un triangle]
    La somme des angles internes d'un triangle vaut \SI{180}{\degree}.
\end{theorem}
<++>

