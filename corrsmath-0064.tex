% This is part of Un soupçon de mathématique sans être agressif pour autant
% Copyright (c) 2014
%   Laurent Claessens
% See the file fdl-1.3.txt for copying conditions.

\begin{corrige}{smath-0064}

    \begin{enumerate}
        \item
            Le point \( D\) est obtenu en translatant le point \( A\) avec le vecteur \( \vect{ BC }\). Ce dernier a pour coordonnées \( \vect{ BC }=\begin{pmatrix}
                -4    \\ 
                2    
            \end{pmatrix}\) et donc \( D=(0+(-4);0+2)=(-4;2)\).
        \item
            Il suffit de calculer les coordonnées des deux vecteurs :
            \begin{equation}
                \vect{ AE }=\begin{pmatrix}
                    -3-0    \\ 
                    -2-0    
                \end{pmatrix}=\begin{pmatrix}
                    -3    \\ 
                    -2    
                \end{pmatrix},
            \end{equation}
            tandis que
            \begin{equation}
                \vect{ FC }=\begin{pmatrix}
                    -2-1    \\ 
                    3-5    
                \end{pmatrix}=\begin{pmatrix}
                    -3    \\ 
                    -2    
                \end{pmatrix}.
            \end{equation}
            Les coordonnées de ces deux vecteurs étant égales, les vecteurs sont égaux.
        \item
            Lorsque \( \vect{ AE }=\vect{ FC }\), le quadrilatère \( AECF\) est un parallélogramme.
        \item
            Il suffit de calculer les milieux :
            \begin{equation}
                \text{milieu de \( [FE]\)}=\left( \frac{ 1-3 }{2};\frac{ 5-2 }{2} \right)=(-1;\frac{ 3 }{2}),
            \end{equation}
            et
            \begin{equation}
                \text{milieu de \( [BD]\)}=\left( \frac{ 2-4}{2};\frac{ 1+2 }{2} \right)=(-1;\frac{ 3 }{2}).
            \end{equation}
            
    \end{enumerate}

\end{corrige}
