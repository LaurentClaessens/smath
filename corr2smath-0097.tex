% This is part of Un soupçon de mathématique sans être agressif pour autant
% Copyright (c) 2015
%   Laurent Claessens
% See the file fdl-1.3.txt for copying conditions.

\begin{corrige}{2smath-0097}

    \begin{enumerate}
        \item
            Mise au même dénominateur :
            \begin{equation}
                \frac{ 3 }{ 5 }=\frac{ 6 }{ 10 },
            \end{equation}
            donc
            \begin{subequations}
                \begin{align}
                    &\frac{ 7 }{ 10 }+\frac{ -3 }{ 5 }\\
                    =&\frac{ 7 }{ 10 }+\frac{ -6 }{ 10 }\\
                    =&\frac{ 7-6 }{ 10 }\\
                    =&\frac{1}{ 10 }.
                \end{align}
            \end{subequations}
        \item
            La mise au même dénominateur est la même : seul le signe de l'opération change. Attention : soustraire un nombre négatif revient à additionner : \( 7-(-3)=7+3=10\) :
            \begin{subequations}
                \begin{align}
                    &\frac{ 7 }{ 10 }-\frac{ -3 }{ 5 }\\
                    =&\frac{ 7 }{ 10 }+\frac{ 6 }{ 10 }\\
                    =&\frac{ 13 }{ 10 }.
                \end{align}
            \end{subequations}
        \item
            La multiplication de fraction est l'opération la plus simple :
            \begin{equation}
                \frac{ 7 }{ 10 }\times \frac{ 3 }{ 5 }=\frac{ 7\times 3 }{ 10\times 5 }=\frac{ 21 }{ 50 }.
            \end{equation}
        \item
            La division revient à multiplier par l'inverse :
            \begin{subequations}
                \begin{align}
                    &\frac{ 7 }{ 10 }\div \frac{ 3 }{ 5 }\\
                    =&\frac{ 7 }{ 10 }\times \frac{ 5 }{ 3 }\\
                    =&\frac{ 7\times 5 }{ 10\times 3 }\\
                    =&\frac{ 35 }{ 3 }\\
                    =&\frac{ 7 }{ 10 }.
                \end{align}
            \end{subequations}
            La dernière ligne est une simplification par \( 5\).
    \end{enumerate}

\end{corrige}
