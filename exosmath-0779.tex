% This is part of Un soupçon de mathématique sans être agressif pour autant
% Copyright (c) 2014
%   Laurent Claessens
% See the file fdl-1.3.txt for copying conditions.

\begin{exercice}\label{exosmath-0779}

    Un charpentier doit couper des poutres de bonne longueur pour créer un triangle isocèle. La poutre transversale horizontale fait \unit{7}{\meter} et l'inclinaison du toit est de \unit{30}{\degree}. Dessiner un schéma à l'échelle (par exemple \unit{1}{\centi\meter} sur la feuille représente \unit{1}{\meter} dans la réalité), et mesurer la longueur des poutres inclinées.

\corrref{smath-0779}
\end{exercice}
