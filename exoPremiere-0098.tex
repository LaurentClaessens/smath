% This is part of Un soupçon de mathématique sans être agressif pour autant
% Copyright (c) 2012
%   Laurent Claessens
% See the file fdl-1.3.txt for copying conditions.

\begin{exercice}\label{exoPremiere-0098}

    Soit la parabole \( f(x)=x^2-6x-7\). Pour cet exercice nous admettons la factorisation \( x^2-6x-7=(x+1)(x-7)\). 
        \begin{enumerate}
            \item
                Dresser le tableau de signe de \( f\) 
            \item
                résoudre l'inéquation \( f(x)\geq 0\).
        \end{enumerate}

\corrref{Premiere-0098}
\end{exercice}
