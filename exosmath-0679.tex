% This is part of Un soupçon de mathématique sans être agressif pour autant
% Copyright (c) 2014
%   Laurent Claessens
% See the file fdl-1.3.txt for copying conditions.

\begin{exercice}\label{exosmath-0679}

\begin{wrapfigure}{r}{7.cm}
   \vspace{-0.5cm}        % à adapter.
   \centering
   \input{Fig_YMslVxg.pstricks}
\end{wrapfigure}

\let\Oldtheenumi\theenumi
\renewcommand{\theenumi}{(\alph{enumi})}
Dire quelle fonction correspond à quelle droite sur le dessin ci-contre (justifier) :
\begin{enumerate}
    \item
        \( f_a(x)=-x+2\)
    \item
        \( f_b(x)=2\)
    \item
        \( f_c(x)=2x-2\)
    \item
        \( f_d(x)=-x-2\)
\end{enumerate}

\let\theenumi\Oldtheenumi

\corrref{smath-0679}
\end{exercice}
