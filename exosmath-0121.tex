% This is part of Un soupçon de mathématique sans être agressif pour autant
% Copyright (c) 2012
%   Laurent Claessens
% See the file fdl-1.3.txt for copying conditions.

\begin{exercice}\label{exosmath-0121}

    % TODO : cet exercice est trop similaire au précédent.
    Un joueur de jeu de rôles doit lancer trois fois un dé à six faces. À chaque test, il réussit si il fait \( 1\) ou \( 2\), et échoue sinon. Nous désignons par \( X\) la variable aléatoire correspondante au nombre de réussites.
    \begin{enumerate}
        \item
            Identifier les paramètres du schéma de Bernoulli (probabilité de réussite, nombre d'essais) sous-jacent à cette situation.
        \item
            Dessiner un arbre.
        \item
            Compléter le tableau
            \begin{equation*}
                \begin{array}[]{|c||c|c|c|c|c|c|}
                    \hline
                    k&0&1&2&3&4&5\\
                    \hline\hline
                    P(X=k)&&&&&&\\
                    \hline
                \end{array}
            \end{equation*}
    \end{enumerate}


\corrref{smath-0121}
\end{exercice}
