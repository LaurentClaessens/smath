% This is part of Un soupçon de mathématique sans être agressif pour autant
% Copyright (c) 2013
%   Laurent Claessens
% See the file fdl-1.3.txt for copying conditions.

\begin{corrige}{smath-0276}

\begin{wrapfigure}{r}{7.0cm}
            \vspace{-2cm}        % à adapter.
                \centering
                    \input{Fig_SWrwUzA.pstricks}
                \end{wrapfigure}

    Nous résolvons séparément les inéquations \( -\frac{1}{ 10 }\leq \frac{1}{ x }\) et \( \frac{1}{ x }\leq\frac{ 3 }{ 4 }\). Sur le dessin ci-contre, nous avons dessiné les solution de l'inéquation \( \frac{1}{ x }\geq -\frac{1}{ 10 }\). Attention : il faut exclure le zéro.

Les solutions de l'inéquation \( -\frac{1}{ 10 }\leq \frac{1}{ x }\) sont :
\begin{equation}
    x\in\mathopen] -\infty , -10 \mathclose]\cup\mathopen] 0 , \infty \mathclose[.
\end{equation}
De la même façon, les solutions de l'autre inéquation sont :
\begin{equation}
    x\in\mathopen] -\infty , 0 \mathclose[\cup\mathopen[ \frac{ 4 }{ 3 } , \infty [.
\end{equation}

Nous mettons les solutions des deux inéquations sur un seul dessin pour voir quels sont les solutions communes.

\end{corrige}
