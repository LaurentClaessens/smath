% This is part of Un soupçon de mathématique sans être agressif pour autant
% Copyright (c) 2014-2015
%   Laurent Claessens
% See the file fdl-1.3.txt for copying conditions.

% This is part of Un soupçon de mathématique sans être agressif pour autant
% Copyright (c) 2014-2015
%   Laurent Claessens
% See the file fdl-1.3.txt for copying conditions.

On croise deux bandes de papier et on s'intéresse à l'intersection.  Quels quadrilatères peut-on obtenir ?
\begin{center}
\includegraphics[width=5cm]{deux_bandes.pdf}
\end{center}

De \cite{NRHooXFvgpp5}

%+++++++++++++++++++++++++++++++++++++++++++++++++++++++++++++++++++++++++++++++++++++++++++++++++++++++++++++++++++++++++++ 
\section{Définitions, propriétés}
%+++++++++++++++++++++++++++++++++++++++++++++++++++++++++++++++++++++++++++++++++++++++++++++++++++++++++++++++++++++++++++

\begin{definition}[\cite{NRHooXFvgpp5}]
    Quelque quadrilatères :
    \begin{enumerate}
        \item
            Un \defe{parallélogramme}{parallélogramme} est un quadrilatère qui a ses côtés opposés parallèles deux à deux.
        \item
            Un \defe{losange}{losange} est un quadrilatère qui possède quatre côtés de même longueur.
        \item
            Un \defe{rectangle}{rectangle} est un quadrilatère qui a quatre angles droits.
        \item
            Un \defe{carré}{carré} est un rectangle qui a ses côtés de même longueur.
    \end{enumerate}
\end{definition}

\begin{propriete}
    Les diagonales d'un carré sont perpendiculaires.
\end{propriete}

\begin{proof}
    Dessinons un carré :
    \begin{center}
        \input{Fig_UOEOooLxhpSC.pstricks}
    \end{center}
    Par définition un carré a ses côtés de même longueurs, donc le triangle \( ADC\) est isocèle en \( D\).
    \begin{center}
        Donc \( a_1=c_2\)
    \end{center}
    Mais comme la somme des angles internes du triangle doit faire \SI{180}{\degree} et que l'angle \( \hat D\) en fait \( 90\), 
    \begin{center}
        \( a_1=a_2=\SI{45}{\degree}\).
    \end{center}
    Le même raisonnement dans les triangles \( ABC\), \( ADB\) et \( CDB\) donne :
    \begin{equation}
        a_1=a_2=b_1=b_2=c_1=c_2=d_1=d_2=\SI{45}{\degree}.
    \end{equation}
    
    Dans le triangle \( DCI\), la somme des angles doit faire \SI{180}{\degree}. Mais les deux angles de base font \SI{45}{\degree}. Donc le troisième angle doit mesurer \SI{90}{\degree}.

\end{proof}

