% This is part of Un soupçon de mathématique sans être agressif pour autant
% Copyright (c) 2012
%   Laurent Claessens
% See the file fdl-1.3.txt for copying conditions.

\begin{corrige}{Premiere-0084}

    Nous devons écrire le tableau de signe, et pour cela nous devons savoir les deux racines. D'abord nous identifions
    \begin{subequations}
        \begin{numcases}{}
            a=-1\\
            b=-4\\
            c=5
        \end{numcases}
    \end{subequations}
    et nous calculons le discriminant
    \begin{equation}
        \Delta=(-4)^2-4\times (-1)\times 5=16+20=36,
    \end{equation}
    et \( \sqrt{\Delta}=\sqrt{36}=6\). Nous avons alors
    \begin{subequations}
        \begin{align}
            x_1&=\frac{ 4+6 }{ -2 }=-5\\
            x_2&=\frac{ 4-6 }{ -2 }=1.
        \end{align}
    \end{subequations}
    La factorisation est donc
    \begin{equation}
        -x^2-4x+5=-(x+5)(x-1).
    \end{equation}
    À ce niveau, vous êtes vivement encouragés à développer le membre de droite pour être sûr qu'il est bien égal au membre de gauche. Le tableau de signe est alors
    \begin{equation*}
        \begin{array}[]{c|ccccc}
            \hline
            x&&-5&&1&\\
            \hline\hline
            -1&-&-&-&-&-\\
            x+5&-&0&+&+&+\\
            x-1&-&-&-&0&+\\
            \hline
            -(x+5)(x-1)&-&0&+&0&-\\
        \end{array}
    \end{equation*}
    Les valeurs de \( x\) pour lesquelles \( f(x)\geq 0\) sont les valeurs sur lesquelles nous lisons un \( +\) ou un \( 0\) dans le tableau. Cela donne :
    \begin{equation}
        x\in\mathopen[ -5 , 1 \mathclose].
    \end{equation}

\end{corrige}
