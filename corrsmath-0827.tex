% This is part of Un soupçon de mathématique sans être agressif pour autant
% Copyright (c) 2014
%   Laurent Claessens
% See the file fdl-1.3.txt for copying conditions.

\begin{corrige}{smath-0827}

    \begin{enumerate}
        \item
            En partant de \( 3\) on obtient \( 9\). L'enchaînement est : \( 3+7=10\), ensuite \( 10\times 3=30\) et enfin \( 30-21=9\).
            
            En partant de \( 10\) on obtient \( 30\) et en parant de \( 12\) on obtient \( 36\).
        \item
            Il semble que le résultat soit toujours le triple du nombre de départ. Pour le démontrer, rien ne vaut un petit peu de \( x\) : si on part du nombre \( x\), l'enchaînement est donné par la formule
            \begin{equation}
                \begin{aligned}[]
                    x\\
                    x+7\\
                    (x+7)\times 3=3\times x+21\\
                    3\times x+21-21=3x.
                \end{aligned}
            \end{equation}
        \item
            Le résultat en partant de \( 7001\) est simple à trouver : c'est le triple de \( 7001\) : \( 7001\times 3=21003\).
        \item
            Le nombre dont le triple est \( 21\) est \( 7\). Il faut donc partir de \( 7\) pour obtenir \( 21\).
    \end{enumerate}
\end{corrige}
