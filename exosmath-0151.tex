% This is part of Un soupçon de mathématique sans être agressif pour autant
% Copyright (c) 2012
%   Laurent Claessens
% See the file fdl-1.3.txt for copying conditions.

\begin{exercice}\label{exosmath-0151}

    \begin{enumerate}
        \item
Donner une fonction affine \( f\) telle que \( f(0)=3\) et \( f(1)=-2\).
\item
    Donner une fonction linéaire telle que \( f(3)=5\).
\item
    Donner une fonction affine telle que \( f(-1)=2\) et \( f(1)=3\).
\item
    Donner une fonction affine telle que \( f(3)=7\) et \( f(9)=7\).
\item
    Donner une fonction affine telle que \( f(10)=10\) et dont le graphe passe par le point \( (1;2)\).
    \end{enumerate}

\corrref{smath-0151}
\end{exercice}
