% This is part of Un soupçon de mathématique sans être agressif pour autant
% Copyright (c) 2012
%   Laurent Claessens
% See the file fdl-1.3.txt for copying conditions.

\begin{exercice}\label{exoPremiere-0012}

    Dans les lignes du tableau suivant, \( A\) est une sous-population de \( E\); nous désignons par \( n_A\) et \( n_E\) leurs effectifs respectifs et par \( p\) la proportion de \( A\) dans \( E\). Compléter le tableau :
    \begin{tabular}{|c|c|c|}
        \hline
        \( n_A\)&\( n_E\)&p\\
        \hline
        200&&0.5\\
        120&170\\
        &1200&60\%\\
        2008&&0.99
    \end{tabular}

\corrref{Premiere-0012}
\end{exercice}
