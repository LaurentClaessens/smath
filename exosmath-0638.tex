% This is part of Un soupçon de mathématique sans être agressif pour autant
% Copyright (c) 2014
%   Laurent Claessens
% See the file fdl-1.3.txt for copying conditions.

\begin{exercice}\label{exosmath-0638}

    Le quadrilatère \( ABCD\) est un rectangle, \( AB=6\) et \( BC=3\). Le point \( M\) est mobile sur \( BC\) et nous notons \( x\) la longueur \( BM\). Répondre aux questions suivantes en fonction de \( x\).
    \begin{enumerate}
        \item
            Donner l'aire du triangle \( ABM\).
        \item
            Donner l'aire de la zone grisée.
        \item
            Donner la longueur du segment \( [MC]\).
    \end{enumerate}
    
    \begin{center}
   \input{Fig_NIGYQHN.pstricks}
    \end{center}

\corrref{smath-0638}
\end{exercice}
