% This is part of Un soupçon de mathématique sans être agressif pour autant
% Copyright (c) 2012
%   Laurent Claessens
% See the file fdl-1.3.txt for copying conditions.

\begin{exercice}\label{exoSeconde-0085}

    Soit un carré \( ABCD\) de \unit{3}{\centi\meter} de côté.
    \begin{enumerate}
        \item   \label{ItemKQDCPw}
            Calculer la longueur de la diagonale \( [BD]\).
        \item
            Le triangle \( ABC\) est-il isocèle ? rectangle ?
        \item
            Relire la définition d'un repère orthonormé et conclure que l'on peut utiliser \( A\), \( B\) et \( D\) comme repère.
        \item
            Exprimer les coordonnées de \( A\), \( B\), \( C\) et \( D\) dans ce repère.
        \item   \label{ItemPZWCuj}
            Calculer la longueur du segment \( [BD]\) en utilisant les coordonnées calculées pour la question précédente.
        \item
            Le résultat des questions \ref{ItemKQDCPw} et \ref{ItemPZWCuj} sont ils contradictoires ?
    \end{enumerate}

\corrref{Seconde-0085}
\end{exercice}
