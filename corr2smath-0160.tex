% This is part of Un soupçon de mathématique sans être agressif pour autant
% Copyright (c) 2015
%   Laurent Claessens
% See the file fdl-1.3.txt for copying conditions.

\begin{corrige}{2smath-0160}

    Quelque éléments de réponse.
    \begin{enumerate}
        \item
            L'angle «à côté» de \( 100\) mesure \SI{80}{\degree}. Propriété utilisée : l'angle plat mesure \SI{180}{\degree}.
        \item
            L'angle «à côté» de \( 30\) mesure \SI{150}{\degree}.
        \item
            De là, tous les angles sur la partie gauche mesurent \SI{100}{\degree} ou \SI{80}{\degree} et tous ceux de la partie droite mesurent soit \SI{30}{\degree} soit \SI{150}{\degree}. Pour déterminer tout cela, deux propriétés sont mise en œuvre :
            \begin{itemize}
                \item Des angles correspondants (et alternes-internes) ont même mesure.
                \item Des angles opposés par le sommet ont même mesure.
            \end{itemize}
        \item
            En ce qui concerne l'angle de somment \( K\), nous considérons le triangle dans lequel il se trouve. Deux angles sont déjà connus : \SI{80}{\degree} et \SI{30}{\degree}. L'angle en \( K\) mesure alors \SI{70}{\degree}.

            Propriété utilisée : la somme des angles internes à un triangle vaut \SI{180}{\degree}.
    \end{enumerate}
    
\end{corrige}
