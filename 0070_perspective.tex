% This is part of Un soupçon de mathématique sans être agressif pour autant
% Copyright (c) 2012-2013
%   Laurent Claessens
% See the file fdl-1.3.txt for copying conditions.

\EpsOrPdfincludegraphics[width=\linewidth]{BO_perspective}

Note : savoir si deux droites dans l'espace sont perpendiculaires n'est pas au programme. Dans ce chapitre nous allons donc nous en tenir qu'à des choses simples.

\setcounter{section}{-1}

%+++++++++++++++++++++++++++++++++++++++++++++++++++++++++++++++++++++++++++++++++++++++++++++++++++++++++++++++++++++++++++ 
\section{Activité : se poser des questions dans un cube}
%+++++++++++++++++++++++++++++++++++++++++++++++++++++++++++++++++++++++++++++++++++++++++++++++++++++++++++++++++++++++++++

% This is part of Un soupçon de mathématique sans être agressif pour autant
% Copyright (c) 2013
%   Laurent Claessens
% See the file fdl-1.3.txt for copying conditions.

\begin{wrapfigure}[3]{r}{5.0cm}
   \vspace{-0.5cm}        % à adapter.
   \centering
   \input{Fig_MEzTDZC.pstricks}
\end{wrapfigure}

À l'aide du cube ci-contre :
\begin{enumerate}
    \item
        Quelle est la nature du triangle \( AEF\) ? 
    \item
        Quelle est la nature du quadrilatère \( ABGH\) ?
    \item
        Est-ce que vous pouvez trouver trois points de ce cube formant un triangle équilatéral ?
    \item
        Si ce cube fait \unit{5}{\meter} de côté, quelle est la longueur de la «grande» diagonale \( [AG]\) ?
\end{enumerate}



%+++++++++++++++++++++++++++++++++++++++++++++++++++++++++++++++++++++++++++++++++++++++++++++++++++++++++++++++++++++++++++ 
\section{Ce qui est dans un plan}
%+++++++++++++++++++++++++++++++++++++++++++++++++++++++++++++++++++++++++++++++++++++++++++++++++++++++++++++++++++++++++++

Les trois suivantes permettent d'affirmer que tel point ou telle droite est dans un plan.
\begin{enumerate}
    \item
        Si \( P\) et \( Q\) sont deux points d'un plan, alors toute la droite \( (PQ)\) est encore dans ce plan.
    \item
        Si \( d\) est une droite et \( P\) est un point d'un plan, alors la parallèle à \( d\) passant par \( P\) est encore dans le plan.
    \item
        Deux droites non parallèles situées dans un même plan sont sécantes.
\end{enumerate}

%---------------------------------------------------------------------------------------------------------------------------
\subsection{Position relative de deux plans}
%---------------------------------------------------------------------------------------------------------------------------

\begin{definition}
    Deux plans sont \defe{parallèles}{parallèle!deux plans} soit si ils sont confondus, soit si ils n'ont aucun point commun. Si ils n'ont aucun point communs, nous disons qu'ils sont \defe{strictement parallèles}{parallèle!strictement}
\end{definition}


\begin{Aretenir}
    Deux plans non parallèles se coupent en une droite.
\end{Aretenir}

\begin{multicols}{2}

    \begin{center}
        Plans parallèles
    \end{center}
   % Des plans parallèles sont soit confondus, soit n'ont aucun point communs.
    

%Les plans \( (AEB)\) et \( (DHC)\) ci-contre sont parallèles.

\columnbreak

%The result is on figure \ref{LabelFigPositionPlansTvKvah}. % From file PositionPlansTvKvah
%\newcommand{\CaptionFigPositionPlansTvKvah}{<+Type your caption here+>}
    \begin{center}
\input{Fig_PositionPlansTvKvah.pstricks}
    \end{center}


\end{multicols}

\begin{multicols}{2}
    \begin{center}
        Plans sécants
    \end{center}
 %   L'intersection de deux plans sécants (non confondus) est une droite.

  %  L'intersection des plans \( (ABG)\) et \( (DCG)\) ci-contre est la droite \( (HG)\).

    \columnbreak

%The result is on figure \ref{LabelFigPositionPlansqSltxa}. % From file PositionPlansqSltxa
%\newcommand{\CaptionFigPositionPlansqSltxa}{<+Type your caption here+>}
    \begin{center}
\input{Fig_PositionPlansqSltxa.pstricks}
    \end{center}

\end{multicols}

%---------------------------------------------------------------------------------------------------------------------------
\subsection{Position relative de deux droites}
%---------------------------------------------------------------------------------------------------------------------------

Deux droites peuvent être soit dans un même plan, soit ne pas être dans le même plan. Deux droites contenues dans un même plan sont dires \defe{coplanaires}{coplanaire}.

%///////////////////////////////////////////////////////////////////////////////////////////////////////////////////////////
\subsubsection{Droites coplanaires}
%///////////////////////////////////////////////////////////////////////////////////////////////////////////////////////////

Deux droites coplanaires respectent la géométrie usuelle. Elles peuvent être parallèles ou sécantes.

\begin{multicols}{2}
    \begin{center}
        Droites parallèles dans le plan \( (EBC)\)
    \end{center}

    \columnbreak

    \begin{center}
        \input{Fig_IDqyzXM.pstricks}
    \end{center}
\end{multicols}

\begin{multicols}{2}
    \begin{center}
        Droites sécantes dans le plan \( (AEH)\)
    \end{center}

    \columnbreak

    \begin{center}
        \input{Fig_ETfnbsh.pstricks}
    \end{center}
\end{multicols}

\begin{Aretenir}
    À propos de droites coplanaires :
    \begin{enumerate}
        \item
            Deux droites sécantes sont toujours coplanaires.
        \item
            Deux droites parallèles sont coplanaires.
    \end{enumerate}
\end{Aretenir}

\begin{proof}

    \begin{enumerate}
        \item

    Soient \( d\) et \( d'\) deux droites sécantes dans l'espace. Soit \( K\) leur point d'intersection; soit \( A\), un autre point sur \( d\) et \( B\) un autre point sur \( d'\).

    Le plan défini par les points \( A\), \( K\) et \( B\) contient les deux droites.

    \item

        Si \( d\) et \( d'\) sont parallèles nous considérons les points \( P\) et \( Q\) sur \( d\) et un point \( A\) sur \( d'\). D'une part droite \( d\) est dans le plan \( (PQA)\) parce que deux de ses points y sont. D'autre part la droite parallèle à \( d\) passant par \( A\) est la droite \( d'\) et est dans le plan comprenant la droite \( d\) et le point \( A\), c'est à dire le plan \( (PQA)\) également.

    \end{enumerate}
\end{proof}

%///////////////////////////////////////////////////////////////////////////////////////////////////////////////////////////
    \subsubsection{Droites non coplanaires}
%///////////////////////////////////////////////////////////////////////////////////////////////////////////////////////////
    
Deux droites non coplanaires ne peuvent pas être sécantes, ni parallèles.

\begin{multicols}{2}
    \begin{center}
        Il est possible pour deux droites dans l'espace d'être ni sécantes ni parallèles.
    \end{center}

    \columnbreak

    \begin{center}
        \input{Fig_ENQhxmG.pstricks}
    \end{center}
\end{multicols}

%---------------------------------------------------------------------------------------------------------------------------
\subsection{Position relative d'une droite et un plan}
%---------------------------------------------------------------------------------------------------------------------------

\begin{definition}
    Une droite est \defe{parallèle}{parallèle!droite et plan} à un plan lorsque soit la droite est contenue dans le plan, soit elle n'a aucun point commun avec le plan.
\end{definition}

%///////////////////////////////////////////////////////////////////////////////////////////////////////////////////////////
\subsubsection{Droite et plan sécants}
%///////////////////////////////////////////////////////////////////////////////////////////////////////////////////////////

\begin{multicols}{2}

    La droite \( (DB)\) intersecte le plan \( (AEF)\).

    \columnbreak
    \begin{center}
\input{Fig_figureBCtCTZo.pstricks}
    \end{center}
\end{multicols}

%///////////////////////////////////////////////////////////////////////////////////////////////////////////////////////////
\subsubsection{Droite et plan parallèles}
%///////////////////////////////////////////////////////////////////////////////////////////////////////////////////////////

\begin{multicols}{2}

    La droite \( (HC)\) et le plan \( (EBF)\) sont parallèles.

    \columnbreak
    \begin{center}
%The result is on figure \ref{LabelFigfigureASkECWS}. % From file figureASkECWS
%\newcommand{\CaptionFigfigureASkECWS}{<+Type your caption here+>}
\input{Fig_figureASkECWS.pstricks}
    \end{center}
\end{multicols}

%///////////////////////////////////////////////////////////////////////////////////////////////////////////////////////////
\subsubsection{Droite contenue dans un plan}
%///////////////////////////////////////////////////////////////////////////////////////////////////////////////////////////

\begin{multicols}{2}

    La droite \( (EB)\) est contenue dans le plan \( (AEF)\).

    \columnbreak
    \begin{center}
%The result is on figure \ref{LabelFigfigureCSIQETx}. % From file figureCSIQETx
%\newcommand{\CaptionFigfigureCSIQETx}{<+Type your caption here+>}
\input{Fig_figureCSIQETx.pstricks}
    \end{center}
\end{multicols}

%\begin{definition}
%    Une droite est \defe{perpendiculaire}{perpendiculaire!droite et plan} à un plan si elle est perpendiculaire à deux droites non confondues du plan.
%\end{definition}

%\begin{Aretenir}
%    Si la droite \( d\) est perpendiculaire au plan \( \Omega\), alors toutes les droites du plan \( \Omega\) sécantes avec \( d\) sont perpendiculaires à \( d\).
%\end{Aretenir}

