% This is part of Un soupçon de mathématique sans être agressif pour autant
% Copyright (c) 2014
%   Laurent Claessens
% See the file fdl-1.3.txt for copying conditions.

\begin{exercice}\label{exosmath-0895}

Soient les deux programmes de calculs suivants :

\begin{framed}
    {\bf Programme 1 :}
\begin{enumerate}
    \item
 Choisir un nombre ;
\item
    multiplier par \( 6\)
\item
    ajouter \( 11\).
\end{enumerate}
\end{framed}
\begin{framed}
    {\bf Programme 2 :}
\begin{enumerate}
    \item
 Choisir un nombre ;
    \item
        ajouter \( 2\)
    \item
        multiplier par \( 6\).
\end{enumerate}
\end{framed}

\begin{enumerate}
    \item
 Tester chacun de ces deux programmes de calculs en partant $2$, $-3$ et $4$.
\item   \label{ItemAMPooMNffsb}
    Quel semble être le lien entre les résultats des deux programmes ?
\item
    Si l'on note $x$ le nombre choisi au départ, écrire une expression $A$ qui traduit le programme $1$.
\item
    Si l'on note $x$ le nombre choisi au départ, écrire une expression $B$ qui traduit le programme $2$.
\item
    Prouver le lien que vous avez \emph{conjecturé} pour la question \ref{ItemAMPooMNffsb}.
\end{enumerate}

\corrref{smath-0895}
\end{exercice}
