% This is part of Un soupçon de mathématique sans être agressif pour autant
% Copyright (c) 2012
%   Laurent Claessens
% See the file fdl-1.3.txt for copying conditions.

\begin{exercice}\label{exosmath-0048}

    Factoriser les expressions suivantes en utilisant les techniques usuelles, mais en plusieurs étapes.
    \begin{multicols}{3}
        \begin{enumerate}
            \item
                \( 4x^3-4x\)
            \item
                \( 3x^2+6x+3\)
            \item
                \( (x+1)^2+3x+3\)
            \item
                \( 16x^2-36\)
            \item
                \( 8x^2-8x+2\)
            \item
                \( (2x-3)(3x+7)-2x+3\)
            \item
                \( x^2+2x+1-a^2\)
            \item
                \( x^2-4+3(x+2)\)
            \item
                \( 4x^2-12x+9\)
            \item
                \( (x-5)^2+3(x+2)(x-5)\)
        \end{enumerate}
    \end{multicols}

\corrref{smath-0048}
\end{exercice}
