% This is part of Un soupçon de mathématique sans être agressif pour autant
% Copyright (c) 2013
%   Laurent Claessens
% See the file fdl-1.3.txt for copying conditions.

\begin{exercice}[\cite{TCTTabY}]\label{exosmath-0310}

    L'unité d'intensité du son est le \wikipedia{fr}{Decibel}{décibel}, noté \unit{}{\deci\bel}. Nous considérons un orchestre symphonique jouant \emph{forte} avec une  intensité de \unit{110}{\deci\bel}.

    Nous souhaitons isoler la salle de concert via une succession de plaques d'isolation phonique, chacune absorbant \( 10\%\) de l'intensité sonore qui lui parvient. Nous notons \( u_n\) l'intensité sonore mesurée après la traversée de \( n\) plaques d'isolation (\( u_0\) est l'intensité sonore de l'orchestre, c'est à dire \( 110\)).

    \begin{enumerate}
        \item
            Calculer \( u_1\), \( u_2\) et \( u_3\).
        \item
            Déterminer la relation entre \( u_{n+1} \) et \( u_n\). Identifier le type de suite à laquelle nous avons affaire : terme initial et raison.
        \item
            Un local est tranquille lorsque l'intensité sonore ne dépasse pas les \unit{50}{\deci\bel}. Combien de couches d'isolation phoniques devons-nous mettre entre la sale de concert et le bar si nous voulons que le bar soit un local tranquille lorsque l'orchestre joue ?
    \end{enumerate}

\corrref{smath-0310}
\end{exercice}
