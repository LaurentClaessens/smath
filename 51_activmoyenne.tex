% This is part of Un soupçon de mathématique sans être agressif pour autant
% Copyright (c) 2014-2015
%   Laurent Claessens
% See the file fdl-1.3.txt for copying conditions.

%--------------------------------------------------------------------------------------------------------------------------- 
\subsection*{Activité : calculer sa moyenne}
%---------------------------------------------------------------------------------------------------------------------------

Le but de cette activité est d'apprendre à calculer la moyenne de plusieurs devoirs avec des coefficients. Dans la suite, toutes les notes sont équivalentes à une note sur \( 20\).
\begin{enumerate}
    \item
        Sans coefficients. Un devoir \( \dfrac{ 12 }{ 20 }\) et un devoir \( \dfrac{ 10 }{ 20 }\).
    \item
        Sans coefficients. Un devoir \( \dfrac{ 12 }{ 20 }\) et un devoir \( \dfrac{ 5 }{ 10 }\).
    \item
        Avec coefficients. Un devoir \( \dfrac{ 12 }{ 20 }\) coefficient \( 2\) et un devoir \( \dfrac{ 10 }{ 20 }\) coefficient \( 1\).
    \item
        Le même que le précédent, mais avec des coefficients \( 0.2\) et \( 0.1\).
    \item
        Encore le même, mais avec des coefficients \( 50\) et \( 25\) (holala les coefficients géants !!)
\end{enumerate}
