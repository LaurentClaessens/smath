% This is part of Un soupçon de mathématique sans être agressif pour autant
% Copyright (c) 2014
%   Laurent Claessens
% See the file fdl-1.3.txt for copying conditions.

\begin{exercice}\label{exosmath-0721}


    Voici le graphe d'une fonction homographique \( f\) :

\begin{center}
   \input{Fig_STSooGUprbS.pstricks}
\end{center}

\begin{enumerate}
    \item
        Ce graphe est-il celui de la fonction \( f_1\), \( f_2\) ou \( f_3\) ?
        \begin{equation}
            \begin{aligned}[]
                f_1(x)=\frac{ x-1 }{ 2x+1 };&&f_2(x)&=\frac{ x-1 }{ 2(x+4) };&f_3(x)&=\frac{ x-2 }{ 2(x+4) }.
            \end{aligned}
        \end{equation}
    \item
        Donner l'ensemble de définition de \( f\).
    \item
        Résoudre par le calcul l'équation \( f(x)=-3\).
    \item
        Donner un antécédent de \( -4\) par la fonction \( f\).
    \item
        Donner l'image de \( -4\) par la fonction \( f\).
    \item
        Résoudre graphiquement l'inéquation \( f(x)\leq 3\). Donner les solutions sous forme d'intervalle.
\end{enumerate}



\corrref{smath-0721}
\end{exercice}
