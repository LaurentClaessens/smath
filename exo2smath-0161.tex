% This is part of Un soupçon de mathématique sans être agressif pour autant
% Copyright (c) 2015
%   Laurent Claessens
% See the file fdl-1.3.txt for copying conditions.

\begin{exercice}\label{exo2smath-0161}
\let\Oldtheenumi\theenumi
\let\Oldtheenumii\theenumii
\renewcommand{\theenumi}{(\arabic{enumi})}
\renewcommand{\theenumii}{(\alph{enumii})}

\begin{enumerate}
    \item
        
Placer les points suivants sur une droite graduée :
\begin{multicols}{2}
\begin{enumerate}
    \item
        le point \( A\) d'abscisse \( 3\)
    \item
        le point \( B\) d'abscisse \( -2\)
    \item
        le point \( C\) d'abscisse \( 1.5\)
    \item
        le point \( D\) d'abscisse \( -1.5\)
\end{enumerate}
\end{multicols}
\item
Donner la distance entre les points \( C\) et \( D\).
\item\label{ItemBRMGooQoNoou}
    Placer un point dont la distance à \( B\) est \( 4\).
\end{enumerate}

\let\theenumi\Oldtheenumi
\let\theenumii\Oldtheenumii
\corrref{2smath-0161}
\end{exercice}
