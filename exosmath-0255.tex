% This is part of Un soupçon de mathématique sans être agressif pour autant
% Copyright (c) 2013-2014
%   Laurent Claessens
% See the file fdl-1.3.txt for copying conditions.

\begin{exercice}[\cite{NAJYXNC}]\label{exosmath-0255}

Vrai ou faux ?
    \begin{enumerate}
        \item
            Pour \( x=0\), nous avons \( x^2-2x=x(x-2)\). 
        \item
            Pour \( x=1\), nous avons \( x^2-2x=x(x-2)\). 
        \item
            Pour tout réel \( x\), nous avons \( x^2-x=x(x+3)\). 
        \item
            Il existe un réel \( x\) tel que \( x^2-x=x(x+3)\). 
    \end{enumerate}

\corrref{smath-0255}
\end{exercice}
