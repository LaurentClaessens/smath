% This is part of Un soupçon de mathématique sans être agressif pour autant
% Copyright (c) 2012-2013
%   Laurent Claessens
% See the file fdl-1.3.txt for copying conditions.

\begin{exercice}\label{exoSeconde-0069}

    \begin{multicols}{2}

À partir du graphique ci-contre, répondre par vrai ou faux aux affirmations suivantes.

\begin{enumerate}
   \item
       L'image de \( 7\) par la fonction \( f\) est \( 4\).
   \item
       \( f(3)=4\).
   \item
       Les antécédents de \( 3\) sont \( 2\) et \( 4\)
   \item
       \( f(2)=5\).
   \item
       L'image de \( 0\) par la fonction \( f\) est \( 1\).
   \item
       \( f(3)>f(5)\).
   \item
       \( f\) est croissante sur \( \mathopen[ 1 , 3 \mathclose]\).
   \item
       \( f\) est décroissante sur \( \mathopen[ 3 , 6 \mathclose]\).
    \item
        \( f\) a un minimum pour \( x=5\) et il vaut \( 2\).
    \item
       L'image de \( 1\) par la fonction \( f\) est \( 0\).
\end{enumerate}

\columnbreak


%\includegraphics[width=9.0cm]{Picture_FIGLabelFigExoIntersectionCourbenzIxXdPICTExoIntersectionCourbenzIxXd-for_eps.png}
\input{Fig_ExoIntersectionCourbenzIxXd.pstricks}

    \end{multicols}

Toujours à partir du même dessin, donner les solutions (approximatives) de l'équation \( f(x)=x\).

%Source : \cite{BPizfV}.
% TODO : remettre cette source.

\corrref{Seconde-0069}
\end{exercice}
