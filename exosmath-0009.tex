% This is part of Un soupçon de mathématique sans être agressif pour autant
% Copyright (c) 2012
%   Laurent Claessens
% See the file fdl-1.3.txt for copying conditions.

\begin{exercice}\label{exosmath-0009}

    Un cylindre a pour hauteur \unit{6}{\centi\meter} et le rayon de sa base est de \unit{2}{\centi\meter}.
    \begin{enumerate}
        \item
            Calculer le volume.
        \item
            Calculer son aire totale (surface latérale plus la base et le «couvercle» )
        \item
            Le patron de la surface latérale est un rectangle. Quelle est la longueur de la diagonale de ce dernier ?
    \end{enumerate}
    Comme toujours, la réponse importante est la réponse exacte !!! Laisser les fractions, les \( \pi\) et les racines. Si après ça vous amuse de donner une valeur approchée pour rentabiliser votre calculatrice, allez-y avec modération\footnote{Si le cylindre en question est un semi-conducteur de qualité spatiale, la quatrième décimale de son volume est \emph{vraiment} importante.}.

\corrref{smath-0009}
\end{exercice}
