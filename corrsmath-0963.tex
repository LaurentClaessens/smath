% This is part of Un soupçon de mathématique sans être agressif pour autant
% Copyright (c) 2014
%   Laurent Claessens
% See the file fdl-1.3.txt for copying conditions.

\begin{corrige}{smath-0963}

    \begin{enumerate}
        \item
            Le plus long côté est \( [FG]\), donc si le triangle est rectangle, l'angle droit est au point \( H\). Nous calculons :
            \begin{equation}
                GH^2+FH^2=9^2+40^2=1681.
            \end{equation}
            et
            \begin{equation}
                FG^2=41^2=1681.
            \end{equation}
            Étant donné que l'égalité \( GH^2+FH^2=FG^2\) est vérifiée, le théorème de Pythagore (sa réciproque) implique que le triangle \( FGH\) est rectangle en \( H\).
        \item
            Le triangle \( KLM\) étant rectangle en \( K\), l'égalité de Pythagore s'écrit
            \begin{equation}
                MK^2+KL^2=ML^2.
            \end{equation}
            En remplaçant par les nombres donnés,
            \begin{equation}
                MK^2+12^2=20^2,
            \end{equation}
            c'est à dire
            \begin{equation}
                MK^2+144=400.
            \end{equation}
            Nous en déduisons que
            \begin{equation}
                MK^2=256
            \end{equation}
            et donc que \( MK=16\) parce que \( 16\times 16=256\).
    \end{enumerate}

\end{corrige}
