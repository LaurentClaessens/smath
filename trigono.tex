% This is part of Un soupçon de mathématique sans être agressif pour autant
% Copyright (c) 2013
%   Laurent Claessens
% See the file fdl-1.3.txt for copying conditions.

%+++++++++++++++++++++++++++++++++++++++++++++++++++++++++++++++++++++++++++++++++++++++++++++++++++++++++++++++++++++++++++ 
\section{Enroulement de la droite numérique sur le cercle}
%+++++++++++++++++++++++++++++++++++++++++++++++++++++++++++++++++++++++++++++++++++++++++++++++++++++++++++++++++++++++++++

\begin{definition}
    Le \defe{cercle trigonométrique}{cercle!trigonométrique} est le cercle de centre \( (0;0)\) et de rayon \( 1\) muni de l'orientation dans le sens direct (le sens inverse des aiguilles d'une montre).
\end{definition}

Nous enroulons la droite réelle sur le cercle, voir la figure \ref{LabelFigYORfWSM}. % From file YORfWSM
\newcommand{\CaptionFigYORfWSM}{Un cercle trigonométrique avec enroulement de la droite rélle.}
\input{Fig_YORfWSM.pstricks}

Étant donné que la circonférence du cercle est \( 2\pi\), le nombre \( 2\pi\) de la droite réelle vient au même endroit que le nombre zéro. Le nombre \( 2\pi+x\) vient alors au même endroit que \( x\) pour tout \( x\).

%+++++++++++++++++++++++++++++++++++++++++++++++++++++++++++++++++++++++++++++++++++++++++++++++++++++++++++++++++++++++++++ 
\section{Longueur d'arc de cercle}
%+++++++++++++++++++++++++++++++++++++++++++++++++++++++++++++++++++++++++++++++++++++++++++++++++++++++++++++++++++++++++++

<++>

%+++++++++++++++++++++++++++++++++++++++++++++++++++++++++++++++++++++++++++++++++++++++++++++++++++++++++++++++++++++++++++ 
\section{Exercices}
%+++++++++++++++++++++++++++++++++++++++++++++++++++++++++++++++++++++++++++++++++++++++++++++++++++++++++++++++++++++++++++


