% This is part of Un soupçon de mathématique sans être agressif pour autant
% Copyright (c) 2014
%   Laurent Claessens
% See the file fdl-1.3.txt for copying conditions.

\begin{exercice}\label{exosmath-0605}

    Dans un repère orthonormé nous posons \( A(4;0)\) et \( C(1;3)\).
    \begin{enumerate}
        \item
            Calculer les coordonnées du point \( B\) tel que \( \vect{ CB }=\vect{ OA }\).
        \item
            Donner les équations des droites \( (OC)\), \( (AB)\) et \( (BC)\).
        \item
            Lesquelles parmi ces trois droites sont parallèles ?
        \item
            Donner l'équation de la droite parallèle à \( (OB)\) passant par \( A\).
        \item
            (plus difficile) Déterminer l'équation de la droite passant par \( C\) et perpendiculaire à \( (AB)\).
    \end{enumerate}

\corrref{smath-0605}
\end{exercice}
