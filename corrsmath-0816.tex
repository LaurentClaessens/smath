% This is part of Un soupçon de mathématique sans être agressif pour autant
% Copyright (c) 2014
%   Laurent Claessens
% See the file fdl-1.3.txt for copying conditions.

\begin{corrige}{smath-0816}

    \begin{enumerate}
        \item
            \( (+5)\times (-12)=-60\). Le produit de deux nombres de signe différents est négatif.
        \item
            \( \dfrac{ 12 }{ -4 }=-\frac{ 12 }{ 4 }=-3\). Le quotient de deux nombres de signes différents est négatif.
        \item
            \( (-4)\times (-5)=20\). Le produit de deux négatifs est positif.
        \item
            \( (+10)-(+14)=10+(-14)\). C'est \( (-14)\) qui a la plus grande distance à zéro et donc lui qui donne le signe au résultat.
        \item
            \( (-9)+(-3)=-12\). La somme de deux nombre de même signe a encore le même signe.
    \end{enumerate}

\end{corrige}
