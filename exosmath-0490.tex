% This is part of Un soupçon de mathématique sans être agressif pour autant
% Copyright (c) 2013
%   Laurent Claessens
% See the file fdl-1.3.txt for copying conditions.

\begin{exercice}\label{exosmath-0490}

    % Cet exercice est inspiré de Liouba.Leroux :
    % http://mutuamath.sesamath.net/node/1017
    % Y compris l'idée de la minipage pour faire en multicols. Merci d'avoir publié le code.

\newpage

\begin{center}
    \emph{La grille présentée ici est une grille de sudoku normale à part qu'elle utilise les nombres de \( -4\) à \( 4\) au lieu des nombres de \( 0\) à \( 9\). Les exercices proposés vous permettent de trouver des indices supplémentaires. Il y a moyen de résoudre la grille sans résoudre tous les exercices.}
\end{center}

\begin{multicols}{2}

    \begin{minipage}{7cm}
        \input{Fig_LNfaWPF.pstricks}
    \end{minipage}

    \vspace{1cm}

    \begin{enumerate}
        \item
            Solution de l'équation \( 2x=8\) dans la case A2.
        \item
            Solution de l'équation \( 7x+8=5x\) dans la case A9.
        \item
            La probabilité de l'événement certain dans la case A8.
        \item
            Solution de l'équation \( \frac{1}{ x }=\frac{1}{ 2 }-\frac{1}{ 6 }\) dans la case A5.
        \item
            Le point d'intersection des droites d'équation \( y=x-7\) et \( y=-2x+2\). Mettre les abscisses en G1 et les ordonnées en G2.
        \item
            Mettre l'image de \( 3\) par la fonction \( x\mapsto -2x^2+6x-3\) dans la case G5.
        \item
            L'abscisse du sommet de la parabole \( y=x^2\) dans la case G9.
        \item
            L'abscisse du sommet de la parabole \( y=2x^2+8x-16\) dans la case G4.
        \item
            Le milieu du segment joignant les points \( (-1;7)\) et \( (5;-1)\). Mettre les abscisses en C7 et les ordonnées en D7.
        \item
            Le coefficient directeur de la droite d'équation \( y=-x+4\) dans la case I7 et son ordonnée à l'origine dans la case H3.
        \item
            Si \( A=(4;4)\) et \( B=(7;0)\), le coordonnées du vecteur \( \vect{ AB }\) vont dans les cases E4 et F4.
        \item
            La droite passant par les points \( (1;1)\) et \( (2;5)\). Son coefficient directeur dans la case I8 et son ordonnée à l'origine dans la case H1.
        \item
            Les cases E1 et G6 sont ont le même nombre.
    \end{enumerate}
\end{multicols}


\corrref{smath-0490}
\end{exercice}
