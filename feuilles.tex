% This is part of Un soupçon de mathématique sans être agressif pour autant
% Copyright (c) 2012-2013
%   Laurent Claessens
% See the file fdl-1.3.txt for copying conditions.

%\documentclass[a4paper,10pt]{book}
\documentclass[a4paper,10pt]{article}
% This is part of Un soupçon de mathématique sans être agressif pour autant
% Copyright (c) 2012-2014
%   Laurent Claessens
% See the file fdl-1.3.txt for copying conditions.


\usepackage{etex}
\usepackage{ifthen}
%\usepackage{pdfsync}       % This package is obsolete : compile with pdflatex -synctex=1 instead.

\usepackage{latexsym}
\usepackage{amsfonts}
\usepackage{amsmath}
\usepackage{amsthm}
\usepackage{amssymb}
\usepackage{bbm}
\usepackage{mathrsfs}           
\usepackage{mathabx}           % Pour \divides

\usepackage{framed} % pour oframed
\usepackage{wrapfig}

\usepackage{calc}   % Les dépendances de phystricks si on n'utilise que le pdf.
%\usepackage{pstricks,pst-eucl,pstricks-add,calc,pst-math}   % Les dépendances de phystricks. Peut être qu'il faut ajouter catchfile


% Les dépendances de phystricks en mode Tikz
\usepackage{tikz}
\usepackage{calc}
\usetikzlibrary{calc}
\usetikzlibrary{patterns}

\usetikzlibrary{external}
\tikzexternalize
%\newcommand{\tikzsetnextfilename}[1]{}
\newcounter{defHatch}
\newcounter{defPattern}
\setcounter{defHatch}{0}
\setcounter{defPattern}{0}


\usepackage{graphicx}                   % Pour l'inclusion d'image en pfd.

%\newcommand{\EpsOrPdfincludegraphics}[2][]{%
%        \ifpdf
%            \includegraphics[#1]{#2.png}
%        \else
%            \includegraphics[#1]{#2.eps}
%        \fi
%        }

\usepackage{subfigure}

\usepackage{fancyvrb}
\usepackage{stmaryrd}       % Pour le \obslash
\usepackage{xstring}        % Utilisé pour les références vers wikipédia
\usepackage{cases}
\usepackage{lscape}         % pour l'environnement landscape, utilisé dans la correction corr0076.tex
\usepackage{multicol}
\setlength{\columnseprule}{0.5pt}
\usepackage{import}         % Pour le hack qui sert à inclure GeomAnal

% TODO : n'en utiliser qu'un
\usepackage[normalem]{ulem}     % Pour le barré, commande \sout
\usepackage{soul}       % Pour le barré, commande \st

\usepackage[all]{xy}

\let\second\undefined      % le paquet amthabx définit \second
\let\degree\undefined       % le paquet amthabx définit \degree
%\usepackage[cdot,thinqspace,amssymb]{SIunits} 
\usepackage[parse-numbers=false,binary-units=true,detect-all=true,per-mode=symbol]{siunitx} 
\newcommand{\unit}[2]{\SI{#1}{#2}}
 % L'option amssymb sert à éviter un conflit avec la commande \square de amssymb. Note qu'elle n'est plus accessible. Si tu en as besoin, faudra RTFM
%ftp://ftp.belnet.be/packages/ctan/macros/latex/contrib/SIunits/SIunits.pdf

\usepackage[nottoc]{tocbibind}
\usepackage[numbers]{natbib}

%%%%%%%%%%%%%%%%%%%%%%%%%%
%
%   Trucs mathématiques
%
%%%%%%%%%%%%%%%%%%%%%%%%

% ENSEMBLES DE NOMBRES
\newcommand{\eA}{\mathbbm{A}}
\newcommand{\eC}{\mathbbm{C}}
\newcommand{\eD}{\mathbbm{D}}
\newcommand{\eE}{\mathbbm{E}}
\newcommand{\eF}{\mathbbm{F}}
\newcommand{\eG}{\mathbbm{G}}
\newcommand{\eH}{\mathbbm{H}}
\newcommand{\eK}{\mathbbm{K}}
\newcommand{\eL}{\mathbbm{L}}
\newcommand{\eM}{\mathbbm{M}}
\newcommand{\eN}{\mathbbm{N}}
\newcommand{\eP}{\mathbbm{P}}
\newcommand{\eQ}{\mathbbm{Q}}
\newcommand{\eR}{\mathbbm{R}}
\newcommand{\eZ}{\mathbbm{Z}}

% ENSEMBLES de fonctions
\newcommand{\aL}{\mathcal{L}}       % Les applications linéaires
\newcommand{\aC}{\mathcal{C}}       % Les fonctions C^1, C^2 etc



\newcommand{\mF}{\mathcal{F}}
\newcommand{\mC}{\mathcal{C}}
\newcommand{\mG}{\mathcal{G}}
\newcommand{\mI}{\mathcal{I}}
\newcommand{\mL}{\mathcal{L}}
\newcommand{\mS}{\mathcal{S}}   % Utilisé pour l'espace des fonctions Schwartz
\newcommand{\mZ}{\mathcal{Z}}


\newcommand{\mtu}{\mathbbm{1}}              % La matrice unité

\DeclareMathOperator{\val}{val}     % valuation d'un polynôme
%\DeclareMathOperator{\opp}{opp}    % Les nombres négatifs


%\newcommand{\efrac}[2]{\frac{ \displaystyle #1 }{\displaystyle #2 }}
%%%%%%%%%%%%%%%%%%%%%%%%%%
%
%   Numérotations en tout genre
%
%%%%%%%%%%%%%%%%%%%%%%%%

\setcounter{tocdepth}{2}        % Profondeur de la table des matières
\setcounter{secnumdepth}{2}     % Profondeur dans le texte

\renewcommand{\thesubsection}{\thesection.\alph{subsection}}

%%%%%%%%%%%%%%%%%%%%%%%%%%
%
%   Les lignes magiques pour le texte en français.
%
%%%%%%%%%%%%%%%%%%%%%%%%

\usepackage[utf8]{inputenc}
\usepackage[T1]{fontenc}

\usepackage{listingsutf8}
%\lstset{language=python,basicstyle=\footnotesize,tabsize=3,numbers=left,numberstyle=\tiny,frame=single,commentstyle=\ttfamily\color[rgb]{0,0,0.5},stringstyle=\color[rgb]{0,0.5,0},title=\lstname,inputencoding=utf8/latin1}
\lstset{language=python,basicstyle=\footnotesize,tabsize=3,frame=single,commentstyle=\ttfamily\color[rgb]{0,0,0.5},stringstyle=\color[rgb]{0,0.5,0},title=\lstname,inputencoding=utf8/latin1}

\usepackage[fr]{exocorr}
\usepackage{textcomp}
\usepackage{lmodern}
\usepackage[a4paper,margin=2cm]{geometry} 
\usepackage[english,frenchb]{babel}


\usepackage{hyperref}                           %Doit être appelé en dernier.
\hypersetup{
colorlinks=true,
linkcolor=blue,
urlcolor=magenta,     % couleur des url
filecolor=magenta   % couleur des textes qui sont des liens
}

% Il me semble que cette commande doit être définie après l'appel à Babel.
\newcommand{\Ieme}{\up{\lowercase{ième}}\xspace}

%%%%%%%%%%%%%%%%%%%%%%%%%%
%
%   Les théorèmes et choses attenantes
%
%%%%%%%%%%%%%%%%%%%%%%%%


\newcounter{numtho}
\newcounter{numprob}

\makeatletter
\@addtoreset{numtho}{chapter}
%\@addtoreset{CountExercice}{chapter}
\@addtoreset{chapter}{part}
\makeatother

\newlength{\EnvSpace}
\setlength{\EnvSpace}{9pt}      % C'est la distance que je veux mettre avant et après les théorèmes, remarques, etc.

\newtheoremstyle{MyTheorems}%
        {\EnvSpace}{\EnvSpace}%
        {\itshape}%
        {}%
        {\bfseries}{.}%
        {\newline}%
        {}%
\newtheoremstyle{MyExamples}%
        {\EnvSpace}{\EnvSpace}%
        {}%
        {}%
        {\bfseries}{.}%
        {\newline}%
        {}%
\newtheoremstyle{MyRemarks}%
        {\EnvSpace}{\EnvSpace}%
        {}%
        {}%
        {\bfseries}{.}%
        {\newline}%
        {}%

%\theoremstyle{MyExamples}   %\newtheorem{exemple}[numtho]{Exemple}      % Pour unification, ne plus utiliser
%                            \newtheorem{example}[numtho]{Exemple}
\newcounter{CounterExample}
\renewcommand{\theCounterExample}{\thechapter.\arabic{CounterExample}}

% J'ai décidé de ne plus numéroter les choses encadrées. 8 avril 2014
\newenvironment{example}{\vspace{\EnvSpace}\refstepcounter{numtho}\noindent{\bf Exemple}\\\nopagebreak}{\phantom{a}\hfill $\triangle$\vspace{\EnvSpace}}
\newenvironment{Aretenir}{\refstepcounter{numtho}\begin{oframed}\noindent{\bf À retenir}\newline}{\end{oframed}\vspace{\EnvSpace}}
\newenvironment{Aprojeter}{\clearpage\phantom{a}\vfill}{\vfill\newpage}
\newenvironment{definition}[1][]{\refstepcounter{numtho}\begin{oframed}\noindent{\bf Définition}#1\newline}{\end{oframed}\vspace{\EnvSpace}}
\newenvironment{propriete}{\refstepcounter{numtho}\begin{oframed}\noindent{\bf Propriété}\newline}{\end{oframed}\vspace{\EnvSpace}}

\newenvironment{Enmini}{\begin{oframed}\noindent{\bf Mini résumé}\newline}{\end{oframed}\vspace{\EnvSpace}}
% Ce bout de code provient de BrunoJ
% https://brunoj.wordpress.com/2009/10/08/latex-the-framed-minipage/
\newsavebox{\fmbox}
 \newenvironment{fmpage}[1]
 {\begin{lrbox}{\fmbox}\begin{minipage}{#1}}
     {\end{minipage}\end{lrbox}
     \fbox{\usebox{\fmbox}}
 }

\theoremstyle{MyRemarks}    \newtheorem{remark}[numtho]{Remarque}

                \newtheorem{amusement}[numtho]{Amusement}
                \newtheorem{erreur}[numtho]{Error}
                \newtheorem{probleme}[numprob]{\fbox{\bf Problèmes et choses à faire}}


\theoremstyle{MyTheorems}
\newtheorem{lemma}[numtho]{Lemme}
\newtheorem{corollary}[numtho]{Corollaire}
\newtheorem{theorem}[numtho]{Théorème}      
\newtheorem{proposition}[numtho]{Proposition}      


\renewcommand{\thenumtho}{\thechapter.\arabic{numtho}}
% La numérotation des équations change dans les corrigés
\renewcommand{\theequation}{\thechapter.\arabic{equation}}
\renewcommand{\theCountExercice}{\arabic{CountExercice}}       % Ce compteur est défini dans SystemeCorr.sty
\newcommand{\defe}[2]{\textbf{#1}\index{#2}}

\renewcommand{\theenumi}{(\alph{enumi})}
\renewcommand{\theenumii}{(\alph{enumi}\arabic{enumii})}

\renewcommand{\labelenumi}{\theenumi}
\renewcommand{\labelenumii}{\theenumii}

\newcommand{\justification}{ {\small \begin{center}    Attention : vous devez laisser sur votre feuille les traces de vos recherches et les étapes intermédiaires de vos calculs !    \end{center}}}
    

    % L'une des deux est avec le nom et l'autre sans.
    \newenvironment{feuilleDS}[1]{\noindent Nom, Prénom : \begin{center}\large #1\\\justification\end{center}\setcounter{CountExercice}{0}  }{\clearpage}
    %\newenvironment{feuilleDS}[1]{\begin{center}\large #1\\\justification\end{center}\setcounter{CountExercice}{0}  }{\clearpage}


    \newenvironment{feuilleExo}[1]{\newpage\begin{center}\large #1\end{center}\setcounter{CountExercice}{0}  }{\clearpage}
    %\newenvironment{feuilleExo}[1]{\begin{center}\large #1\end{center}\setcounter{CountExercice}{0}  }{}

        
        \newcounter{numactivmentale}
        \setcounter{numactivmentale}{0}
        \newcounter{numExoMental}
        \setcounter{numExoMental}{0}
        \newenvironment{MentalActivity}{\setcounter{numExoMental}{0}\newpage\refstepcounter{numactivmentale}\section{Activité mentale \arabic{numactivmentale}}}{\newpage\hphantom{jj}\vfill\large C'est tout pour aujourd'hui\vfill\newpage}
        \newenvironment{mental}{\refstepcounter{numExoMental}\newpage\begin{center}\fbox{\huge Activité mentale \arabic{numactivmentale}}\\\huge Question \arabic{numExoMental}\end{center}\huge\vfill}{\vfill}


\newcommand{\enteteInterro}[3]{
    \begin{center}
        #1\\
        Intérogation #2, sujet #3
    \end{center}
    Nom, prénom, classe : \ldots\\
    \setcounter{CountExercice}{0}
}


%%%%%%%%%%%%%%%%%%%%%%%%%%
%
%   Les macros qui font des choses
%
%%%%%%%%%%%%%%%%%%%%%%%%

\newcommand{\mA}{\mathcal{A}}
\newcommand{\mO}{\mathcal{O}}
\newcommand{\mR}{\mathcal{R}}
\newcommand{\mT}{\mathcal{T}}
\newcommand{\mU}{\mathcal{U}}

\newcommand{\scal}[2]{ \langle #1,#2\rangle }

\newcommand{\tq}{\text{ tel que }}
\newcommand{\tqs}{\text{ tels que }}
\newcommand{\quext}[1]{ \footnote{\textsf{#1}}  }
\newcommand{\info}[1]{\texttt{#1}}
\newcommand{\vect}[1]{\overrightarrow{#1}}    % Cette macro est codée en dur dans phystricksDefVecteurAXDDGP et dans d'autres

\newcommand{\VarAbs}{\text{Var}_{\text{abs}}}
\newcommand{\VarRel}{\text{Var}_{\text{rel}}}

\newcommand{\normal}{\lhd}
\newcommand{\swS}{\mathscr{S}}          % L'ensemble des fonctions Schwartz

%\newcommand{\defD}{\mathscr{D}}     % Ensemble de définition d'une fonction
\newcommand{\defD}{D}                % Le D avec des croles était impossible à comprendre pour les élèves.

\newcommand{\Borelien}{\mathcal{B}\text{or}}       % Les boréliens
\newcommand{\tribA}{\mathcal{A}}            % Une tribu A
\newcommand{\tribB}{\mathcal{B}}            
\newcommand{\tribF}{\mathcal{F}}            % Une tribu F

\newcommand{\affE}{\mathcal{E}}            % Un espace affine E
\newcommand{\affF}{\mathcal{F}}            
\newcommand{\affG}{\mathcal{G}}            

\newcommand{\statS}{\mathcal{S}}            % Un modèle statistique
\newcommand{\partP}{\mathcal{P}}            % L'ensemble des parties d'un ensemble

\newcommand{\polyP}{\mathcal{P}}            % L'ensemble des polynômes

\newcommand{\dB}{\mathscr{B}}       % la distribution de Bernoulli
\newcommand{\dE}{\mathscr{E}}       % la distribution exponentielle
\newcommand{\dG}{\mathscr{G}}       % la distribution géométrique.
\newcommand{\dM}{\mathscr{M}}       % la distribution multinomiale
\newcommand{\dN}{\mathscr{N}}       % la distribution normale.
\newcommand{\dP}{\mathscr{P}}       % la distribution de Poisson.
\newcommand{\dT}{\mathscr{T}}       % la distribution de Student
\newcommand{\dU}{\mathscr{U}}       % la distribution uniforme

\newcommand{\hL}{\mathscr{L}}       
\newcommand{\cL}{\hL}           % Pour la partie Chafai

\newcommand{\modE}{\mathcal{E}}         % Le E des modules
\newcommand{\modF}{\mathcal{F}}         % Le F des modules
\newcommand{\hH}{\mathscr{H}}           % Le H des espaces de Hilbert

%%%%%%%%%%%%%%%%%%%%%%%%%%
%
%   Bibliographie, index et liste des notations
%
%%%%%%%%%%%%%%%%%%%%%%%%

\usepackage{makeidx}
\usepackage[nottoc]{tocbibind}      % Le paquetage qui fait en sorte que la biblio soit inclue correctement dans la table des matières.
\usepackage[refpage]{nomencl}
\renewcommand{\nomname}{Liste des notations}
%
%   Comment introduire des éléments dans l'index des notations.
%
% La liste des tags à mettre pour bien classer mes notations est :
% T     pour la topologie et théorie des ensembles
%
% La syntaxe est facile, par exemple 
%       $\SL(2,\eR)$\nomenclature[G]{$\SL(2,\eR)$}{Le groupe de matrices deux par deux de déterminant 1.}
%\renewcommand{\nomgroup}[1]{%
%    \ifthenelse{\equal{#1}{A}}{\item[\textbf{Algèbre}]}{}%
%    \ifthenelse{\equal{#1}{G}}{\item[\textbf{Géométrie}]}{}%
%    \ifthenelse{\equal{#1}{R}}{\item[\textbf{Théorie des groupes}]}{}%
%    \ifthenelse{\equal{#1}{P}}{\item[\textbf{Probabilités et statistique}]}{}%
%    \ifthenelse{\equal{#1}{Y}}{\item[\textbf{Analyse}]}{}%
%    \ifthenelse{\equal{#1}{M}}{\item[\textbf{Chaînes de Markov}]}{}%
%}

%%%%%%%%%%%%%%%%%%%%%%%%%%
%
%   DeclareMathOperator
%
%%%%%%%%%%%%%%%%%%%%%%%%

\DeclareMathOperator{\signe}{sgn}
\DeclareMathOperator{\Vol}{Vol}
\DeclareMathOperator{\Int}{Int}     % Intérieur d'un ensemble.
\DeclareMathOperator{\Ind}{Ind}     % l'indice d'un chemin en analyse complexe
\DeclareMathOperator{\Diam}{Diam}   
\DeclareMathOperator{\id}{Id}   
\DeclareMathOperator{\Graph}{Graph} 
\DeclareMathOperator{\pr}{\texttt{proj}}
\DeclareMathOperator{\dom}{dom}

\DeclareMathOperator{\Graphe}{Gr}
\DeclareMathOperator{\Spec}{Spec}   % spectre d'un opérateur
\DeclareMathOperator{\arctg}{arctg}
\DeclareMathOperator{\cotg}{cotg}
\DeclareMathOperator{\cosec}{cosec}
\DeclareMathOperator{\arcsinh}{arcsinh}

\DeclareMathOperator{\GL}{GL}   % le groupe linéaire
\DeclareMathOperator{\PGL}{PGL}   % le groupe projectif
\DeclareMathOperator{\SO}{SO}           
\DeclareMathOperator{\SL}{SL}           
\DeclareMathOperator{\PSL}{PSL}   % Le groupe modulaire SL(2,Z)/Z2
\DeclareMathOperator{\gO}{O}           
\DeclareMathOperator{\SU}{SU}           
\DeclareMathOperator{\gU}{U}           

\DeclareMathOperator{\Reel}{Re}        % La partie réelle d'un nombre complexe

\DeclareMathOperator{\Image}{Image}        % ... avec \Image qui donne l'image d'une fonction ou d'un opérateur.
\DeclareMathOperator{\rang}{rg}   
\DeclareMathOperator{\Kernel}{Ker}
\DeclareMathOperator{\Domaine}{Dom}
\DeclareMathOperator{\Span}{Span}
\DeclareMathOperator{\Hom}{Hom}
\DeclareMathOperator{\End}{End}     % L'ensemble des endomorphismes
\DeclareMathOperator{\tr}{Tr}       % la trace
\DeclareMathOperator{\Majorant}{Maj}
\DeclareMathOperator{\codim}{codim} % pour la codimension.
\DeclareMathOperator{\diam}{diam} % le diamètre d'un ensemble.

\DeclareMathOperator{\Var}{Var}     % Variance d'une variable aléatoire.
\DeclareMathOperator{\Fun}{\texttt{Fun}}     % Ensemble des applications d'un ensemble vers l'autre.
\DeclareMathOperator{\Cov}{Cov}     % la covariance.
\DeclareMathOperator{\gr}{gr}     % le groupe engendré
\DeclareMathOperator{\pgcd}{pgcd}     
\DeclareMathOperator{\ppcm}{ppcm}     
\DeclareMathOperator{\Frob}{Frob}     
\DeclareMathOperator{\Card}{Card}       % Le cardinal d'un ensemble.
\DeclareMathOperator{\Stab}{Stab}       % Le stabilisateur d'un point sous l'action d'un groupe.

\DeclareMathOperator{\Frac}{Frac}       % le corps des fractions d'un anneau
\DeclareMathOperator{\Aff}{Aff}         %  l'espace affine engendré

\newenvironment{subproof}{\begin{description}}{\end{description}}

\newcommand{\telque}{\vert\,}
\newcommand{\donc}{\Rightarrow}

\usepackage{mathtools}
\mathtoolsset{showonlyrefs}


\pagestyle{empty}   % Pour éviter les numéros de page.

\begin{document}

%DS MARDI 10 DÉCEMBRE 2013, 2D
\begin{feuilleDS}{Seconde D, DS numéro 3\\ \small mardi 10 décembre 2013}

\Exo{smath-0565}
\Exo{smath-0566}

\end{feuilleDS}

\end{document}

% FEUILLE D'AP pour les 2DE -- logique
\title{Feuille d'AP de seconde, logique}
\date{}
\maketitle

\Exo{smath-0560}
\Exo{smath-0561}
\Exo{smath-0562}
\Exo{smath-0563}
\Exo{smath-0564}    % document d'accompagnement de fonctions.

\end{document}

% FEUILLE D'AP pour les 1ES

\title{Feuille d'AP première ES}
\maketitle

\Exo{smath-0557}
\Exo{smath-0559}
\Exo{Premiere-0006}
\Exo{smath-0558}
\Exo{smath-0399}

\end{document}

%INTERRO STATISTIQUE DESCRIPTIVE
\input{interro_statistique_descriptive_sujet.tex}
\end{document}
% ACTIVITÉ GÉOMÉTRIE DANS L'ESPACE
% This is part of Un soupçon de mathématique sans être agressif pour autant
% Copyright (c) 2013
%   Laurent Claessens
% See the file fdl-1.3.txt for copying conditions.

\begin{wrapfigure}[3]{r}{5.0cm}
   \vspace{-0.5cm}        % à adapter.
   \centering
   \input{Fig_MEzTDZC.pstricks}
\end{wrapfigure}

À l'aide du cube ci-contre :
\begin{enumerate}
    \item
        Quelle est la nature du triangle \( AEF\) ? 
    \item
        Quelle est la nature du quadrilatère \( ABGH\) ?
    \item
        Est-ce que vous pouvez trouver trois points de ce cube formant un triangle équilatéral ?
    \item
        Si ce cube fait \unit{5}{\meter} de côté, quelle est la longueur de la «grande» diagonale \( [AG]\) ?
\end{enumerate}



\vspace{2cm}

% This is part of Un soupçon de mathématique sans être agressif pour autant
% Copyright (c) 2013
%   Laurent Claessens
% See the file fdl-1.3.txt for copying conditions.

\begin{wrapfigure}[3]{r}{5.0cm}
   \vspace{-0.5cm}        % à adapter.
   \centering
   \input{Fig_MEzTDZC.pstricks}
\end{wrapfigure}

À l'aide du cube ci-contre :
\begin{enumerate}
    \item
        Quelle est la nature du triangle \( AEF\) ? 
    \item
        Quelle est la nature du quadrilatère \( ABGH\) ?
    \item
        Est-ce que vous pouvez trouver trois points de ce cube formant un triangle équilatéral ?
    \item
        Si ce cube fait \unit{5}{\meter} de côté, quelle est la longueur de la «grande» diagonale \( [AG]\) ?
\end{enumerate}



\vspace{2cm}

% This is part of Un soupçon de mathématique sans être agressif pour autant
% Copyright (c) 2013
%   Laurent Claessens
% See the file fdl-1.3.txt for copying conditions.

\begin{wrapfigure}[3]{r}{5.0cm}
   \vspace{-0.5cm}        % à adapter.
   \centering
   \input{Fig_MEzTDZC.pstricks}
\end{wrapfigure}

À l'aide du cube ci-contre :
\begin{enumerate}
    \item
        Quelle est la nature du triangle \( AEF\) ? 
    \item
        Quelle est la nature du quadrilatère \( ABGH\) ?
    \item
        Est-ce que vous pouvez trouver trois points de ce cube formant un triangle équilatéral ?
    \item
        Si ce cube fait \unit{5}{\meter} de côté, quelle est la longueur de la «grande» diagonale \( [AG]\) ?
\end{enumerate}




\vspace{2cm}

% This is part of Un soupçon de mathématique sans être agressif pour autant
% Copyright (c) 2013
%   Laurent Claessens
% See the file fdl-1.3.txt for copying conditions.

\begin{wrapfigure}[3]{r}{5.0cm}
   \vspace{-0.5cm}        % à adapter.
   \centering
   \input{Fig_MEzTDZC.pstricks}
\end{wrapfigure}

À l'aide du cube ci-contre :
\begin{enumerate}
    \item
        Quelle est la nature du triangle \( AEF\) ? 
    \item
        Quelle est la nature du quadrilatère \( ABGH\) ?
    \item
        Est-ce que vous pouvez trouver trois points de ce cube formant un triangle équilatéral ?
    \item
        Si ce cube fait \unit{5}{\meter} de côté, quelle est la longueur de la «grande» diagonale \( [AG]\) ?
\end{enumerate}



\vspace{2cm}

% This is part of Un soupçon de mathématique sans être agressif pour autant
% Copyright (c) 2013
%   Laurent Claessens
% See the file fdl-1.3.txt for copying conditions.

\begin{wrapfigure}[3]{r}{5.0cm}
   \vspace{-0.5cm}        % à adapter.
   \centering
   \input{Fig_MEzTDZC.pstricks}
\end{wrapfigure}

À l'aide du cube ci-contre :
\begin{enumerate}
    \item
        Quelle est la nature du triangle \( AEF\) ? 
    \item
        Quelle est la nature du quadrilatère \( ABGH\) ?
    \item
        Est-ce que vous pouvez trouver trois points de ce cube formant un triangle équilatéral ?
    \item
        Si ce cube fait \unit{5}{\meter} de côté, quelle est la longueur de la «grande» diagonale \( [AG]\) ?
\end{enumerate}




\end{document}
% LES FIGURES DE GÉOMÉTRIE DANS L'ESPACE

\renewcommand{\thesection}{\arabic{section}}


%+++++++++++++++++++++++++++++++++++++++++++++++++++++++++++++++++++++++++++++++++++++++++++++++++++++++++++++++++++++++++++ 
\section{Position relatives}
%+++++++++++++++++++++++++++++++++++++++++++++++++++++++++++++++++++++++++++++++++++++++++++++++++++++++++++++++++++++++++++

%--------------------------------------------------------------------------------------------------------------------------- 
\subsection{Positions relatives de deux plans}
%---------------------------------------------------------------------------------------------------------------------------

\begin{definition}
    Deux plans sont \defe{parallèles}{parallèle!deux plans} soit si ils sont confondus, soit si ils n'ont aucun point commun. Si ils n'ont aucun point communs, nous disons qu'ils sont \defe{strictement parallèles}{parallèle!strictement}
\end{definition}

\begin{Aretenir}
    Deux plans non parallèles se coupent en une droite.
\end{Aretenir}

\begin{multicols}{2}

    \begin{center}
        Plans parallèles
    \end{center}
    
\columnbreak

    \begin{center}
\input{Fig_PositionPlansTvKvah.pstricks}
    \end{center}


\end{multicols}

\begin{multicols}{2}
    \begin{center}
        Plans sécants
    \end{center}

    \columnbreak

    \begin{center}
\input{Fig_PositionPlansqSltxa.pstricks}
    \end{center}

\end{multicols}

%---------------------------------------------------------------------------------------------------------------------------
\subsection{Position relative de deux droites}
%---------------------------------------------------------------------------------------------------------------------------

Deux droites peuvent être soit dans un même plan, soit ne pas être dans le même plan. Deux droites contenues dans un même plan sont dires \defe{coplanaires}{coplanaire}.

%///////////////////////////////////////////////////////////////////////////////////////////////////////////////////////////
\subsubsection{Droites coplanaires}
%///////////////////////////////////////////////////////////////////////////////////////////////////////////////////////////

Deux droites coplanaires respectent la géométrie usuelle. Elles peuvent être parallèles ou sécantes.

\vbox{
\begin{multicols}{2}
    \begin{center}
        Droites parallèles dans le plan \( (EBC)\)
    \end{center}

    \columnbreak

    \begin{center}
        \input{Fig_IDqyzXM.pstricks}
    \end{center}
\end{multicols}
}

\vbox{
\begin{multicols}{2}
    \begin{center}
        Droites sécantes dans le plan \( (AEH)\)
    \end{center}

    \columnbreak

    \begin{center}
        \input{Fig_ETfnbsh.pstricks}
    \end{center}
\end{multicols}
}

\begin{Aretenir}
    À propos de droites coplanaires :
    \begin{enumerate}
        \item
            Deux droites sécantes sont toujours coplanaires.
        \item
            Deux droites parallèles sont coplanaires.
    \end{enumerate}
\end{Aretenir}

%///////////////////////////////////////////////////////////////////////////////////////////////////////////////////////////
    \subsubsection{Droites non coplanaires}
%///////////////////////////////////////////////////////////////////////////////////////////////////////////////////////////
    
Deux droites non coplanaires ne peuvent pas être sécantes, ni parallèles.

\vbox{
\begin{multicols}{2}
    \begin{center}
        Il est possible pour deux droites dans l'espace d'être ni sécantes ni parallèles.
    \end{center}

    \columnbreak

    \begin{center}
        \input{Fig_ENQhxmG.pstricks}
    \end{center}
\end{multicols}
}

%---------------------------------------------------------------------------------------------------------------------------
\subsection{Position relative d'une droite et un plan}
%---------------------------------------------------------------------------------------------------------------------------

\begin{definition}
    Une droite est \defe{parallèle}{parallèle!droite et plan} à un plan lorsque soit la droite est contenue dans le plan, soit elle n'a aucun point commun avec le plan.
\end{definition}

%///////////////////////////////////////////////////////////////////////////////////////////////////////////////////////////
\subsubsection{Droite et plan sécants}
%///////////////////////////////////////////////////////////////////////////////////////////////////////////////////////////

\begin{multicols}{2}

    La droite \( (DB)\) intersecte le plan \( (AEF)\).

    \columnbreak
    \begin{center}
\input{Fig_figureBCtCTZo.pstricks}
    \end{center}
\end{multicols}

%///////////////////////////////////////////////////////////////////////////////////////////////////////////////////////////
\subsubsection{Droite et plan parallèles}
%///////////////////////////////////////////////////////////////////////////////////////////////////////////////////////////

\begin{multicols}{2}

    La droite \( (HC)\) et le plan \( (EBF)\) sont parallèles.

    \columnbreak
    \begin{center}
%The result is on figure \ref{LabelFigfigureASkECWS}. % From file figureASkECWS
%\newcommand{\CaptionFigfigureASkECWS}{<+Type your caption here+>}
\input{Fig_figureASkECWS.pstricks}
    \end{center}
\end{multicols}

%///////////////////////////////////////////////////////////////////////////////////////////////////////////////////////////
\subsubsection{Droite contenue dans un plan}
%///////////////////////////////////////////////////////////////////////////////////////////////////////////////////////////

\vbox{
\begin{multicols}{2}

    La droite \( (EB)\) est contenue dans le plan \( (AEF)\).

    \columnbreak
    \begin{center}
\input{Fig_figureCSIQETx.pstricks}
    \end{center}
\end{multicols}
}

\vspace{2cm}
\begin{center}
    Lisez
    
    \url{http://student.ulb.ac.be/~lclaesse/smath.pdf}
\end{center}


\end{document}

% ACTIVITÉ STATISTIQUE DESCRIPTIVE POUR LES 2D


% This is part of Un soupçon de mathématique sans être agressif pour autant
% Copyright (c) 2013
%   Laurent Claessens
% See the file fdl-1.3.txt for copying conditions.

% ATTENTION : les chiffres donnés ici sont repris dans le cours au moment des ECC.
% ATTENTION : il y a trop de 4 dans les valeurs, les effectifs et les ECC. Il faut changer.
%               Il faut changer au moins dans le texte ici, dans la figure et dans le texte des éléments de réponse.

\begin{wrapfigure}[5]{r}{8cm}
   \vspace{-0.5cm}        % à adapter.
   \centering
   \input{Fig_YRQOoPE.pstricks}
\end{wrapfigure}

Le graphique ci-contre illustre le nombre de spam reçus aujourd'hui par les élèves d'une classe.
\begin{enumerate}
    \item
        Combien d'élèves y-a-t-il dans la classe ?
    \item
        Combien d'élèves ont reçu \( 5\) spams ou plus ?
    \item
        En moyenne combien de spam ont reçu les élèves aujourd'hui ?
    \item
        Diviser la classe en 4 groupes suivant le nombre de spams reçus.
\end{enumerate}



\vspace{5cm}

% This is part of Un soupçon de mathématique sans être agressif pour autant
% Copyright (c) 2013
%   Laurent Claessens
% See the file fdl-1.3.txt for copying conditions.

% ATTENTION : les chiffres donnés ici sont repris dans le cours au moment des ECC.
% ATTENTION : il y a trop de 4 dans les valeurs, les effectifs et les ECC. Il faut changer.
%               Il faut changer au moins dans le texte ici, dans la figure et dans le texte des éléments de réponse.

\begin{wrapfigure}[5]{r}{8cm}
   \vspace{-0.5cm}        % à adapter.
   \centering
   \input{Fig_YRQOoPE.pstricks}
\end{wrapfigure}

Le graphique ci-contre illustre le nombre de spam reçus aujourd'hui par les élèves d'une classe.
\begin{enumerate}
    \item
        Combien d'élèves y-a-t-il dans la classe ?
    \item
        Combien d'élèves ont reçu \( 5\) spams ou plus ?
    \item
        En moyenne combien de spam ont reçu les élèves aujourd'hui ?
    \item
        Diviser la classe en 4 groupes suivant le nombre de spams reçus.
\end{enumerate}



\vspace{5cm}

% This is part of Un soupçon de mathématique sans être agressif pour autant
% Copyright (c) 2013
%   Laurent Claessens
% See the file fdl-1.3.txt for copying conditions.

% ATTENTION : les chiffres donnés ici sont repris dans le cours au moment des ECC.
% ATTENTION : il y a trop de 4 dans les valeurs, les effectifs et les ECC. Il faut changer.
%               Il faut changer au moins dans le texte ici, dans la figure et dans le texte des éléments de réponse.

\begin{wrapfigure}[5]{r}{8cm}
   \vspace{-0.5cm}        % à adapter.
   \centering
   \input{Fig_YRQOoPE.pstricks}
\end{wrapfigure}

Le graphique ci-contre illustre le nombre de spam reçus aujourd'hui par les élèves d'une classe.
\begin{enumerate}
    \item
        Combien d'élèves y-a-t-il dans la classe ?
    \item
        Combien d'élèves ont reçu \( 5\) spams ou plus ?
    \item
        En moyenne combien de spam ont reçu les élèves aujourd'hui ?
    \item
        Diviser la classe en 4 groupes suivant le nombre de spams reçus.
\end{enumerate}



\end{document}


%INTERRO GÉOMÉTRIE DANS L'ESPACE
\vbox{1
\emph{Toutes les réponses doivent être justifiées par un calcul accompagné d'un raisonnement.}

\begin{wrapfigure}{r}{5.0cm}
   \vspace{-0.5cm}        % à adapter.
   \centering
   \input{Fig_OKTXHoc.pstricks}
\end{wrapfigure}

    La figure ci-contre est un cube de \unit{2}{\centi\meter}.

    \begin{enumerate}
    \item
    Quelle est la nature du triangle \(EGC\) ? 
    \item
    Quel est son périmètre ?
    \item
    Quelle est son aire ? (pour les rapides)
    \end{enumerate}
    

}
\vspace{2cm}
\vbox{2
\emph{Toutes les réponses doivent être justifiées par un calcul accompagné d'un raisonnement.}

\begin{wrapfigure}{r}{5.0cm}
   \vspace{-0.5cm}        % à adapter.
   \centering
   \input{Fig_OKTXHoc.pstricks}
\end{wrapfigure}

    La figure ci-contre est un cube de \unit{5}{\centi\meter}.

    \begin{enumerate}
    \item
    Quelle est la nature du triangle \(HEB\) ? 
    \item
    Quel est son périmètre ?
    \item
    Quelle est son aire ? (pour les rapides)
    \end{enumerate}
    

}
\vspace{2cm}
\vbox{3
\emph{Toutes les réponses doivent être justifiées par un calcul accompagné d'un raisonnement.}

\begin{wrapfigure}{r}{5.0cm}
   \vspace{-0.5cm}        % à adapter.
   \centering
   \input{Fig_OKTXHoc.pstricks}
\end{wrapfigure}

    La figure ci-contre est un cube de \unit{5}{\centi\meter}.

    \begin{enumerate}
    \item
    Quelle est la nature du triangle \(FBH\) ? 
    \item
    Quel est son périmètre ?
    \item
    Quelle est son aire ? (pour les rapides)
    \end{enumerate}
    

}
\vspace{2cm}
\vbox{4
\emph{Toutes les réponses doivent être justifiées par un calcul accompagné d'un raisonnement.}

\begin{wrapfigure}{r}{5.0cm}
   \vspace{-0.5cm}        % à adapter.
   \centering
   \input{Fig_OKTXHoc.pstricks}
\end{wrapfigure}

    La figure ci-contre est un cube de \unit{6}{\centi\meter}.

    \begin{enumerate}
    \item
    Quelle est la nature du triangle \(BGA\) ? 
    \item
    Quel est son périmètre ?
    \item
    Quelle est son aire ? (pour les rapides)
    \end{enumerate}
    

}
\vspace{2cm}
\vbox{5
\emph{Toutes les réponses doivent être justifiées par un calcul accompagné d'un raisonnement.}

\begin{wrapfigure}{r}{5.0cm}
   \vspace{-0.5cm}        % à adapter.
   \centering
   \input{Fig_OKTXHoc.pstricks}
\end{wrapfigure}

    La figure ci-contre est un cube de \unit{4}{\centi\meter}.

    \begin{enumerate}
    \item
    Quelle est la nature du triangle \(GFC\) ? 
    \item
    Quel est son périmètre ?
    \item
    Quelle est son aire ? (pour les rapides)
    \end{enumerate}
    

}
\vspace{2cm}
\vbox{6
\emph{Toutes les réponses doivent être justifiées par un calcul accompagné d'un raisonnement.}

\begin{wrapfigure}{r}{5.0cm}
   \vspace{-0.5cm}        % à adapter.
   \centering
   \input{Fig_OKTXHoc.pstricks}
\end{wrapfigure}

    La figure ci-contre est un cube de \unit{3}{\centi\meter}.

    \begin{enumerate}
    \item
    Quelle est la nature du triangle \(CAD\) ? 
    \item
    Quel est son périmètre ?
    \item
    Quelle est son aire ? (pour les rapides)
    \end{enumerate}
    

}
\vspace{2cm}
\vbox{7
\emph{Toutes les réponses doivent être justifiées par un calcul accompagné d'un raisonnement.}

\begin{wrapfigure}{r}{5.0cm}
   \vspace{-0.5cm}        % à adapter.
   \centering
   \input{Fig_OKTXHoc.pstricks}
\end{wrapfigure}

    La figure ci-contre est un cube de \unit{4}{\centi\meter}.

    \begin{enumerate}
    \item
    Quelle est la nature du triangle \(EAH\) ? 
    \item
    Quel est son périmètre ?
    \item
    Quelle est son aire ? (pour les rapides)
    \end{enumerate}
    

}
\vspace{2cm}
\vbox{8
\emph{Toutes les réponses doivent être justifiées par un calcul accompagné d'un raisonnement.}

\begin{wrapfigure}{r}{5.0cm}
   \vspace{-0.5cm}        % à adapter.
   \centering
   \input{Fig_OKTXHoc.pstricks}
\end{wrapfigure}

    La figure ci-contre est un cube de \unit{7}{\centi\meter}.

    \begin{enumerate}
    \item
    Quelle est la nature du triangle \(FBD\) ? 
    \item
    Quel est son périmètre ?
    \item
    Quelle est son aire ? (pour les rapides)
    \end{enumerate}
    

}
\vspace{2cm}
\vbox{9
\emph{Toutes les réponses doivent être justifiées par un calcul accompagné d'un raisonnement.}

\begin{wrapfigure}{r}{5.0cm}
   \vspace{-0.5cm}        % à adapter.
   \centering
   \input{Fig_OKTXHoc.pstricks}
\end{wrapfigure}

    La figure ci-contre est un cube de \unit{7}{\centi\meter}.

    \begin{enumerate}
    \item
    Quelle est la nature du triangle \(ACG\) ? 
    \item
    Quel est son périmètre ?
    \item
    Quelle est son aire ? (pour les rapides)
    \end{enumerate}
    

}
\vspace{2cm}
\vbox{10
\emph{Toutes les réponses doivent être justifiées par un calcul accompagné d'un raisonnement.}

\begin{wrapfigure}{r}{5.0cm}
   \vspace{-0.5cm}        % à adapter.
   \centering
   \input{Fig_OKTXHoc.pstricks}
\end{wrapfigure}

    La figure ci-contre est un cube de \unit{4}{\centi\meter}.

    \begin{enumerate}
    \item
    Quelle est la nature du triangle \(ECB\) ? 
    \item
    Quel est son périmètre ?
    \item
    Quelle est son aire ? (pour les rapides)
    \end{enumerate}
    

}
\vspace{2cm}
\vbox{11
\emph{Toutes les réponses doivent être justifiées par un calcul accompagné d'un raisonnement.}

\begin{wrapfigure}{r}{5.0cm}
   \vspace{-0.5cm}        % à adapter.
   \centering
   \input{Fig_OKTXHoc.pstricks}
\end{wrapfigure}

    La figure ci-contre est un cube de \unit{4}{\centi\meter}.

    \begin{enumerate}
    \item
    Quelle est la nature du triangle \(BFD\) ? 
    \item
    Quel est son périmètre ?
    \item
    Quelle est son aire ? (pour les rapides)
    \end{enumerate}
    

}
\vspace{2cm}
\vbox{12
\emph{Toutes les réponses doivent être justifiées par un calcul accompagné d'un raisonnement.}

\begin{wrapfigure}{r}{5.0cm}
   \vspace{-0.5cm}        % à adapter.
   \centering
   \input{Fig_OKTXHoc.pstricks}
\end{wrapfigure}

    La figure ci-contre est un cube de \unit{3}{\centi\meter}.

    \begin{enumerate}
    \item
    Quelle est la nature du triangle \(BEG\) ? 
    \item
    Quel est son périmètre ?
    \item
    Quelle est son aire ? (pour les rapides)
    \end{enumerate}
    

}
\vspace{2cm}
\vbox{13
\emph{Toutes les réponses doivent être justifiées par un calcul accompagné d'un raisonnement.}

\begin{wrapfigure}{r}{5.0cm}
   \vspace{-0.5cm}        % à adapter.
   \centering
   \input{Fig_OKTXHoc.pstricks}
\end{wrapfigure}

    La figure ci-contre est un cube de \unit{4}{\centi\meter}.

    \begin{enumerate}
    \item
    Quelle est la nature du triangle \(BAE\) ? 
    \item
    Quel est son périmètre ?
    \item
    Quelle est son aire ? (pour les rapides)
    \end{enumerate}
    

}
\vspace{2cm}
\vbox{14
\emph{Toutes les réponses doivent être justifiées par un calcul accompagné d'un raisonnement.}

\begin{wrapfigure}{r}{5.0cm}
   \vspace{-0.5cm}        % à adapter.
   \centering
   \input{Fig_OKTXHoc.pstricks}
\end{wrapfigure}

    La figure ci-contre est un cube de \unit{4}{\centi\meter}.

    \begin{enumerate}
    \item
    Quelle est la nature du triangle \(FCE\) ? 
    \item
    Quel est son périmètre ?
    \item
    Quelle est son aire ? (pour les rapides)
    \end{enumerate}
    

}
\vspace{2cm}
\vbox{15
\emph{Toutes les réponses doivent être justifiées par un calcul accompagné d'un raisonnement.}

\begin{wrapfigure}{r}{5.0cm}
   \vspace{-0.5cm}        % à adapter.
   \centering
   \input{Fig_OKTXHoc.pstricks}
\end{wrapfigure}

    La figure ci-contre est un cube de \unit{2}{\centi\meter}.

    \begin{enumerate}
    \item
    Quelle est la nature du triangle \(GBD\) ? 
    \item
    Quel est son périmètre ?
    \item
    Quelle est son aire ? (pour les rapides)
    \end{enumerate}
    

}
\vspace{2cm}
\vbox{16
\emph{Toutes les réponses doivent être justifiées par un calcul accompagné d'un raisonnement.}

\begin{wrapfigure}{r}{5.0cm}
   \vspace{-0.5cm}        % à adapter.
   \centering
   \input{Fig_OKTXHoc.pstricks}
\end{wrapfigure}

    La figure ci-contre est un cube de \unit{5}{\centi\meter}.

    \begin{enumerate}
    \item
    Quelle est la nature du triangle \(EHA\) ? 
    \item
    Quel est son périmètre ?
    \item
    Quelle est son aire ? (pour les rapides)
    \end{enumerate}
    

}
\vspace{2cm}
\vbox{17
\emph{Toutes les réponses doivent être justifiées par un calcul accompagné d'un raisonnement.}

\begin{wrapfigure}{r}{5.0cm}
   \vspace{-0.5cm}        % à adapter.
   \centering
   \input{Fig_OKTXHoc.pstricks}
\end{wrapfigure}

    La figure ci-contre est un cube de \unit{6}{\centi\meter}.

    \begin{enumerate}
    \item
    Quelle est la nature du triangle \(CFD\) ? 
    \item
    Quel est son périmètre ?
    \item
    Quelle est son aire ? (pour les rapides)
    \end{enumerate}
    

}
\vspace{2cm}
\vbox{18
\emph{Toutes les réponses doivent être justifiées par un calcul accompagné d'un raisonnement.}

\begin{wrapfigure}{r}{5.0cm}
   \vspace{-0.5cm}        % à adapter.
   \centering
   \input{Fig_OKTXHoc.pstricks}
\end{wrapfigure}

    La figure ci-contre est un cube de \unit{6}{\centi\meter}.

    \begin{enumerate}
    \item
    Quelle est la nature du triangle \(CFB\) ? 
    \item
    Quel est son périmètre ?
    \item
    Quelle est son aire ? (pour les rapides)
    \end{enumerate}
    

}
\vspace{2cm}
\vbox{19
\emph{Toutes les réponses doivent être justifiées par un calcul accompagné d'un raisonnement.}

\begin{wrapfigure}{r}{5.0cm}
   \vspace{-0.5cm}        % à adapter.
   \centering
   \input{Fig_OKTXHoc.pstricks}
\end{wrapfigure}

    La figure ci-contre est un cube de \unit{7}{\centi\meter}.

    \begin{enumerate}
    \item
    Quelle est la nature du triangle \(DAC\) ? 
    \item
    Quel est son périmètre ?
    \item
    Quelle est son aire ? (pour les rapides)
    \end{enumerate}
    

}
\vspace{2cm}
\vbox{20
\emph{Toutes les réponses doivent être justifiées par un calcul accompagné d'un raisonnement.}

\begin{wrapfigure}{r}{5.0cm}
   \vspace{-0.5cm}        % à adapter.
   \centering
   \input{Fig_OKTXHoc.pstricks}
\end{wrapfigure}

    La figure ci-contre est un cube de \unit{4}{\centi\meter}.

    \begin{enumerate}
    \item
    Quelle est la nature du triangle \(FEC\) ? 
    \item
    Quel est son périmètre ?
    \item
    Quelle est son aire ? (pour les rapides)
    \end{enumerate}
    

}
\vspace{2cm}
\vbox{21
\emph{Toutes les réponses doivent être justifiées par un calcul accompagné d'un raisonnement.}

\begin{wrapfigure}{r}{5.0cm}
   \vspace{-0.5cm}        % à adapter.
   \centering
   \input{Fig_OKTXHoc.pstricks}
\end{wrapfigure}

    La figure ci-contre est un cube de \unit{3}{\centi\meter}.

    \begin{enumerate}
    \item
    Quelle est la nature du triangle \(GCF\) ? 
    \item
    Quel est son périmètre ?
    \item
    Quelle est son aire ? (pour les rapides)
    \end{enumerate}
    

}
\vspace{2cm}
\vbox{22
\emph{Toutes les réponses doivent être justifiées par un calcul accompagné d'un raisonnement.}

\begin{wrapfigure}{r}{5.0cm}
   \vspace{-0.5cm}        % à adapter.
   \centering
   \input{Fig_OKTXHoc.pstricks}
\end{wrapfigure}

    La figure ci-contre est un cube de \unit{3}{\centi\meter}.

    \begin{enumerate}
    \item
    Quelle est la nature du triangle \(ADG\) ? 
    \item
    Quel est son périmètre ?
    \item
    Quelle est son aire ? (pour les rapides)
    \end{enumerate}
    

}
\vspace{2cm}
\vbox{23
\emph{Toutes les réponses doivent être justifiées par un calcul accompagné d'un raisonnement.}

\begin{wrapfigure}{r}{5.0cm}
   \vspace{-0.5cm}        % à adapter.
   \centering
   \input{Fig_OKTXHoc.pstricks}
\end{wrapfigure}

    La figure ci-contre est un cube de \unit{3}{\centi\meter}.

    \begin{enumerate}
    \item
    Quelle est la nature du triangle \(CHF\) ? 
    \item
    Quel est son périmètre ?
    \item
    Quelle est son aire ? (pour les rapides)
    \end{enumerate}
    

}
\vspace{2cm}
\vbox{24
\emph{Toutes les réponses doivent être justifiées par un calcul accompagné d'un raisonnement.}

\begin{wrapfigure}{r}{5.0cm}
   \vspace{-0.5cm}        % à adapter.
   \centering
   \input{Fig_OKTXHoc.pstricks}
\end{wrapfigure}

    La figure ci-contre est un cube de \unit{7}{\centi\meter}.

    \begin{enumerate}
    \item
    Quelle est la nature du triangle \(DBF\) ? 
    \item
    Quel est son périmètre ?
    \item
    Quelle est son aire ? (pour les rapides)
    \end{enumerate}
    

}
\vspace{2cm}
\vbox{25
\emph{Toutes les réponses doivent être justifiées par un calcul accompagné d'un raisonnement.}

\begin{wrapfigure}{r}{5.0cm}
   \vspace{-0.5cm}        % à adapter.
   \centering
   \input{Fig_OKTXHoc.pstricks}
\end{wrapfigure}

    La figure ci-contre est un cube de \unit{6}{\centi\meter}.

    \begin{enumerate}
    \item
    Quelle est la nature du triangle \(HDC\) ? 
    \item
    Quel est son périmètre ?
    \item
    Quelle est son aire ? (pour les rapides)
    \end{enumerate}
    

}
\vspace{2cm}
\vbox{26
\emph{Toutes les réponses doivent être justifiées par un calcul accompagné d'un raisonnement.}

\begin{wrapfigure}{r}{5.0cm}
   \vspace{-0.5cm}        % à adapter.
   \centering
   \input{Fig_OKTXHoc.pstricks}
\end{wrapfigure}

    La figure ci-contre est un cube de \unit{4}{\centi\meter}.

    \begin{enumerate}
    \item
    Quelle est la nature du triangle \(HCA\) ? 
    \item
    Quel est son périmètre ?
    \item
    Quelle est son aire ? (pour les rapides)
    \end{enumerate}
    

}
\vspace{2cm}
\vbox{27
\emph{Toutes les réponses doivent être justifiées par un calcul accompagné d'un raisonnement.}

\begin{wrapfigure}{r}{5.0cm}
   \vspace{-0.5cm}        % à adapter.
   \centering
   \input{Fig_OKTXHoc.pstricks}
\end{wrapfigure}

    La figure ci-contre est un cube de \unit{6}{\centi\meter}.

    \begin{enumerate}
    \item
    Quelle est la nature du triangle \(BEA\) ? 
    \item
    Quel est son périmètre ?
    \item
    Quelle est son aire ? (pour les rapides)
    \end{enumerate}
    

}
\vspace{2cm}
\vbox{28
\emph{Toutes les réponses doivent être justifiées par un calcul accompagné d'un raisonnement.}

\begin{wrapfigure}{r}{5.0cm}
   \vspace{-0.5cm}        % à adapter.
   \centering
   \input{Fig_OKTXHoc.pstricks}
\end{wrapfigure}

    La figure ci-contre est un cube de \unit{2}{\centi\meter}.

    \begin{enumerate}
    \item
    Quelle est la nature du triangle \(EAD\) ? 
    \item
    Quel est son périmètre ?
    \item
    Quelle est son aire ? (pour les rapides)
    \end{enumerate}
    

}
\vspace{2cm}
\vbox{29
\emph{Toutes les réponses doivent être justifiées par un calcul accompagné d'un raisonnement.}

\begin{wrapfigure}{r}{5.0cm}
   \vspace{-0.5cm}        % à adapter.
   \centering
   \input{Fig_OKTXHoc.pstricks}
\end{wrapfigure}

    La figure ci-contre est un cube de \unit{2}{\centi\meter}.

    \begin{enumerate}
    \item
    Quelle est la nature du triangle \(GFA\) ? 
    \item
    Quel est son périmètre ?
    \item
    Quelle est son aire ? (pour les rapides)
    \end{enumerate}
    

}
\vspace{2cm}
\vbox{30
\emph{Toutes les réponses doivent être justifiées par un calcul accompagné d'un raisonnement.}

\begin{wrapfigure}{r}{5.0cm}
   \vspace{-0.5cm}        % à adapter.
   \centering
   \input{Fig_OKTXHoc.pstricks}
\end{wrapfigure}

    La figure ci-contre est un cube de \unit{4}{\centi\meter}.

    \begin{enumerate}
    \item
    Quelle est la nature du triangle \(GDF\) ? 
    \item
    Quel est son périmètre ?
    \item
    Quelle est son aire ? (pour les rapides)
    \end{enumerate}
    

}
\vspace{2cm}
\vbox{31
\emph{Toutes les réponses doivent être justifiées par un calcul accompagné d'un raisonnement.}

\begin{wrapfigure}{r}{5.0cm}
   \vspace{-0.5cm}        % à adapter.
   \centering
   \input{Fig_OKTXHoc.pstricks}
\end{wrapfigure}

    La figure ci-contre est un cube de \unit{4}{\centi\meter}.

    \begin{enumerate}
    \item
    Quelle est la nature du triangle \(BDE\) ? 
    \item
    Quel est son périmètre ?
    \item
    Quelle est son aire ? (pour les rapides)
    \end{enumerate}
    

}
\vspace{2cm}
\vbox{32
\emph{Toutes les réponses doivent être justifiées par un calcul accompagné d'un raisonnement.}

\begin{wrapfigure}{r}{5.0cm}
   \vspace{-0.5cm}        % à adapter.
   \centering
   \input{Fig_OKTXHoc.pstricks}
\end{wrapfigure}

    La figure ci-contre est un cube de \unit{5}{\centi\meter}.

    \begin{enumerate}
    \item
    Quelle est la nature du triangle \(CDB\) ? 
    \item
    Quel est son périmètre ?
    \item
    Quelle est son aire ? (pour les rapides)
    \end{enumerate}
    

}
\vspace{2cm}
\vbox{33
\emph{Toutes les réponses doivent être justifiées par un calcul accompagné d'un raisonnement.}

\begin{wrapfigure}{r}{5.0cm}
   \vspace{-0.5cm}        % à adapter.
   \centering
   \input{Fig_OKTXHoc.pstricks}
\end{wrapfigure}

    La figure ci-contre est un cube de \unit{3}{\centi\meter}.

    \begin{enumerate}
    \item
    Quelle est la nature du triangle \(HBF\) ? 
    \item
    Quel est son périmètre ?
    \item
    Quelle est son aire ? (pour les rapides)
    \end{enumerate}
    

}
\vspace{2cm}
\vbox{34
\emph{Toutes les réponses doivent être justifiées par un calcul accompagné d'un raisonnement.}

\begin{wrapfigure}{r}{5.0cm}
   \vspace{-0.5cm}        % à adapter.
   \centering
   \input{Fig_OKTXHoc.pstricks}
\end{wrapfigure}

    La figure ci-contre est un cube de \unit{7}{\centi\meter}.

    \begin{enumerate}
    \item
    Quelle est la nature du triangle \(BAG\) ? 
    \item
    Quel est son périmètre ?
    \item
    Quelle est son aire ? (pour les rapides)
    \end{enumerate}
    

}
\vspace{2cm}
\vbox{35
\emph{Toutes les réponses doivent être justifiées par un calcul accompagné d'un raisonnement.}

\begin{wrapfigure}{r}{5.0cm}
   \vspace{-0.5cm}        % à adapter.
   \centering
   \input{Fig_OKTXHoc.pstricks}
\end{wrapfigure}

    La figure ci-contre est un cube de \unit{2}{\centi\meter}.

    \begin{enumerate}
    \item
    Quelle est la nature du triangle \(BHE\) ? 
    \item
    Quel est son périmètre ?
    \item
    Quelle est son aire ? (pour les rapides)
    \end{enumerate}
    

}
\vspace{2cm}
\vbox{36
\emph{Toutes les réponses doivent être justifiées par un calcul accompagné d'un raisonnement.}

\begin{wrapfigure}{r}{5.0cm}
   \vspace{-0.5cm}        % à adapter.
   \centering
   \input{Fig_OKTXHoc.pstricks}
\end{wrapfigure}

    La figure ci-contre est un cube de \unit{7}{\centi\meter}.

    \begin{enumerate}
    \item
    Quelle est la nature du triangle \(CEG\) ? 
    \item
    Quel est son périmètre ?
    \item
    Quelle est son aire ? (pour les rapides)
    \end{enumerate}
    

}
\vspace{2cm}
\vbox{37
\emph{Toutes les réponses doivent être justifiées par un calcul accompagné d'un raisonnement.}

\begin{wrapfigure}{r}{5.0cm}
   \vspace{-0.5cm}        % à adapter.
   \centering
   \input{Fig_OKTXHoc.pstricks}
\end{wrapfigure}

    La figure ci-contre est un cube de \unit{3}{\centi\meter}.

    \begin{enumerate}
    \item
    Quelle est la nature du triangle \(ADG\) ? 
    \item
    Quel est son périmètre ?
    \item
    Quelle est son aire ? (pour les rapides)
    \end{enumerate}
    

}
\vspace{2cm}
\vbox{38
\emph{Toutes les réponses doivent être justifiées par un calcul accompagné d'un raisonnement.}

\begin{wrapfigure}{r}{5.0cm}
   \vspace{-0.5cm}        % à adapter.
   \centering
   \input{Fig_OKTXHoc.pstricks}
\end{wrapfigure}

    La figure ci-contre est un cube de \unit{6}{\centi\meter}.

    \begin{enumerate}
    \item
    Quelle est la nature du triangle \(GHD\) ? 
    \item
    Quel est son périmètre ?
    \item
    Quelle est son aire ? (pour les rapides)
    \end{enumerate}
    

}
\vspace{2cm}
\vbox{39
\emph{Toutes les réponses doivent être justifiées par un calcul accompagné d'un raisonnement.}

\begin{wrapfigure}{r}{5.0cm}
   \vspace{-0.5cm}        % à adapter.
   \centering
   \input{Fig_OKTXHoc.pstricks}
\end{wrapfigure}

    La figure ci-contre est un cube de \unit{6}{\centi\meter}.

    \begin{enumerate}
    \item
    Quelle est la nature du triangle \(BCE\) ? 
    \item
    Quel est son périmètre ?
    \item
    Quelle est son aire ? (pour les rapides)
    \end{enumerate}
    

}
\vspace{2cm}

\end{document}

% LES EXOS POUR CEUX QUI ONT TERMINÉ

\title{Des questions en plus}
\date{}
\maketitle


\Exo{smath-0541}
\Exo{smath-0545}

\vspace{2cm}

\Exo{smath-0543}
\Exo{smath-0544}
\Exo{smath-0542}

\end{document}


% LES TICES

\title{Quelque questions d'algorithmique}
\date{}
\maketitle

\Exo{smath-0539}
\Exo{smath-0540}

\end{document}


------------------------------------



% UN PEU D'ALGO


    \Exo{smath-0526}    %algo
    \Exo{smath-0528}    %algo
    \Exo{smath-0527}    %algo


\end{document}

% ACTIVITÉ STATISTIQUE DESCRIPTIVE


% This is part of Un soupçon de mathématique sans être agressif pour autant
% Copyright (c) 2013
%   Laurent Claessens
% See the file fdl-1.3.txt for copying conditions.

% ATTENTION : les chiffres donnés ici sont repris dans le cours au moment des ECC.
% ATTENTION : il y a trop de 4 dans les valeurs, les effectifs et les ECC. Il faut changer.
%               Il faut changer au moins dans le texte ici, dans la figure et dans le texte des éléments de réponse.

\begin{wrapfigure}[5]{r}{8cm}
   \vspace{-0.5cm}        % à adapter.
   \centering
   \input{Fig_YRQOoPE.pstricks}
\end{wrapfigure}

Le graphique ci-contre illustre le nombre de spam reçus aujourd'hui par les élèves d'une classe.
\begin{enumerate}
    \item
        Combien d'élèves y-a-t-il dans la classe ?
    \item
        Combien d'élèves ont reçu \( 5\) spams ou plus ?
    \item
        En moyenne combien de spam ont reçu les élèves aujourd'hui ?
    \item
        Diviser la classe en 4 groupes suivant le nombre de spams reçus.
\end{enumerate}



\vspace{5cm}

% This is part of Un soupçon de mathématique sans être agressif pour autant
% Copyright (c) 2013
%   Laurent Claessens
% See the file fdl-1.3.txt for copying conditions.

% ATTENTION : les chiffres donnés ici sont repris dans le cours au moment des ECC.
% ATTENTION : il y a trop de 4 dans les valeurs, les effectifs et les ECC. Il faut changer.
%               Il faut changer au moins dans le texte ici, dans la figure et dans le texte des éléments de réponse.

\begin{wrapfigure}[5]{r}{8cm}
   \vspace{-0.5cm}        % à adapter.
   \centering
   \input{Fig_YRQOoPE.pstricks}
\end{wrapfigure}

Le graphique ci-contre illustre le nombre de spam reçus aujourd'hui par les élèves d'une classe.
\begin{enumerate}
    \item
        Combien d'élèves y-a-t-il dans la classe ?
    \item
        Combien d'élèves ont reçu \( 5\) spams ou plus ?
    \item
        En moyenne combien de spam ont reçu les élèves aujourd'hui ?
    \item
        Diviser la classe en 4 groupes suivant le nombre de spams reçus.
\end{enumerate}



\vspace{5cm}

% This is part of Un soupçon de mathématique sans être agressif pour autant
% Copyright (c) 2013
%   Laurent Claessens
% See the file fdl-1.3.txt for copying conditions.

% ATTENTION : les chiffres donnés ici sont repris dans le cours au moment des ECC.
% ATTENTION : il y a trop de 4 dans les valeurs, les effectifs et les ECC. Il faut changer.
%               Il faut changer au moins dans le texte ici, dans la figure et dans le texte des éléments de réponse.

\begin{wrapfigure}[5]{r}{8cm}
   \vspace{-0.5cm}        % à adapter.
   \centering
   \input{Fig_YRQOoPE.pstricks}
\end{wrapfigure}

Le graphique ci-contre illustre le nombre de spam reçus aujourd'hui par les élèves d'une classe.
\begin{enumerate}
    \item
        Combien d'élèves y-a-t-il dans la classe ?
    \item
        Combien d'élèves ont reçu \( 5\) spams ou plus ?
    \item
        En moyenne combien de spam ont reçu les élèves aujourd'hui ?
    \item
        Diviser la classe en 4 groupes suivant le nombre de spams reçus.
\end{enumerate}



\end{document}

% ACTIVITÉ GÉOMÉTRIE DANS L'ESPACE
% This is part of Un soupçon de mathématique sans être agressif pour autant
% Copyright (c) 2013
%   Laurent Claessens
% See the file fdl-1.3.txt for copying conditions.

\begin{wrapfigure}[3]{r}{5.0cm}
   \vspace{-0.5cm}        % à adapter.
   \centering
   \input{Fig_MEzTDZC.pstricks}
\end{wrapfigure}

À l'aide du cube ci-contre :
\begin{enumerate}
    \item
        Quelle est la nature du triangle \( AEF\) ? 
    \item
        Quelle est la nature du quadrilatère \( ABGH\) ?
    \item
        Est-ce que vous pouvez trouver trois points de ce cube formant un triangle équilatéral ?
    \item
        Si ce cube fait \unit{5}{\meter} de côté, quelle est la longueur de la «grande» diagonale \( [AG]\) ?
\end{enumerate}



\vspace{2cm}

% This is part of Un soupçon de mathématique sans être agressif pour autant
% Copyright (c) 2013
%   Laurent Claessens
% See the file fdl-1.3.txt for copying conditions.

\begin{wrapfigure}[3]{r}{5.0cm}
   \vspace{-0.5cm}        % à adapter.
   \centering
   \input{Fig_MEzTDZC.pstricks}
\end{wrapfigure}

À l'aide du cube ci-contre :
\begin{enumerate}
    \item
        Quelle est la nature du triangle \( AEF\) ? 
    \item
        Quelle est la nature du quadrilatère \( ABGH\) ?
    \item
        Est-ce que vous pouvez trouver trois points de ce cube formant un triangle équilatéral ?
    \item
        Si ce cube fait \unit{5}{\meter} de côté, quelle est la longueur de la «grande» diagonale \( [AG]\) ?
\end{enumerate}



\vspace{2cm}

% This is part of Un soupçon de mathématique sans être agressif pour autant
% Copyright (c) 2013
%   Laurent Claessens
% See the file fdl-1.3.txt for copying conditions.

\begin{wrapfigure}[3]{r}{5.0cm}
   \vspace{-0.5cm}        % à adapter.
   \centering
   \input{Fig_MEzTDZC.pstricks}
\end{wrapfigure}

À l'aide du cube ci-contre :
\begin{enumerate}
    \item
        Quelle est la nature du triangle \( AEF\) ? 
    \item
        Quelle est la nature du quadrilatère \( ABGH\) ?
    \item
        Est-ce que vous pouvez trouver trois points de ce cube formant un triangle équilatéral ?
    \item
        Si ce cube fait \unit{5}{\meter} de côté, quelle est la longueur de la «grande» diagonale \( [AG]\) ?
\end{enumerate}




\vspace{2cm}

% This is part of Un soupçon de mathématique sans être agressif pour autant
% Copyright (c) 2013
%   Laurent Claessens
% See the file fdl-1.3.txt for copying conditions.

\begin{wrapfigure}[3]{r}{5.0cm}
   \vspace{-0.5cm}        % à adapter.
   \centering
   \input{Fig_MEzTDZC.pstricks}
\end{wrapfigure}

À l'aide du cube ci-contre :
\begin{enumerate}
    \item
        Quelle est la nature du triangle \( AEF\) ? 
    \item
        Quelle est la nature du quadrilatère \( ABGH\) ?
    \item
        Est-ce que vous pouvez trouver trois points de ce cube formant un triangle équilatéral ?
    \item
        Si ce cube fait \unit{5}{\meter} de côté, quelle est la longueur de la «grande» diagonale \( [AG]\) ?
\end{enumerate}



\vspace{2cm}

% This is part of Un soupçon de mathématique sans être agressif pour autant
% Copyright (c) 2013
%   Laurent Claessens
% See the file fdl-1.3.txt for copying conditions.

\begin{wrapfigure}[3]{r}{5.0cm}
   \vspace{-0.5cm}        % à adapter.
   \centering
   \input{Fig_MEzTDZC.pstricks}
\end{wrapfigure}

À l'aide du cube ci-contre :
\begin{enumerate}
    \item
        Quelle est la nature du triangle \( AEF\) ? 
    \item
        Quelle est la nature du quadrilatère \( ABGH\) ?
    \item
        Est-ce que vous pouvez trouver trois points de ce cube formant un triangle équilatéral ?
    \item
        Si ce cube fait \unit{5}{\meter} de côté, quelle est la longueur de la «grande» diagonale \( [AG]\) ?
\end{enumerate}




\end{document}

%DS MARDI 5 NOVEMBRE 2013, seconde D
\begin{feuilleDS}{Seconde D, DS numéro 2\\ \small mardi 5 novembre 2013}

\Exo{smath-0516}
\Exo{smath-0474}
\Exo{smath-0521}
\Exo{smath-0493}
\Exo{smath-0522}

\end{feuilleDS}


%DS MARDI 12 NOVEMBRE 2013, seconde A
\begin{feuilleDS}{Seconde A, DS numéro 2\\ \small mardi 12 novembre 2013}

\Exo{smath-0515}
\Exo{smath-0517}
\Exo{smath-0518}
\Exo{smath-0519}
\Exo{smath-0520}

\end{feuilleDS}

\end{document}

%DS MARDI 1 OCTOBRE 2013, seconde A

\begin{feuilleDS}{Seconde A, DS numéro 1\\ \small mardi 1 octobre 2013}

\Exo{Seconde-0047}
\Exo{smath-0508}
\Exo{smath-0512}
\Exo{smath-0477}
\Exo{smath-0507}

\end{feuilleDS}

\newpage

\setcounter{CountExercice}{0}

%DS MARDI 1 OCTOBRE 2013, seconde D
\begin{feuilleDS}{Seconde D, DS numéro 1\\ \small mardi 1 octobre 2013}


\Exo{smath-0510}
\Exo{smath-0509}
\Exo{smath-0508}
\Exo{smath-0507}
\Exo{smath-0511}

\end{feuilleDS}

\end{document}



%INTERRO REPERE, DISTANCE, MILIEU et sa correction.
\vbox{1
\emph{Toutes les réponses doivent être justifiées par un calcul accompagné d'un raisonnement.}
\begin{enumerate}\item
Placer les points \( A=(6;-1)\), \( B=(7;2)\), \( C=(1;0)\) et \( D=(-10;10)\) dans un repère orthonormé. Calculer la longueur du segment \( [AB]\) et les coordonnées du milieu du segment \( [CD]\).
\item
Est-ce que le triangle formé par les points \( A(8;8)\), \( B(9;7)\) et \( C(17;7)\) est rectangle ?

\end{enumerate}
} 
\vspace{2cm}
\vbox{2
\emph{Toutes les réponses doivent être justifiées par un calcul accompagné d'un raisonnement.}
\begin{enumerate}\item
Placer les points \( A=(9;-5)\), \( B=(9;7)\), \( C=(-9;4)\) et \( D=(1;-3)\) dans un repère orthonormé. Calculer la longueur du segment \( [AB]\) et les coordonnées du milieu du segment \( [CD]\).
\item
Est-ce que le triangle formé par les points \( A(-8;7)\), \( B(-2;1)\) et \( C(-9;0)\) est isocèle ?

\end{enumerate}
} 
\vspace{2cm}
\vbox{3
\emph{Toutes les réponses doivent être justifiées par un calcul accompagné d'un raisonnement.}
\begin{enumerate}\item
Placer les points \( A=(-5;-9)\), \( B=(-3;-1)\), \( C=(8;5)\) et \( D=(-1;-8)\) dans un repère orthonormé. Calculer la longueur du segment \( [AB]\) et les coordonnées du milieu du segment \( [CD]\).
\item
Est-ce que le triangle formé par les points \( A(-7;-3)\), \( B(-3;-7)\) et \( C(1;-5)\) est rectangle ?

\end{enumerate}
} 
\vspace{2cm}
\vbox{4
\emph{Toutes les réponses doivent être justifiées par un calcul accompagné d'un raisonnement.}
\begin{enumerate}\item
Placer les points \( A=(-10;7)\), \( B=(4;3)\), \( C=(0;3)\) et \( D=(10;-1)\) dans un repère orthonormé. Calculer la longueur du segment \( [AB]\) et les coordonnées du milieu du segment \( [CD]\).
\item
Est-ce que le triangle formé par les points \( A(-8;-5)\), \( B(-3;-10)\) et \( C(3;-4)\) est rectangle ?

\end{enumerate}
} 
\vspace{2cm}
\vbox{5
\emph{Toutes les réponses doivent être justifiées par un calcul accompagné d'un raisonnement.}
\begin{enumerate}\item
Placer les points \( A=(-4;2)\), \( B=(4;7)\), \( C=(9;8)\) et \( D=(9;0)\) dans un repère orthonormé. Calculer la longueur du segment \( [AB]\) et les coordonnées du milieu du segment \( [CD]\).
\item
Est-ce que le triangle formé par les points \( A(0;-6)\), \( B(2;-8)\) et \( C(1;-7)\) est isocèle ?

\end{enumerate}
} 
\vspace{2cm}
\vbox{6
\emph{Toutes les réponses doivent être justifiées par un calcul accompagné d'un raisonnement.}
\begin{enumerate}\item
Placer les points \( A=(-10;7)\), \( B=(6;-2)\), \( C=(9;-3)\) et \( D=(7;-3)\) dans un repère orthonormé. Calculer la longueur du segment \( [AB]\) et les coordonnées du milieu du segment \( [CD]\).
\item
Est-ce que le triangle formé par les points \( A(0;3)\), \( B(4;-1)\) et \( C(13;2)\) est isocèle ?

\end{enumerate}
} 
\vspace{2cm}
\vbox{7
\emph{Toutes les réponses doivent être justifiées par un calcul accompagné d'un raisonnement.}
\begin{enumerate}\item
Placer les points \( A=(-4;1)\), \( B=(4;-6)\), \( C=(-9;-3)\) et \( D=(5;-4)\) dans un repère orthonormé. Calculer la longueur du segment \( [AB]\) et les coordonnées du milieu du segment \( [CD]\).
\item
Est-ce que le triangle formé par les points \( A(-3;-10)\), \( B(-7;-6)\) et \( C(10;-7)\) est rectangle ?

\end{enumerate}
} 
\vspace{2cm}
\vbox{8
\emph{Toutes les réponses doivent être justifiées par un calcul accompagné d'un raisonnement.}
\begin{enumerate}\item
Placer les points \( A=(9;1)\), \( B=(-6;-8)\), \( C=(-1;-7)\) et \( D=(-5;-1)\) dans un repère orthonormé. Calculer la longueur du segment \( [AB]\) et les coordonnées du milieu du segment \( [CD]\).
\item
Est-ce que le triangle formé par les points \( A(1;6)\), \( B(-5;12)\) et \( C(9;10)\) est isocèle ?

\end{enumerate}
} 
\vspace{2cm}
\vbox{9
\emph{Toutes les réponses doivent être justifiées par un calcul accompagné d'un raisonnement.}
\begin{enumerate}\item
Placer les points \( A=(-6;-4)\), \( B=(6;10)\), \( C=(0;9)\) et \( D=(-6;8)\) dans un repère orthonormé. Calculer la longueur du segment \( [AB]\) et les coordonnées du milieu du segment \( [CD]\).
\item
Est-ce que le triangle formé par les points \( A(1;3)\), \( B(3;1)\) et \( C(8;0)\) est rectangle ?

\end{enumerate}
} 
\vspace{2cm}
\vbox{10
\emph{Toutes les réponses doivent être justifiées par un calcul accompagné d'un raisonnement.}
\begin{enumerate}\item
Placer les points \( A=(10;-5)\), \( B=(-1;-4)\), \( C=(6;2)\) et \( D=(6;10)\) dans un repère orthonormé. Calculer la longueur du segment \( [AB]\) et les coordonnées du milieu du segment \( [CD]\).
\item
Est-ce que le triangle formé par les points \( A(-10;-10)\), \( B(-7;-13)\) et \( C(0;-10)\) est rectangle ?

\end{enumerate}
} 
\vspace{2cm}
\vbox{11
\emph{Toutes les réponses doivent être justifiées par un calcul accompagné d'un raisonnement.}
\begin{enumerate}\item
Placer les points \( A=(-3;-5)\), \( B=(-6;-2)\), \( C=(-7;-10)\) et \( D=(-8;0)\) dans un repère orthonormé. Calculer la longueur du segment \( [AB]\) et les coordonnées du milieu du segment \( [CD]\).
\item
Est-ce que le triangle formé par les points \( A(6;-5)\), \( B(6;-5)\) et \( C(13;-8)\) est rectangle ?

\end{enumerate}
} 
\vspace{2cm}
\vbox{12
\emph{Toutes les réponses doivent être justifiées par un calcul accompagné d'un raisonnement.}
\begin{enumerate}\item
Placer les points \( A=(2;-1)\), \( B=(-8;3)\), \( C=(5;-8)\) et \( D=(6;6)\) dans un repère orthonormé. Calculer la longueur du segment \( [AB]\) et les coordonnées du milieu du segment \( [CD]\).
\item
Est-ce que le triangle formé par les points \( A(-9;3)\), \( B(-7;1)\) et \( C(-11;1)\) est rectangle ?

\end{enumerate}
} 
\vspace{2cm}
\vbox{13
\emph{Toutes les réponses doivent être justifiées par un calcul accompagné d'un raisonnement.}
\begin{enumerate}\item
Placer les points \( A=(7;2)\), \( B=(-8;-7)\), \( C=(-6;-2)\) et \( D=(-8;-8)\) dans un repère orthonormé. Calculer la longueur du segment \( [AB]\) et les coordonnées du milieu du segment \( [CD]\).
\item
Est-ce que le triangle formé par les points \( A(-6;-1)\), \( B(-8;1)\) et \( C(4;1)\) est isocèle ?

\end{enumerate}
} 
\vspace{2cm}
\vbox{14
\emph{Toutes les réponses doivent être justifiées par un calcul accompagné d'un raisonnement.}
\begin{enumerate}\item
Placer les points \( A=(-2;-5)\), \( B=(6;9)\), \( C=(8;-9)\) et \( D=(-10;4)\) dans un repère orthonormé. Calculer la longueur du segment \( [AB]\) et les coordonnées du milieu du segment \( [CD]\).
\item
Est-ce que le triangle formé par les points \( A(-10;7)\), \( B(-6;3)\) et \( C(4;7)\) est isocèle ?

\end{enumerate}
} 
\vspace{2cm}
\vbox{15
\emph{Toutes les réponses doivent être justifiées par un calcul accompagné d'un raisonnement.}
\begin{enumerate}\item
Placer les points \( A=(8;-4)\), \( B=(-7;-5)\), \( C=(-2;-10)\) et \( D=(3;-9)\) dans un repère orthonormé. Calculer la longueur du segment \( [AB]\) et les coordonnées du milieu du segment \( [CD]\).
\item
Est-ce que le triangle formé par les points \( A(9;-10)\), \( B(5;-6)\) et \( C(12;-7)\) est rectangle ?

\end{enumerate}
} 
\vspace{2cm}
\vbox{16
\emph{Toutes les réponses doivent être justifiées par un calcul accompagné d'un raisonnement.}
\begin{enumerate}\item
Placer les points \( A=(8;-3)\), \( B=(9;-1)\), \( C=(2;0)\) et \( D=(-3;-7)\) dans un repère orthonormé. Calculer la longueur du segment \( [AB]\) et les coordonnées du milieu du segment \( [CD]\).
\item
Est-ce que le triangle formé par les points \( A(-3;-2)\), \( B(-1;-4)\) et \( C(1;0)\) est isocèle ?

\end{enumerate}
} 
\vspace{2cm}
\vbox{17
\emph{Toutes les réponses doivent être justifiées par un calcul accompagné d'un raisonnement.}
\begin{enumerate}\item
Placer les points \( A=(-7;3)\), \( B=(0;-1)\), \( C=(-6;-8)\) et \( D=(5;-4)\) dans un repère orthonormé. Calculer la longueur du segment \( [AB]\) et les coordonnées du milieu du segment \( [CD]\).
\item
Est-ce que le triangle formé par les points \( A(-4;8)\), \( B(0;4)\) et \( C(-4;8)\) est rectangle ?

\end{enumerate}
} 
\vspace{2cm}
\vbox{18
\emph{Toutes les réponses doivent être justifiées par un calcul accompagné d'un raisonnement.}
\begin{enumerate}\item
Placer les points \( A=(1;-3)\), \( B=(5;-9)\), \( C=(-2;-5)\) et \( D=(-6;-6)\) dans un repère orthonormé. Calculer la longueur du segment \( [AB]\) et les coordonnées du milieu du segment \( [CD]\).
\item
Est-ce que le triangle formé par les points \( A(4;-9)\), \( B(10;-15)\) et \( C(19;-10)\) est isocèle ?

\end{enumerate}
} 
\vspace{2cm}
\vbox{19
\emph{Toutes les réponses doivent être justifiées par un calcul accompagné d'un raisonnement.}
\begin{enumerate}\item
Placer les points \( A=(-3;-7)\), \( B=(-6;-5)\), \( C=(4;3)\) et \( D=(3;-10)\) dans un repère orthonormé. Calculer la longueur du segment \( [AB]\) et les coordonnées du milieu du segment \( [CD]\).
\item
Est-ce que le triangle formé par les points \( A(8;0)\), \( B(2;6)\) et \( C(19;7)\) est isocèle ?

\end{enumerate}
} 
\vspace{2cm}
\vbox{20
\emph{Toutes les réponses doivent être justifiées par un calcul accompagné d'un raisonnement.}
\begin{enumerate}\item
Placer les points \( A=(10;5)\), \( B=(2;0)\), \( C=(-10;-5)\) et \( D=(-1;2)\) dans un repère orthonormé. Calculer la longueur du segment \( [AB]\) et les coordonnées du milieu du segment \( [CD]\).
\item
Est-ce que le triangle formé par les points \( A(2;5)\), \( B(0;7)\) et \( C(-3;2)\) est isocèle ?

\end{enumerate}
} 
\vspace{2cm}
\vbox{21
\emph{Toutes les réponses doivent être justifiées par un calcul accompagné d'un raisonnement.}
\begin{enumerate}\item
Placer les points \( A=(4;9)\), \( B=(-5;2)\), \( C=(0;6)\) et \( D=(-6;-5)\) dans un repère orthonormé. Calculer la longueur du segment \( [AB]\) et les coordonnées du milieu du segment \( [CD]\).
\item
Est-ce que le triangle formé par les points \( A(-10;6)\), \( B(-15;11)\) et \( C(-3;3)\) est rectangle ?

\end{enumerate}
} 
\vspace{2cm}
\vbox{22
\emph{Toutes les réponses doivent être justifiées par un calcul accompagné d'un raisonnement.}
\begin{enumerate}\item
Placer les points \( A=(1;3)\), \( B=(6;0)\), \( C=(-10;-6)\) et \( D=(9;-10)\) dans un repère orthonormé. Calculer la longueur du segment \( [AB]\) et les coordonnées du milieu du segment \( [CD]\).
\item
Est-ce que le triangle formé par les points \( A(4;6)\), \( B(7;3)\) et \( C(3;5)\) est rectangle ?

\end{enumerate}
} 
\vspace{2cm}
\vbox{23
\emph{Toutes les réponses doivent être justifiées par un calcul accompagné d'un raisonnement.}
\begin{enumerate}\item
Placer les points \( A=(-6;5)\), \( B=(7;-4)\), \( C=(6;-2)\) et \( D=(7;-10)\) dans un repère orthonormé. Calculer la longueur du segment \( [AB]\) et les coordonnées du milieu du segment \( [CD]\).
\item
Est-ce que le triangle formé par les points \( A(-7;-1)\), \( B(-12;4)\) et \( C(5;1)\) est rectangle ?

\end{enumerate}
} 
\vspace{2cm}
\vbox{24
\emph{Toutes les réponses doivent être justifiées par un calcul accompagné d'un raisonnement.}
\begin{enumerate}\item
Placer les points \( A=(9;-7)\), \( B=(-3;-6)\), \( C=(-10;7)\) et \( D=(-7;9)\) dans un repère orthonormé. Calculer la longueur du segment \( [AB]\) et les coordonnées du milieu du segment \( [CD]\).
\item
Est-ce que le triangle formé par les points \( A(10;7)\), \( B(8;9)\) et \( C(10;9)\) est isocèle ?

\end{enumerate}
} 
\vspace{2cm}
\vbox{25
\emph{Toutes les réponses doivent être justifiées par un calcul accompagné d'un raisonnement.}
\begin{enumerate}\item
Placer les points \( A=(-4;-9)\), \( B=(7;1)\), \( C=(9;-9)\) et \( D=(1;3)\) dans un repère orthonormé. Calculer la longueur du segment \( [AB]\) et les coordonnées du milieu du segment \( [CD]\).
\item
Est-ce que le triangle formé par les points \( A(-2;10)\), \( B(4;4)\) et \( C(4;10)\) est isocèle ?

\end{enumerate}
} 
\vspace{2cm}
\vbox{26
\emph{Toutes les réponses doivent être justifiées par un calcul accompagné d'un raisonnement.}
\begin{enumerate}\item
Placer les points \( A=(4;-1)\), \( B=(6;5)\), \( C=(3;10)\) et \( D=(5;8)\) dans un repère orthonormé. Calculer la longueur du segment \( [AB]\) et les coordonnées du milieu du segment \( [CD]\).
\item
Est-ce que le triangle formé par les points \( A(-5;-8)\), \( B(-9;-4)\) et \( C(-1;-10)\) est isocèle ?

\end{enumerate}
} 
\vspace{2cm}
\vbox{27
\emph{Toutes les réponses doivent être justifiées par un calcul accompagné d'un raisonnement.}
\begin{enumerate}\item
Placer les points \( A=(-4;5)\), \( B=(10;-10)\), \( C=(-8;6)\) et \( D=(6;-10)\) dans un repère orthonormé. Calculer la longueur du segment \( [AB]\) et les coordonnées du milieu du segment \( [CD]\).
\item
Est-ce que le triangle formé par les points \( A(3;4)\), \( B(-1;8)\) et \( C(8;3)\) est isocèle ?

\end{enumerate}
} 
\vspace{2cm}
\vbox{28
\emph{Toutes les réponses doivent être justifiées par un calcul accompagné d'un raisonnement.}
\begin{enumerate}\item
Placer les points \( A=(6;-4)\), \( B=(6;9)\), \( C=(8;9)\) et \( D=(8;-7)\) dans un repère orthonormé. Calculer la longueur du segment \( [AB]\) et les coordonnées du milieu du segment \( [CD]\).
\item
Est-ce que le triangle formé par les points \( A(-6;10)\), \( B(-4;8)\) et \( C(-6;8)\) est isocèle ?

\end{enumerate}
} 
\vspace{2cm}
\vbox{29
\emph{Toutes les réponses doivent être justifiées par un calcul accompagné d'un raisonnement.}
\begin{enumerate}\item
Placer les points \( A=(0;0)\), \( B=(5;10)\), \( C=(9;-10)\) et \( D=(-4;-8)\) dans un repère orthonormé. Calculer la longueur du segment \( [AB]\) et les coordonnées du milieu du segment \( [CD]\).
\item
Est-ce que le triangle formé par les points \( A(0;-10)\), \( B(0;-10)\) et \( C(-3;-13)\) est isocèle ?

\end{enumerate}
} 
\vspace{2cm}
\vbox{30
\emph{Toutes les réponses doivent être justifiées par un calcul accompagné d'un raisonnement.}
\begin{enumerate}\item
Placer les points \( A=(-1;6)\), \( B=(0;10)\), \( C=(0;-8)\) et \( D=(1;-1)\) dans un repère orthonormé. Calculer la longueur du segment \( [AB]\) et les coordonnées du milieu du segment \( [CD]\).
\item
Est-ce que le triangle formé par les points \( A(-6;1)\), \( B(-2;-3)\) et \( C(-2;1)\) est isocèle ?

\end{enumerate}
} 
\vspace{2cm}
\vbox{31
\emph{Toutes les réponses doivent être justifiées par un calcul accompagné d'un raisonnement.}
\begin{enumerate}\item
Placer les points \( A=(6;-3)\), \( B=(-5;-5)\), \( C=(-3;4)\) et \( D=(8;6)\) dans un repère orthonormé. Calculer la longueur du segment \( [AB]\) et les coordonnées du milieu du segment \( [CD]\).
\item
Est-ce que le triangle formé par les points \( A(-5;-1)\), \( B(-7;1)\) et \( C(0;-4)\) est isocèle ?

\end{enumerate}
} 
\vspace{2cm}
\vbox{32
\emph{Toutes les réponses doivent être justifiées par un calcul accompagné d'un raisonnement.}
\begin{enumerate}\item
Placer les points \( A=(-8;4)\), \( B=(1;-9)\), \( C=(-8;2)\) et \( D=(7;5)\) dans un repère orthonormé. Calculer la longueur du segment \( [AB]\) et les coordonnées du milieu du segment \( [CD]\).
\item
Est-ce que le triangle formé par les points \( A(6;1)\), \( B(11;-4)\) et \( C(19;4)\) est rectangle ?

\end{enumerate}
} 
\vspace{2cm}
\vbox{33
\emph{Toutes les réponses doivent être justifiées par un calcul accompagné d'un raisonnement.}
\begin{enumerate}\item
Placer les points \( A=(-10;4)\), \( B=(0;-8)\), \( C=(5;-2)\) et \( D=(-6;-10)\) dans un repère orthonormé. Calculer la longueur du segment \( [AB]\) et les coordonnées du milieu du segment \( [CD]\).
\item
Est-ce que le triangle formé par les points \( A(-1;7)\), \( B(-2;8)\) et \( C(0;8)\) est rectangle ?

\end{enumerate}
} 
\vspace{2cm}
\vbox{34
\emph{Toutes les réponses doivent être justifiées par un calcul accompagné d'un raisonnement.}
\begin{enumerate}\item
Placer les points \( A=(-2;5)\), \( B=(-4;-4)\), \( C=(1;10)\) et \( D=(-5;-6)\) dans un repère orthonormé. Calculer la longueur du segment \( [AB]\) et les coordonnées du milieu du segment \( [CD]\).
\item
Est-ce que le triangle formé par les points \( A(-1;-7)\), \( B(1;-9)\) et \( C(2;-6)\) est isocèle ?

\end{enumerate}
} 
\vspace{2cm}
\vbox{35
\emph{Toutes les réponses doivent être justifiées par un calcul accompagné d'un raisonnement.}
\begin{enumerate}\item
Placer les points \( A=(3;-1)\), \( B=(8;-8)\), \( C=(-3;2)\) et \( D=(-6;-2)\) dans un repère orthonormé. Calculer la longueur du segment \( [AB]\) et les coordonnées du milieu du segment \( [CD]\).
\item
Est-ce que le triangle formé par les points \( A(5;6)\), \( B(3;8)\) et \( C(15;8)\) est isocèle ?

\end{enumerate}
} 
\vspace{2cm}
\vbox{36
\emph{Toutes les réponses doivent être justifiées par un calcul accompagné d'un raisonnement.}
\begin{enumerate}\item
Placer les points \( A=(1;-6)\), \( B=(1;-9)\), \( C=(7;8)\) et \( D=(4;5)\) dans un repère orthonormé. Calculer la longueur du segment \( [AB]\) et les coordonnées du milieu du segment \( [CD]\).
\item
Est-ce que le triangle formé par les points \( A(-7;10)\), \( B(-3;6)\) et \( C(2;5)\) est isocèle ?

\end{enumerate}
} 
\vspace{2cm}
\vbox{37
\emph{Toutes les réponses doivent être justifiées par un calcul accompagné d'un raisonnement.}
\begin{enumerate}\item
Placer les points \( A=(-7;-2)\), \( B=(-5;-10)\), \( C=(-5;10)\) et \( D=(-4;-5)\) dans un repère orthonormé. Calculer la longueur du segment \( [AB]\) et les coordonnées du milieu du segment \( [CD]\).
\item
Est-ce que le triangle formé par les points \( A(1;-7)\), \( B(-3;-3)\) et \( C(2;-6)\) est rectangle ?

\end{enumerate}
} 
\vspace{2cm}
\vbox{38
\emph{Toutes les réponses doivent être justifiées par un calcul accompagné d'un raisonnement.}
\begin{enumerate}\item
Placer les points \( A=(-6;9)\), \( B=(5;-6)\), \( C=(7;0)\) et \( D=(3;7)\) dans un repère orthonormé. Calculer la longueur du segment \( [AB]\) et les coordonnées du milieu du segment \( [CD]\).
\item
Est-ce que le triangle formé par les points \( A(-1;5)\), \( B(4;0)\) et \( C(12;8)\) est rectangle ?

\end{enumerate}
} 
\vspace{2cm}
\vbox{39
\emph{Toutes les réponses doivent être justifiées par un calcul accompagné d'un raisonnement.}
\begin{enumerate}\item
Placer les points \( A=(-2;-5)\), \( B=(6;7)\), \( C=(8;2)\) et \( D=(0;6)\) dans un repère orthonormé. Calculer la longueur du segment \( [AB]\) et les coordonnées du milieu du segment \( [CD]\).
\item
Est-ce que le triangle formé par les points \( A(-9;-1)\), \( B(-15;5)\) et \( C(-5;-1)\) est isocèle ?

\end{enumerate}
} 
\vspace{2cm}
\vbox{40
\emph{Toutes les réponses doivent être justifiées par un calcul accompagné d'un raisonnement.}
\begin{enumerate}\item
Placer les points \( A=(0;5)\), \( B=(2;0)\), \( C=(-2;1)\) et \( D=(10;-5)\) dans un repère orthonormé. Calculer la longueur du segment \( [AB]\) et les coordonnées du milieu du segment \( [CD]\).
\item
Est-ce que le triangle formé par les points \( A(-9;-5)\), \( B(-11;-3)\) et \( C(-1;-5)\) est isocèle ?

\end{enumerate}
} 
\vspace{2cm}
\vbox{41
\emph{Toutes les réponses doivent être justifiées par un calcul accompagné d'un raisonnement.}
\begin{enumerate}\item
Placer les points \( A=(4;-3)\), \( B=(4;4)\), \( C=(9;-5)\) et \( D=(-4;-8)\) dans un repère orthonormé. Calculer la longueur du segment \( [AB]\) et les coordonnées du milieu du segment \( [CD]\).
\item
Est-ce que le triangle formé par les points \( A(-2;2)\), \( B(-4;4)\) et \( C(-1;3)\) est rectangle ?

\end{enumerate}
} 
\vspace{2cm}
\vbox{42
\emph{Toutes les réponses doivent être justifiées par un calcul accompagné d'un raisonnement.}
\begin{enumerate}\item
Placer les points \( A=(4;-1)\), \( B=(0;-5)\), \( C=(2;-1)\) et \( D=(9;7)\) dans un repère orthonormé. Calculer la longueur du segment \( [AB]\) et les coordonnées du milieu du segment \( [CD]\).
\item
Est-ce que le triangle formé par les points \( A(8;7)\), \( B(10;5)\) et \( C(5;2)\) est isocèle ?

\end{enumerate}
} 
\vspace{2cm}
\vbox{43
\emph{Toutes les réponses doivent être justifiées par un calcul accompagné d'un raisonnement.}
\begin{enumerate}\item
Placer les points \( A=(3;8)\), \( B=(-1;6)\), \( C=(2;-7)\) et \( D=(0;-4)\) dans un repère orthonormé. Calculer la longueur du segment \( [AB]\) et les coordonnées du milieu du segment \( [CD]\).
\item
Est-ce que le triangle formé par les points \( A(7;6)\), \( B(11;2)\) et \( C(7;6)\) est rectangle ?

\end{enumerate}
} 
\vspace{2cm}
\vbox{44
\emph{Toutes les réponses doivent être justifiées par un calcul accompagné d'un raisonnement.}
\begin{enumerate}\item
Placer les points \( A=(8;9)\), \( B=(0;10)\), \( C=(-5;10)\) et \( D=(-10;10)\) dans un repère orthonormé. Calculer la longueur du segment \( [AB]\) et les coordonnées du milieu du segment \( [CD]\).
\item
Est-ce que le triangle formé par les points \( A(8;-10)\), \( B(4;-6)\) et \( C(19;-5)\) est isocèle ?

\end{enumerate}
} 
\vspace{2cm}
\vbox{45
\emph{Toutes les réponses doivent être justifiées par un calcul accompagné d'un raisonnement.}
\begin{enumerate}\item
Placer les points \( A=(-1;-10)\), \( B=(-8;2)\), \( C=(-2;0)\) et \( D=(9;-7)\) dans un repère orthonormé. Calculer la longueur du segment \( [AB]\) et les coordonnées du milieu du segment \( [CD]\).
\item
Est-ce que le triangle formé par les points \( A(10;-5)\), \( B(10;-5)\) et \( C(12;-3)\) est rectangle ?

\end{enumerate}
} 
\vspace{2cm}
\vbox{46
\emph{Toutes les réponses doivent être justifiées par un calcul accompagné d'un raisonnement.}
\begin{enumerate}\item
Placer les points \( A=(4;0)\), \( B=(0;-10)\), \( C=(3;8)\) et \( D=(-6;-9)\) dans un repère orthonormé. Calculer la longueur du segment \( [AB]\) et les coordonnées du milieu du segment \( [CD]\).
\item
Est-ce que le triangle formé par les points \( A(3;9)\), \( B(7;5)\) et \( C(2;8)\) est rectangle ?

\end{enumerate}
} 
\vspace{2cm}
\vbox{47
\emph{Toutes les réponses doivent être justifiées par un calcul accompagné d'un raisonnement.}
\begin{enumerate}\item
Placer les points \( A=(2;9)\), \( B=(-6;-2)\), \( C=(0;-6)\) et \( D=(2;-3)\) dans un repère orthonormé. Calculer la longueur du segment \( [AB]\) et les coordonnées du milieu du segment \( [CD]\).
\item
Est-ce que le triangle formé par les points \( A(-8;-8)\), \( B(-10;-6)\) et \( C(0;-8)\) est isocèle ?

\end{enumerate}
} 
\vspace{2cm}
\vbox{48
\emph{Toutes les réponses doivent être justifiées par un calcul accompagné d'un raisonnement.}
\begin{enumerate}\item
Placer les points \( A=(3;5)\), \( B=(2;-1)\), \( C=(1;9)\) et \( D=(5;8)\) dans un repère orthonormé. Calculer la longueur du segment \( [AB]\) et les coordonnées du milieu du segment \( [CD]\).
\item
Est-ce que le triangle formé par les points \( A(8;-5)\), \( B(4;-1)\) et \( C(4;-5)\) est isocèle ?

\end{enumerate}
} 
\vspace{2cm}
\vbox{49
\emph{Toutes les réponses doivent être justifiées par un calcul accompagné d'un raisonnement.}
\begin{enumerate}\item
Placer les points \( A=(8;6)\), \( B=(-8;-7)\), \( C=(9;-5)\) et \( D=(-5;-2)\) dans un repère orthonormé. Calculer la longueur du segment \( [AB]\) et les coordonnées du milieu du segment \( [CD]\).
\item
Est-ce que le triangle formé par les points \( A(4;1)\), \( B(0;5)\) et \( C(10;1)\) est isocèle ?

\end{enumerate}
} 
\vspace{2cm}
\vbox{50
\emph{Toutes les réponses doivent être justifiées par un calcul accompagné d'un raisonnement.}
\begin{enumerate}\item
Placer les points \( A=(2;-10)\), \( B=(-1;2)\), \( C=(0;1)\) et \( D=(-4;-4)\) dans un repère orthonormé. Calculer la longueur du segment \( [AB]\) et les coordonnées du milieu du segment \( [CD]\).
\item
Est-ce que le triangle formé par les points \( A(-9;4)\), \( B(-9;4)\) et \( C(-10;3)\) est isocèle ?

\end{enumerate}
} 
\vspace{2cm}
\vbox{51
\emph{Toutes les réponses doivent être justifiées par un calcul accompagné d'un raisonnement.}
\begin{enumerate}\item
Placer les points \( A=(-9;9)\), \( B=(6;-10)\), \( C=(-8;10)\) et \( D=(5;9)\) dans un repère orthonormé. Calculer la longueur du segment \( [AB]\) et les coordonnées du milieu du segment \( [CD]\).
\item
Est-ce que le triangle formé par les points \( A(-4;0)\), \( B(-10;6)\) et \( C(-7;3)\) est isocèle ?

\end{enumerate}
} 
\vspace{2cm}
\vbox{52
\emph{Toutes les réponses doivent être justifiées par un calcul accompagné d'un raisonnement.}
\begin{enumerate}\item
Placer les points \( A=(-5;0)\), \( B=(8;3)\), \( C=(1;-2)\) et \( D=(-4;5)\) dans un repère orthonormé. Calculer la longueur du segment \( [AB]\) et les coordonnées du milieu du segment \( [CD]\).
\item
Est-ce que le triangle formé par les points \( A(7;-6)\), \( B(1;0)\) et \( C(1;-6)\) est isocèle ?

\end{enumerate}
} 
\vspace{2cm}
\vbox{53
\emph{Toutes les réponses doivent être justifiées par un calcul accompagné d'un raisonnement.}
\begin{enumerate}\item
Placer les points \( A=(0;8)\), \( B=(7;6)\), \( C=(8;4)\) et \( D=(0;6)\) dans un repère orthonormé. Calculer la longueur du segment \( [AB]\) et les coordonnées du milieu du segment \( [CD]\).
\item
Est-ce que le triangle formé par les points \( A(10;-8)\), \( B(10;-8)\) et \( C(11;-7)\) est rectangle ?

\end{enumerate}
} 
\vspace{2cm}
\vbox{54
\emph{Toutes les réponses doivent être justifiées par un calcul accompagné d'un raisonnement.}
\begin{enumerate}\item
Placer les points \( A=(9;-8)\), \( B=(7;-8)\), \( C=(-5;4)\) et \( D=(-9;-5)\) dans un repère orthonormé. Calculer la longueur du segment \( [AB]\) et les coordonnées du milieu du segment \( [CD]\).
\item
Est-ce que le triangle formé par les points \( A(3;-9)\), \( B(-3;-3)\) et \( C(11;-5)\) est isocèle ?

\end{enumerate}
} 
\vspace{2cm}
\vbox{55
\emph{Toutes les réponses doivent être justifiées par un calcul accompagné d'un raisonnement.}
\begin{enumerate}\item
Placer les points \( A=(9;9)\), \( B=(8;9)\), \( C=(9;-4)\) et \( D=(4;-8)\) dans un repère orthonormé. Calculer la longueur du segment \( [AB]\) et les coordonnées du milieu du segment \( [CD]\).
\item
Est-ce que le triangle formé par les points \( A(4;5)\), \( B(5;4)\) et \( C(4;5)\) est rectangle ?

\end{enumerate}
} 
\vspace{2cm}
\vbox{56
\emph{Toutes les réponses doivent être justifiées par un calcul accompagné d'un raisonnement.}
\begin{enumerate}\item
Placer les points \( A=(2;10)\), \( B=(10;-1)\), \( C=(-9;4)\) et \( D=(1;7)\) dans un repère orthonormé. Calculer la longueur du segment \( [AB]\) et les coordonnées du milieu du segment \( [CD]\).
\item
Est-ce que le triangle formé par les points \( A(-7;3)\), \( B(-6;2)\) et \( C(6;6)\) est rectangle ?

\end{enumerate}
} 
\vspace{2cm}
\vbox{57
\emph{Toutes les réponses doivent être justifiées par un calcul accompagné d'un raisonnement.}
\begin{enumerate}\item
Placer les points \( A=(-4;-1)\), \( B=(-8;-3)\), \( C=(-3;10)\) et \( D=(-1;-2)\) dans un repère orthonormé. Calculer la longueur du segment \( [AB]\) et les coordonnées du milieu du segment \( [CD]\).
\item
Est-ce que le triangle formé par les points \( A(-4;8)\), \( B(-6;10)\) et \( C(9;11)\) est rectangle ?

\end{enumerate}
} 
\vspace{2cm}
\vbox{58
\emph{Toutes les réponses doivent être justifiées par un calcul accompagné d'un raisonnement.}
\begin{enumerate}\item
Placer les points \( A=(9;-10)\), \( B=(-4;0)\), \( C=(8;-2)\) et \( D=(4;4)\) dans un repère orthonormé. Calculer la longueur du segment \( [AB]\) et les coordonnées du milieu du segment \( [CD]\).
\item
Est-ce que le triangle formé par les points \( A(-5;7)\), \( B(-3;5)\) et \( C(5;7)\) est rectangle ?

\end{enumerate}
} 
\vspace{2cm}
\vbox{59
\emph{Toutes les réponses doivent être justifiées par un calcul accompagné d'un raisonnement.}
\begin{enumerate}\item
Placer les points \( A=(6;-6)\), \( B=(2;2)\), \( C=(4;4)\) et \( D=(1;6)\) dans un repère orthonormé. Calculer la longueur du segment \( [AB]\) et les coordonnées du milieu du segment \( [CD]\).
\item
Est-ce que le triangle formé par les points \( A(2;9)\), \( B(-3;14)\) et \( C(3;10)\) est rectangle ?

\end{enumerate}
} 
\vspace{2cm}
\vbox{60
\emph{Toutes les réponses doivent être justifiées par un calcul accompagné d'un raisonnement.}
\begin{enumerate}\item
Placer les points \( A=(-10;-10)\), \( B=(5;-8)\), \( C=(10;7)\) et \( D=(3;-4)\) dans un repère orthonormé. Calculer la longueur du segment \( [AB]\) et les coordonnées du milieu du segment \( [CD]\).
\item
Est-ce que le triangle formé par les points \( A(3;6)\), \( B(1;8)\) et \( C(0;5)\) est isocèle ?

\end{enumerate}
} 
\vspace{2cm}
\vbox{61
\emph{Toutes les réponses doivent être justifiées par un calcul accompagné d'un raisonnement.}
\begin{enumerate}\item
Placer les points \( A=(5;-2)\), \( B=(9;-10)\), \( C=(4;-1)\) et \( D=(-10;-9)\) dans un repère orthonormé. Calculer la longueur du segment \( [AB]\) et les coordonnées du milieu du segment \( [CD]\).
\item
Est-ce que le triangle formé par les points \( A(-3;9)\), \( B(1;5)\) et \( C(5;3)\) est isocèle ?

\end{enumerate}
} 
\vspace{2cm}
\vbox{62
\emph{Toutes les réponses doivent être justifiées par un calcul accompagné d'un raisonnement.}
\begin{enumerate}\item
Placer les points \( A=(5;3)\), \( B=(-10;9)\), \( C=(2;10)\) et \( D=(-5;-2)\) dans un repère orthonormé. Calculer la longueur du segment \( [AB]\) et les coordonnées du milieu du segment \( [CD]\).
\item
Est-ce que le triangle formé par les points \( A(-10;-6)\), \( B(-14;-2)\) et \( C(1;-5)\) est rectangle ?

\end{enumerate}
} 
\vspace{2cm}
\vbox{63
\emph{Toutes les réponses doivent être justifiées par un calcul accompagné d'un raisonnement.}
\begin{enumerate}\item
Placer les points \( A=(-8;10)\), \( B=(6;3)\), \( C=(-5;-2)\) et \( D=(6;10)\) dans un repère orthonormé. Calculer la longueur du segment \( [AB]\) et les coordonnées du milieu du segment \( [CD]\).
\item
Est-ce que le triangle formé par les points \( A(6;-8)\), \( B(10;-12)\) et \( C(6;-12)\) est isocèle ?

\end{enumerate}
} 
\vspace{2cm}
\vbox{64
\emph{Toutes les réponses doivent être justifiées par un calcul accompagné d'un raisonnement.}
\begin{enumerate}\item
Placer les points \( A=(9;2)\), \( B=(-5;0)\), \( C=(4;-8)\) et \( D=(8;-5)\) dans un repère orthonormé. Calculer la longueur du segment \( [AB]\) et les coordonnées du milieu du segment \( [CD]\).
\item
Est-ce que le triangle formé par les points \( A(-1;4)\), \( B(-1;4)\) et \( C(12;7)\) est isocèle ?

\end{enumerate}
} 
\vspace{2cm}
\vbox{65
\emph{Toutes les réponses doivent être justifiées par un calcul accompagné d'un raisonnement.}
\begin{enumerate}\item
Placer les points \( A=(-5;3)\), \( B=(-5;-4)\), \( C=(0;-2)\) et \( D=(3;-5)\) dans un repère orthonormé. Calculer la longueur du segment \( [AB]\) et les coordonnées du milieu du segment \( [CD]\).
\item
Est-ce que le triangle formé par les points \( A(3;-2)\), \( B(7;-6)\) et \( C(1;-8)\) est isocèle ?

\end{enumerate}
} 
\vspace{2cm}
\vbox{66
\emph{Toutes les réponses doivent être justifiées par un calcul accompagné d'un raisonnement.}
\begin{enumerate}\item
Placer les points \( A=(2;2)\), \( B=(4;-9)\), \( C=(-10;7)\) et \( D=(0;-7)\) dans un repère orthonormé. Calculer la longueur du segment \( [AB]\) et les coordonnées du milieu du segment \( [CD]\).
\item
Est-ce que le triangle formé par les points \( A(-4;1)\), \( B(-2;-1)\) et \( C(4;-3)\) est isocèle ?

\end{enumerate}
} 
\vspace{2cm}
\vbox{67
\emph{Toutes les réponses doivent être justifiées par un calcul accompagné d'un raisonnement.}
\begin{enumerate}\item
Placer les points \( A=(-2;-9)\), \( B=(-6;8)\), \( C=(-9;6)\) et \( D=(4;-8)\) dans un repère orthonormé. Calculer la longueur du segment \( [AB]\) et les coordonnées du milieu du segment \( [CD]\).
\item
Est-ce que le triangle formé par les points \( A(3;-8)\), \( B(7;-12)\) et \( C(4;-11)\) est isocèle ?

\end{enumerate}
} 
\vspace{2cm}
\vbox{68
\emph{Toutes les réponses doivent être justifiées par un calcul accompagné d'un raisonnement.}
\begin{enumerate}\item
Placer les points \( A=(3;-8)\), \( B=(-1;6)\), \( C=(-5;4)\) et \( D=(10;-9)\) dans un repère orthonormé. Calculer la longueur du segment \( [AB]\) et les coordonnées du milieu du segment \( [CD]\).
\item
Est-ce que le triangle formé par les points \( A(2;-1)\), \( B(4;-3)\) et \( C(6;1)\) est isocèle ?

\end{enumerate}
} 
\vspace{2cm}
\vbox{69
\emph{Toutes les réponses doivent être justifiées par un calcul accompagné d'un raisonnement.}
\begin{enumerate}\item
Placer les points \( A=(5;1)\), \( B=(6;-2)\), \( C=(-9;10)\) et \( D=(2;2)\) dans un repère orthonormé. Calculer la longueur du segment \( [AB]\) et les coordonnées du milieu du segment \( [CD]\).
\item
Est-ce que le triangle formé par les points \( A(-7;-4)\), \( B(-8;-3)\) et \( C(6;-1)\) est rectangle ?

\end{enumerate}
} 
\vspace{2cm}
\vbox{70
\emph{Toutes les réponses doivent être justifiées par un calcul accompagné d'un raisonnement.}
\begin{enumerate}\item
Placer les points \( A=(-6;-6)\), \( B=(-6;-5)\), \( C=(10;-3)\) et \( D=(-9;-2)\) dans un repère orthonormé. Calculer la longueur du segment \( [AB]\) et les coordonnées du milieu du segment \( [CD]\).
\item
Est-ce que le triangle formé par les points \( A(-6;-5)\), \( B(-6;-5)\) et \( C(7;-2)\) est rectangle ?

\end{enumerate}
} 
\vspace{2cm}
\vbox{71
\emph{Toutes les réponses doivent être justifiées par un calcul accompagné d'un raisonnement.}
\begin{enumerate}\item
Placer les points \( A=(-2;-7)\), \( B=(2;-2)\), \( C=(-10;6)\) et \( D=(-7;6)\) dans un repère orthonormé. Calculer la longueur du segment \( [AB]\) et les coordonnées du milieu du segment \( [CD]\).
\item
Est-ce que le triangle formé par les points \( A(10;0)\), \( B(8;2)\) et \( C(19;-1)\) est rectangle ?

\end{enumerate}
} 
\vspace{2cm}
\vbox{72
\emph{Toutes les réponses doivent être justifiées par un calcul accompagné d'un raisonnement.}
\begin{enumerate}\item
Placer les points \( A=(1;6)\), \( B=(-1;9)\), \( C=(3;5)\) et \( D=(0;8)\) dans un repère orthonormé. Calculer la longueur du segment \( [AB]\) et les coordonnées du milieu du segment \( [CD]\).
\item
Est-ce que le triangle formé par les points \( A(10;10)\), \( B(14;6)\) et \( C(10;10)\) est rectangle ?

\end{enumerate}
} 
\vspace{2cm}
\vbox{73
\emph{Toutes les réponses doivent être justifiées par un calcul accompagné d'un raisonnement.}
\begin{enumerate}\item
Placer les points \( A=(-6;-10)\), \( B=(1;8)\), \( C=(-3;8)\) et \( D=(2;-3)\) dans un repère orthonormé. Calculer la longueur du segment \( [AB]\) et les coordonnées du milieu du segment \( [CD]\).
\item
Est-ce que le triangle formé par les points \( A(-7;-6)\), \( B(-13;0)\) et \( C(0;-3)\) est isocèle ?

\end{enumerate}
} 
\vspace{2cm}
\vbox{74
\emph{Toutes les réponses doivent être justifiées par un calcul accompagné d'un raisonnement.}
\begin{enumerate}\item
Placer les points \( A=(1;-1)\), \( B=(4;0)\), \( C=(6;-5)\) et \( D=(5;-4)\) dans un repère orthonormé. Calculer la longueur du segment \( [AB]\) et les coordonnées du milieu du segment \( [CD]\).
\item
Est-ce que le triangle formé par les points \( A(6;1)\), \( B(4;3)\) et \( C(11;-2)\) est isocèle ?

\end{enumerate}
} 
\vspace{2cm}
\vbox{75
\emph{Toutes les réponses doivent être justifiées par un calcul accompagné d'un raisonnement.}
\begin{enumerate}\item
Placer les points \( A=(0;1)\), \( B=(-7;-7)\), \( C=(3;8)\) et \( D=(7;-3)\) dans un repère orthonormé. Calculer la longueur du segment \( [AB]\) et les coordonnées du milieu du segment \( [CD]\).
\item
Est-ce que le triangle formé par les points \( A(10;5)\), \( B(14;1)\) et \( C(14;5)\) est isocèle ?

\end{enumerate}
} 
\vspace{2cm}
\vbox{76
\emph{Toutes les réponses doivent être justifiées par un calcul accompagné d'un raisonnement.}
\begin{enumerate}\item
Placer les points \( A=(10;-4)\), \( B=(-2;-1)\), \( C=(-3;3)\) et \( D=(-3;1)\) dans un repère orthonormé. Calculer la longueur du segment \( [AB]\) et les coordonnées du milieu du segment \( [CD]\).
\item
Est-ce que le triangle formé par les points \( A(-1;6)\), \( B(-6;11)\) et \( C(-3;4)\) est rectangle ?

\end{enumerate}
} 
\vspace{2cm}
\vbox{77
\emph{Toutes les réponses doivent être justifiées par un calcul accompagné d'un raisonnement.}
\begin{enumerate}\item
Placer les points \( A=(-7;-2)\), \( B=(6;10)\), \( C=(0;1)\) et \( D=(1;5)\) dans un repère orthonormé. Calculer la longueur du segment \( [AB]\) et les coordonnées du milieu du segment \( [CD]\).
\item
Est-ce que le triangle formé par les points \( A(-7;-3)\), \( B(-5;-5)\) et \( C(4;-4)\) est isocèle ?

\end{enumerate}
} 
\vspace{2cm}
\vbox{78
\emph{Toutes les réponses doivent être justifiées par un calcul accompagné d'un raisonnement.}
\begin{enumerate}\item
Placer les points \( A=(-7;7)\), \( B=(1;-10)\), \( C=(-7;0)\) et \( D=(-1;-3)\) dans un repère orthonormé. Calculer la longueur du segment \( [AB]\) et les coordonnées du milieu du segment \( [CD]\).
\item
Est-ce que le triangle formé par les points \( A(0;-8)\), \( B(2;-10)\) et \( C(14;-6)\) est isocèle ?

\end{enumerate}
} 
\vspace{2cm}
\vbox{79
\emph{Toutes les réponses doivent être justifiées par un calcul accompagné d'un raisonnement.}
\begin{enumerate}\item
Placer les points \( A=(-4;8)\), \( B=(-7;1)\), \( C=(10;8)\) et \( D=(-7;-1)\) dans un repère orthonormé. Calculer la longueur du segment \( [AB]\) et les coordonnées du milieu du segment \( [CD]\).
\item
Est-ce que le triangle formé par les points \( A(-4;-10)\), \( B(0;-14)\) et \( C(-7;-13)\) est rectangle ?

\end{enumerate}
} 
\vspace{2cm}
\vbox{80
\emph{Toutes les réponses doivent être justifiées par un calcul accompagné d'un raisonnement.}
\begin{enumerate}\item
Placer les points \( A=(-10;-10)\), \( B=(-1;2)\), \( C=(-6;-5)\) et \( D=(0;8)\) dans un repère orthonormé. Calculer la longueur du segment \( [AB]\) et les coordonnées du milieu du segment \( [CD]\).
\item
Est-ce que le triangle formé par les points \( A(-10;4)\), \( B(-11;5)\) et \( C(2;6)\) est rectangle ?

\end{enumerate}
} 
\vspace{2cm}
\vbox{81
\emph{Toutes les réponses doivent être justifiées par un calcul accompagné d'un raisonnement.}
\begin{enumerate}\item
Placer les points \( A=(0;0)\), \( B=(6;9)\), \( C=(-8;-9)\) et \( D=(-8;4)\) dans un repère orthonormé. Calculer la longueur du segment \( [AB]\) et les coordonnées du milieu du segment \( [CD]\).
\item
Est-ce que le triangle formé par les points \( A(2;-1)\), \( B(-4;5)\) et \( C(12;5)\) est isocèle ?

\end{enumerate}
} 
\vspace{2cm}
\vbox{82
\emph{Toutes les réponses doivent être justifiées par un calcul accompagné d'un raisonnement.}
\begin{enumerate}\item
Placer les points \( A=(-3;-9)\), \( B=(1;7)\), \( C=(5;-1)\) et \( D=(7;9)\) dans un repère orthonormé. Calculer la longueur du segment \( [AB]\) et les coordonnées du milieu du segment \( [CD]\).
\item
Est-ce que le triangle formé par les points \( A(-7;-2)\), \( B(-5;-4)\) et \( C(-8;-3)\) est rectangle ?

\end{enumerate}
} 
\vspace{2cm}
\vbox{83
\emph{Toutes les réponses doivent être justifiées par un calcul accompagné d'un raisonnement.}
\begin{enumerate}\item
Placer les points \( A=(4;4)\), \( B=(10;7)\), \( C=(-10;-7)\) et \( D=(-3;-5)\) dans un repère orthonormé. Calculer la longueur du segment \( [AB]\) et les coordonnées du milieu du segment \( [CD]\).
\item
Est-ce que le triangle formé par les points \( A(0;8)\), \( B(0;8)\) et \( C(10;8)\) est rectangle ?

\end{enumerate}
} 
\vspace{2cm}
\vbox{84
\emph{Toutes les réponses doivent être justifiées par un calcul accompagné d'un raisonnement.}
\begin{enumerate}\item
Placer les points \( A=(-1;5)\), \( B=(-10;-9)\), \( C=(10;9)\) et \( D=(-6;-8)\) dans un repère orthonormé. Calculer la longueur du segment \( [AB]\) et les coordonnées du milieu du segment \( [CD]\).
\item
Est-ce que le triangle formé par les points \( A(9;-3)\), \( B(6;0)\) et \( C(12;0)\) est rectangle ?

\end{enumerate}
} 
\vspace{2cm}


-------------------------

\vbox{1
\emph{Toutes les réponses doivent être justifiées par un calcul accompagné d'un raisonnement.}
\begin{enumerate}\item
Placer les points \( A=(-5;3)\), \( B=(1;9)\), \( C=(-9;8)\) et \( D=(6;-1)\) dans un repère orthonormé. Calculer la longueur du segment \( [AB]\) et les coordonnées du milieu du segment \( [CD]\).

 $l^2=72$,$l=6*sqrt(2)$,$M=(-1.5,3.5)$\item
Est-ce que le triangle formé par les points \( A(2;6)\), \( B(6;2)\) et \( C(17;7)\) est isocèle ?

False
\end{enumerate}
} 
\vspace{2cm}
\vbox{2
\emph{Toutes les réponses doivent être justifiées par un calcul accompagné d'un raisonnement.}
\begin{enumerate}\item
Placer les points \( A=(-5;3)\), \( B=(-6;-10)\), \( C=(10;-7)\) et \( D=(4;1)\) dans un repère orthonormé. Calculer la longueur du segment \( [AB]\) et les coordonnées du milieu du segment \( [CD]\).

 $l^2=170$,$l=sqrt(170)$,$M=(7.0,-3.0)$\item
Est-ce que le triangle formé par les points \( A(1;-1)\), \( B(-5;5)\) et \( C(2;6)\) est isocèle ?

True
\end{enumerate}
} 
\vspace{2cm}
\vbox{3
\emph{Toutes les réponses doivent être justifiées par un calcul accompagné d'un raisonnement.}
\begin{enumerate}\item
Placer les points \( A=(0;-9)\), \( B=(6;9)\), \( C=(-1;9)\) et \( D=(8;10)\) dans un repère orthonormé. Calculer la longueur du segment \( [AB]\) et les coordonnées du milieu du segment \( [CD]\).

 $l^2=360$,$l=6*sqrt(10)$,$M=(3.5,9.5)$\item
Est-ce que le triangle formé par les points \( A(-6;-2)\), \( B(-11;3)\) et \( C(1;-5)\) est rectangle ?

False
\end{enumerate}
} 
\vspace{2cm}
\vbox{4
\emph{Toutes les réponses doivent être justifiées par un calcul accompagné d'un raisonnement.}
\begin{enumerate}\item
Placer les points \( A=(4;-6)\), \( B=(-7;-2)\), \( C=(-8;-8)\) et \( D=(-4;-2)\) dans un repère orthonormé. Calculer la longueur du segment \( [AB]\) et les coordonnées du milieu du segment \( [CD]\).

 $l^2=137$,$l=sqrt(137)$,$M=(-6.0,-5.0)$\item
Est-ce que le triangle formé par les points \( A(-3;-10)\), \( B(-7;-6)\) et \( C(4;-9)\) est isocèle ?

False
\end{enumerate}
} 
\vspace{2cm}
\vbox{5
\emph{Toutes les réponses doivent être justifiées par un calcul accompagné d'un raisonnement.}
\begin{enumerate}\item
Placer les points \( A=(6;6)\), \( B=(-10;-9)\), \( C=(-4;9)\) et \( D=(-4;2)\) dans un repère orthonormé. Calculer la longueur du segment \( [AB]\) et les coordonnées du milieu du segment \( [CD]\).

 $l^2=481$,$l=sqrt(481)$,$M=(-4.0,5.5)$\item
Est-ce que le triangle formé par les points \( A(7;10)\), \( B(1;16)\) et \( C(18;17)\) est isocèle ?

False
\end{enumerate}
} 
\vspace{2cm}
\vbox{6
\emph{Toutes les réponses doivent être justifiées par un calcul accompagné d'un raisonnement.}
\begin{enumerate}\item
Placer les points \( A=(-7;7)\), \( B=(-4;0)\), \( C=(4;-2)\) et \( D=(10;-8)\) dans un repère orthonormé. Calculer la longueur du segment \( [AB]\) et les coordonnées du milieu du segment \( [CD]\).

 $l^2=58$,$l=sqrt(58)$,$M=(7.0,-5.0)$\item
Est-ce que le triangle formé par les points \( A(-2;10)\), \( B(-1;9)\) et \( C(-4;8)\) est rectangle ?

True
\end{enumerate}
} 
\vspace{2cm}
\vbox{7
\emph{Toutes les réponses doivent être justifiées par un calcul accompagné d'un raisonnement.}
\begin{enumerate}\item
Placer les points \( A=(8;2)\), \( B=(7;5)\), \( C=(-10;5)\) et \( D=(1;10)\) dans un repère orthonormé. Calculer la longueur du segment \( [AB]\) et les coordonnées du milieu du segment \( [CD]\).

 $l^2=10$,$l=sqrt(10)$,$M=(-4.5,7.5)$\item
Est-ce que le triangle formé par les points \( A(8;-1)\), \( B(4;3)\) et \( C(10;1)\) est rectangle ?

True
\end{enumerate}
} 
\vspace{2cm}
\vbox{8
\emph{Toutes les réponses doivent être justifiées par un calcul accompagné d'un raisonnement.}
\begin{enumerate}\item
Placer les points \( A=(-4;-8)\), \( B=(0;8)\), \( C=(-2;1)\) et \( D=(0;-3)\) dans un repère orthonormé. Calculer la longueur du segment \( [AB]\) et les coordonnées du milieu du segment \( [CD]\).

 $l^2=272$,$l=4*sqrt(17)$,$M=(-1.0,-1.0)$\item
Est-ce que le triangle formé par les points \( A(4;-4)\), \( B(1;-1)\) et \( C(7;-1)\) est rectangle ?

True
\end{enumerate}
} 
\vspace{2cm}
\vbox{9
\emph{Toutes les réponses doivent être justifiées par un calcul accompagné d'un raisonnement.}
\begin{enumerate}\item
Placer les points \( A=(8;-5)\), \( B=(8;-2)\), \( C=(8;2)\) et \( D=(0;10)\) dans un repère orthonormé. Calculer la longueur du segment \( [AB]\) et les coordonnées du milieu du segment \( [CD]\).

 $l^2=9$,$l=3$,$M=(4.0,6.0)$\item
Est-ce que le triangle formé par les points \( A(-6;6)\), \( B(-1;1)\) et \( C(3;5)\) est rectangle ?

False
\end{enumerate}
} 
\vspace{2cm}
\vbox{10
\emph{Toutes les réponses doivent être justifiées par un calcul accompagné d'un raisonnement.}
\begin{enumerate}\item
Placer les points \( A=(4;-2)\), \( B=(7;9)\), \( C=(-10;-9)\) et \( D=(5;-2)\) dans un repère orthonormé. Calculer la longueur du segment \( [AB]\) et les coordonnées du milieu du segment \( [CD]\).

 $l^2=130$,$l=sqrt(130)$,$M=(-2.5,-5.5)$\item
Est-ce que le triangle formé par les points \( A(-1;-2)\), \( B(-4;1)\) et \( C(1;0)\) est rectangle ?

True
\end{enumerate}
} 
\vspace{2cm}
\vbox{11
\emph{Toutes les réponses doivent être justifiées par un calcul accompagné d'un raisonnement.}
\begin{enumerate}\item
Placer les points \( A=(5;9)\), \( B=(-7;5)\), \( C=(7;-1)\) et \( D=(9;-6)\) dans un repère orthonormé. Calculer la longueur du segment \( [AB]\) et les coordonnées du milieu du segment \( [CD]\).

 $l^2=160$,$l=4*sqrt(10)$,$M=(8.0,-3.5)$\item
Est-ce que le triangle formé par les points \( A(3;9)\), \( B(-3;15)\) et \( C(8;10)\) est isocèle ?

False
\end{enumerate}
} 
\vspace{2cm}
\vbox{12
\emph{Toutes les réponses doivent être justifiées par un calcul accompagné d'un raisonnement.}
\begin{enumerate}\item
Placer les points \( A=(-10;-5)\), \( B=(-10;1)\), \( C=(-5;5)\) et \( D=(0;-3)\) dans un repère orthonormé. Calculer la longueur du segment \( [AB]\) et les coordonnées du milieu du segment \( [CD]\).

 $l^2=36$,$l=6$,$M=(-2.5,1.0)$\item
Est-ce que le triangle formé par les points \( A(6;-1)\), \( B(12;-7)\) et \( C(13;0)\) est isocèle ?

True
\end{enumerate}
} 
\vspace{2cm}
\vbox{13
\emph{Toutes les réponses doivent être justifiées par un calcul accompagné d'un raisonnement.}
\begin{enumerate}\item
Placer les points \( A=(10;6)\), \( B=(-5;-7)\), \( C=(-1;7)\) et \( D=(-8;5)\) dans un repère orthonormé. Calculer la longueur du segment \( [AB]\) et les coordonnées du milieu du segment \( [CD]\).

 $l^2=394$,$l=sqrt(394)$,$M=(-4.5,6.0)$\item
Est-ce que le triangle formé par les points \( A(-10;-10)\), \( B(-16;-4)\) et \( C(0;-4)\) est isocèle ?

False
\end{enumerate}
} 
\vspace{2cm}
\vbox{14
\emph{Toutes les réponses doivent être justifiées par un calcul accompagné d'un raisonnement.}
\begin{enumerate}\item
Placer les points \( A=(1;-7)\), \( B=(9;-3)\), \( C=(5;7)\) et \( D=(-6;-5)\) dans un repère orthonormé. Calculer la longueur du segment \( [AB]\) et les coordonnées du milieu du segment \( [CD]\).

 $l^2=80$,$l=4*sqrt(5)$,$M=(-0.5,1.0)$\item
Est-ce que le triangle formé par les points \( A(-3;-8)\), \( B(-9;-2)\) et \( C(-9;-8)\) est isocèle ?

True
\end{enumerate}
} 
\vspace{2cm}
\vbox{15
\emph{Toutes les réponses doivent être justifiées par un calcul accompagné d'un raisonnement.}
\begin{enumerate}\item
Placer les points \( A=(-7;-7)\), \( B=(-1;-9)\), \( C=(10;-8)\) et \( D=(-9;4)\) dans un repère orthonormé. Calculer la longueur du segment \( [AB]\) et les coordonnées du milieu du segment \( [CD]\).

 $l^2=40$,$l=2*sqrt(10)$,$M=(0.5,-2.0)$\item
Est-ce que le triangle formé par les points \( A(2;-10)\), \( B(7;-15)\) et \( C(1;-11)\) est rectangle ?

True
\end{enumerate}
} 
\vspace{2cm}
\vbox{16
\emph{Toutes les réponses doivent être justifiées par un calcul accompagné d'un raisonnement.}
\begin{enumerate}\item
Placer les points \( A=(-3;-10)\), \( B=(-10;-9)\), \( C=(-2;-4)\) et \( D=(-7;3)\) dans un repère orthonormé. Calculer la longueur du segment \( [AB]\) et les coordonnées du milieu du segment \( [CD]\).

 $l^2=50$,$l=5*sqrt(2)$,$M=(-4.5,-0.5)$\item
Est-ce que le triangle formé par les points \( A(-5;-2)\), \( B(0;-7)\) et \( C(-8;-5)\) est rectangle ?

True
\end{enumerate}
} 
\vspace{2cm}
\vbox{17
\emph{Toutes les réponses doivent être justifiées par un calcul accompagné d'un raisonnement.}
\begin{enumerate}\item
Placer les points \( A=(3;6)\), \( B=(7;8)\), \( C=(6;9)\) et \( D=(8;-1)\) dans un repère orthonormé. Calculer la longueur du segment \( [AB]\) et les coordonnées du milieu du segment \( [CD]\).

 $l^2=20$,$l=2*sqrt(5)$,$M=(7.0,4.0)$\item
Est-ce que le triangle formé par les points \( A(6;7)\), \( B(8;5)\) et \( C(21;10)\) est isocèle ?

False
\end{enumerate}
} 
\vspace{2cm}
\vbox{18
\emph{Toutes les réponses doivent être justifiées par un calcul accompagné d'un raisonnement.}
\begin{enumerate}\item
Placer les points \( A=(-1;-10)\), \( B=(4;-4)\), \( C=(-4;-1)\) et \( D=(-5;1)\) dans un repère orthonormé. Calculer la longueur du segment \( [AB]\) et les coordonnées du milieu du segment \( [CD]\).

 $l^2=61$,$l=sqrt(61)$,$M=(-4.5,0.0)$\item
Est-ce que le triangle formé par les points \( A(-10;-4)\), \( B(-11;-3)\) et \( C(-1;-5)\) est rectangle ?

False
\end{enumerate}
} 
\vspace{2cm}
\vbox{19
\emph{Toutes les réponses doivent être justifiées par un calcul accompagné d'un raisonnement.}
\begin{enumerate}\item
Placer les points \( A=(-9;0)\), \( B=(-4;-1)\), \( C=(3;-2)\) et \( D=(7;-6)\) dans un repère orthonormé. Calculer la longueur du segment \( [AB]\) et les coordonnées du milieu du segment \( [CD]\).

 $l^2=26$,$l=sqrt(26)$,$M=(5.0,-4.0)$\item
Est-ce que le triangle formé par les points \( A(10;-6)\), \( B(13;-9)\) et \( C(23;-3)\) est rectangle ?

False
\end{enumerate}
} 
\vspace{2cm}
\vbox{20
\emph{Toutes les réponses doivent être justifiées par un calcul accompagné d'un raisonnement.}
\begin{enumerate}\item
Placer les points \( A=(-5;0)\), \( B=(6;8)\), \( C=(4;5)\) et \( D=(-4;-6)\) dans un repère orthonormé. Calculer la longueur du segment \( [AB]\) et les coordonnées du milieu du segment \( [CD]\).

 $l^2=185$,$l=sqrt(185)$,$M=(0.0,-0.5)$\item
Est-ce que le triangle formé par les points \( A(1;-7)\), \( B(5;-11)\) et \( C(5;-7)\) est isocèle ?

True
\end{enumerate}
} 
\vspace{2cm}
\vbox{21
\emph{Toutes les réponses doivent être justifiées par un calcul accompagné d'un raisonnement.}
\begin{enumerate}\item
Placer les points \( A=(-4;5)\), \( B=(0;-10)\), \( C=(7;-4)\) et \( D=(4;9)\) dans un repère orthonormé. Calculer la longueur du segment \( [AB]\) et les coordonnées du milieu du segment \( [CD]\).

 $l^2=241$,$l=sqrt(241)$,$M=(5.5,2.5)$\item
Est-ce que le triangle formé par les points \( A(5;-4)\), \( B(1;0)\) et \( C(17;-2)\) est rectangle ?

False
\end{enumerate}
} 
\vspace{2cm}
\vbox{22
\emph{Toutes les réponses doivent être justifiées par un calcul accompagné d'un raisonnement.}
\begin{enumerate}\item
Placer les points \( A=(8;8)\), \( B=(-3;-2)\), \( C=(3;2)\) et \( D=(-8;9)\) dans un repère orthonormé. Calculer la longueur du segment \( [AB]\) et les coordonnées du milieu du segment \( [CD]\).

 $l^2=221$,$l=sqrt(221)$,$M=(-2.5,5.5)$\item
Est-ce que le triangle formé par les points \( A(1;-1)\), \( B(-1;1)\) et \( C(1;1)\) est isocèle ?

True
\end{enumerate}
} 
\vspace{2cm}
\vbox{23
\emph{Toutes les réponses doivent être justifiées par un calcul accompagné d'un raisonnement.}
\begin{enumerate}\item
Placer les points \( A=(8;10)\), \( B=(-7;1)\), \( C=(-9;5)\) et \( D=(-5;8)\) dans un repère orthonormé. Calculer la longueur du segment \( [AB]\) et les coordonnées du milieu du segment \( [CD]\).

 $l^2=306$,$l=3*sqrt(34)$,$M=(-7.0,6.5)$\item
Est-ce que le triangle formé par les points \( A(0;9)\), \( B(-4;13)\) et \( C(7;6)\) est rectangle ?

False
\end{enumerate}
} 
\vspace{2cm}
\vbox{24
\emph{Toutes les réponses doivent être justifiées par un calcul accompagné d'un raisonnement.}
\begin{enumerate}\item
Placer les points \( A=(4;-2)\), \( B=(8;1)\), \( C=(4;-3)\) et \( D=(-3;6)\) dans un repère orthonormé. Calculer la longueur du segment \( [AB]\) et les coordonnées du milieu du segment \( [CD]\).

 $l^2=25$,$l=5$,$M=(0.5,1.5)$\item
Est-ce que le triangle formé par les points \( A(1;4)\), \( B(-4;9)\) et \( C(9;2)\) est rectangle ?

False
\end{enumerate}
} 
\vspace{2cm}
\vbox{25
\emph{Toutes les réponses doivent être justifiées par un calcul accompagné d'un raisonnement.}
\begin{enumerate}\item
Placer les points \( A=(2;-5)\), \( B=(-9;4)\), \( C=(5;4)\) et \( D=(-7;4)\) dans un repère orthonormé. Calculer la longueur du segment \( [AB]\) et les coordonnées du milieu du segment \( [CD]\).

 $l^2=202$,$l=sqrt(202)$,$M=(-1.0,4.0)$\item
Est-ce que le triangle formé par les points \( A(-4;3)\), \( B(-2;1)\) et \( C(-3;4)\) est rectangle ?

True
\end{enumerate}
} 
\vspace{2cm}
\vbox{26
\emph{Toutes les réponses doivent être justifiées par un calcul accompagné d'un raisonnement.}
\begin{enumerate}\item
Placer les points \( A=(1;3)\), \( B=(-4;-8)\), \( C=(-6;9)\) et \( D=(-7;9)\) dans un repère orthonormé. Calculer la longueur du segment \( [AB]\) et les coordonnées du milieu du segment \( [CD]\).

 $l^2=146$,$l=sqrt(146)$,$M=(-6.5,9.0)$\item
Est-ce que le triangle formé par les points \( A(-5;3)\), \( B(-11;9)\) et \( C(-6;8)\) est isocèle ?

True
\end{enumerate}
} 
\vspace{2cm}
\vbox{27
\emph{Toutes les réponses doivent être justifiées par un calcul accompagné d'un raisonnement.}
\begin{enumerate}\item
Placer les points \( A=(2;-2)\), \( B=(8;-5)\), \( C=(-1;1)\) et \( D=(10;2)\) dans un repère orthonormé. Calculer la longueur du segment \( [AB]\) et les coordonnées du milieu du segment \( [CD]\).

 $l^2=45$,$l=3*sqrt(5)$,$M=(4.5,1.5)$\item
Est-ce que le triangle formé par les points \( A(1;7)\), \( B(-2;10)\) et \( C(14;10)\) est rectangle ?

False
\end{enumerate}
} 
\vspace{2cm}
\vbox{28
\emph{Toutes les réponses doivent être justifiées par un calcul accompagné d'un raisonnement.}
\begin{enumerate}\item
Placer les points \( A=(7;5)\), \( B=(1;0)\), \( C=(10;1)\) et \( D=(1;-9)\) dans un repère orthonormé. Calculer la longueur du segment \( [AB]\) et les coordonnées du milieu du segment \( [CD]\).

 $l^2=61$,$l=sqrt(61)$,$M=(5.5,-4.0)$\item
Est-ce que le triangle formé par les points \( A(-9;-4)\), \( B(-12;-1)\) et \( C(-11;-6)\) est rectangle ?

True
\end{enumerate}
} 
\vspace{2cm}
\vbox{29
\emph{Toutes les réponses doivent être justifiées par un calcul accompagné d'un raisonnement.}
\begin{enumerate}\item
Placer les points \( A=(-2;-3)\), \( B=(-7;4)\), \( C=(-10;-6)\) et \( D=(-10;-3)\) dans un repère orthonormé. Calculer la longueur du segment \( [AB]\) et les coordonnées du milieu du segment \( [CD]\).

 $l^2=74$,$l=sqrt(74)$,$M=(-10.0,-4.5)$\item
Est-ce que le triangle formé par les points \( A(9;6)\), \( B(10;5)\) et \( C(18;5)\) est rectangle ?

False
\end{enumerate}
} 
\vspace{2cm}
\vbox{30
\emph{Toutes les réponses doivent être justifiées par un calcul accompagné d'un raisonnement.}
\begin{enumerate}\item
Placer les points \( A=(6;9)\), \( B=(1;9)\), \( C=(3;9)\) et \( D=(1;8)\) dans un repère orthonormé. Calculer la longueur du segment \( [AB]\) et les coordonnées du milieu du segment \( [CD]\).

 $l^2=25$,$l=5$,$M=(2.0,8.5)$\item
Est-ce que le triangle formé par les points \( A(10;4)\), \( B(8;6)\) et \( C(19;3)\) est rectangle ?

False
\end{enumerate}
} 
\vspace{2cm}
\vbox{31
\emph{Toutes les réponses doivent être justifiées par un calcul accompagné d'un raisonnement.}
\begin{enumerate}\item
Placer les points \( A=(-1;7)\), \( B=(3;-9)\), \( C=(7;-1)\) et \( D=(-7;-10)\) dans un repère orthonormé. Calculer la longueur du segment \( [AB]\) et les coordonnées du milieu du segment \( [CD]\).

 $l^2=272$,$l=4*sqrt(17)$,$M=(0.0,-5.5)$\item
Est-ce que le triangle formé par les points \( A(-9;3)\), \( B(-5;-1)\) et \( C(-4;4)\) est isocèle ?

True
\end{enumerate}
} 
\vspace{2cm}
\vbox{32
\emph{Toutes les réponses doivent être justifiées par un calcul accompagné d'un raisonnement.}
\begin{enumerate}\item
Placer les points \( A=(7;8)\), \( B=(-5;3)\), \( C=(1;7)\) et \( D=(-10;-10)\) dans un repère orthonormé. Calculer la longueur du segment \( [AB]\) et les coordonnées du milieu du segment \( [CD]\).

 $l^2=169$,$l=13$,$M=(-4.5,-1.5)$\item
Est-ce que le triangle formé par les points \( A(-4;5)\), \( B(-7;8)\) et \( C(5;4)\) est rectangle ?

False
\end{enumerate}
} 
\vspace{2cm}
\vbox{33
\emph{Toutes les réponses doivent être justifiées par un calcul accompagné d'un raisonnement.}
\begin{enumerate}\item
Placer les points \( A=(-1;4)\), \( B=(10;8)\), \( C=(4;-8)\) et \( D=(5;7)\) dans un repère orthonormé. Calculer la longueur du segment \( [AB]\) et les coordonnées du milieu du segment \( [CD]\).

 $l^2=137$,$l=sqrt(137)$,$M=(4.5,-0.5)$\item
Est-ce que le triangle formé par les points \( A(-9;6)\), \( B(-14;11)\) et \( C(-11;4)\) est rectangle ?

True
\end{enumerate}
} 
\vspace{2cm}
\vbox{34
\emph{Toutes les réponses doivent être justifiées par un calcul accompagné d'un raisonnement.}
\begin{enumerate}\item
Placer les points \( A=(-5;-2)\), \( B=(9;10)\), \( C=(7;0)\) et \( D=(-6;8)\) dans un repère orthonormé. Calculer la longueur du segment \( [AB]\) et les coordonnées du milieu du segment \( [CD]\).

 $l^2=340$,$l=2*sqrt(85)$,$M=(0.5,4.0)$\item
Est-ce que le triangle formé par les points \( A(-6;-7)\), \( B(-10;-3)\) et \( C(3;-4)\) est isocèle ?

False
\end{enumerate}
} 
\vspace{2cm}
\vbox{35
\emph{Toutes les réponses doivent être justifiées par un calcul accompagné d'un raisonnement.}
\begin{enumerate}\item
Placer les points \( A=(-4;-9)\), \( B=(-6;6)\), \( C=(-6;8)\) et \( D=(8;-8)\) dans un repère orthonormé. Calculer la longueur du segment \( [AB]\) et les coordonnées du milieu du segment \( [CD]\).

 $l^2=229$,$l=sqrt(229)$,$M=(1.0,0.0)$\item
Est-ce que le triangle formé par les points \( A(-6;-4)\), \( B(-9;-1)\) et \( C(2;-6)\) est rectangle ?

False
\end{enumerate}
} 
\vspace{2cm}
\vbox{36
\emph{Toutes les réponses doivent être justifiées par un calcul accompagné d'un raisonnement.}
\begin{enumerate}\item
Placer les points \( A=(1;-4)\), \( B=(9;1)\), \( C=(-8;-1)\) et \( D=(-1;7)\) dans un repère orthonormé. Calculer la longueur du segment \( [AB]\) et les coordonnées du milieu du segment \( [CD]\).

 $l^2=89$,$l=sqrt(89)$,$M=(-4.5,3.0)$\item
Est-ce que le triangle formé par les points \( A(2;-10)\), \( B(8;-16)\) et \( C(11;-17)\) est isocèle ?

False
\end{enumerate}
} 
\vspace{2cm}
\vbox{37
\emph{Toutes les réponses doivent être justifiées par un calcul accompagné d'un raisonnement.}
\begin{enumerate}\item
Placer les points \( A=(0;5)\), \( B=(7;4)\), \( C=(-2;-6)\) et \( D=(3;-6)\) dans un repère orthonormé. Calculer la longueur du segment \( [AB]\) et les coordonnées du milieu du segment \( [CD]\).

 $l^2=50$,$l=5*sqrt(2)$,$M=(0.5,-6.0)$\item
Est-ce que le triangle formé par les points \( A(9;-3)\), \( B(15;-9)\) et \( C(15;-3)\) est isocèle ?

True
\end{enumerate}
} 
\vspace{2cm}
\vbox{38
\emph{Toutes les réponses doivent être justifiées par un calcul accompagné d'un raisonnement.}
\begin{enumerate}\item
Placer les points \( A=(1;0)\), \( B=(6;6)\), \( C=(7;3)\) et \( D=(1;8)\) dans un repère orthonormé. Calculer la longueur du segment \( [AB]\) et les coordonnées du milieu du segment \( [CD]\).

 $l^2=61$,$l=sqrt(61)$,$M=(4.0,5.5)$\item
Est-ce que le triangle formé par les points \( A(0;-3)\), \( B(-3;0)\) et \( C(-2;-5)\) est rectangle ?

True
\end{enumerate}
} 
\vspace{2cm}
\vbox{39
\emph{Toutes les réponses doivent être justifiées par un calcul accompagné d'un raisonnement.}
\begin{enumerate}\item
Placer les points \( A=(-4;-4)\), \( B=(0;-3)\), \( C=(-6;1)\) et \( D=(-10;9)\) dans un repère orthonormé. Calculer la longueur du segment \( [AB]\) et les coordonnées du milieu du segment \( [CD]\).

 $l^2=17$,$l=sqrt(17)$,$M=(-8.0,5.0)$\item
Est-ce que le triangle formé par les points \( A(0;-6)\), \( B(2;-8)\) et \( C(5;-3)\) est isocèle ?

True
\end{enumerate}
} 
\vspace{2cm}
\vbox{40
\emph{Toutes les réponses doivent être justifiées par un calcul accompagné d'un raisonnement.}
\begin{enumerate}\item
Placer les points \( A=(6;-6)\), \( B=(6;-10)\), \( C=(8;-6)\) et \( D=(2;5)\) dans un repère orthonormé. Calculer la longueur du segment \( [AB]\) et les coordonnées du milieu du segment \( [CD]\).

 $l^2=16$,$l=4$,$M=(5.0,-0.5)$\item
Est-ce que le triangle formé par les points \( A(10;3)\), \( B(8;5)\) et \( C(7;2)\) est isocèle ?

True
\end{enumerate}
} 
\vspace{2cm}
\vbox{41
\emph{Toutes les réponses doivent être justifiées par un calcul accompagné d'un raisonnement.}
\begin{enumerate}\item
Placer les points \( A=(0;0)\), \( B=(6;-2)\), \( C=(-4;-3)\) et \( D=(8;7)\) dans un repère orthonormé. Calculer la longueur du segment \( [AB]\) et les coordonnées du milieu du segment \( [CD]\).

 $l^2=40$,$l=2*sqrt(10)$,$M=(2.0,2.0)$\item
Est-ce que le triangle formé par les points \( A(0;-5)\), \( B(1;-6)\) et \( C(9;-6)\) est rectangle ?

False
\end{enumerate}
} 
\vspace{2cm}
\vbox{42
\emph{Toutes les réponses doivent être justifiées par un calcul accompagné d'un raisonnement.}
\begin{enumerate}\item
Placer les points \( A=(-9;-8)\), \( B=(-5;8)\), \( C=(3;8)\) et \( D=(-4;5)\) dans un repère orthonormé. Calculer la longueur du segment \( [AB]\) et les coordonnées du milieu du segment \( [CD]\).

 $l^2=272$,$l=4*sqrt(17)$,$M=(-0.5,6.5)$\item
Est-ce que le triangle formé par les points \( A(2;-4)\), \( B(0;-2)\) et \( C(0;-4)\) est isocèle ?

True
\end{enumerate}
} 
\vspace{2cm}
\vbox{43
\emph{Toutes les réponses doivent être justifiées par un calcul accompagné d'un raisonnement.}
\begin{enumerate}\item
Placer les points \( A=(-7;-4)\), \( B=(7;-1)\), \( C=(6;-2)\) et \( D=(-9;2)\) dans un repère orthonormé. Calculer la longueur du segment \( [AB]\) et les coordonnées du milieu du segment \( [CD]\).

 $l^2=205$,$l=sqrt(205)$,$M=(-1.5,0.0)$\item
Est-ce que le triangle formé par les points \( A(1;-6)\), \( B(6;-11)\) et \( C(3;-4)\) est rectangle ?

True
\end{enumerate}
} 
\vspace{2cm}
\vbox{44
\emph{Toutes les réponses doivent être justifiées par un calcul accompagné d'un raisonnement.}
\begin{enumerate}\item
Placer les points \( A=(6;-3)\), \( B=(-9;-8)\), \( C=(8;-9)\) et \( D=(-6;-7)\) dans un repère orthonormé. Calculer la longueur du segment \( [AB]\) et les coordonnées du milieu du segment \( [CD]\).

 $l^2=250$,$l=5*sqrt(10)$,$M=(1.0,-8.0)$\item
Est-ce que le triangle formé par les points \( A(8;1)\), \( B(13;-4)\) et \( C(21;4)\) est rectangle ?

False
\end{enumerate}
} 
\vspace{2cm}
\vbox{45
\emph{Toutes les réponses doivent être justifiées par un calcul accompagné d'un raisonnement.}
\begin{enumerate}\item
Placer les points \( A=(7;1)\), \( B=(9;5)\), \( C=(-8;6)\) et \( D=(-3;-5)\) dans un repère orthonormé. Calculer la longueur du segment \( [AB]\) et les coordonnées du milieu du segment \( [CD]\).

 $l^2=20$,$l=2*sqrt(5)$,$M=(-5.5,0.5)$\item
Est-ce que le triangle formé par les points \( A(-6;-1)\), \( B(-2;-5)\) et \( C(7;2)\) est rectangle ?

False
\end{enumerate}
} 
\vspace{2cm}
\vbox{46
\emph{Toutes les réponses doivent être justifiées par un calcul accompagné d'un raisonnement.}
\begin{enumerate}\item
Placer les points \( A=(8;9)\), \( B=(5;-9)\), \( C=(-9;-5)\) et \( D=(-1;7)\) dans un repère orthonormé. Calculer la longueur du segment \( [AB]\) et les coordonnées du milieu du segment \( [CD]\).

 $l^2=333$,$l=3*sqrt(37)$,$M=(-5.0,1.0)$\item
Est-ce que le triangle formé par les points \( A(-3;8)\), \( B(-5;10)\) et \( C(9;12)\) est isocèle ?

False
\end{enumerate}
} 
\vspace{2cm}
\vbox{47
\emph{Toutes les réponses doivent être justifiées par un calcul accompagné d'un raisonnement.}
\begin{enumerate}\item
Placer les points \( A=(-8;-2)\), \( B=(10;-8)\), \( C=(0;-6)\) et \( D=(8;4)\) dans un repère orthonormé. Calculer la longueur du segment \( [AB]\) et les coordonnées du milieu du segment \( [CD]\).

 $l^2=360$,$l=6*sqrt(10)$,$M=(4.0,-1.0)$\item
Est-ce que le triangle formé par les points \( A(10;-10)\), \( B(12;-12)\) et \( C(9;-11)\) est rectangle ?

True
\end{enumerate}
} 
\vspace{2cm}
\vbox{48
\emph{Toutes les réponses doivent être justifiées par un calcul accompagné d'un raisonnement.}
\begin{enumerate}\item
Placer les points \( A=(4;-1)\), \( B=(-1;-8)\), \( C=(2;6)\) et \( D=(-5;-7)\) dans un repère orthonormé. Calculer la longueur du segment \( [AB]\) et les coordonnées du milieu du segment \( [CD]\).

 $l^2=74$,$l=sqrt(74)$,$M=(-1.5,-0.5)$\item
Est-ce que le triangle formé par les points \( A(-3;-2)\), \( B(-9;4)\) et \( C(0;-3)\) est isocèle ?

False
\end{enumerate}
} 
\vspace{2cm}
\vbox{49
\emph{Toutes les réponses doivent être justifiées par un calcul accompagné d'un raisonnement.}
\begin{enumerate}\item
Placer les points \( A=(9;9)\), \( B=(10;-3)\), \( C=(9;0)\) et \( D=(9;-9)\) dans un repère orthonormé. Calculer la longueur du segment \( [AB]\) et les coordonnées du milieu du segment \( [CD]\).

 $l^2=145$,$l=sqrt(145)$,$M=(9.0,-4.5)$\item
Est-ce que le triangle formé par les points \( A(3;-9)\), \( B(7;-13)\) et \( C(2;-10)\) est rectangle ?

True
\end{enumerate}
} 
\vspace{2cm}
\vbox{50
\emph{Toutes les réponses doivent être justifiées par un calcul accompagné d'un raisonnement.}
\begin{enumerate}\item
Placer les points \( A=(5;7)\), \( B=(-7;0)\), \( C=(-7;1)\) et \( D=(0;9)\) dans un repère orthonormé. Calculer la longueur du segment \( [AB]\) et les coordonnées du milieu du segment \( [CD]\).

 $l^2=193$,$l=sqrt(193)$,$M=(-3.5,5.0)$\item
Est-ce que le triangle formé par les points \( A(-6;-3)\), \( B(-11;2)\) et \( C(-4;-1)\) est rectangle ?

True
\end{enumerate}
} 
\vspace{2cm}
\vbox{51
\emph{Toutes les réponses doivent être justifiées par un calcul accompagné d'un raisonnement.}
\begin{enumerate}\item
Placer les points \( A=(-6;1)\), \( B=(-10;-3)\), \( C=(8;6)\) et \( D=(-8;2)\) dans un repère orthonormé. Calculer la longueur du segment \( [AB]\) et les coordonnées du milieu du segment \( [CD]\).

 $l^2=32$,$l=4*sqrt(2)$,$M=(0.0,4.0)$\item
Est-ce que le triangle formé par les points \( A(-7;7)\), \( B(-6;6)\) et \( C(-4;10)\) est rectangle ?

True
\end{enumerate}
} 
\vspace{2cm}
\vbox{52
\emph{Toutes les réponses doivent être justifiées par un calcul accompagné d'un raisonnement.}
\begin{enumerate}\item
Placer les points \( A=(3;-2)\), \( B=(5;8)\), \( C=(-3;10)\) et \( D=(4;4)\) dans un repère orthonormé. Calculer la longueur du segment \( [AB]\) et les coordonnées du milieu du segment \( [CD]\).

 $l^2=104$,$l=2*sqrt(26)$,$M=(0.5,7.0)$\item
Est-ce que le triangle formé par les points \( A(-4;8)\), \( B(-8;12)\) et \( C(-5;11)\) est isocèle ?

True
\end{enumerate}
} 
\vspace{2cm}
\vbox{53
\emph{Toutes les réponses doivent être justifiées par un calcul accompagné d'un raisonnement.}
\begin{enumerate}\item
Placer les points \( A=(10;-2)\), \( B=(-10;-4)\), \( C=(6;1)\) et \( D=(5;-4)\) dans un repère orthonormé. Calculer la longueur du segment \( [AB]\) et les coordonnées du milieu du segment \( [CD]\).

 $l^2=404$,$l=2*sqrt(101)$,$M=(5.5,-1.5)$\item
Est-ce que le triangle formé par les points \( A(-7;7)\), \( B(-3;3)\) et \( C(1;1)\) est isocèle ?

False
\end{enumerate}
} 
\vspace{2cm}
\vbox{54
\emph{Toutes les réponses doivent être justifiées par un calcul accompagné d'un raisonnement.}
\begin{enumerate}\item
Placer les points \( A=(-3;9)\), \( B=(5;-6)\), \( C=(-9;-8)\) et \( D=(-10;-4)\) dans un repère orthonormé. Calculer la longueur du segment \( [AB]\) et les coordonnées du milieu du segment \( [CD]\).

 $l^2=289$,$l=17$,$M=(-9.5,-6.0)$\item
Est-ce que le triangle formé par les points \( A(2;6)\), \( B(1;7)\) et \( C(14;8)\) est rectangle ?

False
\end{enumerate}
} 
\vspace{2cm}
\vbox{55
\emph{Toutes les réponses doivent être justifiées par un calcul accompagné d'un raisonnement.}
\begin{enumerate}\item
Placer les points \( A=(-9;-9)\), \( B=(-6;-2)\), \( C=(-1;-4)\) et \( D=(7;10)\) dans un repère orthonormé. Calculer la longueur du segment \( [AB]\) et les coordonnées du milieu du segment \( [CD]\).

 $l^2=58$,$l=sqrt(58)$,$M=(3.0,3.0)$\item
Est-ce que le triangle formé par les points \( A(1;10)\), \( B(2;9)\) et \( C(8;7)\) est rectangle ?

False
\end{enumerate}
} 
\vspace{2cm}
\vbox{56
\emph{Toutes les réponses doivent être justifiées par un calcul accompagné d'un raisonnement.}
\begin{enumerate}\item
Placer les points \( A=(6;-6)\), \( B=(9;-8)\), \( C=(9;7)\) et \( D=(-9;4)\) dans un repère orthonormé. Calculer la longueur du segment \( [AB]\) et les coordonnées du milieu du segment \( [CD]\).

 $l^2=13$,$l=sqrt(13)$,$M=(0.0,5.5)$\item
Est-ce que le triangle formé par les points \( A(-2;7)\), \( B(2;3)\) et \( C(3;8)\) est isocèle ?

True
\end{enumerate}
} 
\vspace{2cm}
\vbox{57
\emph{Toutes les réponses doivent être justifiées par un calcul accompagné d'un raisonnement.}
\begin{enumerate}\item
Placer les points \( A=(-9;4)\), \( B=(-2;9)\), \( C=(8;0)\) et \( D=(-10;-2)\) dans un repère orthonormé. Calculer la longueur du segment \( [AB]\) et les coordonnées du milieu du segment \( [CD]\).

 $l^2=74$,$l=sqrt(74)$,$M=(-1.0,-1.0)$\item
Est-ce que le triangle formé par les points \( A(-4;7)\), \( B(-8;11)\) et \( C(3;8)\) est isocèle ?

False
\end{enumerate}
} 
\vspace{2cm}
\vbox{58
\emph{Toutes les réponses doivent être justifiées par un calcul accompagné d'un raisonnement.}
\begin{enumerate}\item
Placer les points \( A=(-1;2)\), \( B=(-7;-5)\), \( C=(1;1)\) et \( D=(-10;7)\) dans un repère orthonormé. Calculer la longueur du segment \( [AB]\) et les coordonnées du milieu du segment \( [CD]\).

 $l^2=85$,$l=sqrt(85)$,$M=(-4.5,4.0)$\item
Est-ce que le triangle formé par les points \( A(-4;2)\), \( B(-2;0)\) et \( C(9;5)\) est rectangle ?

False
\end{enumerate}
} 
\vspace{2cm}
\vbox{59
\emph{Toutes les réponses doivent être justifiées par un calcul accompagné d'un raisonnement.}
\begin{enumerate}\item
Placer les points \( A=(6;3)\), \( B=(7;-7)\), \( C=(10;-1)\) et \( D=(3;0)\) dans un repère orthonormé. Calculer la longueur du segment \( [AB]\) et les coordonnées du milieu du segment \( [CD]\).

 $l^2=101$,$l=sqrt(101)$,$M=(6.5,-0.5)$\item
Est-ce que le triangle formé par les points \( A(-10;10)\), \( B(-6;6)\) et \( C(6;12)\) est isocèle ?

False
\end{enumerate}
} 
\vspace{2cm}
\vbox{60
\emph{Toutes les réponses doivent être justifiées par un calcul accompagné d'un raisonnement.}
\begin{enumerate}\item
Placer les points \( A=(3;-7)\), \( B=(-9;7)\), \( C=(-8;-5)\) et \( D=(-7;-5)\) dans un repère orthonormé. Calculer la longueur du segment \( [AB]\) et les coordonnées du milieu du segment \( [CD]\).

 $l^2=340$,$l=2*sqrt(85)$,$M=(-7.5,-5.0)$\item
Est-ce que le triangle formé par les points \( A(5;7)\), \( B(4;8)\) et \( C(12;4)\) est rectangle ?

False
\end{enumerate}
} 
\vspace{2cm}
\vbox{61
\emph{Toutes les réponses doivent être justifiées par un calcul accompagné d'un raisonnement.}
\begin{enumerate}\item
Placer les points \( A=(10;-5)\), \( B=(-5;-9)\), \( C=(4;2)\) et \( D=(-4;-3)\) dans un repère orthonormé. Calculer la longueur du segment \( [AB]\) et les coordonnées du milieu du segment \( [CD]\).

 $l^2=241$,$l=sqrt(241)$,$M=(0.0,-0.5)$\item
Est-ce que le triangle formé par les points \( A(2;3)\), \( B(-2;7)\) et \( C(-1;4)\) est isocèle ?

True
\end{enumerate}
} 
\vspace{2cm}
\vbox{62
\emph{Toutes les réponses doivent être justifiées par un calcul accompagné d'un raisonnement.}
\begin{enumerate}\item
Placer les points \( A=(-6;1)\), \( B=(-1;-3)\), \( C=(-8;-3)\) et \( D=(8;-6)\) dans un repère orthonormé. Calculer la longueur du segment \( [AB]\) et les coordonnées du milieu du segment \( [CD]\).

 $l^2=41$,$l=sqrt(41)$,$M=(0.0,-4.5)$\item
Est-ce que le triangle formé par les points \( A(-3;7)\), \( B(-5;9)\) et \( C(10;10)\) est rectangle ?

False
\end{enumerate}
} 
\vspace{2cm}
\vbox{63
\emph{Toutes les réponses doivent être justifiées par un calcul accompagné d'un raisonnement.}
\begin{enumerate}\item
Placer les points \( A=(-7;-4)\), \( B=(-9;-8)\), \( C=(-5;2)\) et \( D=(-3;-1)\) dans un repère orthonormé. Calculer la longueur du segment \( [AB]\) et les coordonnées du milieu du segment \( [CD]\).

 $l^2=20$,$l=2*sqrt(5)$,$M=(-4.0,0.5)$\item
Est-ce que le triangle formé par les points \( A(-1;10)\), \( B(1;8)\) et \( C(9;8)\) est isocèle ?

False
\end{enumerate}
} 
\vspace{2cm}
\vbox{64
\emph{Toutes les réponses doivent être justifiées par un calcul accompagné d'un raisonnement.}
\begin{enumerate}\item
Placer les points \( A=(-9;-6)\), \( B=(10;-3)\), \( C=(7;-10)\) et \( D=(7;-1)\) dans un repère orthonormé. Calculer la longueur du segment \( [AB]\) et les coordonnées du milieu du segment \( [CD]\).

 $l^2=370$,$l=sqrt(370)$,$M=(7.0,-5.5)$\item
Est-ce que le triangle formé par les points \( A(1;-4)\), \( B(5;-8)\) et \( C(10;-5)\) est rectangle ?

False
\end{enumerate}
} 
\vspace{2cm}
\vbox{65
\emph{Toutes les réponses doivent être justifiées par un calcul accompagné d'un raisonnement.}
\begin{enumerate}\item
Placer les points \( A=(-3;9)\), \( B=(7;6)\), \( C=(-4;3)\) et \( D=(1;-1)\) dans un repère orthonormé. Calculer la longueur du segment \( [AB]\) et les coordonnées du milieu du segment \( [CD]\).

 $l^2=109$,$l=sqrt(109)$,$M=(-1.5,1.0)$\item
Est-ce que le triangle formé par les points \( A(5;-7)\), \( B(3;-5)\) et \( C(13;-7)\) est isocèle ?

False
\end{enumerate}
} 
\vspace{2cm}
\vbox{66
\emph{Toutes les réponses doivent être justifiées par un calcul accompagné d'un raisonnement.}
\begin{enumerate}\item
Placer les points \( A=(-8;-7)\), \( B=(0;1)\), \( C=(8;2)\) et \( D=(-8;8)\) dans un repère orthonormé. Calculer la longueur du segment \( [AB]\) et les coordonnées du milieu du segment \( [CD]\).

 $l^2=128$,$l=8*sqrt(2)$,$M=(0.0,5.0)$\item
Est-ce que le triangle formé par les points \( A(-1;-7)\), \( B(3;-11)\) et \( C(12;-4)\) est rectangle ?

False
\end{enumerate}
} 
\vspace{2cm}
\vbox{67
\emph{Toutes les réponses doivent être justifiées par un calcul accompagné d'un raisonnement.}
\begin{enumerate}\item
Placer les points \( A=(3;-4)\), \( B=(5;-2)\), \( C=(-1;9)\) et \( D=(5;-6)\) dans un repère orthonormé. Calculer la longueur du segment \( [AB]\) et les coordonnées du milieu du segment \( [CD]\).

 $l^2=8$,$l=2*sqrt(2)$,$M=(2.0,1.5)$\item
Est-ce que le triangle formé par les points \( A(-7;4)\), \( B(-10;7)\) et \( C(1;2)\) est rectangle ?

False
\end{enumerate}
} 
\vspace{2cm}
\vbox{68
\emph{Toutes les réponses doivent être justifiées par un calcul accompagné d'un raisonnement.}
\begin{enumerate}\item
Placer les points \( A=(9;2)\), \( B=(-6;6)\), \( C=(-1;-8)\) et \( D=(-7;9)\) dans un repère orthonormé. Calculer la longueur du segment \( [AB]\) et les coordonnées du milieu du segment \( [CD]\).

 $l^2=241$,$l=sqrt(241)$,$M=(-4.0,0.5)$\item
Est-ce que le triangle formé par les points \( A(5;-6)\), \( B(9;-10)\) et \( C(20;-5)\) est isocèle ?

False
\end{enumerate}
} 
\vspace{2cm}
\vbox{69
\emph{Toutes les réponses doivent être justifiées par un calcul accompagné d'un raisonnement.}
\begin{enumerate}\item
Placer les points \( A=(-1;2)\), \( B=(0;-4)\), \( C=(-9;-4)\) et \( D=(8;5)\) dans un repère orthonormé. Calculer la longueur du segment \( [AB]\) et les coordonnées du milieu du segment \( [CD]\).

 $l^2=37$,$l=sqrt(37)$,$M=(-0.5,0.5)$\item
Est-ce que le triangle formé par les points \( A(7;-9)\), \( B(13;-15)\) et \( C(9;-13)\) est isocèle ?

True
\end{enumerate}
} 
\vspace{2cm}
\vbox{70
\emph{Toutes les réponses doivent être justifiées par un calcul accompagné d'un raisonnement.}
\begin{enumerate}\item
Placer les points \( A=(-1;0)\), \( B=(-1;2)\), \( C=(-2;5)\) et \( D=(9;-8)\) dans un repère orthonormé. Calculer la longueur du segment \( [AB]\) et les coordonnées du milieu du segment \( [CD]\).

 $l^2=4$,$l=2$,$M=(3.5,-1.5)$\item
Est-ce que le triangle formé par les points \( A(9;-2)\), \( B(6;1)\) et \( C(22;1)\) est rectangle ?

False
\end{enumerate}
} 
\vspace{2cm}
\vbox{71
\emph{Toutes les réponses doivent être justifiées par un calcul accompagné d'un raisonnement.}
\begin{enumerate}\item
Placer les points \( A=(5;-7)\), \( B=(8;6)\), \( C=(-8;9)\) et \( D=(-5;-6)\) dans un repère orthonormé. Calculer la longueur du segment \( [AB]\) et les coordonnées du milieu du segment \( [CD]\).

 $l^2=178$,$l=sqrt(178)$,$M=(-6.5,1.5)$\item
Est-ce que le triangle formé par les points \( A(6;-4)\), \( B(9;-7)\) et \( C(17;-3)\) est rectangle ?

False
\end{enumerate}
} 
\vspace{2cm}
\vbox{72
\emph{Toutes les réponses doivent être justifiées par un calcul accompagné d'un raisonnement.}
\begin{enumerate}\item
Placer les points \( A=(1;-6)\), \( B=(-1;-1)\), \( C=(-4;10)\) et \( D=(-2;-1)\) dans un repère orthonormé. Calculer la longueur du segment \( [AB]\) et les coordonnées du milieu du segment \( [CD]\).

 $l^2=29$,$l=sqrt(29)$,$M=(-3.0,4.5)$\item
Est-ce que le triangle formé par les points \( A(-5;9)\), \( B(-3;7)\) et \( C(0;12)\) est isocèle ?

True
\end{enumerate}
} 
\vspace{2cm}
\vbox{73
\emph{Toutes les réponses doivent être justifiées par un calcul accompagné d'un raisonnement.}
\begin{enumerate}\item
Placer les points \( A=(2;3)\), \( B=(0;4)\), \( C=(-6;-8)\) et \( D=(6;-4)\) dans un repère orthonormé. Calculer la longueur du segment \( [AB]\) et les coordonnées du milieu du segment \( [CD]\).

 $l^2=5$,$l=sqrt(5)$,$M=(0.0,-6.0)$\item
Est-ce que le triangle formé par les points \( A(7;-9)\), \( B(6;-8)\) et \( C(20;-6)\) est rectangle ?

False
\end{enumerate}
} 
\vspace{2cm}
\vbox{74
\emph{Toutes les réponses doivent être justifiées par un calcul accompagné d'un raisonnement.}
\begin{enumerate}\item
Placer les points \( A=(-1;1)\), \( B=(1;7)\), \( C=(-7;-7)\) et \( D=(8;-2)\) dans un repère orthonormé. Calculer la longueur du segment \( [AB]\) et les coordonnées du milieu du segment \( [CD]\).

 $l^2=40$,$l=2*sqrt(10)$,$M=(0.5,-4.5)$\item
Est-ce que le triangle formé par les points \( A(6;-6)\), \( B(10;-10)\) et \( C(20;-6)\) est isocèle ?

False
\end{enumerate}
} 
\vspace{2cm}
\vbox{75
\emph{Toutes les réponses doivent être justifiées par un calcul accompagné d'un raisonnement.}
\begin{enumerate}\item
Placer les points \( A=(2;-7)\), \( B=(-1;-9)\), \( C=(-4;10)\) et \( D=(4;-4)\) dans un repère orthonormé. Calculer la longueur du segment \( [AB]\) et les coordonnées du milieu du segment \( [CD]\).

 $l^2=13$,$l=sqrt(13)$,$M=(0.0,3.0)$\item
Est-ce que le triangle formé par les points \( A(10;0)\), \( B(4;6)\) et \( C(13;-1)\) est isocèle ?

False
\end{enumerate}
} 
\vspace{2cm}
\vbox{76
\emph{Toutes les réponses doivent être justifiées par un calcul accompagné d'un raisonnement.}
\begin{enumerate}\item
Placer les points \( A=(-2;-8)\), \( B=(-10;-5)\), \( C=(3;-10)\) et \( D=(-8;7)\) dans un repère orthonormé. Calculer la longueur du segment \( [AB]\) et les coordonnées du milieu du segment \( [CD]\).

 $l^2=73$,$l=sqrt(73)$,$M=(-2.5,-1.5)$\item
Est-ce que le triangle formé par les points \( A(6;-5)\), \( B(10;-9)\) et \( C(11;-4)\) est isocèle ?

True
\end{enumerate}
} 
\vspace{2cm}
\vbox{77
\emph{Toutes les réponses doivent être justifiées par un calcul accompagné d'un raisonnement.}
\begin{enumerate}\item
Placer les points \( A=(9;-1)\), \( B=(-10;-1)\), \( C=(-3;4)\) et \( D=(6;4)\) dans un repère orthonormé. Calculer la longueur du segment \( [AB]\) et les coordonnées du milieu du segment \( [CD]\).

 $l^2=361$,$l=19$,$M=(1.5,4.0)$\item
Est-ce que le triangle formé par les points \( A(0;-9)\), \( B(4;-13)\) et \( C(1;-12)\) est isocèle ?

True
\end{enumerate}
} 
\vspace{2cm}
\vbox{78
\emph{Toutes les réponses doivent être justifiées par un calcul accompagné d'un raisonnement.}
\begin{enumerate}\item
Placer les points \( A=(-7;-3)\), \( B=(7;-5)\), \( C=(3;-6)\) et \( D=(0;-3)\) dans un repère orthonormé. Calculer la longueur du segment \( [AB]\) et les coordonnées du milieu du segment \( [CD]\).

 $l^2=200$,$l=10*sqrt(2)$,$M=(1.5,-4.5)$\item
Est-ce que le triangle formé par les points \( A(6;-9)\), \( B(9;-12)\) et \( C(4;-11)\) est rectangle ?

True
\end{enumerate}
} 
\vspace{2cm}
\vbox{79
\emph{Toutes les réponses doivent être justifiées par un calcul accompagné d'un raisonnement.}
\begin{enumerate}\item
Placer les points \( A=(0;-8)\), \( B=(8;0)\), \( C=(-8;0)\) et \( D=(-7;6)\) dans un repère orthonormé. Calculer la longueur du segment \( [AB]\) et les coordonnées du milieu du segment \( [CD]\).

 $l^2=128$,$l=8*sqrt(2)$,$M=(-7.5,3.0)$\item
Est-ce que le triangle formé par les points \( A(10;8)\), \( B(6;12)\) et \( C(17;5)\) est rectangle ?

False
\end{enumerate}
} 
\vspace{2cm}
\vbox{80
\emph{Toutes les réponses doivent être justifiées par un calcul accompagné d'un raisonnement.}
\begin{enumerate}\item
Placer les points \( A=(-2;2)\), \( B=(-3;-1)\), \( C=(-9;4)\) et \( D=(-1;5)\) dans un repère orthonormé. Calculer la longueur du segment \( [AB]\) et les coordonnées du milieu du segment \( [CD]\).

 $l^2=10$,$l=sqrt(10)$,$M=(-5.0,4.5)$\item
Est-ce que le triangle formé par les points \( A(9;-10)\), \( B(5;-6)\) et \( C(18;-7)\) est isocèle ?

False
\end{enumerate}
} 
\vspace{2cm}
\vbox{81
\emph{Toutes les réponses doivent être justifiées par un calcul accompagné d'un raisonnement.}
\begin{enumerate}\item
Placer les points \( A=(10;3)\), \( B=(10;-8)\), \( C=(-2;2)\) et \( D=(-9;7)\) dans un repère orthonormé. Calculer la longueur du segment \( [AB]\) et les coordonnées du milieu du segment \( [CD]\).

 $l^2=121$,$l=11$,$M=(-5.5,4.5)$\item
Est-ce que le triangle formé par les points \( A(-10;7)\), \( B(-12;9)\) et \( C(-12;5)\) est rectangle ?

True
\end{enumerate}
} 
\vspace{2cm}
\vbox{82
\emph{Toutes les réponses doivent être justifiées par un calcul accompagné d'un raisonnement.}
\begin{enumerate}\item
Placer les points \( A=(-8;-6)\), \( B=(-9;-8)\), \( C=(-6;8)\) et \( D=(6;0)\) dans un repère orthonormé. Calculer la longueur du segment \( [AB]\) et les coordonnées du milieu du segment \( [CD]\).

 $l^2=5$,$l=sqrt(5)$,$M=(0.0,4.0)$\item
Est-ce que le triangle formé par les points \( A(6;-9)\), \( B(3;-6)\) et \( C(7;-8)\) est rectangle ?

True
\end{enumerate}
} 
\vspace{2cm}
\vbox{83
\emph{Toutes les réponses doivent être justifiées par un calcul accompagné d'un raisonnement.}
\begin{enumerate}\item
Placer les points \( A=(0;-9)\), \( B=(9;-1)\), \( C=(2;8)\) et \( D=(10;-8)\) dans un repère orthonormé. Calculer la longueur du segment \( [AB]\) et les coordonnées du milieu du segment \( [CD]\).

 $l^2=145$,$l=sqrt(145)$,$M=(6.0,0.0)$\item
Est-ce que le triangle formé par les points \( A(2;4)\), \( B(4;2)\) et \( C(9;-1)\) est isocèle ?

False
\end{enumerate}
} 
\vspace{2cm}
\vbox{84
\emph{Toutes les réponses doivent être justifiées par un calcul accompagné d'un raisonnement.}
\begin{enumerate}\item
Placer les points \( A=(1;-4)\), \( B=(7;-5)\), \( C=(-4;-4)\) et \( D=(-8;4)\) dans un repère orthonormé. Calculer la longueur du segment \( [AB]\) et les coordonnées du milieu du segment \( [CD]\).

 $l^2=37$,$l=sqrt(37)$,$M=(-6.0,0.0)$\item
Est-ce que le triangle formé par les points \( A(-10;3)\), \( B(-4;-3)\) et \( C(-6;1)\) est isocèle ?

True
\end{enumerate}
} 
\vspace{2cm}


\end{document}

% FONCTION : MODE GRAPHIQUE

%This is part of Un soupçon de mathématique sans être agressif pour autant
% Copyright (c) 2012-2013
%   Laurent Claessens
% See the file fdl-1.3.txt for copying conditions.

    Un berger syldave s'entraine pour le championnat national du lancer de chèvre. L'épreuve consiste à lancer une chèvre vers le haut depuis le bord d'une falaise située au bord d'un lac tranquille. La hauteur de la chèvre en fonction du temps par rapport à la surface du lac tranquille est une fonction \( f\) donnée par le graphique suivant.

    \begin{center}
        \input{Fig_WRXbDCo.pstricks}
    \end{center}
    La dernière partie du graphique correspond à la chèvre que l'on remonte rapidement hors de l'eau.
    À partir du graphique :
    \begin{enumerate}
        \item
            À quelle hauteur se trouve la chèvre au moment du lancer ?
        \item
            Pendant combien de temps la chèvre reste à une hauteur supérieure à celle à laquelle elle a été lancée ?
        \item
            À quel moment la chèvre atteint-elle sa hauteur maximale ? Quelle est cette hauteur ?
        \item
            À quelle hauteur se trouve la chèvre après \( 2.5\) secondes de vol ?
        \item
            Résumer toutes ces informations en dressant le tableau de variation de la fonction \( f\).
    \end{enumerate}

\vspace{1cm}
%This is part of Un soupçon de mathématique sans être agressif pour autant
% Copyright (c) 2012-2013
%   Laurent Claessens
% See the file fdl-1.3.txt for copying conditions.

    Un berger syldave s'entraine pour le championnat national du lancer de chèvre. L'épreuve consiste à lancer une chèvre vers le haut depuis le bord d'une falaise située au bord d'un lac tranquille. La hauteur de la chèvre en fonction du temps par rapport à la surface du lac tranquille est une fonction \( f\) donnée par le graphique suivant.

    \begin{center}
        \input{Fig_WRXbDCo.pstricks}
    \end{center}
    La dernière partie du graphique correspond à la chèvre que l'on remonte rapidement hors de l'eau.
    À partir du graphique :
    \begin{enumerate}
        \item
            À quelle hauteur se trouve la chèvre au moment du lancer ?
        \item
            Pendant combien de temps la chèvre reste à une hauteur supérieure à celle à laquelle elle a été lancée ?
        \item
            À quel moment la chèvre atteint-elle sa hauteur maximale ? Quelle est cette hauteur ?
        \item
            À quelle hauteur se trouve la chèvre après \( 2.5\) secondes de vol ?
        \item
            Résumer toutes ces informations en dressant le tableau de variation de la fonction \( f\).
    \end{enumerate}


\end{document}


\clearpage

% FONCTIONS AFFINES

%This is part of Un soupçon de mathématique sans être agressif pour autant
% Copyright (c) 2012-2013
%   Laurent Claessens
% See the file fdl-1.3.txt for copying conditions.

    Le taxi Besacdanslesac divise son prix en deux parties : $0.2$ euros de frais de prise en charge plus un euro par km parcouru. Le taxi Ledoubstoudoux par contre divise son prix en $1$ euro de frais de prise en charge plus $0.8$ euros par kilomètre parcouru.

    \begin{enumerate}
        \item
            Combien coûte un trajet de \SI{5}{\kilo\meter} avec Besacdanslesac ?
        \item
            Donner une expression algébrique du prix d'une course en fonction du nombre de kilomètres parcourus.
        \item
            Combien de kilomètres peut-t-on effectuer dans Ledoubstoudoux avec \( 10\) euros ?
        \item
            Exprimer les prix en fonction du nombre de kilomètres parcourus sur un graphique (les deux taxis sur le même graphique).
        \item
            À partir de combien de kilomètres parcourus vaut-il mieux prendre Ledoubstoudoux ?
    \end{enumerate}


\vspace{1cm}

%This is part of Un soupçon de mathématique sans être agressif pour autant
% Copyright (c) 2012-2013
%   Laurent Claessens
% See the file fdl-1.3.txt for copying conditions.

    Le taxi Besacdanslesac divise son prix en deux parties : $0.2$ euros de frais de prise en charge plus un euro par km parcouru. Le taxi Ledoubstoudoux par contre divise son prix en $1$ euro de frais de prise en charge plus $0.8$ euros par kilomètre parcouru.

    \begin{enumerate}
        \item
            Combien coûte un trajet de \SI{5}{\kilo\meter} avec Besacdanslesac ?
        \item
            Donner une expression algébrique du prix d'une course en fonction du nombre de kilomètres parcourus.
        \item
            Combien de kilomètres peut-t-on effectuer dans Ledoubstoudoux avec \( 10\) euros ?
        \item
            Exprimer les prix en fonction du nombre de kilomètres parcourus sur un graphique (les deux taxis sur le même graphique).
        \item
            À partir de combien de kilomètres parcourus vaut-il mieux prendre Ledoubstoudoux ?
    \end{enumerate}


\vspace{1cm}
%This is part of Un soupçon de mathématique sans être agressif pour autant
% Copyright (c) 2012-2013
%   Laurent Claessens
% See the file fdl-1.3.txt for copying conditions.

    Le taxi Besacdanslesac divise son prix en deux parties : $0.2$ euros de frais de prise en charge plus un euro par km parcouru. Le taxi Ledoubstoudoux par contre divise son prix en $1$ euro de frais de prise en charge plus $0.8$ euros par kilomètre parcouru.

    \begin{enumerate}
        \item
            Combien coûte un trajet de \SI{5}{\kilo\meter} avec Besacdanslesac ?
        \item
            Donner une expression algébrique du prix d'une course en fonction du nombre de kilomètres parcourus.
        \item
            Combien de kilomètres peut-t-on effectuer dans Ledoubstoudoux avec \( 10\) euros ?
        \item
            Exprimer les prix en fonction du nombre de kilomètres parcourus sur un graphique (les deux taxis sur le même graphique).
        \item
            À partir de combien de kilomètres parcourus vaut-il mieux prendre Ledoubstoudoux ?
    \end{enumerate}


\vspace{1cm}
%This is part of Un soupçon de mathématique sans être agressif pour autant
% Copyright (c) 2012-2013
%   Laurent Claessens
% See the file fdl-1.3.txt for copying conditions.

    Le taxi Besacdanslesac divise son prix en deux parties : $0.2$ euros de frais de prise en charge plus un euro par km parcouru. Le taxi Ledoubstoudoux par contre divise son prix en $1$ euro de frais de prise en charge plus $0.8$ euros par kilomètre parcouru.

    \begin{enumerate}
        \item
            Combien coûte un trajet de \SI{5}{\kilo\meter} avec Besacdanslesac ?
        \item
            Donner une expression algébrique du prix d'une course en fonction du nombre de kilomètres parcourus.
        \item
            Combien de kilomètres peut-t-on effectuer dans Ledoubstoudoux avec \( 10\) euros ?
        \item
            Exprimer les prix en fonction du nombre de kilomètres parcourus sur un graphique (les deux taxis sur le même graphique).
        \item
            À partir de combien de kilomètres parcourus vaut-il mieux prendre Ledoubstoudoux ?
    \end{enumerate}




\end{document}
