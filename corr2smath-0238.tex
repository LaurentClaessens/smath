% This is part of Un soupçon de mathématique sans être agressif pour autant
% Copyright (c) 2015
%   Laurent Claessens
% See the file fdl-1.3.txt for copying conditions.

\begin{corrige}{2smath-0238}

    Pour rappel, \( \frac{ 3 }{ 4 }=0.75\). Voici les points sur un axe gradué :
\begin{center}
   \input{Fig_XKZSooPNxqjW.pstricks}
\end{center}

En ce qui concerne la seconde partie,
\begin{enumerate}
    \item   \label{ItemKNJTooVpnCJz}
        \( 4-(-3)=7\). C'est la distance entre les points \( E\) et \( A\) ou \( B\) et \( C\).
    \item
        \( \frac{ 3 }{ 4 }+3=3.75\) est la distance entre \( D\) et \( A\).
    \item
        \( 4-3=1\) est la distance entre \( A\) et \( C\) ou \( E\) et \( B\)
    \item
        \( 4+3=7\), même points que \ref{ItemKNJTooVpnCJz}
\end{enumerate}

\end{corrige}
