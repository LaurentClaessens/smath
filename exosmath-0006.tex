% This is part of Un soupçon de mathématique sans être agressif pour autant
% Copyright (c) 2012
%   Laurent Claessens
% See the file fdl-1.3.txt for copying conditions.

\begin{exercice}\label{exosmath-0006}

    \begin{multicols}{2}
    Soit la fonction \( f\) donnée par le graphique ci-contre.
        \begin{enumerate}
            \item
                Donner le domaine de \( f\).
            \item
                Donner les racines de \( f\).
            \item 
                Tracer la droite \( (AB)\).
            \item
                Si \( g\) est la fonction dont le graphe est la droite \( (AB)\), résoudre \( f(x)=g(x)\).
        \end{enumerate}


        \columnbreak
%The result is on figure \ref{LabelFigGraphInterfQVfSf}. % From file GraphInterfQVfSf
%\newcommand{\CaptionFigGraphInterfQVfSf}{<+Type your caption here+>}
\input{Fig_GraphInterfQVfSf.pstricks}


    \end{multicols}
    Les réponses étant lues sur le graphique, elles sont forcément approximatives et aucun calcul n'est nécessaire. Quelque mots de justification sont \emph{toujours} nécessaires.

\corrref{smath-0006}
\end{exercice}
