% This is part of Un soupçon de mathématique sans être agressif pour autant
% Copyright (c) 2014
%   Laurent Claessens
% See the file fdl-1.3.txt for copying conditions.

%+++++++++++++++++++++++++++++++++++++++++++++++++++++++++++++++++++++++++++++++++++++++++++++++++++++++++++++++++++++++++++ 
\section*{Deux cinquièmes ou un cinquième de deux ?}
%+++++++++++++++++++++++++++++++++++++++++++++++++++++++++++++++++++++++++++++++++++++++++++++++++++++++++++++++++++++++++++

Dans laquelle de ces deux situations vous avez le plus à manger ? Dans la première situation vous êtes cinq à vous partager un gâteau, mais votre voisin vous donne sa part. Dans la seconde situation, il y a deux gâteaux à partager en cinq, mais votre voisin garde sa part.

%+++++++++++++++++++++++++++++++++++++++++++++++++++++++++++++++++++++++++++++++++++++++++++++++++++++++++++++++++++++++++++ 
\section*{Secteurs d'émissions}
%+++++++++++++++++++++++++++++++++++++++++++++++++++++++++++++++++++++++++++++++++++++++++++++++++++++++++++++++++++++++++++

Sur cette planète, un cinquième des émissions de dioxyde de carbone sont dus aux processus industriels, un autre cinquième aux transports et un dixième aux bâtiments. Quelle fraction du total des émissions de \( CO_2\) est due à ces trois activités ?

Les centrales énergétiques sont responsables d'un tiers des émissions de dioxyde de carbone. Quel est le total de ces quatre activités ?

Il semble que l'agriculture ne soit pour rien dans cette histoire ? En fait elle est responsable de \( 40\%\) des émissions de méthane et de \( 62\%\) des émissions d'oxydes d'azote. Au total, l'agriculture produit \( 12\%\) des émissions de gaz à effet de serre. Comment est-ce possible que ce \( 12\) soit plus petit que le \( 40\) et le \( 62\) ?

