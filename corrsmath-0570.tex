% This is part of Un soupçon de mathématique sans être agressif pour autant
% Copyright (c) 2013
%   Laurent Claessens
% See the file fdl-1.3.txt for copying conditions.

\begin{corrige}{smath-0570}

        \begin{center}
            \input{Fig_EJbcoxO.pstricks}
        \end{center}

\begin{enumerate}
    \item
        Pour résoudre graphiquement \( f(x)=1\) il faut trouver tous les points du graphe de \( f\) situés à la hauteur \( 1\). Pour cela nous traçons la droite \( y=1\) (celle en rouge) et nous trouvons les intersections avec le graphe. Ce sont les points \( A\) et \( B\). Les solutions à l'équation \( f(x)=1\) sont les abscisses de ces points :
        \begin{equation}
            S=\{ 0.5;1.25 \}.
        \end{equation}
        
        En ce qui concerne l'équation \( f(x)=2\), il n'y a pas de solutions parce qu'à aucun moment le graphe de \( f\) n'arrive à la hauteur \( 2\).

    \item

        Entre les abscisses \( -2\) et \( 1\), le graphe de \( f\) est une courbe qui descend, donc la fonction est décroissante.

    \item

        Les antécédents de zéro sont les solutions de \( f(x)=0\). Il s'agit de trouver les points où la courbe coupe l'axe horizontal; ce sont les points \( Z_1\), \( Z_2\) et \( Z_3\) sur le dessin. Les solutions sont
        \begin{equation}
            S=\{ -1.9;0;1.9 \}.
        \end{equation}
        
    \item

        Le point le plus haut d'abscisse entre \( -1\) et \( 1\) est le point noté \( M\). Sa hauteur (ordonnée) est environ \( 1.25\). Donc le maximum recherché est \( 1.25\) et il est atteint pour \( x=0.8\).

    \item

        \( f(0)\) vaut zéro. La courbe passe par l'origine.

    \item

        Le tableau de variations est celui avec les flèches (à ne pas confondre avec celui de signe) :

        \begin{equation*}
            \begin{array}[]{c|ccccccc}
                x&2&&-0.8&&0.8&&2\\
                \hline
                &0.1&&&&1.3&&\\
                f(x)&&\searrow&&\nearrow&&\searrow&\\
                &&&-1.3&&&&-0.1\\
            \end{array}
        \end{equation*}
        
    \item

        Certes le graphe de la fonction passe par les points \( (0;0)\) et \( (\frac{ 1 }{2};1)\), mais le graphe n'étant manifestement pas une droite, la fonction n'est ni linéaire ni affine. Albert se trompe du tout au tout.

\end{enumerate}


\end{corrige}
