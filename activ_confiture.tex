% This is part of Un soupçon de mathématique sans être agressif pour autant
% Copyright (c) 2014
%   Laurent Claessens
% See the file fdl-1.3.txt for copying conditions.

%--------------------------------------------------------------------------------------------------------------------------- 
\subsection*{Confiture sucrée}
%---------------------------------------------------------------------------------------------------------------------------

Après un bel été bien ensoleillé, Philippe souhaite faire de la confiture pas trop sucrée. En regardant sur Internet, elle trouve trois recettes.

\begin{center}
    \begin{tabular}[]{|c|c|}
        \hline
        Confiture de fraises&«\unit{450}{\gram} de sucre pour \unit{750}{\gram} de fraises» \\
        \hline
        Confiture d'abricots& «\unit{500}{\gram} de sucre pour \unit{1}{\kilo\gram} de confiture» \\
        \hline
        Confiture de cerises&  «\unit{800}{\gram} de sucre pour \unit{2400}{\gram} de cerises» \\ 
        \hline
    \end{tabular}
\end{center}


\begin{enumerate}
    \item
Pour chaque recette, exprimer la proportion de sucre ajouté dans la confiture sous forme de fraction.
b. Simplifie le plus possible les fractions obtenues à la question précédente.
1
c. Que signifie une proportion de sucre ajouté supérieure à ?
2
2. Émilie cherche à savoir quelle est la recette avec le moins de sucre ajouté. Elle fait le
raisonnement suivant : « C'est dans la confiture de fraises qu'on retrouve la masse de sucre
ajouté la moins importante (450 g), c'est donc dans la confiture de fraises qu'il y a le moins
de sucre ajouté. ». Que penses-tu de son raisonnement ?
3. La moins sucrée
a. Pour chaque fruit, indique le poids de sucre ajouté nécessaire pour réaliser un
kilogramme de confiture.
b. Pour chaque confiture, écris la proportion de sucre ajouté sous forme d'une fraction de
dénominateur 1 000.
c. Quelle est la confiture qui contient le moins de sucre ajouté en proportion ?
4. En reprenant les fractions obtenues à la question 1. b., trouve le plus petit
dénominateur commun permettant de comparer les trois fractions.

\end{enumerate}
<++>
