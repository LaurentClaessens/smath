% This is part of Un soupçon de mathématique sans être agressif pour autant
% Copyright (c) 2014
%   Laurent Claessens
% See the file fdl-1.3.txt for copying conditions.

\begin{corrige}{smath-0886}

    Pour construire ce triangle, le mieux est de commencer par dessiner un segment \( [FG]\) de \SI{6}{\centi\meter} et de construire un angle de \SI{45}{\degree} sur le point \( G\). Dans un premier temps nous dessinons donc

\begin{center}
   \input{Fig_QAHooOhyyHI0.pstricks}
\end{center}
Il faut ensuite placer le point \( H\) sur la ligne pointillée, à une distance \SI{7}{\centi\meter} de \( G\). Cela se fait à la règle.
\begin{center}
   \input{Fig_QAHooOhyyHI1.pstricks}
\end{center}
Il suffit maintenant de tracer le triangle et de mesurer les angles au rapporteur. Ils mesurent respectivement (environ) \( 57\) et \( 78\) degrés.
\begin{center}
   \input{Fig_QAHooOhyyHI2.pstricks}
\end{center}

\end{corrige}
