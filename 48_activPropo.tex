% This is part of Un soupçon de mathématique sans être agressif pour autant
% Copyright (c) 2015
%   Laurent Claessens
% See the file fdl-1.3.txt for copying conditions.

Lu sur une étiquette pâte à tartiner au chocolat et noisettes :
\begin{oframed}
    
\begin{center}
Valeurs nutritionnelles moyennes pour \SI{100}{\gram}
\end{center}

\begin{itemize}
    \item Énergie : \SI{609}{\kilo\cal}; (\SI{2528.4}{\kilo\joule})
    \item Matières grasses : \SI{46.7}{\gram}; dont acides gras saturés : \SI{16.3}{\gram}
    \item Glucides : \SI{39.6}{\gram}; dont sucres :  \SI{38.8}{\gram}
    \item Fibres : \SI{4.3}{\gram}
    \item Protéines : \SI{5.5}{\gram}
    \item Sel : \SI{0.08}{\gram}.
\end{itemize}

\end{oframed}
\begin{enumerate}
    \item
        Quelle est la proportion de fibres dans cette délicieuse pâte ?
    \item
        Bertande la gourmande tartine \SI{20}{\gram} de cette pâte sur \SI{50}{\gram} de pain. Quelle est la proportion de pain dans ce repas ?
    \item
        Sachant que ledit pain contient \SI{57.7}{\gram} de glucides pour \SI{100}{\gram}, quel est le pourcentage de glucides dans ce repas ?
\end{enumerate}
