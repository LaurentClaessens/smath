% This is part of Un soupçon de mathématique sans être agressif pour autant
% Copyright (c) 2012-2013
%   Laurent Claessens
% See the file fdl-1.3.txt for copying conditions.

\begin{exercice}\label{exoSeconde-0070}

\begin{wrapfigure}{r}{8.0cm}
   \vspace{-1.5cm}        % à adapter.
   \centering
   \input{Fig_ExResolutionOSiaMS.pstricks}
\end{wrapfigure}

        À l'aide du graphique ci-contre, donner les solutions (approximatives) des équations
        \begin{enumerate}
            \item
                \( f(x)=1\)
            \item
                \( g(x)=3\)
            \item
                \( f(x)=-2\)
            \item
                $f(x)=g(x)$.
        \end{enumerate}

        %À partir du même graphique, dire quels sont les points du type \( (2,a)\) à être sur le graphe de \( g\) ?

\corrref{Seconde-0070}
\end{exercice}
