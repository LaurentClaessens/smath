% This is part of Un soupçon de mathématique sans être agressif pour autant
% Copyright (c) 2013
%   Laurent Claessens
% See the file fdl-1.3.txt for copying conditions.

\begin{corrige}{smath-0568}

    \begin{enumerate}
        \item
            Nous pouvons commencer par simplifier tout par $3$ :
            \begin{equation}
                \frac{ 3x^2+6x }{ 18x }=\frac{ x^2+2x }{ 6x },
            \end{equation}
            et ensuite par \( x\) :
            \begin{equation}
                \frac{ 3x^2+6x }{ 18x }=\frac{ x^2+2x }{ 6x }=\frac{ x+2 }{ 6 }.
            \end{equation}
        \item
            Il n'y a pas grand chose à dire : \( \sqrt{23}\). C'est une valeur exacte. Inutile de donner des approximations numériques.
        \item
            C'est un produit remarquable :
            \begin{equation}
                (x+3)^2=x^2+6x+9.
            \end{equation}
        \item
            Règle du produit nul : si un produit est nul, alors un des deux facteurs est nul. Pour résoudre il faut donc résoudre séparément \( x-3=0\) et \( x+12=0\). Les deux solutions sont donc
            \begin{equation}
                S=\{ 3,-12 \}.
            \end{equation}
        \item
            Il faut remplacer \( x\) par \( -4\) en n'oubliant pas que \( (-4)^2=16\) :
            \begin{equation}
                f(-4)=16-8=8.
            \end{equation}
            

    \end{enumerate}
    

\end{corrige}
