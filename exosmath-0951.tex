% This is part of Un soupçon de mathématique sans être agressif pour autant
% Copyright (c) 2014
%   Laurent Claessens
% See the file fdl-1.3.txt for copying conditions.

\begin{exercice}\label{exosmath-0951}

    Compléter les pointillés.
    \begin{multicols}{2}
        \begin{enumerate}
            \item
                \( \dfrac{ 1 }{ 2 }=\dfrac{ 10 }{ \ldots }\)
            \item
                \( \dfrac{ 4 }{ 5 }=\dfrac{ \ldots }{ 15 }\)
            \item
                \( \dfrac{ 7 }{ \ldots }=\dfrac{ 1 }{ 3 }\)
            \item
                \( \dfrac{ 12 }{ 100 }=\dfrac{ 120 }{ \ldots }\)
            \item
                \( \dfrac{ 34 }{ 20 }=\dfrac{ \ldots }{ 100 }\)
            \item
                \( \dfrac{ 25 }{ 45 }=\dfrac{ \ldots }{ 100 }\)
            \item
                \( \dfrac{ 12 }{ 15 }=\dfrac{ \ldots }{ 100 }\)
        \end{enumerate}
    \end{multicols}

\corrref{smath-0951}
\end{exercice}
