% This is part of Un soupçon de mathématique sans être agressif pour autant
% Copyright (c) 2012
%   Laurent Claessens
% See the file fdl-1.3.txt for copying conditions.

\begin{exercice}\label{exosmath-0112}

    Soit un triangle \( ABC\). Nous nommons \( I\), \( J\) et \( K\) les milieux respectifs des côtés \( [AB]\), \( [AC]\) et \( [BC]\).
    \begin{enumerate}
        \item
            Prouver que \( (IJ)\) est parallèle à \( (BC)\) en partant de l'égalité vectorielle \( \vect{ BC }=\vect{ BI }+\vect{ IJ }+\vect{ JC }\).
        \item
            Prouver que \( \vect{ BK }=\vect{ KC }=\vect{ IJ }\).
        \item
            Pour quelle valeur de \( s\) a-t-on \( \vect{ BC }=s\vect{ IJ }\) ?
    \end{enumerate}

\corrref{smath-0112}
\end{exercice}
