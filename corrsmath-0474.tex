% This is part of Un soupçon de mathématique sans être agressif pour autant
% Copyright (c) 2013
%   Laurent Claessens
% See the file fdl-1.3.txt for copying conditions.

\begin{corrige}{smath-0474}

    \begin{wrapfigure}[4]{r}{5.0cm}
   \vspace{-0.5cm}        % à adapter.
   \centering
   \input{Fig_APYEkIv.pstricks}
\end{wrapfigure}

Le dessin ci-contre n'est pas à l'échelle, mais il montre comment il faut utiliser le théorème de Pythagore pour calculer la distance entre le joueur et l'ennemi. Le point \( J\) représente le joueur et le point \( E\), l'ennemi. Le théorème de Pythagore donne que la distance est
\begin{equation}
    EJ=\sqrt{15^2+230^2}=\sqrt{53125}\simeq=230.5.
\end{equation}
La distance étant de \unit{230}{\meter}, le fusil est capable de le toucher.

\end{corrige}
