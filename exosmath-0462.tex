% This is part of Un soupçon de mathématique sans être agressif pour autant
% Copyright (c) 2013
%   Laurent Claessens
% See the file fdl-1.3.txt for copying conditions.

% Cet exo était pour une interro.
\begin{exercice}\label{exosmath-0462}

    \vspace{2cm}

    Résoudre \( (x+3)(2x-7)\geq 0\). Indice : un tableau de signe peut vous aider.
    \vspace{2cm}

    Résoudre \( (x-3)(5x-7)\leq 0\). Indice : un tableau de signe peut vous aider.
    \vspace{2cm}

    Résoudre \( (x+2)(2x-5)\geq 0\). Indice : un tableau de signe peut vous aider.
    \vspace{2cm}

    Résoudre \( (x-2)(2x+5)\leq 0\). Indice : un tableau de signe peut vous aider.
    \vspace{2cm}

    Résoudre \( (-x+6)(4x+5)\geq 0\). Indice : un tableau de signe peut vous aider.
    \vspace{2cm}

    Factoriser l'expression \( 3x+5-(x+2)(3x+5)\). Pour quelle valeur de \( x\) est-elle nulle ?
    \vspace{2cm}

    Factoriser l'expression \( (x+2)(3x+1)+3x+1\). Pour quelle valeur de \( x\) est-elle nulle ?
    \vspace{2cm}

    Factoriser l'expression \( 2x+5-(x+2)(2x+5)\). Pour quelle valeur de \( x\) est-elle nulle ?
    \vspace{2cm}

    Factoriser l'expression \( (x+2)(6x+4)+x+2\). Pour quelle valeur de \( x\) est-elle nulle ?
    \vspace{2cm}

\corrref{smath-0462}
\end{exercice}
