% This is part of Un soupçon de mathématique sans être agressif pour autant
% Copyright (c) 2015
%   Laurent Claessens
% See the file fdl-1.3.txt for copying conditions.
%--------------------------------------------------------------------------------------------------------------------------- 
\subsection*{Activité : prix en solde}
%---------------------------------------------------------------------------------------------------------------------------

Un magasin fait des soldes «\( -20\%\)» et donne le tableau suivant pour aider le clients à savoir les prix soldés en fonction du prix non soldé :
\begin{equation*}
    \begin{array}[]{|c||c|c|c|c|}
        \hline
        \text{prix non soldé}&50&70&130&200\\
        \hline\hline
        \text{prix soldé}&40&56&104&160\\
        \hline
    \end{array}
\end{equation*}
\begin{enumerate}
    \item
        Le prix soldé est-il proportionnel au prix non soldé ?
    \item
        Combien coûterait un article dont le prix non soldé est de \( 120\)€, et \( 150\)€ ?
\end{enumerate}
