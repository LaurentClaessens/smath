% This is part of Un soupçon de mathématique sans être agressif pour autant
% Copyright (c) 2012
%   Laurent Claessens
% See the file fdl-1.3.txt for copying conditions.

\begin{exercice}\label{exoPremiere-0087}

    Un élève a un DS sur les tables de multiplications\footnote{C'était le temps avant les calculatrices; aujourd'hui, ça ne sert plus à rien.}. Il a \( 20\) calculs à faire et on note sur \( 20\), un point gagné par réponse correcte. Il répond systématiquement une réponse au hasard entre \( 0\) et \( 100\).
    \begin{enumerate}
        \item
            Déterminer la loi binomiale sous-jacente. Attention : entre \( 0\) et \( 100\) compris, il n'y a pas \( 100\) nombres ! Il y en a \( 101\).
        \item
            Quelle est la probabilité qu'il obtienne exactement  \( 10/20\) ?
        \item
            Quelle est sa probabilité de réussir ? C'est à dire d'obtenir \( 10\) ou plus.
        \item
            Quelle est sa probabilité d'obtenir \( 3/20\) ou moins ?
    \end{enumerate}
    Pour cet exercice, vous avez besoin de bouquiner le mode d'emploi de votre calculatrice pour trouver comment on calcule une probabilité cumulée.

\corrref{Premiere-0087}
\end{exercice}
