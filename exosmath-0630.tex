% This is part of Un soupçon de mathématique sans être agressif pour autant
% Copyright (c) 2014
%   Laurent Claessens
% See the file fdl-1.3.txt for copying conditions.

\begin{exercice}\label{exosmath-0630}

    Pour chacune des questions suivantes, une seule réponse est correcte.
    \begin{multicols}{2}
    \begin{enumerate}
        \item
           L'algorithme

    \begin{fmpage}{0.9\linewidth}

        Demander \( x\)

    Si  \( x\leq 10\) , alors :

    \hspace{0.5cm} \( k=3x\)

    \hspace{0.5cm} Écrire \( k\) 

    Sinon :

    \hspace{0.5cm} \( k=-x+7\)

    \hspace{0.5cm} Écrire \( k\) 

\end{fmpage}

    \begin{enumerate}
        \item
            Écrit \( -3\) si l'utilisateur rentre \( -1\).
        \item
            Écrit \( 60 \) si l'utilisateur rentre \( 20\).
        \item
            Ne fait rien si l'utilisateur rentre \( 10\).
    \end{enumerate}
    
\item

    Si \( A\) est le point de coordonnées \( (3;-1)\) et \( X\) celui de coordonnées \( (7;-4)\) alors le vecteur \( v=\vect{ AX }\) a pour coordonnées
    \begin{enumerate}
        \item
            \( \begin{pmatrix}
                -4    \\ 
                3    
            \end{pmatrix}\)
        \item
            \( \begin{pmatrix}
                4    \\ 
                -3    
            \end{pmatrix}\)
        \item
            \( \begin{pmatrix}
                10    \\ 
                -5    
            \end{pmatrix}\)
    \end{enumerate}
    
\item

    L'équation
    \begin{equation}
        (x+2)(x-10)=0
    \end{equation}
    \begin{enumerate}
        \item
            a pour solutions \( 2\) et \( -10\)
        \item
            n'a pas de solutions
        \item
            a pour solutions \( -2\) et \( 10\).
    \end{enumerate}
    <++>

    \end{enumerate}
    \end{multicols}
    <++>

\corrref{smath-0630}
\end{exercice}
