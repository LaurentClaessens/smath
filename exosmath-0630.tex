% This is part of Un soupçon de mathématique sans être agressif pour autant
% Copyright (c) 2014
%   Laurent Claessens
% See the file fdl-1.3.txt for copying conditions.

\begin{exercice}[\ldots/4]\label{exosmath-0630}

    Pour chacune des questions suivantes, une seule réponse est correcte; dire laquelle (et expliquer le choix)
    \begin{multicols}{2}
    \begin{enumerate}
        \item
           L'algorithme

    \begin{fmpage}{0.9\linewidth}

        Demander \( x\)

    Si  \( x< 10\) , alors :

    \hspace{0.5cm} \( k=3x\)

    Sinon :

    \hspace{0.5cm} \( k=-x+7\)

    Écrire \( k\) 

\end{fmpage}

    \begin{enumerate}
        \item
            Écrit \( -3\) si l'utilisateur rentre \( -1\).
        \item
            Écrit \( 60 \) si l'utilisateur rentre \( 20\).
        \item
            Ne fait rien si l'utilisateur rentre \( 10\).
    \end{enumerate}
    
\item

    L'équation
    \begin{equation*}
        (x+2)(x-10)=0
    \end{equation*}
    \begin{enumerate}
        \item
            a pour solutions \( 2\) et \( -10\)
        \item
            n'a pas de solutions
        \item
            a pour solutions \( -2\) et \( 10\).
    \end{enumerate}
    
\item

    La fraction
    \begin{equation*}
        \frac{ 9a }{ 3a^2+6 }
    \end{equation*}
    
    \begin{enumerate}
        \item
            peut être simplifiée par \( a\)
        \item
            peut être simplifiée par \( 3\)
        \item
            peut être simplifiée par \( 3\) et par \( a\).
    \end{enumerate}

\item

    Les représentations graphiques des fonctions \( f(x)=3x+7\) et \( g(x)=3x^2+5x-7\)
    \begin{enumerate}
        \item
            sont des droites parallèles
        \item
            sont des droites sécantes
        \item
            ne sont pas toutes les deux des droites
    \end{enumerate}

    \end{enumerate}
    \end{multicols}

\corrref{smath-0630}
\end{exercice}
