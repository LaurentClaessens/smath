% This is part of Un soupçon de mathématique sans être agressif pour autant
% Copyright (c) 2014
%   Laurent Claessens
% See the file fdl-1.3.txt for copying conditions.

\begin{corrige}{smath-0974}

    Rappel rapide de la règle : pour additionner deux fractions, il faut passer par un dénominateur commun. Pour multiplier deux fraction, il suffit de multiplier les numérateurs et les dénominateurs entre eux.
    \begin{enumerate}
        \item
            Pour \( \dfrac{ 1 }{ 2 }+\dfrac{ 1 }{ 4 }\), le dénominateur commun est \( 4\) parce qu'il est «facile» de transformer un \( 2\) en \( 4\) par une multiplication par \( 2\) :
            \begin{equation}
                \frac{ 1 }{2}=\frac{ 2 }{ 4 }.
            \end{equation}
            L'addition se fait donc de la façon suivante :
            \begin{equation}
                \frac{ 1 }{2}+\frac{ 1 }{ 4 }=\frac{ 2 }{ 4 }+\frac{ 1 }{ 4 }=\frac{ 1+2 }{ 4 }=\frac{ 3 }{ 4 }.
            \end{equation}
        \item
            Lorsque les dénominateurs sont \( 12\) et \( 3\), le dénominateur commun est \( 12\) parce qu'un \( 3\) se transforme facilement en \( 12\) \emph{via} une multiplication par \( 4\) :
            \begin{equation}
                \frac{ 11 }{ 12 }-\frac{ 2 }{ 3 }=\frac{ 11 }{ 12 }-\frac{ 2\times 4 }{ 3\times 4 }=\frac{ 11 }{ 12}-\frac{ 8 }{ 12 }=\frac{ 3 }{ 12 }.
            \end{equation}
            C'est terminé pour la soustraction des deux fractions. Il est possible de simplifier la réponse :
            \begin{equation}
                \frac{ 3 }{ 12 }=\frac{ 3\div 3 }{ 12\div 3 }=\frac{1}{ 4 }.
            \end{equation}
        \item
            Il suffit de multiplier les numérateurs et les dénominateurs entre eux :
            \begin{equation}
                \frac{ 3 }{ 5 }\times \frac{ 6 }{ 7 }=\frac{ 3\times 6 }{ 5\times 7 }=\frac{ 15 }{ 35 }.
            \end{equation}
            Cette fraction peut être simplifiée par \( 5\) :
            \begin{equation}
                \frac{ 15 }{ 35 }=\frac{ 15\div 5 }{ 35\div 5 }=\frac{ 3 }{ 7 }.
            \end{equation}
        \item
            \begin{equation}
                \frac{ 8 }{ 9 }\times \frac{ 3 }{ 8 }=\frac{ 8\times 3 }{ 9\times 8 }=\frac{ 24 }{ 72 }.
            \end{equation}
            Cette fraction peut encore être simplifiée :
            \begin{equation}
                \frac{ 24 }{ 72 }=\frac{ 24 }{ 72 }.
            \end{equation}
    \end{enumerate}

\end{corrige}
