% This is part of Un soupçon de mathématique sans être agressif pour autant
% Copyright (c) 2012-2013
%   Laurent Claessens
% See the file fdl-1.3.txt for copying conditions.

\begin{exercice}\label{exosmath-0146}

\begin{wrapfigure}{r}{8.0cm}
   \vspace{-0.5cm}        % à adapter.
   \centering
   \input{Fig_figureSCkAAJI.pstricks}
\end{wrapfigure}
    Afin de préserver son anonymat, la fonction du second degré ci-contre s'est parée d'un grand carré noir. Nous savons que les points \( (2;1)\) et \( (5;1)\) sont sur la courbe. À défaut de pouvoir reconnaître la fonction, donner l'abscisse de son sommet.

    Justifier en utilisant un calcul ou des propriétés des paraboles.

\corrref{smath-0146}
\end{exercice}
