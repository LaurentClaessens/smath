% This is part of Un soupçon de mathématique sans être agressif pour autant
% Copyright (c) 2015
%   Laurent Claessens
% See the file fdl-1.3.txt for copying conditions.

\begin{exercice}\label{exo2smath-0210}

    D'abord sur papier, ensuite au tableur.

    \begin{enumerate}
        \item
            Développer l'expression \( 3\times (a+4)\).
        \item
            Faire calculer dans deux colonnes face à face les valeurs de \( 3\times (a+4)\) et de \( 3\times a+12\) pour une centaine de valeurs de \( a\) entre \( -10\) et \( 10\).
        \item
            Que remarque-t-on ?
        \item
            Faire de même avec \( 10\times (3-y)\)
    \end{enumerate}

\corrref{2smath-0210}
\end{exercice}
