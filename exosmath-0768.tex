% This is part of Un soupçon de mathématique sans être agressif pour autant
% Copyright (c) 2014
%   Laurent Claessens
% See the file fdl-1.3.txt for copying conditions.

\begin{exercice}[\cite{WCBooDyumvE}]\label{exosmath-0768}

    La fausse équerre ou sauterelle est une équerre mobile, composée de deux règles de même longueur et assemblées, par l'un de leurs bouts, en charnière, comme un compas, de sorte que les deux éléments étant mobiles, elle sert à prendre et tracer toutes sortes d'angles.

    Quelle est la distance maximale que l'on puisse mettre entre les deux extrémités d'une fausse équerre dont les bras ont une longueur de \unit{25}{\centi\meter} ?

    Anatole a une fausse équerre cassé dont un bras fait \unit{25}{\centi\meter} et l'autre seulement \unit{10}{\centi\meter}. Quelle est la distance maximale qu'il puisse mettre entre les deux extémités ? Et la distance minimale ?

\corrref{smath-0768}
\end{exercice}
