% This is part of Un soupçon de mathématique sans être agressif pour autant
% Copyright (c) 2014
%   Laurent Claessens
% See the file fdl-1.3.txt for copying conditions.

\begin{exercice}\label{exosmath-0655}

    Soit la fonction \( f\) définie sur \( \mathopen[ -1 , \infty [\) par \( f(x)=x^2+2x-8\).
        \begin{enumerate}
            \item
                Montrer que \( f(x)=(x-2)(x+4)\) et résoudre \( f(x)=0\).
            \item
                Étudier les variations de \( f\).
            \item
                Calculer les images de \( -1\), \( 0\), \( 1\) et \( 3\) par \( f\).
            \item
                Quels sont les antécédents de \( 1\) par \( f\) ?
        \end{enumerate}

\corrref{smath-0655}
\end{exercice}
