% This is part of Un soupçon de mathématique sans être agressif pour autant
% Copyright (c) 2014
%   Laurent Claessens
% See the file fdl-1.3.txt for copying conditions.

\begin{corrige}{smath-0813}

    \begin{enumerate}
        \item
            \( 8-3\times 2=8-6=2\). Il faut d'abord effectuer le produit.
        \item
            \( \dfrac{ 16+4 }{ 5 }= \dfrac{ 20 }{ 5 }=4 \). Il faut effectuer le numérateur en priorité.
        \item
            \( \ldots \times 4+12=40\). Combien de fois quatre plus douze est égal à quarante ? Il faut que \( \ldots\times 4=40-12\), c'est à dire qu'il fait chercher \( 28\) dans la table de \( 4\). La réponse est \( 6\times 4+12=40\).
        \item
            \( (8-3)\times 2= 5\times 2=10\). Il faut effectuer la parenthèse en priorité.
        \item
            \( 7\times 35+7\times 24=7\times(35 +24)\). C'est le règle de factorisation.
    \end{enumerate}


\end{corrige}
