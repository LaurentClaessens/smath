% This is part of Un soupçon de mathématique sans être agressif pour autant
% Copyright (c) 2013
%   Laurent Claessens
% See the file fdl-1.3.txt for copying conditions.

\begin{exercice}\label{exosmath-0553}

    Nous donnons le tableau de valeurs suivant pour une fonction \( f\) définie sur \( \mathopen[ -5 ;10 \mathclose]\) :
    \begin{equation*}
        \begin{array}[]{|c||c|c|c|c|c|c|}
            \hline
            x&-5&-2&0&1&6&10\\
            \hline\hline
            f(x)&20&-1&-5&-4&31&95\\
            \hline
        \end{array}
    \end{equation*}
    Répondre aux affirmations suivantes par «vrai», «faux» ou «on ne peut pas savoir».
    \begin{enumerate}
        \item
            L'image de \( -5\) est \( 20\).
        \item
            Un antécédent de \( 1\) est \( -4\).
        \item
            \( -5\) admet un unique antécédent.
        \item
            La représentation graphique de \( f\) passe par le point \( A(10;95)\).
        \item
            La fonction \( f\) est définie par la formule \( x\mapsto x^2-5\).
        \item
            Pour tout \( x\) dans \( \mathopen[ -5 ;10 \mathclose]\), le nombre \( f(x)\) est strictement négatif.
        \item
            Pour tout \( x\) dans \( \mathopen[ -2 ;1 \mathclose]\), le nombre \( f(x)\) est strictement négatif.
    \end{enumerate}

\corrref{smath-0553}
\end{exercice}
