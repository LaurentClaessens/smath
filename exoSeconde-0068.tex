% This is part of Un soupçon de mathématique sans être agressif pour autant
% Copyright (c) 2012
%   Laurent Claessens
% See the file fdl-1.3.txt for copying conditions.

\begin{exercice}\label{exoSeconde-0068}

    L'énergie produite par une éolienne est proportionnelle au cube de la vitesse du vent\footnote{Pour savoir pourquoi, demandez au prof.}. Nous notons
    \begin{equation}
        E(v)=av^3
    \end{equation}
    où \( a\) est un coefficient sur lequel nous ne donnons pas de précisions. 
    \begin{enumerate}
        \item
            Quelle est l'énergie produite pour une vitesse \( v=1\) ?
        \item
            Quelle est l'énergie produite pour une vitesse \( v=3\) ?
        \item
            Soit \( v_0\) la vitesse du vent le \( 13\) janvier. Le \( 14\) janvier, la vitesse du vent a doublé et atteint donc \( 2v_0\). Écrire, en fonction de \( a\) et de \( v_0\) l'énergie produite par l'éolienne les \( 13\) et \( 14\) janvier.
        \item
            Quel est le rapport d'énergie produite entre le \( 14\) et le \( 13\) ?

    \end{enumerate}

\corrref{Seconde-0068}
\end{exercice}
