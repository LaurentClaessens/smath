% This is part of Un soupçon de mathématique sans être agressif pour autant
% Copyright (c) 2012
%   Laurent Claessens
% See the file fdl-1.3.txt for copying conditions.

\begin{exercice}\label{exoSeconde-0068}

    L'énergie produite par une éolienne est proportionnelle au cube de la vitesse du vent\footnote{Pour savoir pourquoi, demandez au prof.}. Nous notons
    \begin{equation}
        E(v)=av^3
    \end{equation}
    où \( a\) est un coefficient sur lequel nous ne donnons pas de précisions. Par combien l'énergie produite est multipliée si la vitesse du vent est multipliée par deux ? Autrement dit, calculer
    \begin{equation}
        \frac{ E(2v) }{ E(v) }
    \end{equation}
    La réponse en dépend ni de \( v\) ni de \( a\).

\corrref{Seconde-0068}
\end{exercice}
