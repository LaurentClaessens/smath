% This is part of Un soupçon de mathématique sans être agressif pour autant
% Copyright (c) 2013
%   Laurent Claessens
% See the file fdl-1.3.txt for copying conditions.

\begin{corrige}{smath-0547}


    \begin{center}
   \input{Fig_QJLooNAToEq.pstricks}
    \end{center}

    \begin{enumerate}
        \item
            L'ensemble de définition de la fonction est \( \mathopen[ -4 ; 4 \mathclose]\).
        \item
            \( f(2)=3\) : c'est l'ordonnée du point d'abscisse \( 2\) : le point \( A\).
        \item
            Les antécédents de \( 0\) sont les abscisses des points situés sur l'axe des abscisses : \( 3\), \( 0\) et \( 4\). Cela correspond aux points \( Z_1\), \( Z_2\) et \( Z_3\).
        \item
            Le nombre \( 3\) a exactement deux antécédents : il y a exactement deux points sur le graphe de \( f\) à avoir une ordonnée \( 3\) : les points \( (2;3)\) et \( (2.7;3)\).
        \item
            Le tableau de variations de \( f\) est :
            \begin{equation*}
                \begin{array}[]{c|ccccccccccc}
                    x&-4&&-4.2&&-2.1&&0&&-2.4&&4\\
                    \hline
                    &-2&&&&2.3&&&&3.3&&\\
                    f(x)&&\searrow&&\nearrow&&\searrow&&\nearrow&&\searrow&\\
                    &&&2.2&&&&0&&&&0\\
                \end{array}
            \end{equation*}

        \item
            Il faut trouver toutes les abscisses au-dessus desquelles la fonction est entre \( 0\) et \( 1\). Autrement dit les abscisses pour lesquelles le graphe est entre l'axe des abscisses et la ligne horizontale. Les valeurs de \( x\) qui satisfont à l'inéquation proposée sont :
            \begin{equation}
                x\in\mathopen[ 3 ; 3.2 \mathclose]\cup\mathopen[ -0.9 ; 1 \mathclose]\cup\mathopen[ 3.5 ; 4 \mathclose].
            \end{equation}
        \item
            \begin{itemize}
                \item  Le maximum de \( f\) sur \( D_f\) est \( 3.3\) et il est atteint pour \( x=2.4\).
                \item  Le minimum de \( f\) sur \( D_f\) est \( -2.3\) et il est atteint pour \( x=-3.7\).
                \item  Le maximum de \( f\) sur \( \mathopen[ -3 ; 1 \mathclose]\) est \( 2.3\) et il est atteint pour \( x=-2\).
                \item  Le minimum de \( f\) sur \( \mathopen[ -3 ; 1 \mathclose]\) est \( 0\) et il est atteint pour \( x=-3\) et \( x=0\).
            \end{itemize}
        \item
            Le tableau de signes de \( f\) est :
            \begin{equation*}
                \begin{array}[]{c|ccccccc}
                    x&-4&&-3&&0&&4\\
                    \hline
                    f(x)&-&+&0&+&0&+&+\\
                \end{array}
            \end{equation*}
            
    \end{enumerate} 

\end{corrige}
