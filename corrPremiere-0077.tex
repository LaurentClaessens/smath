% This is part of Un soupçon de mathématique sans être agressif pour autant
% Copyright (c) 2012
%   Laurent Claessens
% See the file fdl-1.3.txt for copying conditions.

\begin{corrige}{Premiere-0077}

    \begin{enumerate}
        \item
            La nombre \( P(X=10)\) est la probabilité que parmi les \( 2000\) vaches, \( 10\) sont malades; ni plus ni moins.
        \item
            Nous savons que les probabilités sont à peu près symétriques par rapport à la moyenne, c'est à dire par rapport à \( 2000\times \frac{1}{ 100 }\). La moyenne est \( 20\); le nombre \( 18\) est à \( 2\) de distance alors que \( 25\) est à \( 5\) de distance, donc \( P(X=18)\) est plus grand que \( P(X=25)\).
    \end{enumerate}

\end{corrige}
