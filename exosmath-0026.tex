% This is part of Un soupçon de mathématique sans être agressif pour autant
% Copyright (c) 2012
%   Laurent Claessens
% See the file fdl-1.3.txt for copying conditions.

\begin{exercice}\label{exosmath-0026}

    \begin{enumerate}
        \item
            Rappeler la formule donnant la distance entre les points \( A\) et \( B\) de coordonnées \( A=(x_A;y_B)\) et \( B=(x_B;y_B)\). Illustrer par un dessin.
        \item
            Calculer la distance entre les points \( A=(-1;7)\) et \( B=(5;1)\).
        \item
            Soit \( ABC\) un triangle rectangle en \( B\). Calculer la longueur du segment \( [BC]\) sachant que \( AC=\unit{17}{\centi\meter}\) et \( AB=\unit{8}{\centi\meter}\) ?
    \end{enumerate}

\corrref{smath-0026}
\end{exercice}
