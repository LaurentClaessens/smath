% This is part of Un soupçon de mathématique sans être agressif pour autant
% Copyright (c) 2012
%   Laurent Claessens
% See the file fdl-1.3.txt for copying conditions.

\begin{exercice}\label{exosmath-0106}

        Soient les parallélogrammes \( ABCD\) et \( ABDE\). Faire un dessin.
        \begin{enumerate}
            \item
                À partir des points \( A\), \( B\), \( C\), \( D\) et \( E\), donner deux vecteurs égaux à \( \vect{ AB }\).
            \item
                Où se trouve le point \( D\) par rapport au segment \( [EC]\) ? Conjecturer et prouver.
            \item
                Soit \( F\) le symétrique de \( C\) par rapport à \( D\) (donc \( D\) est le milieu de \( [CF]\)). Montrer que \( DFEA\) est un parallélogramme.
        \end{enumerate}

\corrref{smath-0106}
\end{exercice}
