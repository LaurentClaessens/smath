% This is part of Un soupçon de mathématique sans être agressif pour autant
% Copyright (c) 2014
%   Laurent Claessens
% See the file fdl-1.3.txt for copying conditions.

\begin{exercice}\label{exosmath-0763}

Quatre ordinateurs doivent être mis en réseau. En disposant d'un modem central, il suffit de quatre câbles : chacun se connecte au modem.

Pour des raisons de performances de transfert et de confidentialité, il est décidé que chaque ordinateur devra être directement relié à chacun des autres. Combien de câbles partent de chacun des ordinateurs ? Combien en faut-il au total ?

Mêmes questions si il faut mettre \( 50\) ordinateurs en réseau.

Revenons aux quatre ordinateurs. Martin l'informaticien commence par connecter les ordinateurs \( 2\), \( 3\) et \( 4\) à l'ordinateur \( 1\). Ensuite il passe à l'ordinateur numéro \( 2\) et y connecte les ordinateurs qui doivent encore être connectés (le numéro \( 1\) est déjà connecté au \( 2\)). Et ainsi de suite.

Combien fait la somme \( 1+2+3\) ? Comparer au nombre de câbles à utiliser pour mettre \( 4\) ordinateurs en réseau.

Combien fait la somme \( 1+2+3+4+5+\ldots+100\) ?
\corrref{smath-0763}
\end{exercice}
