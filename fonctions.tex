% This is part of Un soupçon de mathématique sans être agressif pour autant
% Copyright (c) 2012
%   Laurent Claessens, Pauline Klein
% See the file fdl-1.3.txt for copying conditions.

%+++++++++++++++++++++++++++++++++++++++++++++++++++++++++++++++++++++++++++++++++++++++++++++++++++++++++++++++++++++++++++
\section{Introduction.}
%+++++++++++++++++++++++++++++++++++++++++++++++++++++++++++++++++++++++++++++++++++++++++++++++++++++++++++++++++++++++++++


\begin{example} \label{ExemVmCkIH}
    Un petit tour de magie. Choisissez un nombre entre \( 1\) et \( 10\). Ajoutez \( 5\), multipliez par \( 2\), ajoutez \( 7\), enlevez le double du nombre de départ.

    Le nombre que vous avez maintenant est \( 17\).
\end{example}


\newpage

\Exo{Seconde-0042}
\Exo{Seconde-0046}
\Exo{Seconde-0047}
\Exo{Seconde-0044}

\newpage

\begin{definition}
    Soit \( D\) un ensemble de nombres. On définit une \defe{fonction}{fonction} \( f\) sur \( D\) en associant à chaque nombre \( x\) dans \( D\) un seul nombre \( y\). Dans ce cas nous disons que \( f\) est une fonction de la \defe{variable}{variable} \( x\).
\end{definition}

En ce qui concerne les notations, nous écrivons
\begin{equation}
    \begin{aligned}
        f\colon D&\to \eR \\
        x&\mapsto y. 
    \end{aligned}
\end{equation}
Cela signifie que \( D\) est l'\defe{ensemble de définition de \( f\)}{ensemble!de définition}, c'est à dire l'ensemble des nombres sur lesquels \( f\) peut être appliquée. Le symbole \( \eR\) désigne l'ensemble de tous les nombres.

Nous écrivons \( y=f(x)\).

Nous verrons de nombreux exemples.

%+++++++++++++++++++++++++++++++++++++++++++++++++++++++++++++++++++++++++++++++++++++++++++++++++++++++++++++++++++++++++++
\section{Domaine de définition}
%+++++++++++++++++++++++++++++++++++++++++++++++++++++++++++++++++++++++++++++++++++++++++++++++++++++++++++++++++++++++++++

Nous avons déjà mentionné le fait que le domaine de définition d'une fonction est l'ensemble des nombres sur lesquels la fonction peut être calculée.

\begin{example}
    Soit la fonction qui à la longueur d'un segment fait correspondre la surface du carré construit sur ce segment. Cette fonction n'est définie que sur les nombres positifs (parce qu'il n'existe pas de segments de longueurs négatives). Nous écrivons donc
    \begin{equation}
        \begin{aligned}
            f\colon \mathopen[ 0 , \infty [&\to \eR \\
            x&\mapsto x^2,
        \end{aligned}
    \end{equation}
    et nous avons \( f(x)=x^2\).
\end{example}

\begin{example}
    Soit la fonction qui a un nombre entier fait correspondre la somme de ses chiffres. Par exemple \( f(0)=0\) et \( f(123)=6\). Cette fonction est définie sur les entiers et retourne un entier. Nous pouvons écrire
    \begin{equation}
        \begin{aligned}
            f\colon \eN&\to \eN \\
            x&\mapsto f(x). 
        \end{aligned}
    \end{equation}
    Ici il est compliqué de donner une forme explicite pour \( f\).
\end{example}

%+++++++++++++++++++++++++++++++++++++++++++++++++++++++++++++++++++++++++++++++++++++++++++++++++++++++++++++++++++++++++++
\section{Courbe représentative d'une fonction}
%+++++++++++++++++++++++++++++++++++++++++++++++++++++++++++++++++++++++++++++++++++++++++++++++++++++++++++++++++++++++++++

Soit $f$ une fonction définie sur un ensemble $\mathscr{D}$. \\
Le plan est muni d'un repère $(O,I,J)$.

\begin{definition}
  \fbox{
    \begin{minipage}[t]{0.85\linewidth}
      On appelle \emph{représentation graphique} de $f$, ou courbe
      représentative de $f$, l'ensemble des points $M(x;y)$ tels que
      $x\in\mathscr{D}$ et $y=f(x)$.
    \end{minipage}
  }
\end{definition}

Cette courbe est notée $\mathscr{C}$ ou $\mathscr{C}_f$.\\

{\centering 
  \includegraphics[width=8cm]{F_Axes.pdf}    
\par}

\begin{definition}
  \fbox{
    \begin{minipage}[t]{0.85\linewidth}
      Si une fonction $f$ a pour représentation graphique la courbe
      $\mathscr{C}$, on dit que cette courbe $\mathscr{C}$ \emph{a pour
        équation} $y=f(x)$ dans le repère choisi.
    \end{minipage}
  }
\end{definition}

\begin{itemize}
\item[$\bullet$] Pour une valeur $x$ sur l'axe des abscisses, il y a un et un
  seul point d'abscisse $x$ sur la courbe.
\item[$\bullet$] Pour tracer une courbe, il faut placer des points. Plus on
  choisit de points, plus la courbe sera précise.
\end{itemize}

\begin{rem}
  \begin{minipage}[t]{0.45\linewidth}
    Cette courbe ne représente pas une fonction, car le nombre $x$ a
    deux images $y$.    
  \end{minipage}
  \qquad
  \begin{minipage}[c]{0.4\linewidth}
    \includegraphics[width=5cm]{F_NonFct.pdf}
  \end{minipage}
\end{rem}


\paragraph{Cas particulier :} 
\begin{minipage}[t]{0.8\linewidth}
  Si $f$ est une fonction affine définie pour $x\in\R$ par 
  $f(x)=ax+b$, et si la droite $d$ est sa représentation graphique,
  une équation de la droite $d$ est \ $y=ax+b$.
\end{minipage}

\paragraph{Application :} 
\begin{minipage}[t]{0.8\linewidth}
  Pour $x\in\R$, $f(x)=-x^2+2x$, de courbe représentative
  $\mathscr{P}$.\\
  Le point $A(2;0)$ appartient-il à $\mathscr{P}$ ? De même pour
  $B(-2;-7)$.
  \vspace{4cm}
\end{minipage}



\section{Lecture d'un graphique. Résolution graphique d'(in)équations} 

\subsection{Lecture graphique d'images / antécédents}

\begin{minipage}[c]{0.4\linewidth}
  On considère 
  $
  \begin{array}[t]{cl}
    f : & \R \longrightarrow \R \\
    & x \longmapsto x^2
  \end{array}
  $.  \\

  A $x$, on associe $y=f(x)$. 

  \vspace{1cm}

  \begin{itemize}
  \item[\textbullet] Image de $1,5$ ? \\[2em]
  \item[\textbullet] Antécédent de $3,5$ ?  \\[2em]
  \item[\textbullet] Antécédent de $-1$ ? \\[2em]
  \item[\textbullet] Antécédent de $0$ ?  \\[2em]
  \end{itemize}
  
\end{minipage}
\begin{minipage}[c]{0.6\linewidth}
  \includegraphics[width=9cm]{F_Carre2.pdf}  
\end{minipage} \\


Fonction $f(x)=x^2$ \\[2ex]
\begin{tabular}{|>{$}c<{$}|*{11}{>{\centering$}p{1cm}<{$}|}c}
  \cline{1-12}
  \displaystyle \vphantom{\int} \ x \ \ 
  & -4 & -3 & -2 & -1 & 0 & 1 & 2 & 3 & 4 & -0,5 & 0,5 & \\
  \cline{1-12}
  \displaystyle \vphantom{\int} y
  & & & & & & & & & & & \\
  \cline{1-12}
\end{tabular}

\medskip


\subsection{Résolution graphique d'équations}

$k$ est un réel fixé. \\

\noindent
\fbox{
  \textbullet \quad
  Résoudre l'équation $f(x)=k$ revient à chercher les antécédents par
  $f$ du nombre $k$.
} \\

Le nombre de solutions de l'équation $f(x)=k$ est égal au nombre de
points d'intersection de la courbe $\mathscr{C}$ avec la droite $d$
d'équation $y=k$. Les solutions sont les abscisses de ces points
d'intersection. 

\paragraph{Exemple :} 
\begin{minipage}[t]{0.55\linewidth}
  On cherche à résoudre $f(x)=4$. \\
  On trace la droite d'équation \hspace{1.5cm} et
  on observe les points d'intersection de cette droite avec la courbe
  $\mathscr{C}$. \\[1ex]
  $\mathscr{S}=\{\hspace{1.5cm}\}$. 
\end{minipage}
\qquad
\begin{minipage}[c]{0.25\linewidth}
  \includegraphics[width=\textwidth]{F_reseq_f.pdf}
\end{minipage}

\pagebreak[4]

\noindent
\fbox{
  \begin{minipage}[t]{1.0\linewidth}
    \textbullet \quad 
    Résoudre l'équation $f(x)=g(x)$ revient à déterminer les abscisses
    des points d'intersection des courbes $\mathscr{C}_f$ et
    $\mathscr{C}_g$.
  \end{minipage}
} \\[1em]

\begin{minipage}[c]{0.5\linewidth}
  \includegraphics[width=\textwidth]{F_resineq_fg.pdf}
\end{minipage}
\qquad
\begin{minipage}[t]{0.4\linewidth}
  Sur cette figure, l'équation $f(x)=g(x)$ a \hspace{1cm} solutions,
  qui sont
\end{minipage} \\[1em]



\subsection{Résolution graphique d'inéquations}

\paragraph{\textbullet \ Inéquation du type $f(x)<k$} \ \par
Les solutions de l'inéquation $f(x)<k$ sont les abscisses des points
de $\mathscr{C}$ situés en-dessous de la droite d'équation $y=k$.

En particulier, les solutions de l'inéquation $f(x)<0$ sont les
abscisses des points de la courbe $\mathscr{C}$ qui sont situés
en-dessous de l'axe des abscisses, c'est-à-dire ayant une ordonnée
strictement négative.\\

\medskip

{\centering
\includegraphics[width=0.5\textwidth]{F_resineq_f.pdf}
\par}

\medskip

On procède de la même manière pour les inégalités du type $f(x)>k$,
$f(x)\geq k$, $f(x)\leq k$. 

\paragraph{\textbullet \ Inéquation du type $f(x)<g(x)$} \ \par
Les solutions de l'inéquation $f(x)<g(x)$ sont les abscisses des
points pour lesquels la courb $\mathscr{C}_f$ et en-dessous de la
courbe $\mathscr{C}_g$.

\medskip

{\centering
  \includegraphics[width=0.5\textwidth]{F_resineq_fg.pdf}
\par}

%\clearpage


\section{Sens de variation d'une fonction}
\label{sec:variations}

\subsection{Notion intuitive}
\label{ssec:notion_intuitive}

\vspace{2cm}

\begin{center}
  \includegraphics[width=0.6\textwidth]{F_Variations.pdf}
\end{center}

\vspace{2cm}



\subsection{Définition}

\begin{definition}
  \fbox{
    \begin{minipage}[t]{0.82\linewidth}
      Soit $f$ une fonction définie sur $\mathscr{D}$ et $I$ un
      intervalle de $\mathscr{D}$.\\[-2ex]
      \begin{itemize}
      \item[\textbullet] On dit que $f$ est \emph{croissante} sur $I$
        si et seulement si pour tous réels $a$ et $b$ de $I$, \\
        \ \quad si $a\leq b$ \ alors \ $f(a)\leq f(b)$.\\[-2ex]
      \item[\textbullet] On dit que $f$ est \emph{décroissante} sur $I$
        si et seulement si pour tous réels $a$ et $b$ de $I$, \\
        \ \quad si $a\leq b$ \ alors \ $f(a)\geq f(b)$. 
      \end{itemize}
    \end{minipage}
  }
\end{definition}

\smallskip

\begin{rem}
  \begin{minipage}[t]{0.8\linewidth}
    Une fonction croissante range les images dans le même ordre que
    les antécédents. \\
    Une fonction décroissante inverse cet ordre. \\
  \end{minipage}
\end{rem}

\begin{center}
  \begin{tabular}{c@{\qquad \qquad}c}
    \begin{minipage}[c]{0.36\linewidth}
      \includegraphics[width=\textwidth]{F_Croissante.pdf}
    \end{minipage}
    &
    \begin{minipage}[c]{0.36\linewidth}
      \includegraphics[width=\textwidth]{F_Decroissante.pdf}
    \end{minipage}  \\
    & \\
    $f$ croissante sur $I$
    &
    $f$ décroissante sur $I$
  \end{tabular}
\end{center}


\begin{rem}
  \begin{minipage}[t]{0.84\linewidth}
  On dit que $f$ est \emph{\underline{strictement} croissante}
  (respectivement \emph{\underline{strictement} décroissante}) sur~$I$
  si pour tous réels $a$ et $b$ de $I$ tels que $a<b$, on a
  $f(a)<f(b)$ (respectivement $f(a)>f(b)$).  \\
  \end{minipage}
\end{rem}

\begin{rem}
  \begin{minipage}[t]{0.8\linewidth}
    Dans cette définition, il est important que $I$ soit un
    intervalle. \\ $]-\infty;0[\ \cup \ ]0;+\infty[$ n'est pas un
    intervalle. \\
  \end{minipage}
\end{rem}

\begin{definition}
  \fbox{
    \begin{minipage}[t]{0.82\linewidth}
      On appelle fonction \emph{monotone} sur $I$ une fonction qui est
      soit croissante sur $I$, soit décroissante sur $I$.
    \end{minipage}
  }
\end{definition}

\smallskip

La fonction représentée en \,IV.\,1) n'est pas
monotone sur $[-4;3]$. Elle est monotone sur $[-4;-2,5]$
(décroissante), sur $[-2,5;1]$ (croissante), et sur $[1;3]$ (décroissante).\\

\begin{definition}
  \fbox{
    \begin{minipage}[t]{0.82\linewidth}
      On dit que $f$ est \emph{constante} sur $I$ lorsque pour tous
      les réels $a$ et $b$ de $I$, on a $f(a)=f(b)$. (Tous les réels de
      $I$ ont la même image par $f$).
    \end{minipage}
  }
\end{definition}

\begin{minipage}[t]{0.4\linewidth}
  Dans ce cas, il existe $k\in\R$ tel que \\ pour tout $a\in I$, $f(a)=k$.
\end{minipage}
\qquad
\begin{minipage}[c]{0.4\linewidth}
  \includegraphics[width=\textwidth]{F_Constante.pdf}
\end{minipage}



\subsection{Tableau de variations}

\og Étudier les variations de $f$ \fg{} signifie {\ldots}


Ces résultats sont résumés dans un tableau de variations. \\

%\begin{center}
  %\begin{minipage}[c]{0.4\linewidth}
    %\includegraphics[width=\textwidth]{F_TabVar.pdf}
  %\end{minipage}
  %\qquad \qquad
  %\begin{minipage}[c]{0.5\linewidth}
    %\centering
    %\begin{variations}
      %x & & 0 & & 1 & & 3 & & 4  \; \\
      %\filet
      %\m f & \; & -1 & \c & \h3 & \d & -1 & \c & \h3 \\
    %\end{variations}        
  %\end{minipage} \\[1em]
  %\begin{minipage}[c]{0.4\linewidth}
    %\centering
    %$f$ définie sur $[0;4]$
  %\end{minipage}
  %\qquad \qquad
  %\begin{minipage}[c]{0.5\linewidth}
    %\centering
    %$f$ est croissante sur $[0;1]$ et sur $[3;4]$. \\
    %$f$ est décroissante sur $[1;3]$
  %\end{minipage} 
%\end{center}

%TODO : refaire le tableau de variations.

\subsection{Sens de variation d'une fonction affine}

\begin{proposition}
  \fbox{
    \begin{minipage}[t]{0.7\linewidth}
      Soit $f$ la fonction affine $x\mapsto ax+b$.
      \begin{itemize}
      \item[\textbullet] Si $a>0$, alors $f$ est croissante sur $\R$.
      \item[\textbullet] Si $a<0$, alors $f$ est décroissante sur $\R$.
      \item[\textbullet] Si $a=0$, alors $f$ est constante sur $\R$
        (et $f(x)=b$ pour tout $x\in\R$).
      \end{itemize}
    \end{minipage}
  }  
\end{proposition}

\paragraph{Démonstration :} \ \\

\vspace{6cm}


\begin{rem}
  $f(x)=0$ pour $x=\hspace{1cm}$. On connaît donc le signe de $f(x)$
  selon les valeurs de $x$.
\end{rem}


\begin{minipage}[c]{0.4\linewidth}
  \centering
  \includegraphics[width=0.6\linewidth]{F_Affine_a.pdf}
\end{minipage}
\quad
\begin{minipage}[c]{0.4\linewidth}
  \centering
  \includegraphics[width=0.6\linewidth]{F_Affine_b.pdf}
\end{minipage}



\begin{proposition}
  \fbox{
    \begin{minipage}[t]{0.85\linewidth}
      Règle du signe de $ax+b$ \\
      
      \begin{tabular}{cc}
        $a<0$ & $a>0$ \\
        & \\
        $\begin{array}{|c|ccccc|}
          \hline
          x & -\infty & & -\dfrac{b}{a} & & +\infty \\[2ex]
          \hline
          ax+b & & + \quad \ & 0 & \quad - & \\[1ex]
          \hline
        \end{array}$
        &
        $\begin{array}{|c|ccccc|}
          \hline
          x & -\infty & & -\dfrac{b}{a} & & +\infty \\[2ex]
          \hline
          ax+b & & - \quad \ & 0 & \quad + & \\[1ex]
          \hline
        \end{array}$ \\
      \end{tabular}  \\[1ex] 
    \end{minipage} \\
  }    
\end{proposition}


\section{Minimum et maximum}

\begin{definition}
  \fbox{
    \begin{minipage}[t]{0.8\linewidth}
      Soit $f$ une fonction définie sur un intervalle $I$. \\[-2ex] 
      \begin{itemize}
      \item[\textbullet] On dit que $f$ admet le réel $m$ pour minimum
        sur $I$ ssi il existe $c\in I$ tel que $f(c)=m$
        et pour tout $x\in I$, $f(x)\geq m$. \\[-2ex]
      \item[\textbullet] On dit que $f$ admet le réel $M$ pour maximum
        sur $I$ ssi il existe $d\in I$ tel que $f(d)=M$ et pour tout
        $x\in I$, $f(x)\leq M$.
      \end{itemize}
    \end{minipage}
  }    
\end{definition}


\begin{minipage}[c]{0.4\linewidth}
  \centering
  \includegraphics[width=6cm]{F_Extrema.pdf}
\end{minipage}
\quad
\begin{minipage}[c]{0.4\linewidth}
  \centering
  \ \\
  $f$ admet un \hspace{2cm} en $a$  \\[2ex]
  $\phantom{f(a)}a\approx$ \\[2ex]
  $\phantom{a}f(a)\approx $ \\[2em]
  
  \bigskip

  $f$ admet un \hspace{2cm} en $b$  \\[2ex]
  $\phantom{f(b)}b\approx$ \\[2ex]
  $\phantom{b}f(b)\approx $   
\end{minipage}



%+++++++++++++++++++++++++++++++++++++++++++++++++++++++++++++++++++++++++++++++++++++++++++++++++++++++++++++++++++++++++++
\section{Exercices}
%+++++++++++++++++++++++++++++++++++++++++++++++++++++++++++++++++++++++++++++++++++++++++++++++++++++++++++++++++++++++++++

\Exo{Seconde-0043}

\Exo{Premiere-0018}                                                                                                                                
