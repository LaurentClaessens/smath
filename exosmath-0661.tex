% This is part of Un soupçon de mathématique sans être agressif pour autant
% Copyright (c) 2014
%   Laurent Claessens
% See the file fdl-1.3.txt for copying conditions.

\begin{exercice}\label{exosmath-0661}

    Nous considérons les points \( A(0;2)\), \( B(4.5;4.5)\) et \( D(1.5;-2)\) ainsi que le vecteur \( \vect{ u }=\begin{pmatrix}
        6    \\ 
        -1.5    
    \end{pmatrix}\).
    \begin{enumerate}
        \item
            Placer ces points dans un repère et construire le point \( C\) tel que \( \vect{ AC }=\vect{ u }\).
        \item
            Construire l'image de \( A\) par la translation de vecteur \( \vect{ BD }\). Nous nommons \( E\) ce point.
        \item
            Calculer les coordonnées du vecteur \( \vect{ AB }\) ainsi que la distance \( AB\).
        \item
            Démontrer que \( \vect{ AC }=\vect{ AB }+\vect{ AD }\).
        \item
            Pourquoi peut-on affirmer que \( ABCD\) est un parallélogramme ?
        \item
            Calculer les coordonnées du point \( E\).
    \end{enumerate}

\corrref{smath-0661}
\end{exercice}
