% This is part of Un soupçon de mathématique sans être agressif pour autant
% Copyright (c) 2013
%   Laurent Claessens
% See the file fdl-1.3.txt for copying conditions.

\begin{exercice}\label{exosmath-0415}

    Soit \( z_0=3 e^{i\pi/3}\) et la suite définie par récurrence par
    \begin{equation}
        z_{n+1}=\frac{ z_n }{ 4 }\left( \sqrt{2}+i\sqrt{2} \right).
    \end{equation}
    \begin{enumerate}
        \item
            Calculer les termes \( z_0\), \( z_1\), \( z_2\) et \( z_3\).
        \item
            Nous nommons \( r_n\) le module de \( z_n\). Donner une expression de \( z_n\) en fonction de \( z\).
        \item
            Quelle est la limite de la suite définie par les \( r_n\) ?
    \end{enumerate}

\corrref{smath-0415}
\end{exercice}
