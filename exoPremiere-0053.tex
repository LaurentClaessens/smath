% This is part of Un soupçon de mathématique sans être agressif pour autant
% Copyright (c) 2012
%   Laurent Claessens
% See the file fdl-1.3.txt for copying conditions.

\begin{exercice}\label{exoPremiere-0053}

    Mettre dans la variable \info{texte\_reference} un texte assez long (quelque centaines de caractères que vous irez pomper n'importe où). Écrire un programme qui affiche un texte de 20 lettres prises au hasard dans cette variable. Procéder comme suit.
    \begin{enumerate}
        \item
            Créer une variable contenant un texte vide : \info{mon\_texte=``''}.
        \item
            Créer une boucle \info{for} sur \info{range(1,21)}
        \item
            À chaque étape, prendre une lettre au hasard dans \info{texte\_reference}, la mettre dans la variable \info{a}, et l'ajouter à \info{mon\_texte} en utilisant
            \begin{quote}
                \info{mon\_texte=mon\_texte+a}
            \end{quote}
    \end{enumerate}

\corrref{Premiere-0053}
\end{exercice}
