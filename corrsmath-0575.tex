% This is part of Un soupçon de mathématique sans être agressif pour autant
% Copyright (c) 2013
%   Laurent Claessens
% See the file fdl-1.3.txt for copying conditions.

\begin{corrige}{smath-0575}

    \begin{enumerate}
        \item
            Le nombre total de courriers interceptés est l'effectif total : \( 50+20+65+40+\ldots +13+5=285\).
        \item
            Pour calculer les fréquences cumulées croissantes, il est bon de commencer par faire la ligne des effectifs cumulés croissants; ensuite il suffit de diviser par l'effectif total :
    \begin{equation*}
        \begin{array}[]{|c||c|c|c|c|c|c|c|c|c|c|c|}
            \hline
            \text{Nombre de caractères}&200&500&1000&1500&2000&2500&3000&3500&4000&4500&5000\\
            \hline\hline
            \text{Effectifs}&50&20&65&40&15&12&20&17&28&13&5\\
            \hline
            \text{ECC}&50&70&135&175&190&202&222&239&267&280&285\\
            \hline
            \text{FCC}&0.17&0.24&0.47&0.61&0.66&0.70&0.77&0.83&0.93&0.98&1\\
            \hline
        \end{array}
    \end{equation*}
        \item
            Le premier quartile est la valeur à partir de laquelle les fréquences cumulées dépassent \( 0.25\). Dans ce cas c'est
            \begin{equation}
                Q_1=1000.
            \end{equation}
            La médiane est la valeur sur laquelle tombe le \( 0.5\). Ici : 
            \begin{equation}
                \text{médiane}=1500.
            \end{equation}
            Le troisième quartile est la valeur pour laquelle les fréquences cumulées dépassent \( 0.75\). Dans notre cas c'est
            \begin{equation}
                Q_3=3000.
            \end{equation}

    \end{enumerate}

\end{corrige}
