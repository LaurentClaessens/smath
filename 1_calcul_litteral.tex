% This is part of Un soupçon de mathématique sans être agressif pour autant
% Copyright (c) 2014
%   Laurent Claessens
% See the file fdl-1.3.txt for copying conditions.

%--------------------------------------------------------------------------------------------------------------------------- 
\subsection*{Carré sans coins}
%---------------------------------------------------------------------------------------------------------------------------

% This is part of Un soupçon de mathématique sans être agressif pour autant
% Copyright (c) 2014
%   Laurent Claessens
% See the file fdl-1.3.txt for copying conditions.

Un carreleur veut créer le motif suivant :
\begin{center}
   \input{Fig_TCBLooKXvOaZ.pstricks}
\end{center}
\begin{enumerate}
    \item
Combien de carreaux de couleur lui faudra-t-il ? 
\item
Reproduire le dessin pour une fresque de \( 6\times 6\) au lieu de \( 5\times 5\). Combien de carreaux de couleur faut-il alors ? 
\item
Et pour une fresque de taille \( 100\times 100\) que l'on crée encore selon le même motif ?
\item
Le professeur appelle $x$ le nombre de carreaux d'un côté de la fresque et $G$ le nombre de cases roses. Des élèves ont obtenu les expressions suivantes :
\begin{multicols}{3}
    \begin{itemize}
        \item
            Anis : \( G=x\times 4-2\)
        \item
            Basile : \( G=x-2\times 4\)
        \item
            Chloé : \( G=4\times (x-2)\)
        \item
             Dalila : \( (x-2)\times 4 \)
         \item
             Enzo : \( G=4\times x-8\)
         \item
             René : \( G=4\times x-4\)
    \end{itemize}
\end{multicols}
Parmi ces expressions, lesquelles sont fausses ? Pourquoi ? Y a-t-il plusieurs bonnes réponses ? Justifier.
\end{enumerate}



De \cite{NRHooXFvgpp5}

%+++++++++++++++++++++++++++++++++++++++++++++++++++++++++++++++++++++++++++++++++++++++++++++++++++++++++++++++++++++++++++ 
\section{Expression «en fonction de \( x\)» }
%+++++++++++++++++++++++++++++++++++++++++++++++++++++++++++++++++++++++++++++++++++++++++++++++++++++++++++++++++++++++++++

\begin{definition}
    Une expression dans laquelle certains nombres sont représentés par des lettres est une \defe{expression littérale}{expression littérale}.
\end{definition}

\begin{example}
    Le triple d'un nombre \( x\) est noté «\( 3\times x\)».
\end{example}

\begin{example}
    La longueur d'un wagon est de \( 7.35\) mètres. La longueur de \( x\) wagons est \( 7.35\times x\) mètres.
\end{example}

La triple de \( x\) et la longueur des \( x\) wagons ont été exprimés \defe{en fonction de \( x\)}{en fonction de}.

%+++++++++++++++++++++++++++++++++++++++++++++++++++++++++++++++++++++++++++++++++++++++++++++++++++++++++++++++++++++++++++ 
\section{Un tour de magie}
%+++++++++++++++++++++++++++++++++++++++++++++++++++++++++++++++++++++++++++++++++++++++++++++++++++++++++++++++++++++++++++

Amandine dit à Arnaud :
\begin{itemize}
    \item Choisi un nombre entre \( 1\) et \( 10\)
    \item multiplie par \( 2\)
    \item fais \( +6\)
    \item divise par \( 2\)
    \item soustrais le nombre choisis.
\end{itemize}
Quel que soit le nombre de départ, Arnaud obtient \( 3\). Pourquoi ?
