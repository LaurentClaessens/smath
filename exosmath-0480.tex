% This is part of Un soupçon de mathématique sans être agressif pour autant
% Copyright (c) 2013
%   Laurent Claessens
% See the file fdl-1.3.txt for copying conditions.

\begin{exercice}\label{exosmath-0480}

    Dessiner un repère orthonormé \( (O,I,J)\) en prenant \( OI=\unit{2}{\centi\meter}\) et \( OJ=\unit{2}{\centi\meter}\). Y placer les points \( A(2;-2)\), \( B(1;3)\), \( C(-1;-2)\), \( D(4;-2)\) et \( E(1;-3)\).

\corrref{smath-0480}
\end{exercice}
