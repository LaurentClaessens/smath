% This is part of Un soupçon de mathématique sans être agressif pour autant
% Copyright (c) 2013
%   Laurent Claessens
% See the file fdl-1.3.txt for copying conditions.

\begin{corrige}{smath-0569}

    \begin{enumerate}
        \item
            Vu que l'ordonnée à l'origine de la droite tracée est \( -1\), cela ne peut être que la fonction \( f\).
        \item
            Le dessin complet est :
            \begin{center}
                \input{Fig_JLUTBlD.pstricks}
            \end{center}
        \item
            Il y a deux façons de résoudre l'équation \( f(x)=g(x)\). La première, graphique, est de donner l'abscisse du point d'intersection entre les deux droites. C'est directement visible sur le dessin : \( x=1\). 

            La seconde est de résoudre l'équation
            \begin{equation}
                3x-1=-x+3.
            \end{equation}
            On passe tout les \( x\) à gauche et le reste à droite :
            \begin{equation}
                3x+x=3+1
            \end{equation}
            Donc \( 4x=4\) et finalement \( x=1\).

            Notons que les deux méthodes donnent le même résultat. Si les deux méthodes ne donnent pas le même résultat, c'est qu'on se trompe.

    \end{enumerate}

\end{corrige}
