% This is part of Un soupçon de mathématique sans être agressif pour autant
% Copyright (c) 2014
%   Laurent Claessens
% See the file fdl-1.3.txt for copying conditions.

\begin{corrige}{smath-0778}

    Compléter les égalités :
        \begin{enumerate}
            \item
                \( (-5)+3=-2\)
            \item
                \( 4-(-7)=4+7=11\) parce que soustraire revient à additionner l'opposé
            \item
                \( 1+(-\ldots)=-1\) Nous savons que \( 1-2=-1\), donc \( 1+(-2)=-1\).
            \item
                \( (-2)\times (-3)=6\)
            \item
                \( \dfrac{ -6 }{ -3 }=2\). Ne pas oublier le signe.
        \end{enumerate}

\end{corrige}
