% This is part of Un soupçon de mathématique sans être agressif pour autant
% Copyright (c) 2012
%   Laurent Claessens
% See the file fdl-1.3.txt for copying conditions.

\begin{corrige}{Premiere-0082}

    \begin{enumerate}
        \item
            Chaque pomme a une chance sur \( 20\) d'être mangée par une chenille. Donc pour chacune des pommes prises, nous avons une chance sur \( 20\) de trouver une chenille et en tout nous devrions avoir autour de \( 100\times \frac{1}{ 20 }=5\) pommes avec chenille.
        \item
            Le verger compte \( 150\times 5000=750000\) pommes. La probabilité de reprendre deux fois la même étant très petite, nous pouvons considérer que le tirage est avec remise.
    \end{enumerate}

\end{corrige}
