% This is part of Un soupçon de mathématique sans être agressif pour autant
% Copyright (c) 2014
%   Laurent Claessens
% See the file fdl-1.3.txt for copying conditions.

\begin{exercice}[\ldots/5]\label{exosmath-0636}

    Soit une fonction \( g\) définie sur \( [-3;7]\) telle que
    \begin{enumerate}
        \item
            le graphe de \( g\) passe par le point \( A(1;2)\)
        \item
            \( g\) est croissante sur \( [-3;0]\) et décroissante sur \( [1;7]\).
        \item
            \( g(-3)=3\).
        \item
            Le maximum de \( g\) est atteint pour \( x=0\).
    \end{enumerate}

    Questions :
    \begin{enumerate}
        \item
            À partir de ces données, comparer \( g(2)\) et \( g(-1)\).
        \item
            Dessiner une représentation graphique possible de \( g\) et en vous basant sur votre dessin :                
            \begin{itemize}
        \item
            Dresser le tableau de variations de la fonction que vous avez dessinée.
        \item
            Dresser le tableau de signes de la fonction que vous l'avez dessinée.
            \end{itemize}
    \end{enumerate}

\corrref{smath-0636}
\end{exercice}
