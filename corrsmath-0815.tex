% This is part of Un soupçon de mathématique sans être agressif pour autant
% Copyright (c) 2014
%   Laurent Claessens
% See the file fdl-1.3.txt for copying conditions.

\begin{corrige}{smath-0815}

    Calculer :
    \begin{enumerate}
        \item
            \( (-1)+(-7)=-8\). Règle d'addition de deux nombres de même signe.
        \item
            \( (-4)\times 4=-16\)  multiplication de deux nombres de signes différent : négatif.
        \item
            \( (+5)-(+7)= 5+(-7) = -2  \)  la plus grande distance à zéro est le \( -7\); nous prenons donc son signe.
        \item
            \( \dfrac{ 9 }{ -3 }=-\dfrac{ 9 }{ 3 }=-3\). Règle du signe d'un quotient : la même que celle du produit.
        \item
            \( (-12)\times (-2)=24\). Le produit de deux nombres négatifs est positif.
    \end{enumerate}

\end{corrige}
