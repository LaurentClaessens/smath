% This is part of Un soupçon de mathématique sans être agressif pour autant
% Copyright (c) 2014
%   Laurent Claessens
% See the file fdl-1.3.txt for copying conditions.

% This is part of Un soupçon de mathématique sans être agressif pour autant
% Copyright (c) 2014
%   Laurent Claessens
% See the file fdl-1.3.txt for copying conditions.

%--------------------------------------------------------------------------------------------------------------------------- 
\subsection*{Activité : tableaux et représentation graphique}
%---------------------------------------------------------------------------------------------------------------------------

Associer à chaque tableau le graphique qui correspond. Quels sont ceux qui correspondent à des situations de proportionnalité ?

\begin{equation}
    \input{Fig_ARKZooKZOuAkconvprix.latex}\quad
    \input{ARKZooKZOuAkfar.latex}
\end{equation}
\begin{equation}
    \input{ARKZooKZOuAkmilm.latex}\quad
    \input{ARKZooKZOuAkctlibre.latex}
\end{equation}

\begin{center}
   \input{Fig_ARKZooKZOuAk0.pstricks}
   \input{Fig_ARKZooKZOuAk3.pstricks}
   \input{Fig_ARKZooKZOuAk2.pstricks}
   \input{Fig_ARKZooKZOuAk1.pstricks}
\end{center}


%+++++++++++++++++++++++++++++++++++++++++++++++++++++++++++++++++++++++++++++++++++++++++++++++++++++++++++++++++++++++++++ 
\section{Représentation graphique}
%+++++++++++++++++++++++++++++++++++++++++++++++++++++++++++++++++++++++++++++++++++++++++++++++++++++++++++++++++++++++++++

\begin{propriete}
    Une situation de proportionnalité est représentée dans un repère par des points alignés avec l'origine.
\end{propriete}

\begin{example}
    Le périmètre d'un carré est proportionnel à son côté parce que \( p=4\times c\). Représenter graphiquement le périmètre en fonction du côté.

    \begin{enumerate}
        \item
            On choisit quelque valeurs pour le côté \( c\)
        \item
            On calcule les valeurs correspondantes pour le périmètre \( p\) :
            \begin{equation*}
                \begin{array}[]{|c||c|c|c|c|}
                    \hline
                    \text{côté}&1&2&3&4\\
                    \hline\hline
                    \text{périmètre}&4&8&12&16\\
                    \hline
                \end{array}
            \end{equation*}
            La deuxième ligne est quatre fois la première.
        \item
            On place ces points dans un repère :
            \begin{center}
\input{Fig_VJZKooGlvyPP.pstricks}
            \end{center}
            et \SI{p}{\centi\meter}.
    \end{enumerate}

\end{example}

Exercice \ref{exosmath-0941}.

%+++++++++++++++++++++++++++++++++++++++++++++++++++++++++++++++++++++++++++++++++++++++++++++++++++++++++++++++++++++++++++ 
\section{Quatrième proportionnelle}
%+++++++++++++++++++++++++++++++++++++++++++++++++++++++++++++++++++++++++++++++++++++++++++++++++++++++++++++++++++++++++++

%--------------------------------------------------------------------------------------------------------------------------- 
\subsection{Tableau}
%---------------------------------------------------------------------------------------------------------------------------

\begin{definition}
    Dans une situation de proportionnalité, la \defe{quatrième proportionnelle}{quatrième proportionnelle} est le quatrième nombre calculé à partir de trois nombres déjà connus.
\end{definition}

Faire copier dans le cahier de leçon et résoudre dans un premier temps dans le cahier d'exercices. La correction au tableau sera prise directement dans le cahier de leçon.

\begin{example}
    Calculer le prix \( x\) de trois baguettes à partir du tableau de proportionnalité suivant :
    \begin{equation*}
        \begin{array}[]{|c|c|c|}
            \hline
            \text{nombre de baguettes}&5&3\\
            \hline
            \text{prix (€)}&4.25&x\\
            \hline
        \end{array}
    \end{equation*}
    Le facteur de proportionnalité est \( \frac{ 4.25 }{ 5 }\). Donc
    \begin{equation}
        x=\frac{ 4.25 }{ 5 }\times 3=\frac{ 12.75 }{ 5 }=2.55
    \end{equation}
\end{example}

Exercices : \ref{exosmath-0944}

%--------------------------------------------------------------------------------------------------------------------------- 
\subsection{Produit en croix}
%---------------------------------------------------------------------------------------------------------------------------

% This is part of Un soupçon de mathématique sans être agressif pour autant
% Copyright (c) 2014
%   Laurent Claessens
% See the file fdl-1.3.txt for copying conditions.

%--------------------------------------------------------------------------------------------------------------------------- 
\subsection*{Activité : quatrième proportionnelle}
%---------------------------------------------------------------------------------------------------------------------------

Compléter les tableaux de proportionnalité suivants (on pourra appeler \( x\) la valeur manquante)
\begin{multicols}{3}
    \begin{enumerate}
        \item
            \begin{equation*}
                \begin{array}[]{|c|c|}
                    \hline
                    6&10\\
                    \hline
                    \ldots&30\\
                    \hline
                \end{array}
            \end{equation*}
        \item
            \begin{equation*}
                \begin{array}[]{|c|c|}
                    \hline
                    10&\ldots\\
                    \hline
                    30&45\\
                    \hline
                \end{array}
            \end{equation*}
        \item
            \begin{equation*}
                \begin{array}[]{|c|c|}
                    \hline
                    3&\ldots\\
                    \hline
                    4&10\\
                    \hline
                \end{array}
            \end{equation*}
    \end{enumerate}
\end{multicols}




\begin{propriete}
    Il y a égalité des quotients
    \begin{equation}
        \frac{ a }{ b }=\frac{ x }{ y },
    \end{equation}
    si et seulement si
    \begin{equation}
        a\times y=b\times x.
    \end{equation}
    Cette égalité s'appelle l'égalité du \defe{produit en croix}{produit en croix}.
\end{propriete}

\begin{proof}
    \begin{subproof}
        \item[Sens direct]
            Nous partons de l'égalité
            \begin{equation}
                \frac{ a }{ b }-\frac{ x }{ y }=0
            \end{equation}
            et nous mettons les fractions au même dénominateur :
            \begin{equation}
                \frac{ a\times y }{ b\times y }-\frac{ x\times b }{ y\times b }=0,
            \end{equation}
            ou encore :
            \begin{equation}
                a\times y=b\times x.
            \end{equation}
        \item[Réciproque]
            Nous partons de l'égalité
            \begin{equation}
                a\times y-b\times x=0
            \end{equation}
            Lorsque quelque chose est nul, cela reste nul lorsqu'on divise par n'importe que nombre. Ici nous divisons par \( b\times y\) :
            \begin{equation}
                \frac{ a\times y }{ b\times y }-\frac{ b\times x }{ b\times y }
            \end{equation}
            et nous simplifions chacune des deux fractions par ce qui convient :
            \begin{equation}
                \frac{ a }{ b }-\frac{ x }{ y }=0.
            \end{equation}
    \end{subproof}
\end{proof}

\begin{example}
    Compléter le tableau de proportionnalité :
    \begin{equation*}
        \begin{array}[]{|c|c|c|c|c|}
            \hline
            \text{temps}&\SI{2}{\second}&\SI{3}{\second}&\SI{4}{\second}&\SI{6}{\second}\\
            \hline
            \text{distance}&\SI{5}{\meter}&\SI{7.5}{\meter}&\ldots&\SI{15}{\meter}\\
            \hline
        \end{array}
    \end{equation*}
    Le coefficient de proportionnalité est le même quelle que soit la colonne sur laquelle on le calcule, donc si nous nommons \( x\) la nombre à mettre dans la case manquante,
    \begin{equation}
        \frac{ x }{ 4 }=\frac{ 5 }{ 2 }.
    \end{equation}
    L'égalité du produit en croix donne : 
    \begin{equation}
        2x=5\times 4
    \end{equation}
    et donc \( x=10\).
\end{example}

\begin{example}
    Compléter le tableau de proportionnalité suivant
    \begin{equation*}
        \begin{array}[]{|c|c|c|c|}
            \hline
             6&12&\ldots&27\\
              \hline
              8&16&28&36\\ 
              \hline 
               \end{array}
    \end{equation*}
    Nous pouvons écrire
    \begin{equation}
        \frac{ 8 }{ 6 }=\frac{ 28 }{ x }.
    \end{equation}
    L'égalité du produit en croix donne :
    \begin{equation}
        8x=6\times 28,
    \end{equation}
    c'est à dire
    \begin{equation}
        x=\frac{ 6\times 28 }{ 8 }=21.
    \end{equation}
\end{example}

Exercices : \ref{exosmath-0945}.

%+++++++++++++++++++++++++++++++++++++++++++++++++++++++++++++++++++++++++++++++++++++++++++++++++++++++++++++++++++++++++++ 
\section{Vitesse}
%+++++++++++++++++++++++++++++++++++++++++++++++++++++++++++++++++++++++++++++++++++++++++++++++++++++++++++++++++++++++++++

\begin{definition}
    Si un mobile parcourt une distance $d$ en un temps $t$, alors la \defe{vitesse moyenne}{vitesse moyenne} $v$ de ce mobile est le quotient
    \begin{equation}
        v=\frac{ d }{ t }.
    \end{equation}
\end{definition}

\begin{Aretenir}
    Nous avons aussi les relations
    \begin{equation}
        d=v\times t
    \end{equation}
    et
    \begin{equation}
        t=\frac{ d }{ v }.
    \end{equation}
\end{Aretenir}

\begin{example}
    Un cycliste roule à \SI{25}{\kilo\meter\per\hour}. Remplir le tableau suivant qui donne la distance parcourue en fonction du temps :
    \begin{equation*}
        \begin{array}[]{|c||c|c|c|c|}
            \hline
            \text{temps de parcours}&\SI{\frac{ 1 }{2}}{\hour}&\SI{1}{\hour}&\SI{3}{\hour}&\SI{3.5}{\hour}\\
            \hline
            \text{distance parcourue}&&&&\\
            \hline
        \end{array}
    \end{equation*}
    Attention : \( \SI{3.5}{\hour}\) ne sont pas trois heures et cinq minutes mais trois heures et demi, c'est à dire trois heures et trente minutes.
\end{example}

Exercices : \ref{exosmath-0943}.

%+++++++++++++++++++++++++++++++++++++++++++++++++++++++++++++++++++++++++++++++++++++++++++++++++++++++++++++++++++++++++++ 
\section{Pourcentage}
%+++++++++++++++++++++++++++++++++++++++++++++++++++++++++++++++++++++++++++++++++++++++++++++++++++++++++++++++++++++++++++

% This is part of Un soupçon de mathématique sans être agressif pour autant
% Copyright (c) 2014
%   Laurent Claessens
% See the file fdl-1.3.txt for copying conditions.
% et by the way, ce texte est recopié du sésamath de quatrième.

$25$ filles et $20$ garçons de deux classes de 4\ieme\ ont effectué un devoir commun.  60\% des filles et 50\% des garçons ont obtenu la moyenne. Calculer le pourcentage d'élèves qui ont obtenu la moyenne dans l'ensemble de ces deux classes.

De \cite{NRHooXFvgpp4}

Indications possibles
\begin{itemize}
    \item Combien de filles ont réussi ?
    \item Combien d'élève au total ont passé ce devoir ?
\end{itemize}

Éléments de réponse :
\begin{enumerate}
    \item
        Commençons par déterminer le nombre de filles ayant obtenu la moyenne : \( 60\%\) de \( 40\) font
        \begin{equation}
            \frac{ 60 }{ 100 }\times 40=\frac{ 60\times 40 }{ 100 }=24.
        \end{equation}
    \item
        Déterminons ensuite le nombre de garçons ayant obtenu la moyenne : \( 50\%\) de \( 14\) font
        \begin{equation}
            \frac{ 50 }{ 100 }\times 14=\frac{ 1 }{2}\times 14=7.
        \end{equation}
    \item
        Donc sur les \( 54\) élèves ayant passé le devoir, \( 31\) ont réussi. Cela fait une proportion de
        \begin{equation}
            \frac{ 31 }{ 54 }\simeq 0.57
        \end{equation}
        ou un pourcentage de \( 57\%\).
\end{enumerate}

Exercice \ref{exosmath-0957}.
