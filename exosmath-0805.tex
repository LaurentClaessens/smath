% This is part of Un soupçon de mathématique sans être agressif pour autant
% Copyright (c) 2014
%   Laurent Claessens
% See the file fdl-1.3.txt for copying conditions.

\begin{exercice}[\cite{NRHooXFvgpp4}]\label{exosmath-0805}

    Nous avons le rectangle suivant dont les mesures sont données en fonction de \( x\) :

\begin{center}
   \input{Fig_OJDooPOzSiC.pstricks}
\end{center}

\begin{enumerate}
    \item
        Écrire son périmètre sous forme d'une expression réduite.
    \item
        Écrire son aire sous forme d'une expression factorisée.
    \item
        Écrire son aire sous forme d'une expression réduite
    \item
        Quels sont les périmètre et l'aire de ce rectangle si \( x=7\) ?
\end{enumerate}

\corrref{smath-0805}
\end{exercice}
