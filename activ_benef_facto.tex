% This is part of Un soupçon de mathématique sans être agressif pour autant
% Copyright (c) 2014
%   Laurent Claessens
% See the file fdl-1.3.txt for copying conditions.

%--------------------------------------------------------------------------------------------------------------------------- 
\subsection*{Activité : vente de gâteaux}
%---------------------------------------------------------------------------------------------------------------------------

Bob et Alice vendent des gâteaux pour financer un voyage. Chaque gâteau coûte deux euros à produire et est vendu cinq euros. Bob fait le raisonnement suivant : «si on vend \( n\) gâteaux, cela nous a coûté \( 2n\) euros à produire et rapporté \( 5n\) euros. Donc le bénéfice est \( 5n-2n\)». Alice par contre dit : «chaque gâteau rapporte \( 3\) euros; donc si on en vend \( n\), on gagne \( 3n\) euros». 
\begin{enumerate}
    \item
        Quel est le bénéfice obtenu en vendant \( 10\) gâteaux suivant le raisonnement de Bob ?
    \item
        Quel est le bénéfice obtenu en vendant \( 10\) gâteaux suivant le raisonnement d'Alice ?
    \item
        Qui a raison ?
\end{enumerate}
