% This is part of Un soupçon de mathématique sans être agressif pour autant
% Copyright (c) 2014
%   Laurent Claessens
% See the file fdl-1.3.txt for copying conditions.

\begin{corrige}{smath-0636}

    \begin{enumerate}
        \item
            Nous avons certainement \( g(2)<2\) parce que \( g\) est décroissante entre \( 1\) et \( 7\) alors que \( g(1)=2\). Par contre \( g(-1)>3\) parce que \( g(-3)=3\) et que \( g\) est croissante sur \( \mathopen[ -3 , 0 \mathclose]\).

            Donc \( g(-1)>g(2)\).
        \item
            Il y a de nombreuses possibilités. En voici une :
            \begin{center}
                \input{Fig_SWFywZG.pstricks}
            \end{center}
        \item
            Le tableau de variations de la fonction dessinée est :
            \begin{equation*}
                \begin{array}[]{c|ccccc}
                    x&-3&&0&&7\\
                    \hline
                    &&&5&&\\
                    g(x)&&\nearrow&&\searrow&\\
                    &3&&&&1\\
                \end{array}
            \end{equation*}

        \item
            Le tableau de signes est encore plus simple parce que la fonction dessinée est toujours positive :
            \begin{equation*}
                \begin{array}[]{c|ccc}
                     x&-3&&7\\
                      \hline
                      f(x)&+&+&+\\ 
                       \end{array}
                   \end{equation*}

    \end{enumerate}
    

\end{corrige}
