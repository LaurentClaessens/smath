% This is part of Un soupçon de mathématique sans être agressif pour autant
% Copyright (c) 2014
%   Laurent Claessens
% See the file fdl-1.3.txt for copying conditions.

\begin{corrige}{smath-0823}

    \begin{enumerate}
        \item
            La technique consiste à tracer le segment \( KL\) de longueur \SI{10}{\centi\meter} et de tracer des cercles de rayons \SI{6}{\centi\meter} à partir de chacune des deux extrémités. Le point \( M\) est une des intersections (au choix).

\begin{center}
   \input{Fig_PSZooRphUET.pstricks}
\end{center}

        \item
            La mesure des angles doit donner environ \SI{33.5}{\degree} pour les angles de la base et \SI{113}{\degree} pour l'angle \( \hat M\).

        \item
            Le devoir donné à Alysée n'est pas possible parce que le triangle demandé ne respecte pas l'inégalité triangulaire : le plus long côté est plus long que la somme des deux autres :
            \begin{equation}
                25>10+12.
            \end{equation}
    \end{enumerate}

\end{corrige}
