% This is part of Un soupçon de mathématique sans être agressif pour autant
% Copyright (c) 2012
%   Laurent Claessens
% See the file fdl-1.3.txt for copying conditions.

\begin{exercice}\label{exosmath-0000}

    Nous munissons le plan d'un repère orthonormé \( (O,I,J)\) et nous considérons les droites \( d\) et \( d'\) d'équations
    \begin{subequations}
        \begin{align}
            y=-2x-3\\
            y=\frac{ 1 }{2}x+2.
        \end{align}
    \end{subequations}
    \begin{multicols}{2}
    \begin{enumerate}
        \item
            Dessiner \( d\) et \( d'\).
        \item
            Calculer le point d'intersection de \( d\) et \( d'\).
        \item
            Est-ce que le point \( B=(1;-5)\) est sur \( d\) ?
        \item
            Donner l'équation de la parallèle à \( d'\) passant par \( B\).
        \item
            Est-ce que \( D=(4;4)\) est sur la droite \( d'\) ?
        \item
            Donner l'équation de la parallèle à \( d\) passant par \( D\).
        \item
            Pour que point \( C\) le quadrilatère \( ABCD\) est un parallélogramme. Donner les coordonnées de \( C\).
        \item
            Calculer les longueurs \( AB\), \( AD\) et \( BD\).
        \item
            Quelle est la nature du triangle \( ABD\) ?
        \item
            En déduire quelle est précisément la nature du quadrilatère \( ABCD\).
    \end{enumerate}
    \end{multicols}

\corrref{smath-0000}
\end{exercice}
