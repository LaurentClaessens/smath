% This is part of Un soupçon de mathématique sans être agressif pour autant
% Copyright (c) 2014
%   Laurent Claessens
% See the file fdl-1.3.txt for copying conditions.

\begin{corrige}{smath-0675}

    La plus simple à trouver est \( d_4\) qui correspond à la fonction \( f_b(x)=-2x+2\) : c'est la seule décroissante (\( m=-2\)). L'autre directement visible est \( d_3\) qui correspond à \( f_d(x)=3x\) : c'est la seule droite passant par l'origine et donc la seule fonction linéaire.

    Les deux restantes sont \( f_a(x)=2x+2\) et \( f_c(x)=2x-2\), qui sont parallèles et qui sont donc \( d_1\) et \( d_2\). La fonction \( f_a\) est \( d_1\) parce que l'ordonnée à l'origine est \( 2\) (positive) et la fonction \( f_c\) est la droite \( d_2\) parce que l'ordonnée à l'origine est négative (\( -2\)).

\end{corrige}
