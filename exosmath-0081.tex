% This is part of Un soupçon de mathématique sans être agressif pour autant
% Copyright (c) 2012
%   Laurent Claessens
% See the file fdl-1.3.txt for copying conditions.

\begin{exercice}\label{exosmath-0081}

    \begin{multicols}{2}
        Soit la pyramide régulière de base carrée dessinée ci-contre. Le point \( I\) est sur le segment \( AS\) et tel que \( [AF]=\frac{ 3 }{ 4 }[AS]\); le point \( J\) est sur \( [SC]\) et tel que \( [CH]=\frac{1}{ 3 }[CS]\).

        \begin{enumerate}
            \item
        Est-ce que le droites \( (AC)\) et \( (IJ)\) sont sécantes ?
    \item
        Où placer le point \( I\) sur \( [AS]\) pour que les droites \( (AC)\) et \( (IJ)\) ne soient pas sécantes ?
        \end{enumerate}

        \columnbreak

        \begin{center}
\input{Fig_figureTFaRFVd.pstricks}
        \end{center}
%The result is on figure \ref{LabelFigfigureTFaRFVd}. % From file figureTFaRFVd
%\newcommand{\CaptionFigfigureTFaRFVd}{<+Type your caption here+>}

    \end{multicols}
    

\corrref{smath-0081}
\end{exercice}
