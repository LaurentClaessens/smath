% This is part of Un soupçon de mathématique sans être agressif pour autant
% Copyright (c) 2012
%   Laurent Claessens
% See the file fdl-1.3.txt for copying conditions.

\begin{exercice}[\cite{ZNloWAM}]\label{exosmath-0240}

    Résoudre les systèmes suivants.
    \begin{multicols}{2}
        \begin{enumerate}
            \item
                \begin{subequations}
                    \begin{numcases}{}
                        3x+5y=1\\
                        -2x+7y=0
                    \end{numcases}
                \end{subequations}
            \item
                \begin{subequations}
                    \begin{numcases}{}
                        \frac{1}{ 3 }x+\frac{ 4 }{ 5 }y=-1\\
                        5x+12y=3
                    \end{numcases}
                \end{subequations}
            \item
                \begin{subequations}
                    \begin{numcases}{}
                        x+y=2x+3\\
                        2x+2y=x-y+1
                    \end{numcases}
                \end{subequations}
            \item
                Attention : le suivant est difficile :
                \begin{subequations}
                    \begin{numcases}{}
                        3x^2-5y^2=-2\\
                        -2x^2+4y^2=1
                    \end{numcases}
                \end{subequations}
                Pour le résoudre, poser \( u=x^2\), \( v=y^2\), résoudre par rapport à \( u\) et \( v\). Cela vous donne \( 4\) solutions pour \( x\) et \( y\) vu que \( x=\pm\sqrt{u}\) et \( y=\pm\sqrt{v}\). Lesquelles sont réellement solutions du système de départ ?
                
        \end{enumerate}
    \end{multicols}

    Question supplémentaire : dessiner des figures illustrant ces systèmes.

\corrref{smath-0240}
\end{exercice}
