% This is part of Un soupçon de mathématique sans être agressif pour autant
% Copyright (c) 2012
%   Laurent Claessens
% See the file fdl-1.3.txt for copying conditions.

\begin{exercice}\label{exoPremiere-0005}

\begin{quote}
    Près de 850 000 personnes travaillent en Lorraine fin 2010, ce qui représente 3,2\% de l’emploi métropolitain. Les salariés constituent 92\% de ces travailleurs. Le secteur tertiaire pourvoit 75,5\% des emplois en Lorraine, soit une proportion proche de celle de la France de province (75,3\%). Toutefois, l’industrie régionale tient encore une place importante (16\% de l’emploi, contre 14,5\% pour la France de province). 
\end{quote}
Citation de l'\href{http://www.insee.fr/fr/regions/lor/reg-dep.asp?theme=3&suite=1}{INSEE}.

\begin{enumerate}
    \item
        Combien de personnes travaillaient en France métropolitaine en \( 2010\) ?
    \item
        Combien de personnes travaillaient dans le tertiaire ?
    \item
        Combien de personnes travaillaient ni dans le tertiaire, ni dans l'industrie ?
\end{enumerate}

\corrref{Premiere-0005}
\end{exercice}
