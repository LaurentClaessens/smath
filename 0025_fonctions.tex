%This is part of Un soupçon de mathématique sans être agressif pour autant
% Copyright (c) 2012-2013
%   Laurent Claessens, Pauline Klein
% See the file fdl-1.3.txt for copying conditions.


\section{Sens de variation d'une fonction}

\subsection{Notion intuitive}

\begin{multicols}{2}

    La fonction ci-contre descend jusqu'à \( -2.5\) puis monte entre \( -2.5\) et \( 2\) pour ensuite descendre.

    Nous disons qu'elle est
    \begin{itemize}
        \item 
            \emph{décroissante} sur \( \mathopen] \infty , -2.5 \mathclose]\);
        \item
            \emph{croissante} sur \( \mathopen[ -2.5 , 2 \mathclose]\);
        \item
            à nouveau décroissante sur \( \mathopen[ 2 , \infty [\).
    \end{itemize}
    
    \columnbreak

%    \includegraphics{Picture_FIGLabelFigExVariationRXTkocPICTExVariationRXTkoc-for_eps.pdf}
%The result is on figure \ref{LabelFigExVariationRXTkoc}.
%\newcommand{\CaptionFigExVariationRXTkoc}{<+Type your caption here+>}
\input{Fig_ExVariationRXTkoc.pstricks}

\end{multicols}


\subsection{Définition}

\begin{definition}
      Soit $f$ une fonction définie sur $\defD$ et $I$ un
      intervalle de $\defD$.\\[-2ex]
      \begin{enumerate}
          \item On dit que $f$ est \defe{croissante}{croissante (fonction)} sur $I$
        si et seulement si pour tous réels $a$ et $b$ de $I$, 
        si $a\leq b$ \ alors \ $f(a)\leq f(b)$.
    \item On dit que $f$ est \defe{décroissante}{décroissante (fonction)} sur $I$
        si et seulement si pour tous réels $a$ et $b$ de $I$, 
        si $a\leq b$ \ alors \ $f(a)\geq f(b)$. 
      \end{enumerate}
\end{definition}


\begin{remark}
    Une fonction croissante range les images dans le même ordre que les antécédents. Une fonction décroissante inverse cet ordre. 
\end{remark}


\begin{definition}
    On dit que $f$ est \defe{strictement croissante}{strictement croissante} sur~$I$
  si pour tous réels $a$ et $b$ de $I$ tels que $a<b$, on a $f(a)<f(b)$.

  La fonction \( f\) est \defe{strictement décroissante}{strictement décroissante} sur \( I\) si pour tous réels $a$ et $b$ de $I$ tels que $a<b$, on a $f(a)>f(b)$.
\end{definition}
La différence entre la croissance et la \emph{stricte} croissance est que l'inégalité est stricte.

\begin{definition}
    Soit \( I\) un intervalle de \( \eR\). Nous disons que la fonction \( f\) est \defe{monotone}{monotone} sur $I$ si elle est soit croissante sur $I$, soit décroissante sur $I$.
\end{definition}

\begin{multicols}{2}

    La fonction dessinée ci-contre n'est pas monotone sur l'intervalle \( \mathopen[ -2 , 0 \mathclose]\). Elle est
    \begin{itemize}
        \item 
            monotone décroissante sur \( \mathopen[ -2.5 , -1 \mathclose]\);
        \item
            monotone croissante sur \( \mathopen[ -1 , 1 \mathclose]\);
        \item
            monotone décroissante sur \( \mathopen[ 1 , 1.5 \mathclose]\).
    \end{itemize}

\columnbreak

%\includegraphics{Picture_FIGLabelFigGrapheVarndvdQMPICTGrapheVarndvdQM-for_eps.pdf}
%The result is on figure \ref{LabelFigGrapheVarndvdQM}.
%\newcommand{\CaptionFigGrapheVarndvdQM}{<+Type your caption here+>}
\input{Fig_GrapheVarndvdQM.pstricks}

\end{multicols}




\begin{definition}
    Soit \( I\) un intervalle. On dit que $f$ est \defe{constante}{constante (fonction)} sur $I$ lorsque pour tous les réels $a$ et $b$ de $I$, on a $f(a)=f(b)$. (Tous les réels de $I$ ont la même image par $f$).
\end{definition}

\begin{multicols}{2}
    Dans ce cas, il existe $k\in\eR$ tel que pour tout $a\in I$, $f(a)=k$. 
    
    La figure ci-contre donne le graphe de la fonction \( f(x)=1.5\) entre \( x=-3\) et \( x=3\).

\columnbreak

%\includegraphics{Picture_FIGLabelFigFoncConstFdDkhWPICTFoncConstFdDkhW-for_eps.pdf}
%The result is on figure \ref{LabelFigFoncConstFdDkhW}.
%\newcommand{\CaptionFigFoncConstFdDkhW}{<+Type your caption here+>}
\input{Fig_FoncConstFdDkhW.pstricks}

\end{multicols}
