%This is part of Un soupçon de mathématique sans être agressif pour autant
% Copyright (c) 2012-2013
%   Laurent Claessens, Pauline Klein
% See the file fdl-1.3.txt for copying conditions.

Dans ce chapitre :
\begin{enumerate}
    \item
        Résoudre graphiquement des inéquations : \( f(x)\leq k\) ou \( f(x)\geq g(x)\).
    \item
        Lien tableau de variations, tableau de valeurs et dessin.
    \item
        Comparer des images depuis un tableau de variation.
    \item
        Donner des fonctions sous forme de programmes ou algos.
    \item
        Équation produit.
\end{enumerate}

%+++++++++++++++++++++++++++++++++++++++++++++++++++++++++++++++++++++++++++++++++++++++++++++++++++++++++++++++++++++++++++ 
\section{Différentes manières de parler d'une fonction}
%+++++++++++++++++++++++++++++++++++++++++++++++++++++++++++++++++++++++++++++++++++++++++++++++++++++++++++++++++++++++++++

%--------------------------------------------------------------------------------------------------------------------------- 
\subsection{Tableau de valeurs}
%---------------------------------------------------------------------------------------------------------------------------

Le tableau de valeurs consiste à donner quelque valeurs connues de la fonction.

\begin{example}
    Soit une fonction \( f\) définie sur \( \mathopen[ -5 ; 10 \mathclose]\) et dont nous connaissons le tableau de valeurs suivant :
    \begin{equation}
        \begin{array}[h]{|c||c|c|c|c|c|c|}
            \hline
            x&-5&-1&1&3&9&10\\
            \hline
            f(x)&0&2&-3&7&2&5\\
            \hline
        \end{array}
    \end{equation}
    Nous pouvons dire que
    \begin{enumerate}
        \item
            L'image de \( 1\) par \( f\) est \( -3\).
        \item
            Les nombres \( -1\) et \( 9\) sont tous deux des antécédents de \( 2\).
    \end{enumerate}
    Nous ne pouvons pas dire 
    \begin{enumerate}
        \item
            L'image de \( 2\).
        \item
            Le nombre \( 3\) a d'autres antécédents que \( 7\).
        \item
            Si la fonction est croissante sur \( \mathopen[ -5 ;0 \mathclose]\).
    \end{enumerate}


    En réalité n'importe quelle courbe qui passe par les points suivants peut être \( f\) :
    \begin{center}
   \input{Fig_WYeESAN.pstricks}
    \end{center}

\end{example}

%--------------------------------------------------------------------------------------------------------------------------- 
\subsection{Tableau de signe}
%---------------------------------------------------------------------------------------------------------------------------

Le tableau de signe ne donne que le signe de la fonction. Prenons une fonction \( f\) définie sur \( \mathopen[ -4 , 10 \mathclose]\) dont nous savons le tableau de signe :
\begin{equation*}
    \begin{array}[]{c|ccccccccc}
        x&&-1&&2&&3&&7&\\
        \hline
        f(x)&+&0&-&0&-&0&+&0&-\\
    \end{array}
\end{equation*}
Nous savons que
\begin{enumerate}
    \item
        \( 0\) a pour antécédents \( -1\), \( 2\), \( 3\), \( 7\).
    \item
        \( f(1)<0\).
\end{enumerate}
Nous ne savons pas si 
\begin{enumerate}
    \item
        \( f\) est croissante sur \( \mathopen[ 0 ; 2 \mathclose]\).
    \item
        Quelle est l'image de \( 5\).
\end{enumerate}

Une forme possible de \( f\) est donnée ci-dessous :
\begin{center}
\input{Fig_EIxhcRb.pstricks}
\end{center}

Et de nombreuses autres sont possibles.

%--------------------------------------------------------------------------------------------------------------------------- 
\subsection{Tableau de variations}
%---------------------------------------------------------------------------------------------------------------------------

Le tableau de variations dit pour chaque abscisse si la fonction est croissante ou décroissante ainsi que certaines valeurs.

\begin{example}
    \begin{equation*}
        \begin{array}[]{c|ccccccc}
            x&4&&-1&&3&&6\\
            \hline
            &5&&&&3&&\\
            f(x)&&\searrow&&\nearrow&&\searrow&\\
            &&&-2&&&&-1\\
        \end{array}
    \end{equation*}

Ce que le tableau dit :
\begin{enumerate}
    \item
        \( f(-1)=-2\).
    \item
        Entre \( x=3\) et \( x=6\), la fonction est décroissante.
    \item
        \( f(-2)\geq f(-1)\).
    \item
        \( f(5)\) est entre \( 3\) et \( 6\).
\end{enumerate}
Ce que le tableau ne dit pas :
\begin{enumerate}
    \item
        La valeur de \( f(1)\).
    \item
        Si \( f(4)\) est plus grand ou plus petit que \( f(-3)\).
\end{enumerate}

\end{example}

%---------------------------------------------------------------------------------------------------------------------------
\section{Résolution graphique d'(in)équations} 
%---------------------------------------------------------------------------------------------------------------------------

\begin{Aretenir}
    Résoudre l'équation $f(x)=g(x)$ revient à déterminer les abscisses des points d'intersection des courbes représentatives de \( f\) et \( g\).
\end{Aretenir}


En pratique pour déterminer graphiquement les solutions de \( f(x)=g(x)\) il faut faire :
\begin{enumerate}
    \item
        Trouver les points d'intersection entre les courbes de \( f\) et de \( g\).
    \item
        Les solutions sont les abscisses de ces points.
    \item
        Écrire \( S=\{ \ldots;\ldots;\ldots \}\).
\end{enumerate}

Dans le cas du dessin ci-dessous, les solutions sont approximativement \( x=-3.6\), \( x=-1.1\) et \( x=1.5\).

\begin{center}
\input{Fig_ExEquationIntersectioniSHPTw.pstricks}
\end{center}

Nous notons alors
\begin{equation}
    S=\{ -3.6;-1.1;1.5 \}.
\end{equation}

\subsection{Résolution graphique d'inéquations}


%///////////////////////////////////////////////////////////////////////////////////////////////////////////////////////////
\subsubsection{Inéquation du type $f(x)<k$}
%///////////////////////////////////////////////////////////////////////////////////////////////////////////////////////////

\begin{Aretenir}
Les solutions de l'inéquation $f(x)<k$ sont les abscisses des points de la courbe représentative de \( f\) situés en-dessous de la droite d'équation $y=k$.
\end{Aretenir}

Sur la figure ci-dessous, nous résolvons \( f(x)\leq 2\). La procédure à suivre est la suivante.
\begin{enumerate}
    \item
        Tracer la droite horizontale \( y=2\).
    \item
        Trouver les points d'intersection avec le graphe de \( f\).
    \item
        Les solutions sont les abscisses pour lesquelles le graphe de \( f\) est au-dessus du graphe de la droite \( y=2\).
    \item
        Donner les solutions sous forme d'intervalle.
\end{enumerate}


\begin{center}
    \input{Fig_ExIneqOcAWMq.pstricks}
\end{center}
Ici nous écrivons les solutions sous la forme
\begin{equation}
    x\in\mathopen[ -4.4 ; 1.4 \mathclose].
\end{equation}
