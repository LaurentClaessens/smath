% This is part of Un soupçon de mathématique sans être agressif pour autant
% Copyright (c) 2012
%   Laurent Claessens
% See the file fdl-1.3.txt for copying conditions.

\begin{exercice}\label{exosmath-0107}

    Soit, dans un repère \( (0,I,J)\), les points \( A=(2;1)\), \( B=(-3,6)\), \( C=(10,37)\).
    \begin{enumerate}
        \item
            Donner les coordonnées des points \( O\), \( I\) et \( J\).
        \item
            Calculer les coordonnées des vecteurs \( \vect{ OI }\), \( \vect{ OJ }\), \( \vect{ IA }\), \( \vect{ BC }\) et \( \vect{ IJ }\).
        \item
            Calculer les coordonnées des vecteurs \( 2\vect{ AB  }\), \( -\vect{ OC }\), \( -34\vect{ AC }\).
        \item
            Construire le point \( M\) tel que \( \vect{ AM }=\vect{ AB }+2\vect{ AO }\).
        \item
            Calculer les coordonnées du point \( N\) tel que \( \vect{ ON }=\vect{ AB }-3\vect{ AC }\).
    \end{enumerate}

\corrref{smath-0107}
\end{exercice}
