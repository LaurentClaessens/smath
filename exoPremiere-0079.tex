% This is part of Un soupçon de mathématique sans être agressif pour autant
% Copyright (c) 2012
%   Laurent Claessens
% See the file fdl-1.3.txt for copying conditions.

\begin{exercice}\label{exoPremiere-0079}

    Dans cet exercice nous allons perfectionner le programme précédent pour que que la fonction accepte de prendre la probabilité de succès en argument. Pour cela, remarquez que \info{bernoulli} a une paire de parenthèses vide.

    Remplacer
    \begin{quote}
        \info{def bernoulli():}
    \end{quote}
    par
    \begin{quote}
        \info{def bernoulli(p):}
    \end{quote}
    et remplacer les \info{0.8} par \info{p} partout où il faut.

    Lorsque le programme fonctionne, enregistrez le soigneusement dans votre répertoire personnel parce que vous en aurez encore besoin.

\corrref{Premiere-0079}
\end{exercice}
