% This is part of Un soupçon de mathématique sans être agressif pour autant
% Copyright (c) 2014
%   Laurent Claessens
% See the file fdl-1.3.txt for copying conditions.

%--------------------------------------------------------------------------------------------------------------------------- 
\subsection*{Napoléon se place}
%---------------------------------------------------------------------------------------------------------------------------

% This is part of Un soupçon de mathématique sans être agressif pour autant
% Copyright (c) 2014
%   Laurent Claessens
% See the file fdl-1.3.txt for copying conditions.


Les Anglais et les Autrichiens ont pris position en deux points \( A\) et \( B\) distants de \( \SI{10}{\kilo\meter}\). Napoléon veut pouvoir être à même de les attaquer tous les deux d'égale manière et décide donc de se positionner en un point $N$ qui serait à égale distance de \( A\) que de \( B\).

Bien entendu il pourrait se placer au milieu du segment \( [AB]\), mais ainsi il se ferait trop facilement attaquer des deux côtés à la fois (pas fou le Corse!). Où peut-il se placer ? Faire un dessin pour l'aider.


%+++++++++++++++++++++++++++++++++++++++++++++++++++++++++++++++++++++++++++++++++++++++++++++++++++++++++++++++++++++++++++ 
\section{Médiatrice}
%+++++++++++++++++++++++++++++++++++++++++++++++++++++++++++++++++++++++++++++++++++++++++++++++++++++++++++++++++++++++++++

\begin{definition}
    La \defe{médiatrice}{médiatrice} d'un segment est la droite perpendiculaire à ce segment en son milieu.
\end{definition}

\begin{center}
   \input{Fig_KSQooHHfEpe.pstricks}
\end{center}

\begin{Aretenir}
    La médiatrice du segment \( [AB]\) est l'ensemble des points équidistants de \( A\) et \( B\).
\end{Aretenir}

Pour tracer la médiatrice du segment \( [AB]\), 
\begin{itemize}
    \item Tracer un arc de cercle de centre \( A\) (peu importe le rayon);
    \item en gardant le même rayon, tracer un arc de cercle de centre \( B\);
    \item la droite passant par les deux intersections est la médiatrice.
\end{itemize}
Attention : il faut choisir le rayon assez grand pour qu'il y ait des intersections.

\begin{center}
   \input{Fig_BKWooKbAHbD.pstricks}
\end{center}

%+++++++++++++++++++++++++++++++++++++++++++++++++++++++++++++++++++++++++++++++++++++++++++++++++++++++++++++++++++++++++++ 
\section{Cercle circonscrit}
%+++++++++++++++++++++++++++++++++++++++++++++++++++++++++++++++++++++++++++++++++++++++++++++++++++++++++++++++++++++++++++

\begin{definition}
    Le \defe{cercle circonscrit}{cercle!circonscrit} à un triangle est le cercle passant par les trois sommets de ce triangle.
\end{definition}


\begin{center}
   \input{Fig_ZSAooVHmHWd0.pstricks}
\end{center}

En ce qui concerne la construction.

\begin{center}
   \input{Fig_ZSAooVHmHWd1.pstricks}
\end{center}


