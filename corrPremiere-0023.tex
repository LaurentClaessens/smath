% This is part of Un soupçon de mathématique sans être agressif pour autant
% Copyright (c) 2012
%   Laurent Claessens
% See the file fdl-1.3.txt for copying conditions.

\begin{corrige}{Premiere-0023}

    Soit \( x\) le prix de l'ancien assureur. Le nouvel assureur dit que \( 30\%\) de \( x\) font \( 240\) euros. Donc
    \begin{equation}
        \frac{ 30 }{ 100 }x=240,
    \end{equation}
    et par conséquent 
    \begin{equation}
        x=\frac{ 240\times 100 }{ 30 }=800.
    \end{equation}

\end{corrige}
