% This is part of Un soupçon de mathématique sans être agressif pour autant
% Copyright (c) 2012
%   Laurent Claessens
% See the file fdl-1.3.txt for copying conditions.

\begin{exercice}\label{exosmath-0238}

    Un athlète Syldave s'entraine pour le championnat national du lancer de chèvre. L'épreuve consiste à lancer une chèvre vers le haut depuis le bord d'une falaise au sommet d'une falaise située au bord d'un lac tranquille. La hauteur de la chèvre en fonction du temps (en secondes) par rapport à la surface du lac tranquille est une fonction \( f\) donnée par le graphique suivant.

    \begin{center}
%The result is on figure \ref{LabelFigfigureXCScSiP}. % From file figureXCScSiP
%\newcommand{\CaptionFigfigureXCScSiP}{<+Type your caption here+>}
\input{Fig_figureXCScSiP.pstricks}
    \end{center}
    À partir du graphique :
    \begin{enumerate}
        \item
            À quelle hauteur se trouve la chèvre au moment du lancer ?
        \item
            Pendant combien de temps la chèvre reste à une hauteur suppérieure à celle à laquelle elle a été lancée ?
        \item
            À quel moment la chèvre atteint-elle sa hauteur maximale ? Quelle est cette hauteur ?
        \item
            Au bout de combien de temps la chèvre touche-t-elle la surface de l'eau ?
        \item
            Résumer toutes ces informations en dressant le tableau de variation de la fonction \( f\).
    \end{enumerate}
    Quelque questions théoriques.

\corrref{smath-0238}
\end{exercice}
