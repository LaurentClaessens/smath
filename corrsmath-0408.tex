% This is part of Un soupçon de mathématique sans être agressif pour autant
% Copyright (c) 2013
%   Laurent Claessens
% See the file fdl-1.3.txt for copying conditions.

\begin{corrige}{smath-0408}

    Il s'agit de faire le tableau de signe des trois monômes dans le même tableau en faisant attention à
    \begin{enumerate}
        \item
            Mettre les «valeurs pivot» dans l'ordre croissant.
        \item
            Bien noter les valeurs interdites.
    \end{enumerate}
    Ici la seule valeur interdite est \( x=-3\) parce que \( x+3\) est au dénominateur.

    Le tableau est :
    \begin{equation*}
        \begin{array}[]{c|ccccccc}
            x&&-3&&-1&&1&\\
            \hline
            x+1&-&|&-&0&+&+&+\\
            1-x&+&|&+&+&+&0&-\\
            x+3&-&|&+&+&+&+&+\\
            \hline
            f(x)&+&|&-&0&+&0&-\\
        \end{array}
    \end{equation*}
    Nous avons mis une barre verticale à travers toute la colonne du \( -3\) parce que c'est la valeur interdite.

    Pour votre culture générale, la figure \ref{LabelFigOIHVjmO} montre le graphe de cette fonction. 

%The result is on figure \ref{LabelFigOIHVjmO}. % From file OIHVjmO
\newcommand{\CaptionFigOIHVjmO}{La fonction dont on étudie le signe dans l'exercice \ref{exosmath-0408}.}
\input{Fig_OIHVjmO.pstricks}

\end{corrige}
