% This is part of Un soupçon de mathématique sans être agressif pour autant
% Copyright (c) 2014
%   Laurent Claessens
% See the file fdl-1.3.txt for copying conditions.

\begin{exercice}\label{exosmath-0770}


    \begin{equation*}
        \begin{array}[]{|c|c|c|c|}
            \hline
            \unit{8}{\centi\meter}&\unit{5}{\centi\meter}&\unit{12}{\centi\meter}&\unit{2}{\centi\meter}\\
            \hline
            \unit{10}{\centi\meter}&\unit{12}{\centi\meter}&\unit{15}{\centi\meter}&\unit{10}{\centi\meter}\\
            \hline
            \unit{9}{\centi\meter}&\unit{3}{\centi\meter}&\unit{5}{\centi\meter}&\unit{7}{\centi\meter}\\
            \hline
        \end{array}
    \end{equation*}

Choisis trois nombres du tableau correspondant aux longueurs des côtés d'un triangle :
\begin{multicols}{2}
    \begin{enumerate}
        \item
non constructible ;
\item
 quelconque ;
 \item isocèle ;
\item
    de périmètre \unit{13}{\centi\meter}
    \end{enumerate}
\end{multicols}

\corrref{smath-0770}
\end{exercice}
