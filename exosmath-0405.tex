% This is part of Un soupçon de mathématique sans être agressif pour autant
% Copyright (c) 2013
%   Laurent Claessens
% See the file fdl-1.3.txt for copying conditions.

\begin{exercice}\label{exosmath-0405}


%The result is on figure \ref{LabelFigFLnDVHh}. % From file FLnDVHh
%\newcommand{\CaptionFigFLnDVHh}{<+Type your caption here+>}
\begin{minipage}{0.485\textwidth}
    Le dessin ci-contre montre le graphe de la fonction \( f(x)=-\frac{ 1 }{ 2 }x^2+x+3\). Tracer la tangente à ce graphe au point d'abscisse \( x=2\).  Justifier le dessin par un calcul de dérivée.
                    \end{minipage}
                    \hspace{1mm}    
                    \begin{minipage}{0.485\textwidth}
                            \begin{center}
                            \input{Fig_FLnDVHh.pstricks}
                            \end{center}
                    \end{minipage}

\corrref{smath-0405}
\end{exercice}
