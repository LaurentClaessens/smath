% This is part of Un soupçon de mathématique sans être agressif pour autant
% Copyright (c) 2014
%   Laurent Claessens
% See the file fdl-1.3.txt for copying conditions.

% This is part of Un soupçon de mathématique sans être agressif pour autant
% Copyright (c) 2014
%   Laurent Claessens
% See the file fdl-1.3.txt for copying conditions.

\begin{enumerate}
    \item
        Compléter la figure suivante.
    \item
        Quel est la médiatrice du segment \( [AA']\) ?
\end{enumerate}
 

\begin{center}
   \input{Fig_KREWooIrMfCQ.pstricks}
\end{center}



%+++++++++++++++++++++++++++++++++++++++++++++++++++++++++++++++++++++++++++++++++++++++++++++++++++++++++++++++++++++++++++ 
\section{Symétrie axiale}
%+++++++++++++++++++++++++++++++++++++++++++++++++++++++++++++++++++++++++++++++++++++++++++++++++++++++++++++++++++++++++++

\begin{definition}
    Soit un point \( P\) et une droite \( d\). Le \defe{symétrique}{symétrie axiale} de \( P\) par rapport à la droite \( d\) est le point \( P'\) tel que \( d\) soit la médiatrice de \( PP'\).
\end{definition}

Pour construire (ne pas faire écrire au cahier de leçon) :
\begin{itemize}
    \item Tracer une droite \( k\) perpendiculaire à \( d\) passant par \( P\).
    \item Cette droite coupe \( d\) un un point \( Q\).
    \item Tracer un cercle de rayon \( PQ\) centré en \( Q\).
    \item \( P'\) est le point d'intersection entre le cercle et la droite \( k\).
\end{itemize}

%+++++++++++++++++++++++++++++++++++++++++++++++++++++++++++++++++++++++++++++++++++++++++++++++++++++++++++++++++++++++++++ 
\section{Symétrie par rapport à un point}
%+++++++++++++++++++++++++++++++++++++++++++++++++++++++++++++++++++++++++++++++++++++++++++++++++++++++++++++++++++++++++++

Une fois n'est pas coutume, nous partons de la définition et nous faisons l'activité après pour la construction.
\begin{definition}
    Deux point \( P\) et \( P'\) sont \defe{symétriques par rapport au point \( O\)}{symétrie!centrale} si \( O\) est le milieu du segment \( [PP']\). 
\end{definition}


% This is part of Un soupçon de mathématique sans être agressif pour autant
% Copyright (c) 2014
%   Laurent Claessens
% See the file fdl-1.3.txt for copying conditions.

%--------------------------------------------------------------------------------------------------------------------------- 
\subsection*{Activité : ne pas tout mesurer}
%---------------------------------------------------------------------------------------------------------------------------


Mesurer les côtés de ces triangles et calculer l'aire totale de la figure grisée.
\begin{center}
   \input{Fig_RAACooAwsaVw.pstricks}
\end{center}
Comparer la longueur \( AB\) avec \( A'B'\).



