% This is part of Un soupçon de mathématique sans être agressif pour autant
% Copyright (c) 2014
%   Laurent Claessens
% See the file fdl-1.3.txt for copying conditions.

% This is part of Un soupçon de mathématique sans être agressif pour autant
% Copyright (c) 2014
%   Laurent Claessens
% See the file fdl-1.3.txt for copying conditions.

\begin{enumerate}
    \item
        Compléter la figure suivante.
    \item
        Quel est la médiatrice du segment \( [AA']\) ?
\end{enumerate}
 

\begin{center}
   \input{Fig_KREWooIrMfCQ.pstricks}
\end{center}



%+++++++++++++++++++++++++++++++++++++++++++++++++++++++++++++++++++++++++++++++++++++++++++++++++++++++++++++++++++++++++++ 
\section{Symétrie axiale}
%+++++++++++++++++++++++++++++++++++++++++++++++++++++++++++++++++++++++++++++++++++++++++++++++++++++++++++++++++++++++++++

\begin{definition}
    Soit un point \( P\) et une droite \( d\). Le \defe{symétrique}{symétrie axiale} de \( P\) par rapport à la droite \( d\) est le point \( P'\) tel que \( d\) soit la médiatrice de \( PP'\).
\end{definition}

Pour construire (ne pas faire écrire au cahier de leçon) :
\begin{itemize}
    \item Tracer une droite \( k\) perpendiculaire à \( d\) passant par \( P\).
    \item Cette droite coupe \( d\) un un point \( Q\).
    \item Tracer un cercle de rayon \( PQ\) centré en \( Q\).
    \item \( P'\) est le point d'intersection entre le cercle et la droite \( k\).
\end{itemize}

%+++++++++++++++++++++++++++++++++++++++++++++++++++++++++++++++++++++++++++++++++++++++++++++++++++++++++++++++++++++++++++ 
\section{Symétrie par rapport à un point}
%+++++++++++++++++++++++++++++++++++++++++++++++++++++++++++++++++++++++++++++++++++++++++++++++++++++++++++++++++++++++++++

% This is part of Un soupçon de mathématique sans être agressif pour autant
% Copyright (c) 2014
%   Laurent Claessens
% See the file fdl-1.3.txt for copying conditions.

%--------------------------------------------------------------------------------------------------------------------------- 
\subsection*{Activité : ne pas tout mesurer}
%---------------------------------------------------------------------------------------------------------------------------


Mesurer les côtés de ces triangles et calculer l'aire totale de la figure grisée.
\begin{center}
   \input{Fig_RAACooAwsaVw.pstricks}
\end{center}
Comparer la longueur \( AB\) avec \( A'B'\).



\begin{definition}
    Deux point \( P\) et \( P'\) sont \defe{symétriques par rapport au point \( O\)}{symétrie!centrale} lorsque \( O\) est le milieu du segment \( [PP']\). 
\end{definition}

Pour la construction du symétrique du point \( A\) par rapport à \( O\) :
\begin{itemize}
    \item Tracer la droite \( (AO)\)
    \item Pointer le compas en \( O\)
    \item Tracer le cercle de rayon \( AO\)
    \item Le point \( A'\) est l'intersection.
    \item Ne pas oublier de coder : la distance \( AO\) est égale à la distance \( OA'\).
\end{itemize}

\begin{center}
    \input{Fig_QDPKooICzieh.pstricks}
\end{center}

%+++++++++++++++++++++++++++++++++++++++++++++++++++++++++++++++++++++++++++++++++++++++++++++++++++++++++++++++++++++++++++ 
\section{Symétrique d'une droite}
%+++++++++++++++++++++++++++++++++++++++++++++++++++++++++++++++++++++++++++++++++++++++++++++++++++++++++++++++++++++++++++

% This is part of Un soupçon de mathématique sans être agressif pour autant
% Copyright (c) 2014
%   Laurent Claessens
% See the file fdl-1.3.txt for copying conditions.

%--------------------------------------------------------------------------------------------------------------------------- 
\subsection*{Activité : symétrique d'une droite}
%---------------------------------------------------------------------------------------------------------------------------

Placer trois points alignés \( A\), \( B\), \( C\) ainsi qu'un point \( O\) n'appartenant pas à la droite \( (AB)\). Nous nommons \( A'\), \( B'\) et \( C'\) les symétriques de \( A\), \( B\) et \( C\) par rapport à \( O\). Sont-ils alignés ?


Une propriété admise :
\begin{propriete}
    Le symétrique d'une droite par rapport à un point est une droite parallèle.
\end{propriete}

