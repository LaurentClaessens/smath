% This is part of Un soupçon de mathématique sans être agressif pour autant
% Copyright (c) 2014
%   Laurent Claessens
% See the file fdl-1.3.txt for copying conditions.

% This is part of Un soupçon de mathématique sans être agressif pour autant
% Copyright (c) 2014
%   Laurent Claessens
% See the file fdl-1.3.txt for copying conditions.

\begin{enumerate}
    \item
        Compléter la figure suivante.
    \item
        Quel est la médiatrice du segment \( [AA']\) ?
\end{enumerate}
 

\begin{center}
   \input{Fig_KREWooIrMfCQ.pstricks}
\end{center}



------

À moins que l'activité se soit mal passée, on ne fait pas la symétrie axiale en classe.

%+++++++++++++++++++++++++++++++++++++++++++++++++++++++++++++++++++++++++++++++++++++++++++++++++++++++++++++++++++++++++++ 
\section*{Symétrie axiale}
%+++++++++++++++++++++++++++++++++++++++++++++++++++++++++++++++++++++++++++++++++++++++++++++++++++++++++++++++++++++++++++

\begin{definition}
    Soit un point \( P\) et une droite \( (d)\). Le \defe{symétrique}{symétrie axiale} de \( P\) par rapport à la droite \( (d)\) est le point \( P'\) tel que \( (d)\) soit la médiatrice du segment \( [PP']\).
\end{definition}

Pour construire (ne pas faire écrire au cahier de leçon) :
\begin{itemize}
    \item Tracer la demi-droite \( k\) perpendiculaire à \( d\) et d'orignie \( P\).
    \item Reporter la distance de \( P\) à l'axe de l'autre côté.
\end{itemize}

%+++++++++++++++++++++++++++++++++++++++++++++++++++++++++++++++++++++++++++++++++++++++++++++++++++++++++++++++++++++++++++ 
\section{Symétrie par rapport à un point}
%+++++++++++++++++++++++++++++++++++++++++++++++++++++++++++++++++++++++++++++++++++++++++++++++++++++++++++++++++++++++++++

% This is part of Un soupçon de mathématique sans être agressif pour autant
% Copyright (c) 2014
%   Laurent Claessens
% See the file fdl-1.3.txt for copying conditions.

%--------------------------------------------------------------------------------------------------------------------------- 
\subsection*{Activité : ne pas tout mesurer}
%---------------------------------------------------------------------------------------------------------------------------


Mesurer les côtés de ces triangles et calculer l'aire totale de la figure grisée.
\begin{center}
   \input{Fig_RAACooAwsaVw.pstricks}
\end{center}
Comparer la longueur \( AB\) avec \( A'B'\).



\begin{definition}
    Deux point \( P\) et \( P'\) sont \defe{symétriques par rapport au point \( O\)}{symétrie!centrale} lorsque \( O\) est le milieu du segment \( [PP']\). 
\end{definition}

Pour la construction du symétrique du point \( A\) par rapport à \( O\) :
\begin{itemize}
    \item Tracer la demi-droite \(  [AO) \)
    \item Reporter la distance \( AO\) à partir du point \( O\).
    \item Ne pas oublier de coder : la distance \( AO\) est égale à la distance \( OA'\).
\end{itemize}

\begin{center}
    \input{Fig_QDPKooICzieh.pstricks}
\end{center}

\begin{propriete}
    La symétrie centrale conserve
    \begin{enumerate}
        \item
            les angles,
        \item
            les longueurs,
        \item
            les aires.
    \end{enumerate}
\end{propriete}

%+++++++++++++++++++++++++++++++++++++++++++++++++++++++++++++++++++++++++++++++++++++++++++++++++++++++++++++++++++++++++++ 
\section{Symétrique d'une droite}
%+++++++++++++++++++++++++++++++++++++++++++++++++++++++++++++++++++++++++++++++++++++++++++++++++++++++++++++++++++++++++++

% This is part of Un soupçon de mathématique sans être agressif pour autant
% Copyright (c) 2014
%   Laurent Claessens
% See the file fdl-1.3.txt for copying conditions.

%--------------------------------------------------------------------------------------------------------------------------- 
\subsection*{Activité : symétrique d'une droite}
%---------------------------------------------------------------------------------------------------------------------------

Placer trois points alignés \( A\), \( B\), \( C\) ainsi qu'un point \( O\) n'appartenant pas à la droite \( (AB)\). Nous nommons \( A'\), \( B'\) et \( C'\) les symétriques de \( A\), \( B\) et \( C\) par rapport à \( O\). Sont-ils alignés ?


Deux propriétés admises :
\begin{propriete}
    \begin{enumerate}
        \item
            Le symétrique d'une droite par rapport à un point est une droite parallèle.
        \item
            Le symétrique d'un segment par rapport à un point est une segment parallèle de même longueur.
    \end{enumerate}
\end{propriete}
