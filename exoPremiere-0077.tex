% This is part of Un soupçon de mathématique sans être agressif pour autant
% Copyright (c) 2012
%   Laurent Claessens
% See the file fdl-1.3.txt for copying conditions.

\begin{exercice}\label{exoPremiere-0077}

    Nous considérons le schéma de Bernoulli qui consiste à prendre \( 2000\) fois une vache au hasard et vérifier si elle a une maladie. 
    \begin{enumerate}
        \item
            Que signifie \( P(X=10)\) ?
        \item
            Si nous savons qu'en moyenne une vache sur \( 100\) est malade, est-ce que \( P(X=18)\) est plus grand ou plus petit que \( P(X=25)\) ?
    \end{enumerate}

\corrref{Premiere-0077}
\end{exercice}
