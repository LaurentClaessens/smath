% This is part of Un soupçon de mathématique sans être agressif pour autant
% Copyright (c) 2013
%   Laurent Claessens
% See the file fdl-1.3.txt for copying conditions.

\begin{corrige}{smath-0495}

    Nous écrivons les fonctions qui au nombre de semaines fait correspondre le nombre de bouteilles présentes dans chacune des deux caves. Vu que la première cave commence à \( 200\) et perd \( 10\) bouteilles par semaine, la première fonction est :
    \begin{equation}
        f_1(x)=200-10x.
    \end{equation}
    En ce qui concerne la seconde cave, elle gagne \( 9\) bouteilles par semaine. Donc :
    \begin{equation}
        f_2(x)=9x.
    \end{equation}
    Pour trouver à quel moment on arrive à égalité, il faut résoudre l'équation \( f_1(x)=f_2(x)\), c'est à dire
    \begin{equation}
        200-10x=9x
    \end{equation}
    dont la solution est \( x=\frac{ 200 }{ 19 }\simeq 10.52\).

    En conclusion, après \( 10\) semaines, il y aura encore quelque bouteilles de plus dans la première cave et après la onzième semaine il en aura un peu plus dans la seconde. Notre buveur doit donc s'arrêter après \( 10\) ou \( 11\) semaines suivant ce qu'il préfère.

\end{corrige}
