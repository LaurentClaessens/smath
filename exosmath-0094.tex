% This is part of Un soupçon de mathématique sans être agressif pour autant
% Copyright (c) 2012-2013
%   Laurent Claessens
% See the file fdl-1.3.txt for copying conditions.

\begin{exercice}\label{exosmath-0094}

\begin{wrapfigure}{r}{0.2\textwidth}
    \centering
    \vspace{-0.5cm}
\input{Fig_figureWFDTzSN.pstricks}
\end{wrapfigure}

    Nous considérons une sphère de centre \( O\) et de rayon \( \unit{65}{\centi\meter}\).
        \begin{enumerate}
            \item
                Quel est le rayon de la section horizontale à \( \unit{56}{\centi\meter}\) de haut ?
            \item
                Le dessin ci-contre représente en coupe un cylindre inclus à la sphère.
            \item
                Calculer le volume du cylindre.
        \end{enumerate}

        %The result is on figure \ref{LabelFigfigureWFDTzSN}. % From file figureWFDTzSN
%\newcommand{\CaptionFigfigureWFDTzSN}{<+Type your caption here+>}
%        \begin{center}
%\input{Fig_figureWFDTzSN.pstricks}
%        \end{center}

        Faire à la maison l'exercice 10 de la page 234.

\corrref{smath-0094}
\end{exercice}
