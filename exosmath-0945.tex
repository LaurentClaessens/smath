% This is part of Un soupçon de mathématique sans être agressif pour autant
% Copyright (c) 2014
%   Laurent Claessens
% See the file fdl-1.3.txt for copying conditions.

\begin{exercice}\label{exosmath-0945}

    Compléter les tableaux de proportionnalité suivants :
    \begin{multicols}{2}
    \begin{enumerate}
        \item
            \begin{equation*}
                \begin{array}[]{|c|c|c|c|}
                    \hline
                    10&15&20&40\\
                    \hline
                    \ldots&30&\ldots&\ldots\\
                    \hline
                \end{array}
            \end{equation*}
        \item
            \begin{equation*}
                \begin{array}[]{|c|c|c|c|}
                    \hline
                     2&\ldots&9&\ldots\\
                      \hline
                      \ldots&56&63&77\\ 
                      \hline 
                       \end{array}
                   \end{equation*}
               \item
                   \begin{equation*}
                       \begin{array}[]{|c|c|c|c|c|}
                           \hline
                           1&\ldots&8&10&\ldots\\
                           \hline
                           \ldots&\dfrac{ 9 }{2}&9&\ldots&900\\
                           \hline
                       \end{array}
                   \end{equation*}
                   \item
                       \begin{equation*}
                           \begin{array}[]{|c|c|c|c|}
                               \hline
                                2&\ldots&\ldots&60\\
                                 \hline
                                 \ldots&4&10&15\\ 
                                 \hline 
                                  \end{array}
                              \end{equation*}
    \end{enumerate}
    \end{multicols}

\corrref{smath-0945}
\end{exercice}
