%This is part of Un soupçon de mathématique sans être agressif pour autant
% Copyright (c) 2012-2013
%   Laurent Claessens, Pauline Klein
% See the file fdl-1.3.txt for copying conditions.

%+++++++++++++++++++++++++++++++++++++++++++++++++++++++++++++++++++++++++++++++++++++++++++++++++++++++++++++++++++++++++++
\section{Courbe représentative d'une fonction}
%+++++++++++++++++++++++++++++++++++++++++++++++++++++++++++++++++++++++++++++++++++++++++++++++++++++++++++++++++++++++++++

\begin{Aprojeter}
    %This is part of Un soupçon de mathématique sans être agressif pour autant
% Copyright (c) 2012-2013
%   Laurent Claessens
% See the file fdl-1.3.txt for copying conditions.

    Un berger syldave s'entraine pour le championnat national du lancer de chèvre. L'épreuve consiste à lancer une chèvre vers le haut depuis le bord d'une falaise située au bord d'un lac tranquille. La hauteur de la chèvre en fonction du temps par rapport à la surface du lac tranquille est une fonction \( f\) donnée par le graphique suivant.

    \begin{center}
        \input{Fig_WRXbDCo.pstricks}
    \end{center}
    La dernière partie du graphique correspond à la chèvre que l'on remonte rapidement hors de l'eau.
    À partir du graphique :
    \begin{enumerate}
        \item
            À quelle hauteur se trouve la chèvre au moment du lancer ?
        \item
            Pendant combien de temps la chèvre reste à une hauteur supérieure à celle à laquelle elle a été lancée ?
        \item
            À quel moment la chèvre atteint-elle sa hauteur maximale ? Quelle est cette hauteur ?
        \item
            À quelle hauteur se trouve la chèvre après \( 2.5\) secondes de vol ?
        \item
            Résumer toutes ces informations en dressant le tableau de variation de la fonction \( f\).
    \end{enumerate}

\end{Aprojeter}

%TODO : il faut essayer de refaire une figure pour tous les dessins de Pauline.

\begin{definition}
Soit $f$ une fonction définie sur un ensemble $\defD$.
    On appelle \defe{représentation graphique}{représentation graphique (d'une fonction)}, ou le \defe{graphe}{graphe} de $f$, l'ensemble des points $(x,y)$ tels que $x\in\defD$ et $y=f(x)$.

    Lorsque \( y=f(x)\), le nombre \( y\) est l'\defe{image}{image par une fonction} de \( x\) par la fonction \( f\) et \( x\) est \emph{un} \defe{antécédent}{antécédent} de \( y\).
\end{definition}

\begin{Aprojeter}

    \begin{Aretenir}
        La règle d'or des graphiques : le point de coordonnées \( (a;b)\) est sur le graphique de la fonction \( f\) si et seulement si \( f(a)=b\).
    \end{Aretenir}

    \begin{multicols}{2}

    À propos du graphe ci-contre :
    \begin{enumerate}
        \item
            Quel est l'ensemble de définition de la fonction \( f\) ?
        \item
            Quelle est l'image de \( 1\) par \( f\) ?
        \item
            Donner un antécédent de \( 3\).
        \item
            Que vaut \( f(-3)\) ?
        \item
            Quels sont les antécédents de \( -1\) ?
    \end{enumerate}
        
        \columnbreak

    \begin{center}
       \input{Fig_AHAbqhj.pstricks}
    \end{center}

    \end{multicols}
\end{Aprojeter}

%Quelque conseils pour dessiner.
%\begin{itemize}
%    \item
%        Pour une valeur $x$ sur l'axe des abscisses, il y a un et un seul point d'abscisse $x$ sur la courbe.
%    \item
%        Pour tracer une courbe, il faut placer des points. Plus on choisit de points, plus la courbe sera précise.
%    \item
%        Si possible, trouver quelque valeurs clefs. Par exemple on cherchera les points d'intersection entre les axes et les courbe. Le point \( (0,f(0)) \) est intéressant à mettre, ainsi que les points \( x\) tels que \( f(x)=0\).
%\end{itemize}


%\newcommand{\CaptionFigExFonction}{Comment tracer la fonction \( f\colon x\to 2x+1\) ?}
%\input{Fig_ExFonction.pstricks}

%Nous donnons à la figure \ref{LabelFigExFonction} le tracé de la fonction \( f(x)=2x+1\). La figure \ref{LabelFigExFonctionssLabelSubFigExFonction0} donne quelque points du graphe de la fonction. La figure \ref{LabelFigExFonctionssLabelSubFigExFonction1} donne le graphe complet de la fonction. Comment le construit-on ? Par définition pour chaque \( x\) sur l'axe des abscisses (il y en a une infinité), il faut calculer le nombre \( f(x)\) et mettre dans le plan le point de coordonnées \( \big( x,f(x) \big)\).

%En pratique, il n'est pas possible de calculer \( f(x)\) pour \emph{tous} les \( x\) réels\footnote{Chuck Norris peut le faire.}. C'est pourquoi nous nous contentons qu'en calculer quelque uns, et nous les relions «le plus intelligemment possible».


%---------------------------------------------------------------------------------------------------------------------------
\subsection{Ce qui n'est pas une fonction}
%---------------------------------------------------------------------------------------------------------------------------

%    \includegraphics[width=5cm]{F_NonFct.pdf}

\begin{multicols}{2}
    Cette courbe ne représente pas une fonction, car à partir de \( x=-4\), les nombres ont deux images. Les courbes données par des fonctions sont des courbes acceptant une seule ordonnée pour chaque abscisse.

\columnbreak

%\includegraphics{Picture_FIGLabelFigPasFonctionYoQfSuPICTPasFonctionYoQfSu-for_eps.pdf}

\input{Fig_PasFonctionYoQfSu.pstricks}

\end{multicols}

%+++++++++++++++++++++++++++++++++++++++++++++++++++++++++++++++++++++++++++++++++++++++++++++++++++++++++++++++++++++++++++ 
\section{Pour tracer}
%+++++++++++++++++++++++++++++++++++++++++++++++++++++++++++++++++++++++++++++++++++++++++++++++++++++++++++++++++++++++++++

Pour tracer le graphe d'une fonction affine.
\begin{itemize}
    \item
        Vu que le graphe est une droite, il suffit de deux points.
    \item
        Les fonction linéaires passent par l'origine \( (0;0)\).
    \item 
        Pour un tracé à la règle, il est plus précis de prendre deux points relativement éloignés.
    \item
        La droite \( f(x)=mx+p\) monte si \( m>0\) et descend si \( m<0\). La pente est d'autant plus raide que \( m\) est grand.
\end{itemize}

Quelque exemples à la figure \ref{LabelFigGrapheAffinHqXJGx}.
\newcommand{\CaptionFigGrapheAffinHqXJGx}{Des graphes de fonctions linéaires et affines.}
\input{Fig_GrapheAffinHqXJGx.pstricks}

