% This is part of Un soupçon de mathématique sans être agressif pour autant
% Copyright (c) 2013
%   Laurent Claessens
% See the file fdl-1.3.txt for copying conditions.

%+++++++++++++++++++++++++++++++++++++++++++++++++++++++++++++++++++++++++++++++++++++++++++++++++++++++++++++++++++++++++++ 
\section{Introduction}
%+++++++++++++++++++++++++++++++++++++++++++++++++++++++++++++++++++++++++++++++++++++++++++++++++++++++++++++++++++++++++++

\begin{minipage}{0.485\textwidth}
Prenons la parabole \( f(x)=x^2-2x-3\) dessinée ci-contre. Nous savons qu'elle est tournée vers le haut, et qu'elle a un sommet en \( x=1\). Son tableau de variations est :
\begin{equation*}
    \begin{array}[]{c|ccccc}
        x&-\infty&&1&&\infty\\
        \hline
        &\infty&&&&\infty\\
        f(x)&&\searrow&&\nearrow&\\
        &&&-4&&\\
    \end{array}
\end{equation*}
\end{minipage}
\hspace{1mm}
%The result is on figure \ref{LabelFigCZAVGrm}. % From file CZAVGrm
%\newcommand{\CaptionFigCZAVGrm}{<+Type your caption here+>}
                    \begin{minipage}{0.485\textwidth}
                            \begin{center}
                            \input{Fig_CZAVGrm.pstricks}
                            \end{center}
                    \end{minipage}

%+++++++++++++++++++++++++++++++++++++++++++++++++++++++++++++++++++++++++++++++++++++++++++++++++++++++++++++++++++++++++++ 
\section{Tangente}
%+++++++++++++++++++++++++++++++++++++++++++++++++++++++++++++++++++++++++++++++++++++++++++++++++++++++++++++++++++++++++++

\begin{Aretenir}
    Pour trouver la tangente au graphe de la fonction \( f\) au point \( A=\big( a,f(a) \big)\), nous faisons :
    \begin{enumerate}
        \item
            Calculer \( m\), le nombre dérivé de $f$ en \( x=a\).
        \item
            La droite tangente est celle de coefficient angulaire \( m\) et passant par le point \( A=\big( a,f(a) \big)\), c'est à dire
            \begin{equation}
                y=mx+p
            \end{equation}
            où \( p\) doit encore être trouvé.
        \item
            Le \( p\) se trouve en résolvant l'équation
            \begin{equation}
                ma+p=f(a).
            \end{equation}
    \end{enumerate}
    Au final, la tangente a pour équation
    \begin{equation}
        y=m(x-a)+f(a).
    \end{equation}
\end{Aretenir}

Étant donné que la tangente est la droite qui colle le mieux à la courbe, nous avons les interprétations suivantes. Soit \( m\) la dérivée de la fonction \( f\) au point \( x=a\).
\begin{enumerate}
    \item
        Si \( m>0\) alors la fonction est croissante en \( a\).
    \item
        Si \( m<0\) alors la fonction est décroissante en \( a\).
\end{enumerate}
Et surtout : plus \( m\) est grand, plus la pente de la courbe est forte.

%+++++++++++++++++++++++++++++++++++++++++++++++++++++++++++++++++++++++++++++++++++++++++++++++++++++++++++++++++++++++++++ 
\section{Exercices}
%+++++++++++++++++++++++++++++++++++++++++++++++++++++++++++++++++++++++++++++++++++++++++++++++++++++++++++++++++++++++++++

\Exo{smath-0300}
\Exo{smath-0301}
\Exo{smath-0397}
\Exo{smath-0387}
\Exo{smath-0351}
\Exo{smath-0400}
\Exo{smath-0401}
\Exo{smath-0402}
\Exo{smath-0403}
\Exo{smath-0407}
