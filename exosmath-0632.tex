% This is part of Un soupçon de mathématique sans être agressif pour autant
% Copyright (c) 2014
%   Laurent Claessens
% See the file fdl-1.3.txt for copying conditions.

\begin{exercice}\label{exosmath-0632}

\begin{wrapfigure}{r}{5.0cm}
   \vspace{-5.5cm}        % à adapter.
   \centering
   \input{Fig_QGRiHbb.pstricks}
\end{wrapfigure}

    Un oiseau pêche son poisson en plongeant dans un lac depuis une falaise. Nous modélisons sa trajectoire (hauteur par rapport à la surface de l'eau en fonction de la distance horizontale de la falaise) par la fonction
    \begin{equation*}
        h(x)=x^2-6x+5
    \end{equation*}
    dessinée ci-contre.
    \begin{enumerate}
        \item
            De quelle hauteur l'oiseau a-t-il sauté ?
        \item

    \end{enumerate}
    <++>

\corrref{smath-0632}
\end{exercice}
