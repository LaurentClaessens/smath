% This is part of Un soupçon de mathématique sans être agressif pour autant
% Copyright (c) 2013
%   Laurent Claessens
% See the file fdl-1.3.txt for copying conditions.

%+++++++++++++++++++++++++++++++++++++++++++++++++++++++++++++++++++++++++++++++++++++++++++++++++++++++++++++++++++++++++++ 
\section{Univers}
%+++++++++++++++++++++++++++++++++++++++++++++++++++++++++++++++++++++++++++++++++++++++++++++++++++++++++++++++++++++++++++

Nous allons fonder notre étude sur l'étude des trois expériences aléatoires.
\begin{enumerate}
    \item
        Lancer un dé à \( 6\) faces.
    \item
        Tirer une carte d'un jeu de \( 52\) cartes.
    \item
        Tirer une boule d'une urne contenant \( 16\) boules dont trois vertes, cinq jaunes, deux bleues et six rouges.
\end{enumerate}

\begin{definition}
    Une \defe{expérience aléatoire}{expérience aléatoire} est une expérience qui a plusieurs issues possibles, et pour laquelle on ne peut ni prévoir ni calculer laquelle des issues sera réalisée.
\end{definition}

\begin{definition}
    L'ensemble des issues possibles d'une expérience aléatoire est l'\defe{univers}{univers} de l'expérience. Il sera noté \( E\).
\end{definition}

\begin{enumerate}
    \item
        Lancer un dé à \( 6\) faces. L'univers est \( \{ 1,2,3,4,5,6 \}\).
    \item
        Tirer une carte d'un jeu de \( 52\) cartes. L'univers est \( \{ \text{as de pique},\text{cinq de carreau},\text{valet de trèfle}, \ldots \}\).
    \item
        Tirer une boule d'une urne qui contient par exemple trois boules vertes, cinq jaunes, deux bleues et six rouges. L'univers est \( \{ V,J,B,R \}\).
\end{enumerate}


\begin{definition}
    Un \defe{événement}{événement} d'une expérience aléatoire est une partie de l'univers : un sous-ensemble de l'ensemble des issues possibles.
\end{definition}

\begin{definition}
    Lorsque toutes les issues possibles ont la même probabilité, nous disons que nous sommes dans une situation d'\defe{équiprobabilité}{équiprobabilité}.
\end{definition}

%+++++++++++++++++++++++++++++++++++++++++++++++++++++++++++++++++++++++++++++++++++++++++++++++++++++++++++++++++++++++++++ 
\section{Intersection, union, complémentaire}
%+++++++++++++++++++++++++++++++++++++++++++++++++++++++++++++++++++++++++++++++++++++++++++++++++++++++++++++++++++++++++++

%TODO : des dessins pour expliquer la réunion et l'intersection.

Si \( A\) et \( B\) sont des événements, 
\begin{enumerate}
    \item
nous notons \( A\cup B\) l'événement qui est réalisé lorsque \( A\) \emph{ou} \( B\) se réalise;
\item
nous notons \( A\cap B\) l'événement qui est réalisé lorsque \( A\) \emph{et} \( B\) se réalisent.
\end{enumerate}

\begin{Aretenir}
    Si \( A\) et \( B\) sont deux événements d'une expérience aléatoire, alors
    \begin{equation}
        p(A\cup B)=p(A)+p(B)-p(A\cap B).
    \end{equation}
\end{Aretenir}

\begin{example}
    Soient \( A\) l'événement «obtenir un roi» et \( B\) l'événement «obtenir un \( \heartsuit\)». Pour calculer \( p(A\cup B)\), on suit le raisonnement suivant.
    \begin{enumerate}
        \item
            La probabilité de \( A\) est \( p(A)=\frac{ 4 }{ 52 }=\frac{1}{ 13 }\). 
        \item
            La probabilité de \( B\) est \( p(B)=\frac{ 13 }{ 52 }=\frac{1}{ 4 }\).
        \item
            L'événement \( A\cup B\) est l'événement «avoir un roi ou un cœur». Les cartes qui réalisent cet événement sont toutes les \( 13\) cartes de cœur plus les quatre rois (dont un est de cœur!!). 
        \item
            L'événement \( A\cap B\) est réalisé par les cartes qui sont à la fois cœur et roi, c'est à dire uniquement par la carte roi de \( \heartsuit\). Nous avons donc \( p(A\cap B)=\frac{1}{ 52 }\).
        \item
            Pour savoir la probabilité \( p(A\cup B)\), on utilise la formule
            \begin{equation}
                p(A\cup B)=p(A)+p(B)-p(A\cap B)=\frac{1}{ 13 }+\frac{1}{ 4 }-\frac{1}{ 52 }=\frac{ 4 }{ 13 }.
            \end{equation}
    \end{enumerate}
    Notons que ce résultat peut aussi être obtenu en réfléchissant. L'événement \( A\cup B\) est constitué de toutes les cartes \( \heartsuit\) et de tous les rois, c'est à dire les treize cartes \( \heartsuit\) plus les rois de \( \clubsuit\), \( \spadesuit\) et \( \diamondsuit\), soient \( 13+3=16\) cartes en tout. Donc
    \begin{equation}
        p(A\cup B)=\frac{ 16 }{ 52 }=\frac{ 4 }{ 13 }.
    \end{equation}
\end{example}

%+++++++++++++++++++++++++++++++++++++++++++++++++++++++++++++++++++++++++++++++++++++++++++++++++++++++++++++++++++++++++++ 
\section{Probabilité}
%+++++++++++++++++++++++++++++++++++++++++++++++++++++++++++++++++++++++++++++++++++++++++++++++++++++++++++++++++++++++++++

Si nous recommençons plusieurs fois une même expérience aléatoire, nous pouvons parler de \defe{fréquence}{fréquence} d'apparition de l'événement \( A\).

\begin{definition}
    Lorsqu'on effectue un grand nombre de fois une expérience aléatoire, la fréquence d'apparition d'un événement \( A\) tend à se stabiliser autour d'une «fréquence théorique» appelée la \defe{probabilité}{probabilité} de \( A\) et notée \( p(A)\).
\end{definition}

\begin{example}
    Si \( A\) désigne l'événement «obtenir la face \( 4\) » lors du lancé du dé à \( 6\) faces, la probabilité est \( p(A)=\frac{1}{ 6 }\) parce que nous nous attendons à obtenir le \( 4\) une fois sur six.

    Si \( B\) désigne l'événement «obtenir un nombre pair», alors la probabilité est \( p(A)=\frac{ 1 }{2}\) parce qu'un lancer sur deux doit en moyenne tomber sur un nombre pair.
\end{example}

\begin{example}
    Si l'événement \( A\) désigne «obtenir l'as de $\spadesuit$», alors la probabilité est de \( 1/52\).
    
    Si \( B\) désigne l'événement «obtenir une dame», alors la probabilité est de \( 4/52=1/13\).
\end{example}

\begin{example}
    L'urne contient en tout \( 16\) boules. Nous avons
    \begin{itemize}
        \item \( p(V)=\frac{ 3 }{ 16 }\);
        \item
            \( p(R)=\frac{ 6 }{ 16 }=\frac{ 3 }{ 8 }\);
        \item
            \( p(B)=\frac{ 2 }{ 16 }=\frac{1}{ 8 }\).
    \end{itemize}
    
\end{example}


\begin{propriete}
    \begin{enumerate}
        \item
            Une probabilité est un nombre compris entre \( 0\) et \( 1\) : \( 0\leq p(A)\leq 1\).
        \item
            Un événement impossible a une probabilité \( 0\) : \( p(\emptyset)=0\).
        \item
            Un événement dont la probabilité est \( 1\) est appelé \defe{événement certain}{événement!certain}.
        \item
            La somme des probabilités de tous les événements élémentaires vaut toujours \( 1\).
    \end{enumerate}
\end{propriete}

\begin{propriete}
    Si une expérience possède \( n\) issues équiprobables, alors :
    \begin{enumerate}
        \item
            La probabilité de chaque événement primaire est \( \frac{1}{ n }\).
        \item
            La probabilité d'un événement \( A\) est donnée par
            \begin{equation}
                p(A)=\frac{ \Card(A) }{ n }
            \end{equation}
            où \( \Card(A)\) est le \defe{cardinal}{cardinal} de \( A\), c'est à dire le nombre d'éléments dans \( A\).
    \end{enumerate}
\end{propriete}

\begin{example}
    \begin{description}
        \item[Le dé à 6 faces]
            La probabilité d'un événement élémentaire est \( \frac{1}{ 6 }\).

            Si \( A\) est l'événement «obtenir un nombre pair», alors \( p(A)=\frac{ 3 }{ 6 }=\frac{ 1 }{2}\) parce qu'il y a trois nombres pairs : \( 2\), \( 4\) et \( 6\). Il y a donc trois issues possibles qui réalisent l'événement \( A\).
        \item[Le jeu de carte]
            Les événements élémentaires sont chacune des cartes et leurs probabilités sont \( \frac{1}{ 52 }\). 
            
            
            Si \( A\) est l'événement «Tirer une carte de \( \heartsuit\)», alors \( p(A)=\frac{ 13 }{ 52 }=\frac{1}{ 4 }\) parce qu'il y a \( 13\) cartes de \( \heartsuit\) sur \( 52\) cartes en tout.

        \item[L'urne] Les événements élémentaires (tirer une boule verte, jaune, bleue ou rouge) ne sont pas équiprobables. Par exemple la probabilité d'avoir une rouge est \( p(R)=\frac{ 6 }{ 16 }=\frac{ 3 }{ 8 }\).

            Si \( A\) est l'événement «tirer une boule jaune ou verte», alors nous avons \( p(A)=\frac{ 5+3 }{ 16 }=\frac{ 1 }{ 2 }\).
    \end{description}
\end{example}

\begin{definition}
    Si \( A\) est un événement, l'\defe{événement contraire}{événement!contraire}, noté \( \bar A\), est l'événement constitué de l'ensemble des événement élémentaires n'appartenant pas à \( A\). On a l'égalité
    \begin{equation}
        p(\bar A)=1-p(A).
    \end{equation}
\end{definition}

\begin{example}
    \begin{description}
        \item[Le dé à six face] Si \( A\) est l'événement «obtenir \( 3\) ou \( 4\)», alors \( \bar A\) est l'événement «ne pas obtenir \( 3\) ou \( 4\)», c'est à dire «obtenir \( 1\), \( 2\), \( 5\) ou \( 6\)».
        \item[Le jeu de cartes] Si \( A\) est l'événement «tirer un \( \spadesuit\)», alors l'événement contraire \( \bar A\) est l'événement «Tirer un \( \heartsuit\), \( \diamondsuit\) ou \( \clubsuit\)».
        \item[L'urne] Si \( A\) est l'événement «Tirer une boule verte ou jaune», alors le complémentaire est \( \bar A\) «Tirer une boule bleue ou rouge».
    \end{description}
\end{example}

%+++++++++++++++++++++++++++++++++++++++++++++++++++++++++++++++++++++++++++++++++++++++++++++++++++++++++++++++++++++++++++ 
\section{Événement élémentaire}
%+++++++++++++++++++++++++++++++++++++++++++++++++++++++++++++++++++++++++++++++++++++++++++++++++++++++++++++++++++++++++++

\begin{definition}
    Un événement est \defe{élémentaire}{événement=élémentaire} si il n'est réalisé que par une seule issue de l'expérience.
\end{definition}

\begin{example}
    \begin{description}
        \item[Pour le dé]
            \begin{itemize}
                \item Obtenir le six (événement élémentaire).
                \item
                    Obtenir un nombre pair.
                \item
                    Obtenir \( 2\) ou \( 5\).
            \end{itemize}
        \item[Pour le jeu de cartes]
    \begin{itemize}
        \item
            Tirer le six de cœur (événement élémentaire).
        \item Tirer une carte noire.
        \item
            Tirer une dame.
    \end{itemize}
\item[Pour l'urne]
    \begin{itemize}
        \item Tirer une boule rouge (événement élémentaire).
        \item
            Tirer une boule rouge ou verte.
        \item
            Tirer une boule d'une couleur autre que verte.
    \end{itemize}
    \end{description}
\end{example}
