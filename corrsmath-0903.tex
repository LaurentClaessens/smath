% This is part of Un soupçon de mathématique sans être agressif pour autant
% Copyright (c) 2014
%   Laurent Claessens
% See the file fdl-1.3.txt for copying conditions.

\begin{corrige}{smath-0903}

    La proportion d'élèves ayant raté est de \( 8\) sur \( 24\), c'est à dire un tiers. Le troisième diagramme est donc bon, et le second est faux.

    Visuellement, le premier a l'air correct aussi parce que l'aire remplie semble être le tiers. En comptant, on remarque que \( 4\) tranches sur \( 12\) sont remplies, mais
    \begin{equation}
        \frac{ 4 }{ 12 }=\frac{ 4\div 4 }{ 12\div 4 }=\frac{ 1 }{ 3 }.
    \end{equation}
    Le premier diagramme est donc également correct.

    En ce qui concerne la proportion de garçons dans la classe, elle est de dix garçons sur \( 24\) élèves, c'est à dire dix vingt-quatrièmes.

    La première possibilité est fausse parce qu'elle est seulement huit vingt-quatrièmes. La seconde est également fausse.

    Le fait que \( \frac{ 14 }{ 10 }\) ne soit pas égal à \( \frac{ 10 }{ 24 }\) n'est en réalité par très facile à justifier. Il y a plusieurs méthodes. Une première est de remarquer que \( \frac{ 14 }{ 10 }\) est plus grand que \( 1\) alors que \( \frac{ 10 }{ 24 }\) est plus petit que \( 1\) (pour le voir, effectuer la division \( 14\div 10\) et \( 10\div 24\)). Une autre façon est de comparer les fractions simplifiées.

\end{corrige}
