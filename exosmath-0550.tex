% This is part of Un soupçon de mathématique sans être agressif pour autant
% Copyright (c) 2013
%   Laurent Claessens
% See the file fdl-1.3.txt for copying conditions.

\begin{exercice}\label{exosmath-0550}

    Soit \( h\) une fonction définie sur \( \mathopen[ -2 ; 10 \mathclose]\) telle que 
    \begin{enumerate}
        \item
            l'image de \( -2\) est \( 0\) et un antécédent de \( 7\) est \( 10\).,
        \item
            \( h\) est croissante sur \( \mathopen[ 1 ; 4 \mathclose]\), décroissante sur \( \mathopen[ -2 ; 1 \mathclose]\) et sur \( \mathopen[ 4 ; 10 \mathclose]\).,
        \item
            le minimum de \( h\) est \( -4\) et son maximum est \( 11\).
    \end{enumerate}
    À partir de  ces informations, répondre aux questions suivantes :
    \begin{enumerate}
        \item
            Dresser le tableau de variations de la fonction \( h\).
        \item
            Comparer (si possible) les nombres suivants : \( 0\) et \( h(0)\); \( h(-1)\) et \( h(3)\); \( h(7)\) et \( h(8)\).
        \item
            Construire une courbe représentative de \( h\) possible.
    \end{enumerate}

\corrref{smath-0550}
\end{exercice}
