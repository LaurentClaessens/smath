% This is part of Un soupçon de mathématique sans être agressif pour autant
% Copyright (c) 2013
%   Laurent Claessens
% See the file fdl-1.3.txt for copying conditions.

\begin{exercice}[\cite{NAuSlZN}]\label{exosmath-0315}

Quatre régions ont le même nombre d'habitants. Mais dans la première région, il y a un médecin pour \( 100\) habitants, dans la deuxième région, il y a un médecin pour \( 200\) habitants, dans la troisième, un médecin pour \( 250\) habitants, et dans la quatrième, un médecin pour \( 500\) habitants.

Quel est le nombre moyen d'habitants par médecins pour l'ensemble des quatre régions ?

\corrref{smath-0315}
\end{exercice}
