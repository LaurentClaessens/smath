% This is part of Un soupçon de mathématique sans être agressif pour autant
% Copyright (c) 2015
%   Laurent Claessens
% See the file fdl-1.3.txt for copying conditions.

\begin{exercice}\label{exo2smath-0096}

    Un train avance à vitesse constante (la distance parcourue est donc proportionnelle au temps de voyage). Lequel de ces trois graphiques correspond à cette situation ?

\begin{center}
   \input{Fig_QXYWooOWHshl0.pstricks}
   \input{Fig_QXYWooOWHshl1.pstricks}
   \input{Fig_QXYWooOWHshl2.pstricks}
\end{center}

\corrref{2smath-0096}
\end{exercice}
