% This is part of Un soupçon de mathématique sans être agressif pour autant
% Copyright (c) 2014
%   Laurent Claessens
% See the file fdl-1.3.txt for copying conditions.

\begin{corrige}{smath-0895}

    Le premier programme s'exprime avec le diagramme suivant :
    \begin{equation}
        \boxed{\ldots}\stackrel{\times 6}{\longrightarrow}\boxed{\ldots}\stackrel{+11}{\longrightarrow}\boxed{\ldots}
    \end{equation}
    et le second par
    \begin{equation}
        \boxed{\ldots}\stackrel{+2}{\longrightarrow}\boxed{\ldots}\stackrel{\times 6}{\longrightarrow}\boxed{\ldots}
    \end{equation}
    \begin{enumerate}
        \item
            \begin{description}
                \item[Programme 1]
                    \begin{itemize}
                        \item En choisissant \( 2\), on obtient \( 23\).
                        \item En choisissant \( -3\), on obtient \( -18+11=-7\).
                        \item En choisissant \( 4\), on obtient \( 4\times 6+11=24+11=35\).
                    \end{itemize}
                \item[Programme 2]
                    \begin{itemize}
                        \item En choisissant \( 2\), on obtient \( 24\).
                        \item En choisissant \( -3\), on obtient \( (-3+2)\times 6=-1\times 6=-6 \).
                        \item En choisissant \( 4\), on obtient \( 6\times 6=36\).
                    \end{itemize}
            \end{description}
        \item
            Nous voyons que le second programme donne un résultat de un plus grand que le premier.
        \item
            Le plus simple est de suivre le chemin suivant :
    \begin{equation}
        \boxed{x}\stackrel{\times 6}{\longrightarrow}\boxed{6x}\stackrel{+11}{\longrightarrow}\boxed{6x+11}
    \end{equation}
    Nous avons donc \( A=6x+11\). Les réponses suivante sont également correctes: \( 6\times x+11\) ou \( x\times 6+11\).
\item
    Nous suivant le chemin suivant :
    \begin{equation}
        \boxed{x}\stackrel{+2}{\longrightarrow}\boxed{x+2}\stackrel{\times 6}{\longrightarrow}\boxed{ (x+2)\times 6  }
    \end{equation}
    Donc \( B=(x+2)\times 6\). Il y a aussi moyen d'écrire cela sous la forme \( 6(x+2)\).

\item
    Il s'agit de prouver que \( B\) est toujours \( A\) plus un. Pour cela nous pouvons développer l'expression de \( B\) :
    \begin{equation}
        B=(x+2)\times 6=6x+12.
    \end{equation}
    La comparaison entre \( A=6x+11\) et \( B=6x+12\) montre que \( B\) vaut toujours \( 1\) de plus que \( A\) :
    \begin{equation}
        A+1=6x+11+1=6x+12,
    \end{equation}
    ce qui signifie que \( A+1\) est \( B\). Ou encore que le résultat du deuxième programme vaut un de plus que celui du premier.
            
    \end{enumerate}
    

\end{corrige}
