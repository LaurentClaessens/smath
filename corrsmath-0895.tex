% This is part of Un soupçon de mathématique sans être agressif pour autant
% Copyright (c) 2014
%   Laurent Claessens
% See the file fdl-1.3.txt for copying conditions.

\begin{corrige}{smath-0895}

    Le premier programme s'exprime avec le diagramme suivant :
    \begin{equation}
        \fbox{\ldots}\stackrel{\times 6}{\longrightarrow}\fbox{\ldots}\stackrel{+11}{\longrightarrow}\fbox{\ldots}
    \end{equation}
    et le second par
    \begin{equation}
        \fbox{\ldots}\stackrel{+2}{\longrightarrow}\fbox{\ldots}\stackrel{\times 6}{\longrightarrow}\fbox{\ldots}
    \end{equation}
    <++>

\end{corrige}
