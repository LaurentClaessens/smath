% This is part of Un soupçon de mathématique sans être agressif pour autant
% Copyright (c) 2014-2015
%   Laurent Claessens
% See the file fdl-1.3.txt for copying conditions.

% LES ĚVALUATIONS DE QUATRIÈME

% Les exercices en réserve sont dans 19_autresquestions

%+++++++++++++++++++++++++++++++++++++++++++++++++++++++++++++++++++++++++++++++++++++++++++++++++++++++++++++++++++++++++++ 
\section{Petite interro}
%+++++++++++++++++++++++++++++++++++++++++++++++++++++++++++++++++++++++++++++++++++++++++++++++++++++++++++++++++++++++++++

\Exo{smath-0815}
\Exo{smath-0816}

%+++++++++++++++++++++++++++++++++++++++++++++++++++++++++++++++++++++++++++++++++++++++++++++++++++++++++++++++++++++++++++ 
\section{Devoir surveillé, 29 septembre 2014}
%+++++++++++++++++++++++++++++++++++++++++++++++++++++++++++++++++++++++++++++++++++++++++++++++++++++++++++++++++++++++++++

\Exo{smath-0772}
\Exo{smath-0778}
\Exo{smath-0827}
\Exo{smath-0828}
\Exo{smath-0831}
\Exo{smath-0834}

%+++++++++++++++++++++++++++++++++++++++++++++++++++++++++++++++++++++++++++++++++++++++++++++++++++++++++++++++++++++++++++ 
\section{Devoir surveillé, 17 octobre 2014}
%+++++++++++++++++++++++++++++++++++++++++++++++++++++++++++++++++++++++++++++++++++++++++++++++++++++++++++++++++++++++++++

\Exo{smath-0898}
\Exo{smath-0896}
\Exo{smath-0897}
\Exo{smath-0899}
\Exo{smath-0895}

%+++++++++++++++++++++++++++++++++++++++++++++++++++++++++++++++++++++++++++++++++++++++++++++++++++++++++++++++++++++++++++ 
\section{Petite interro 17 novembre 2014}
%+++++++++++++++++++++++++++++++++++++++++++++++++++++++++++++++++++++++++++++++++++++++++++++++++++++++++++++++++++++++++++

\Exo{smath-0953}
\Exo{smath-0954}
\Exo{smath-0955}
\Exo{smath-0956}

%+++++++++++++++++++++++++++++++++++++++++++++++++++++++++++++++++++++++++++++++++++++++++++++++++++++++++++++++++++++++++++ 
\section{Devoir surveillé, 21 novembre 2014}
%+++++++++++++++++++++++++++++++++++++++++++++++++++++++++++++++++++++++++++++++++++++++++++++++++++++++++++++++++++++++++++

\Exo{smath-0966}
\Exo{smath-0969}
\Exo{smath-0970}
\Exo{smath-0963}
\Exo{smath-0965}

%+++++++++++++++++++++++++++++++++++++++++++++++++++++++++++++++++++++++++++++++++++++++++++++++++++++++++++++++++++++++++++ 
\section{Petite interro du 9 décembre 2014}
%+++++++++++++++++++++++++++++++++++++++++++++++++++++++++++++++++++++++++++++++++++++++++++++++++++++++++++++++++++++++++++

Interrogation à propos de la proportionnalité et du calcul d'une quatrième proportionnelle.

\Exo{2smath-0011}       % Dans cet exercice, les nombres tombent trop juste : une réponse est 7.2.
                        % Il faut que ça ne tombe pas sur un décimal
\Exo{2smath-0012}  
\Exo{2smath-0013}  
\Exo{2smath-0014}  

%+++++++++++++++++++++++++++++++++++++++++++++++++++++++++++++++++++++++++++++++++++++++++++++++++++++++++++++++++++++++++++ 
\section{Devoir surveillé du 19 décembre 2014}
%+++++++++++++++++++++++++++++++++++++++++++++++++++++++++++++++++++++++++++++++++++++++++++++++++++++++++++++++++++++++++++

\Exo{2smath-0078}
\Exo{2smath-0018}
\Exo{2smath-0008}  
\Exo{2smath-0020}
\Exo{2smath-0021}
\Exo{2smath-0036}

%+++++++++++++++++++++++++++++++++++++++++++++++++++++++++++++++++++++++++++++++++++++++++++++++++++++++++++++++++++++++++++ 
\section{Petite interro du 19 janvier 2015}
%+++++++++++++++++++++++++++++++++++++++++++++++++++++++++++++++++++++++++++++++++++++++++++++++++++++++++++++++++++++++++++

\Exo{2smath-0089}
\Exo{2smath-0092}
\Exo{2smath-0090}
\Exo{2smath-0091}

%+++++++++++++++++++++++++++++++++++++++++++++++++++++++++++++++++++++++++++++++++++++++++++++++++++++++++++++++++++++++++++ 
\section{Devoir surveillé du 23 janvier 2015}
%+++++++++++++++++++++++++++++++++++++++++++++++++++++++++++++++++++++++++++++++++++++++++++++++++++++++++++++++++++++++++++

\Exo{2smath-0093}
\Exo{smath-0964}
\Exo{2smath-0095}
\Exo{2smath-0096}
\Exo{2smath-0097}

%+++++++++++++++++++++++++++++++++++++++++++++++++++++++++++++++++++++++++++++++++++++++++++++++++++++++++++++++++++++++++++ 
\section{Interrogation du 9 février 2015}
%+++++++++++++++++++++++++++++++++++++++++++++++++++++++++++++++++++++++++++++++++++++++++++++++++++++++++++++++++++++++++++

\Exo{2smath-0132}
\Exo{2smath-0133}
\Exo{2smath-0134}
\Exo{2smath-0135}

%+++++++++++++++++++++++++++++++++++++++++++++++++++++++++++++++++++++++++++++++++++++++++++++++++++++++++++++++++++++++++++ 
\section{Devoir surveillé du 20 février 2015}
%+++++++++++++++++++++++++++++++++++++++++++++++++++++++++++++++++++++++++++++++++++++++++++++++++++++++++++++++++++++++++++

\Exo{2smath-0155}
\Exo{2smath-0152}
\Exo{2smath-0153}
\Exo{2smath-0154}
\Exo{2smath-0159}



Juste pour les références

\ref{exo2smath-0089}
\ref{exo2smath-0092}
\ref{exo2smath-0090}
\ref{exo2smath-0091}
\ref{exo2smath-0132}
\ref{exo2smath-0133}
\ref{exo2smath-0134}
\ref{exo2smath-0135}
\ref{exo2smath-0138}
\ref{exo2smath-0139}
\ref{exo2smath-0140}       % Avant de le remettre, il faut changer la figure pour qu'elle soit plus proche de vraiment 10 degrés
\ref{exo2smath-0141}
\ref{exo2smath-0143}
\ref{exo2smath-0141}
\ref{exo2smath-0145}
\ref{exo2smath-0139}
\ref{exosmath-0815}
\ref{exosmath-0816}
\ref{exosmath-0953}
\ref{exosmath-0954}
\ref{exosmath-0955}
\ref{exosmath-0956}
\ref{exo2smath-0001}  
\ref{exo2smath-0002}  
\ref{exo2smath-0011}       % Dans cet exercice, les nombres tombent trop juste : une réponse est 7.2.
\ref{exo2smath-0012}  
\ref{exo2smath-0013}  
\ref{exo2smath-0014}  
\ref{exo2smath-0072}
\ref{exo2smath-0073}
\ref{exo2smath-0074}
\ref{exo2smath-0075}
\ref{exo2smath-0072}
\ref{exo2smath-0077}
\ref{exo2smath-0074}
\ref{exo2smath-0079}
\ref{exo2smath-0155}
\ref{exo2smath-0152}
\ref{exo2smath-0153}
\ref{exo2smath-0154}
\ref{exo2smath-0159}
\ref{exo2smath-0161}
\ref{exo2smath-0162}
\ref{exo2smath-0160}
\ref{exo2smath-0156}
\ref{exo2smath-0167}
\ref{exo2smath-0161}
\ref{exo2smath-0162}
\ref{exo2smath-0156}
\ref{exo2smath-0164}
\ref{exo2smath-0165}
\ref{exo2smath-0093}
\ref{exosmath-0964}
\ref{exo2smath-0095}
\ref{exo2smath-0096}
\ref{exo2smath-0097}
\ref{exo2smath-0082}
\ref{exo2smath-0083}
\ref{exo2smath-0084}
\ref{exo2smath-0086}
\ref{exo2smath-0087}
\ref{exo2smath-0081}
\ref{exo2smath-0087}
\ref{exo2smath-0088}
\ref{exo2smath-0099}
\ref{exo2smath-0100}
\ref{exo2smath-0063}
\ref{exo2smath-0064}
\ref{exo2smath-0065}
\ref{exo2smath-0005}  
\ref{exo2smath-0066}
\ref{exo2smath-0006} 
\ref{exo2smath-0018}
\ref{exo2smath-0008}  
\ref{exo2smath-0020}
\ref{exo2smath-0021}
\ref{exo2smath-0036}
\ref{exosmath-0973}    % Ce serait bien de changer les nombres pour que la part totale ne soit pas un demi.
\ref{exo2smath-0009}  
\ref{exo2smath-0005}  
\ref{exo2smath-0006} 
\ref{exo2smath-0007}  
\ref{exo2smath-0003}  
\ref{exo2smath-0009}  
\ref{exosmath-0973}
\ref{exo2smath-0005}  
\ref{exo2smath-0017}
\ref{exosmath-0966}
\ref{exosmath-0969}
\ref{exosmath-0970}
\ref{exosmath-0963}
\ref{exosmath-0965}
\ref{exosmath-0898}
\ref{exosmath-0896}
\ref{exosmath-0897}
\ref{exosmath-0899}
\ref{exosmath-0821}
\ref{exosmath-0772}
\ref{exosmath-0778}
\ref{exosmath-0827}
\ref{exosmath-0828}
\ref{exosmath-0831}
\ref{exosmath-0834}

