% This is part of Un soupçon de mathématique sans être agressif pour autant
% Copyright (c) 2013
%   Laurent Claessens
% See the file fdl-1.3.txt for copying conditions.

\begin{corrige}{smath-0431}

Si \( X\) est le nombre de candidats se présentant, \( X\) est une variable aléatoire de type binomiale de paramètres \( n=200\) et \( p=0.6\). La probabilité que moins de \( 100\) candidats se présentent est
    \begin{equation}
        P(X\leq 100)=0.00263,
    \end{equation}
et la probabilité que plus de \( 140\) se présentent est de 
\begin{equation}
    P(X\geq 140)=1-P(X\leq 139)=1-0.99786=0.00213.
\end{equation}

Donc le service de recrutement, en convoquant tous les candidats aura trop peu de candidats dans environ \( 2.6\%\) des cas et trop de candidats dans environ \( 2.1\%\) des cas. En tout ils auront des problèmes d'organisation avec une probabilité d'un peu moins de \( 5\%\).

\end{corrige}
