% This is part of Un soupçon de mathématique sans être agressif pour autant
% Copyright (c) 2014
%   Laurent Claessens
% See the file fdl-1.3.txt for copying conditions.

\begin{exercice}\label{exosmath-0901}

    Le but de cet exercice est de trouver un nombre \( x\) tel que \( x^2=14\). Ou, à défaut de trouver un tel nombre, en donner une approximation numérique.
    \begin{enumerate}
        \item
            Donner un encadrement de \( x\) par deux entiers : \( x\) est plus grand que \( \ldots\) et plus petit que \( \ldots\).
        \item
            Lancer n'importe quel logiciel permettant d'effectuer des calculs.
        \item
            Essayer d'affiner l'encadrement. (avec un tableur, il y a moyen de faire assez bien)
    \end{enumerate}

\corrref{smath-0901}
\end{exercice}
