% This is part of Un soupçon de mathématique sans être agressif pour autant
% Copyright (c) 2015
%   Laurent Claessens
% See the file fdl-1.3.txt for copying conditions.

\begin{exercice}\label{exo2smath-0152}

    \begin{multicols}{2}
    Un téléphérique parcours une distance de \SI{1}{\kilo\meter} pour monter de \( \unit{600}{\meter}\). Un des pylônes a une hauteur de \unit{150}{\meter}.
    \begin{enumerate}
        \item
            Faire un dessin à main levée de la situation.
        \item
    À quelle distance du point de départ faut-il le placer ?
    \end{enumerate}

    \columnbreak
    \includegraphics[width=4cm]{telepherique.pdf}
    \end{multicols}

\corrref{2smath-0152}
\end{exercice}
