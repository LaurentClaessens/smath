% This is part of Un soupçon de mathématique sans être agressif pour autant
% Copyright (c) 2012
%   Laurent Claessens
% See the file fdl-1.3.txt for copying conditions.

\begin{exercice}\label{exoSeconde-0054}

    Soit \( f\), la fonction qui à chaque entier positif fait correspondre le nombre de chiffres nécessaires à l'écrire.
    \begin{multicols}{2}
        \begin{enumerate}
            \item
            Combien vaut \( f(5)\) ?
        \item
            Combien vaut \( f(5200)\) ?
        \item
            Donner un antécédent de \( 5\).
        \item
            Combien d'antécédents a le nombre \( 1\) ?
        \item
            Combien d'antécédents a le nombre \( 3\) ?
        \item
            Donner un exemple de \( n\in \eN\) tel que \( f(n+1)=f(n)\).
        \item
            Donner un exemple de \( n\in \eN\) tel que \( f(n+1)\neq f(n)\).
        \end{enumerate}
    \end{multicols}

\corrref{Seconde-0054}
\end{exercice}
