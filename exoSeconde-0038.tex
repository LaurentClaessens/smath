% This is part of Un soupçon de mathématique sans être agressif pour autant
% Copyright (c) 2012
%   Laurent Claessens
% See the file fdl-1.3.txt for copying conditions.

\begin{exercice}\label{exoSeconde-0038}

La répartition des exploitations agricoles françaises selon la taille en 1995, en pourcentage du nombre d'exploitations, est donnée dans le tableau suivant \emph{(Source : Ministère de l'Agriculture)} :

\begin{center}
  \begin{tabular}{|c||c|c|c|c|c|}
      \hline
      {\bf Surface}
    & $<10$ & [10;35[ & [35;50[ & [50;100[ & $\geq 100$ \\
        \hline
        \textbf{Quantité (\%)}   & 36 & 24 & 10 & 19 & 11 \\
    \hline
  \end{tabular}
\end{center}

\begin{enumerate}
\item Représenter cette répartition par un histogramme. \emph{N.B. : On prendra comme échelle : un grand carreau pour 4\,\%, et on choisira une largeur de 100 ha pour le dernier effectif.}
\item Que peut-on dire de la représentation des petites exploitations en France ?
\end{enumerate}

\corrref{Seconde-0038}
\end{exercice}



