% This is part of Un soupçon de mathématique sans être agressif pour autant
% Copyright (c) 2014
%   Laurent Claessens
% See the file fdl-1.3.txt for copying conditions.

\begin{exercice}[\cite{IEDooSJNUQD}]\label{exosmath-0759}

    La planète Venus est une planète rocheuse semblable à la Terre quant à ses proportions et sa densité. Elle serait formée de trois couches successives : le noyau, le manteau et la croute. Le noyau (peut-être entièrement composé de métaux liquides) aurait un rayon de \unit{2850}{\kilo\meter}. Le manteau rocheux d'environ \unit{3175}{\kilo\meter} d'épaisseur serait essentiellement composé de silicates et d'oxydes de métaux. Enfin une croute dure et froide de \unit{20}{\kilo\meter} recouvrirait le tout.

    Quelle est la longueur de l'équateur de Vénus ?

\corrref{smath-0759}
\end{exercice}
