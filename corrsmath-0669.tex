% This is part of Un soupçon de mathématique sans être agressif pour autant
% Copyright (c) 2014
%   Laurent Claessens
% See the file fdl-1.3.txt for copying conditions.

\begin{corrige}{smath-0669}

    \begin{enumerate}
        \item
            Pour savoir si le quadrilatère \( ABCD\) est un parallélogramme, il faut vérifier si \( \vect{ AB }=\vect{ DC }\). Ici nous avons
            \begin{equation}
                \vect{ AB }=\begin{pmatrix}
                    7-(-2)    \\ 
                    1-1    
                \end{pmatrix}=\begin{pmatrix}
                    9    \\ 
                    0    
                \end{pmatrix}.
            \end{equation}
            et
            \begin{equation}
                \vect{ DC }=\begin{pmatrix}
                    25-15    \\ 
                    1-1    
                \end{pmatrix}=\begin{pmatrix}
                    10    \\ 
                    0    
                \end{pmatrix}.
            \end{equation}
            Étant donné que \( \vect{ AB }\neq \vect{ DC }\), le quadrilatère \( ABCD\) n'est pas un parallélogramme.
        \item
            Inutile de vérifier si \( ABCD\) est un rectangle : si ce n'est pas un parallélogramme, ça n'a aucune chance d'être un rectangle.

            Où aurait-on dû placer \( D\) pour avoir un parallélogramme ? Il faut le placer de telle façon à avoir \( \vect{ DC }=\begin{pmatrix}
                9    \\ 
                0    
            \end{pmatrix}\) au lieu de \( \begin{pmatrix}
                10    \\ 
                    0
            \end{pmatrix}\). Il faut donc placer \( D\) une unité plus à droite, c'est à dire remplacer le \( 15\) par un \( 16\) : \( D(16;3)\).
        \item
            Nous commençons par calculer 
            \begin{equation}
                \vect{ AB }+\vect{ CB }=\begin{pmatrix}
                    9    \\ 
                    0    
                \end{pmatrix}+\begin{pmatrix}
                    -18\\ 
                    -2    
                \end{pmatrix}=\begin{pmatrix}
                    -9    \\ 
                    -2    
                \end{pmatrix}.
            \end{equation}
            Si nous posons \( E(x_E;y_E)\) alors nous devons fixer \( x_E\) et \( y_E\) de telle sorte à avoir
            \begin{equation}
                \begin{pmatrix}
                    -9    \\ 
                    -2    
                \end{pmatrix}=\begin{pmatrix}
                    x_E-(-2)    \\ 
                    y_E-1    
                \end{pmatrix},
            \end{equation}
            c'est à dire \( x_E+2=-9\) et \( y_E-1=-2\), et donc \( E=(-11;-1)\).
        \item
            Il y a deux façons de prouver que \( \vect{ BC }=\vect{ EB }\). La première est de calculer les coordonnées de ces deux vecteurs avec les nombres trouvés :
            \begin{equation}
                \vect{ EB }=\begin{pmatrix}
                    7-(-11)    \\ 
                    1-(-1)    
                \end{pmatrix}=\begin{pmatrix}
                    18    \\ 
                    2    
                \end{pmatrix}.
            \end{equation}
            Pour le vecteur \( \vect{ BC }\) il ne faut pas faire de calculs parce que nous avons déjà fait \( \vect{ CB }\) qui est \( -\vect{ BC }\).

            L'autre méthode est d'utiliser les relations de Chasles de façon intelligente :
            \begin{subequations}
                \begin{align}
                    \vect{ AE }=\vect{ AB }+\vect{ CB }\\
                    \vect{ AE }-\vect{ AB }=\vect{ CB }\\
                    \vect{ AE }+\vect{ BA }=\vect{ CB }\\
                    \vect{ BE }=\vect{ CB }.
                \end{align}
            \end{subequations}
        \item
            Le point \( B\) est au milieu du segment \( [CE]\) parce que le déplacement de \( E\) à \( B\) est le même que celui de \( B\) à \( E\).
    \end{enumerate}

\end{corrige}
