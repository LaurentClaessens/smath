% This is part of Un soupçon de mathématique sans être agressif pour autant
% Copyright (c) 2014
%   Laurent Claessens
% See the file fdl-1.3.txt for copying conditions.

\begin{corrige}{smath-0669}

    \begin{enumerate}
        \item
            Pour savoir si le quadrilatère \( ABCD\) est un parallélogramme, il faut vérifier si \( \vect{ AB }=\vect{ DC }\). Ici nous avons
            \begin{equation}
                \vect{ AB }=\begin{pmatrix}
                    7-(-2)    \\ 
                    1-1    
                \end{pmatrix}=\begin{pmatrix}
                    9    \\ 
                    0    
                \end{pmatrix}.
            \end{equation}
            et
            \begin{equation}
                \vect{ DC }=\begin{pmatrix}
                    25-15    \\ 
                    1-1    
                \end{pmatrix}=\begin{pmatrix}
                    10    \\ 
                    0    
                \end{pmatrix}.
            \end{equation}
            Étant donné que \( \vect{ AB }\neq \vect{ DC }\), le quadrilatère \( ABCD\) n'est pas un parallélogramme.
        \item
            Inutile de vérifier si \( ABCD\) est un rectangle : si ce n'est pas un parallélogramme, ça n'a aucune chance d'être un rectangle.

            Où aurait-on dû placer \( D\) pour avoir un parallélogramme ? Il faut le placer de telle façon à avoir \( \vect{ DC }=\begin{pmatrix}
                9    \\ 
                0    
            \end{pmatrix}\) au lieu de \( \begin{pmatrix}
                10    \\ 
                    0
            \end{pmatrix}\).
    \end{enumerate}
    <++>

\end{corrige}
