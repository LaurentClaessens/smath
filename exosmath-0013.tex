% This is part of Un soupçon de mathématique sans être agressif pour autant
% Copyright (c) 2012
%   Laurent Claessens
% See the file fdl-1.3.txt for copying conditions.

\begin{exercice}\label{exosmath-0013}

    Soit la fonction \( f(x)=3x^2\) et deux nombres \( a\) et \( b\). Calculer les valeurs suivantes. Il peut rester \( a\) et \( b\) dans les réponses.
    \begin{multicols}{3}
        \begin{enumerate}
            \item
                \( f(2)\) 
            \item
                \( f(-2)\)
            \item
                \( f(a)\)
            \item
                \( f(a+1)\)
            \item
                \( f(a+b)\)
            \item
                \( f(ab)\)
            \item
                \( f(2a)\)
            \item
                \( f(a^2)\)
            \item
                \( f(6b)\)
            \item
                \( f(1.25)\). Donner le résultat sous forme de fraction.
            \item
                \( f(a-1)\)
        \end{enumerate}
    \end{multicols}

\corrref{smath-0013}
\end{exercice}
