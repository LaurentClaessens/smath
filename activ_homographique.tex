%This is part of Un soupçon de mathématique sans être agressif pour autant
% Copyright (c) 2014
%   Laurent Claessens
%See the file fdl-1.3.txt for copying conditions.


    \begin{enumerate}
        \item
    Richard veut parcourir \unit{60}{\kilo\meter} à vélo en deux heures et demie. À quelle vitesse doit-il pédaler ? À quelle vitesse doit-il pédaler pour effectuer ce trajet en deux heures et dix minutes ?

\item
    Écrire la fonction qui à \( x\) fait correspondre la vitesse à laquelle il faut se déplacer pour effectuer \unit{60}{\kilo\meter} en une heure plus \( x\) minutes.
\item
    Retrouver les réponses de la première question en utilisant la fonction trouvée à la seconde question.            
\item
    Toujours en utilisant cette fonction, à quelle vitesse faut-il avancer pour faire le voyage en \( 50\) minutes ? \( 10\) minutes ? \( 1\) minute ?
    \end{enumerate}




%Un muret vertical de deux mètres de haut est placé trois mètres devant la façade d'une maison. Nous voulons placer un projecteur au sol de telle sorte que l'ombre du muret couvre la fenêtre de la chambre des enfants. Le sommet de cette fenêtre se situe à \unit{2.5}{\meter}.
    
%Où placer le projecteur ?

%Voir la figure \ref{LabelFigBIlgjwy}. % From file BIlgjwy
%\newcommand{\CaptionFigBIlgjwy}{La figure de l'exercice \ref{exosmqth-0379}.}
%    \begin{center}
%\input{Fig_BIlgjwy.pstricks}
%    \end{center}

