% This is part of Un soupçon de mathématique sans être agressif pour autant
% Copyright (c) 2014
%   Laurent Claessens
% See the file fdl-1.3.txt for copying conditions.

\begin{exercice}\label{exosmath-0906}

    Un lac de \SI{33.554432}{\kilo\meter\squared} est pris par un nénuphar très particulier : son aire double tous les jours. Le premier jour, il faisait \SI{0.5}{\meter\squared}, le second jour il faisait donc \SI{1}{\meter\squared} et le troisième jour notre nénuphar atteignait la taille de \SI{2}{\meter\squared}. À ce train là, il a fallu \( 26\) jours au nénuphar pour recouvrir exactement toute la surface du lac. Après combien de jour est-ce que le nénuphar couvrait la moitié du lac ?

\corrref{smath-0906}
\end{exercice}
