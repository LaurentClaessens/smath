% This is part of Un soupçon de mathématique sans être agressif pour autant
% Copyright (c) 2013
%   Laurent Claessens
% See the file fdl-1.3.txt for copying conditions.

\begin{exercice}\label{exosmath-0293}

    Une ligne de chemin de fer à grande vitesse doit être construite entre les villes \( V\) et \( W\). La ville \( W\) est située à \unit{120}{\kilo\meter} au sud et à \unit{210}{\kilo\meter} à l'est de \( V\).
    \begin{enumerate}
        \item
            Quelle est la distance à vol d'oiseau entre les deux villes ?
        \item
            Si un kilomètre de ligne à grande vitesse coûte \href{http://fr.wikipedia.org/wiki/Lgv}{17 millions d'euros}, quel sera le coût des travaux ?
    \end{enumerate}
    Au moment d'arrondir vos résultats, pensez qu'une erreur de \unit{1}{\kilo\meter} coûte dix-sept millions d'euros. Soyez raisonnables.

\corrref{smath-0293}
\end{exercice} 
