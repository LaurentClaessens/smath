% This is part of Un soupçon de mathématique sans être agressif pour autant
% Copyright (c) 2015
%   Laurent Claessens
% See the file fdl-1.3.txt for copying conditions.

\begin{exercice}[\ldots\ldots/4]\label{exo2smath-0164}

    Vrai ou faux ?
    \begin{enumerate}
        \item
            Si \( y\) est n'importe quel entier positif, alors \( 4y\) est dans la table de \( 2\).
        \item
            Si \( A'\) et \( B'\) sont les symétriques des points \( A\) et \( B\) par rapport au point \( O\), alors la droite \( (AA')\) est parallèle à la droite \( (BB')\).
        \item
            Si \( x\) est un nombre dont la distance à zéro est plus grande que \( 5\) alors \( x>3\).
    \end{enumerate}

\corrref{2smath-0164}
\end{exercice}
