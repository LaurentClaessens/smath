% This is part of Un soupçon de mathématique sans être agressif pour autant
% Copyright (c) 2014
%   Laurent Claessens
% See the file fdl-1.3.txt for copying conditions.

\begin{corrige}{smath-0824}

    Cet exercice est très similaire à l'activité «mesure astronomique». Si nous choisissons l'échelle \SI{1}{\centi\meter} représente \SI{1}{\meter} alors il s'agit de tracer un triangle dont un côté est de longueur \SI{8}{\centi\meter} et les deux angles sur ce côté sont de \SI{40}{\degree}. Cela se fait à la règle et au rapporteur.

\begin{center}
   \input{Fig_CVKooPKKMNG.pstricks}
\end{center}

Ensuite il faut mesurer à la règle la longueur \( AC\), qui devrait faire environ \( \SI{5.2}{\centi\meter}\). En repassant à l'échelle, nous voyons que la poutre transversale fait \SI{5.2}{\meter}.

Vous verrez dans les années à venir que l'on peut résoudre cet exercice sans dessins ni mesures, grâce à la trigonométrie.

\end{corrige}
