% This is part of Un soupçon de mathématique sans être agressif pour autant
% Copyright (c) 2012
%   Laurent Claessens
% See the file fdl-1.3.txt for copying conditions.

\begin{exercice}\label{exosmath-0157}

    Un policier poursuit un bandit. Si ce dernier arrive à sa voiture plus de 10 secondes avant le policier, il pourra s'échapper. Au départ de la poursuite, le bandit était à \( 200\) mètres de sa voiture avec une avance de \( 40\) mètres sur le policier. Le bandit court à \unit{7}{\meter\per\second} tandis que le policier court à \unit{8}{\meter\per\second}.
    \begin{enumerate}
        \item
            Écrire la fonction qui donnent en fonction du temps les distance entre le bandit et sa voiture.
        \item
            Écrire la fonction qui donnent en fonction du temps les distance entre le policier et la voiture.
        \item
            Est-ce que le policier rejoint le bandit avant ou après que ce dernier ait rejoint sa voiture ?
        \item
            Le bandit pourra-t-il s'échapper ?
    \end{enumerate}
\corrref{smath-0157}
\end{exercice}
