% This is part of Un soupçon de mathématique sans être agressif pour autant
% Copyright (c) 2012
%   Laurent Claessens
% See the file fdl-1.3.txt for copying conditions.

\begin{corrige}{smath-0111}

    Il y a plusieurs façon de résoudre l'exercice parlant du poisson. En voici une originale. 

    La vitesse du poisson est composée d'une composante verticale de \unit{15}{\kilo\meter\per\hour} et d'une horizontale de \unit{5}{\kilo\meter\per\hour}. Donc sa vitesse est donnée par le vecteur
    \begin{equation}
        \vect{ v }=\begin{pmatrix}
            5    \\ 
            15    
        \end{pmatrix}.
    \end{equation}
    Il s'agit donc de savoir à quel point de la seconde berge arrive la droite portée par le vecteur \( \vect{ v }\).

    %The result is on figure \ref{LabelFigfigureTJkpHLv}. % From file figureTJkpHLv
%\newcommand{\CaptionFigfigureTJkpHLv}{<+Type your caption here+>}
    \begin{center}
\input{Fig_figureTJkpHLv.pstricks}
    \end{center}
    Pour cela nous pouvons utiliser le théorème de Thalès.

    \begin{center}
%The result is on figure \ref{LabelFigfigureDTzvwiz}. % From file figureDTzvwiz
%\newcommand{\CaptionFigfigureDTzvwiz}{<+Type your caption here+>}
\input{Fig_figureDTzvwiz.pstricks}
    \end{center}

    Il nous donne 
    \begin{equation}
        \frac{ A'O }{ AO }=\frac{ B'O }{ BO }=\frac{ A'B' }{ AB },
    \end{equation}
    et donc
    \begin{equation}
        \frac{ 0.025 }{ 15 }=\frac{ A'B' }{ 5 },
    \end{equation}
    ce qui donne \( A'B'=\unit{8.3}{\meter}\).

\end{corrige}
