% This is part of Un soupçon de mathématique sans être agressif pour autant
% Copyright (c) 2014
%   Laurent Claessens
% See the file fdl-1.3.txt for copying conditions.

\begin{exercice}[\cite{DiopDEYooBREjUb}]\label{exosmath-0853}

    Voici quelque énoncés :
    \begin{enumerate}
        \item
            Si deux angles sont opposés par le sommet, alors leurs mesures sont égales.
        \item
            Si un triangle est équilatéral, alors chacun de ses angles mesure \SI{60}{\degree}.
        \item
            Si un quadrilatère a ses diagonales de même longueurs alors c'est un rectangle.
    \end{enumerate}
    
    Pour chacun de ces trois énoncés, répondre aux questions suivantes.
    \begin{enumerate}
        \item
            Est-il vrai ? Si non, donner un contre-exemple.
        \item
            Écrire la réciproque.
        \item
            Est-ce que la réciproque est vraie ?
    \end{enumerate}

\corrref{smath-0853}
\end{exercice}
