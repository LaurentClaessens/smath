% This is part of Un soupçon de mathématique sans être agressif pour autant
% Copyright (c) 2012
%   Laurent Claessens
% See the file fdl-1.3.txt for copying conditions.

\begin{exercice}\label{exoSeconde-0042}

    Nous considérons le programme de calcul en quatre étapes suivant :
    \begin{itemize}
        \item
            choisir un nombre;
        \item
            ajouter \( 5\)
        \item
            multiplier par \( 2\)
        \item
            retrancher \( 4\).
    \end{itemize}
    Compléter le tableau suivant
    \begin{center}
        \begin{tabular}[h]{|c||c|c|c|c|c|}
            \hline
            \( x\)&0&1&-2&$2/3$&\\
            \hline
            image de \( x\)& & & & & 16\\
            \hline
        \end{tabular}
    \end{center}

    Résoudre l'équation \( 2(x+5)-4=20\).

\corrref{Seconde-0042}
\end{exercice}
