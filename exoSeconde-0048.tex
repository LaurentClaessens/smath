% This is part of Un soupçon de mathématique sans être agressif pour autant
% Copyright (c) 2012
%   Laurent Claessens
% See the file fdl-1.3.txt for copying conditions.

\begin{exercice}\label{exoSeconde-0048}

    Un magicien en herbe voudrait créer un tour du même type de celui de l'exemple \ref{ExemVmCkIH}. Il voudrait commencer par
    \begin{itemize}
        \item choisir un nombre,
        \item faire \( +1\),
        \item multiplier par \( 4\),
        \item \( \ldots\)
    \end{itemize}
    et il voudrait que le nombre trouvé au final soit \( 8\). Quelles étapes il peut ajouter dans les pointillés ? Plusieurs réponses sont possibles.

\corrref{Seconde-0048}
\end{exercice}
