% This is part of Un soupçon de mathématique sans être agressif pour autant
% Copyright (c) 2013
%   Laurent Claessens
% See the file fdl-1.3.txt for copying conditions.

\begin{corrige}{smath-0289}

    \begin{enumerate}
        \item
    Il y a \( 6\) élèves de 18 ans (\( 20\%\) de 30) et donc \( 3\) élèves de 16 ans.
\item
    Si nous raisonnons par exemple sur \( Q=100\), le nouveau prix est \( 0.75\times 100=75\). Il s'agit donc d'une diminution de \( 25\%\).
\item
    Si \( V_I\) est le prix de départ, nous avons 
    \begin{equation}
        1000=V_I\times \frac{ 95 }{ 100 },
    \end{equation}
    donc 
    \begin{equation}
        V_I=1000\times\frac{ 100 }{ 95 }=1052.63.
    \end{equation}
    Donc la réponse la plus proche était \( 1053\).

    \end{enumerate}

\end{corrige}
