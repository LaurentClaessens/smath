% This is part of Un soupçon de mathématique sans être agressif pour autant
% Copyright (c) 2013
%   Laurent Claessens
% See the file fdl-1.3.txt for copying conditions.

\begin{exercice}\label{exosmath-0378}

    Une usine possède deux machines à ensacher des boulons par sacs de \( 112\).  Aucune machine n'étant parfaites, elles ne font pas que des sacs d'exactement \( 112\) boulons chacun. Voici la composition des \( 200\) derniers sacs pour les deux machines :
    \begin{equation*}
        \begin{array}[]{|c||c|c|c|c|c|c|c|c|}
            \hline
            \text{Nombre de boulons}&108&109&110&111&112&113&114&115\\
            \hline
            \text{Nombre de sacs (machine A)}&3&3&10&34&93&40&16&1\\
            \hline
            \text{Nombre de sacs (machine B)}&7&23&18&29&36&37&25&25\\
            \hline
        \end{array}
    \end{equation*}
    Selon vous, quelle est la machine la plus fiable. Quel arguments donneriez-vous si vous deviez la vendre ?


\corrref{smath-0378}
\end{exercice}
