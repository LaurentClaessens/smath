% This is part of Un soupçon de mathématique sans être agressif pour autant
% Copyright (c) 2013
%   Laurent Claessens
% See the file fdl-1.3.txt for copying conditions.

\begin{exercice}\label{exosmath-0294}

    Donner l'ensemble de définition des fonctions suivantes :
    \begin{equation}
        f\colon x\to \frac{ x+1 }{ x-3 }
    \end{equation}
    et
    \begin{equation}
        g\colon x\to \frac{ x-1 }{ 5-3x }.
    \end{equation}
    Pour chacune des fonctions suivantes, dire si elle est égale à \( f\), à \( g\) ou à aucune des deux. Justifier par du calcul.
    \begin{multicols}{2}
        \begin{enumerate}
            \item
                \( k(x)=1+\frac{ 4 }{ x-3 }\) 
            \item
                \( l(x)=1-\frac{ 4 }{ x-3 }\)
            \item
                \( p(x)=1-\frac{ 6-4x }{ 5-3x }\)
            \item
                \( q(x)=1+\frac{ 4x-6 }{ 5-3x }\)
        \end{enumerate}
    \end{multicols}

\corrref{smath-0294}
\end{exercice}
