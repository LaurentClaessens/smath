% This is part of Un soupçon de mathématique sans être agressif pour autant
% Copyright (c) 2013
%   Laurent Claessens
% See the file fdl-1.3.txt for copying conditions.

\begin{exercice}\label{exosmath-0402}

    \begin{enumerate}
        \item
            Est-il possible que la tangente au sommet d'une parabole ait pour équation \( y=\frac{ 1 }{2}x-1\) ?
        \item
            Soit un polynôme du second degré dont le tableau de variation est
            \begin{equation*}
                \begin{array}[]{c|ccccc}
                    x&-\infty&&3&&+\infty\\
                    \hline
                    &\infty&&&&\infty\\
                    f(x)&&\searrow&&\nearrow&\\
                    &&&-5&&\\
                \end{array}
            \end{equation*}
            Est-ce que la dérivée de ce polynôme en l'abscisse \( x=2\) est positive ou négative ?
    \end{enumerate}

\corrref{smath-0402}
\end{exercice}
