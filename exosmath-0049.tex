% This is part of Un soupçon de mathématique sans être agressif pour autant
% Copyright (c) 2012
%   Laurent Claessens
% See the file fdl-1.3.txt for copying conditions.

\begin{exercice}\label{exosmath-0049}

    Lu sur \wikipedia{fr}{Crue_centennale}{wikipédia} :
    \begin{quote}
    Une crue centennale est une crue dont la probabilité d'apparition sur une année est de $1/100$, en termes de débit. Autrement dit, la probabilité que son débit soit atteint ou dépassé est chaque année de $1/100$. Ceci s'appliquant sur la base des crues constatées, cette dénomination n'a donc aucune valeur prédictive.
    \end{quote}

    \begin{enumerate}
        \item
            Quelle est la probabilité d'avoir \emph{exactement} une crue centennale sur une période de \( 100\) ans ?
        \item
            Quelle est la probabilité d'avoir \emph{au moins} une crue centennale sur une période de \( 100\) ans ?
    \end{enumerate}

\corrref{smath-0049}
\end{exercice}
