% This is part of Un soupçon de mathématique sans être agressif pour autant
% Copyright (c) 2014
%   Laurent Claessens
% See the file fdl-1.3.txt for copying conditions.

\begin{exercice}\label{exosmath-0708}

    Un sac contient trois boules bleues \( B_1\), \( B_2\) et \( B_3\) ainsi que deux boules rouges \( R_1\) et \( R_2\). Nous tirons succéssivement et sans remises deux boules au hasard du sac et nous en notons les couleurs.
    \begin{enumerate}
        \item
            Construire l'arbre correspondant à la situation.
        \item
            À l'aide de l'arbre, déterminer les probabilités des événements suivants
            \begin{enumerate}
                \item
                    Obtenir exactement deux boules bleues.
                \item
                    Obtenir au moins une boule bleue.
                \item
                    Obtenir au plus une boule rouge.
            \end{enumerate}
    \end{enumerate}

\corrref{smath-0708}
\end{exercice}
