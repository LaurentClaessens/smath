% This is part of Un soupçon de mathématique sans être agressif pour autant
% Copyright (c) 2014
%   Laurent Claessens
% See the file fdl-1.3.txt for copying conditions.

\begin{exercice}\label{exosmath-0733}


    Rafaël a fait installer des panneaux solaires et une citerne de récupération d'eau de pluie dans sa maison. À la fin de l'année, son système solaire combiné avec du gaz lui a permis d'économiser $642.52$ € en eau chaude et chauffage. En un an, il a aussi utilisé \unit{65}{\cubic\meter} d'eau de pluie de sa citerne de récupération.  Dans sa ville, un mètre cube d'eau de distribution coûte $5,44$ €.
    
    \begin{enumerate}
        \item
            
    Écris une expression qui permet de calculer l'économie réalisée chaque mois. Calcule-la.  
    
\item
     Tous ses travaux lui ont coûté $9 837.94$ €. Au bout de combien de mois aura-t-il économisé cette somme ?
    
 \item
     Quelle hypothèse sur les prix as-tu faite ?
    \end{enumerate}

\corrref{smath-0733}
\end{exercice}
