% This is part of Un soupçon de mathématique sans être agressif pour autant
% Copyright (c) 2013
%   Laurent Claessens
% See the file fdl-1.3.txt for copying conditions.

\begin{exercice}\label{exosmath-0565}

    Soient les points \( A(-1;1)\), \( B(3;13)\) et \( C(7;5)\). Faire un dessin et montrer par le calcul que \( C\) est sur la médiatrice de \( [AB]\). Rappel : la médiatrice d'un segment est la droite passant perpendiculairement par le milieu du segment.

\corrref{smath-0565}
\end{exercice}
