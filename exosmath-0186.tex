% This is part of Un soupçon de mathématique sans être agressif pour autant
% Copyright (c) 2012
%   Laurent Claessens
% See the file fdl-1.3.txt for copying conditions.

\begin{exercice}\label{exosmath-0186}

            Un poisson nage perpendiculairement d'une rive à l'autre d'un fleuve à la vitesse de \unit{15}{\kilo\meter\per\hour}. Le fleuve a une largeur de \unit{25}{\meter} et le poisson est emporté par un courant de \unit{5}{\kilo\meter\per\hour}. Nous voulons savoir de combien de mètres il aura dévié dans le sens du courant durant sa traversée.
            \begin{enumerate}
                \item
                    Nous supposons que la rive de départ soit la droite \( y=0\) et que la rive d'arrivée soit \( y=25\). Le poisson part du point \( (0;0)\) Dessiner.
                \item
                    Calculer les coordonnées du vecteur vitesse du poisson, résultant de la nage et du courant.
                \item
                    Écrire l'équation de la droite que suivra le poisson.
                \item
                    Calculer le point d'intersection avec l'autre rive du fleuve.
                \item
                    Conclure.
            \end{enumerate}

\corrref{smath-0186}
\end{exercice}
