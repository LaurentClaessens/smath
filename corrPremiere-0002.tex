% This is part of Un soupçon de mathématique sans être agressif pour autant
% Copyright (c) 2012
%   Laurent Claessens
% See the file fdl-1.3.txt for copying conditions.

\begin{corrige}{Premiere-0002}

    Si \( a\) est le chiffre d'affaire de départ, au bout de la première année, le chiffre d'affaire est de 
    \begin{equation}
        a\frac{ 105 }{ 100}
    \end{equation}
    Durant la seconde année, l'entreprise gagne encore \( 5\%\), c'est à dire qu'elle vaudra à la fin de la seconde année \( 105\%\) de ce qu'elle valait au début de la seconde année, c'est à dire
    \begin{equation}
        a\frac{ 105 }{ 100}\frac{ 105 }{ 100 }=\frac{ 441 }{ 400 }\simeq 1.1025a.
    \end{equation}
    Que signifie ce nombre \( 1.1025\) ?

\end{corrige}
