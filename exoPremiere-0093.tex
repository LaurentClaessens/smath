% This is part of Un soupçon de mathématique sans être agressif pour autant
% Copyright (c) 2012
%   Laurent Claessens
% See the file fdl-1.3.txt for copying conditions.

\begin{exercice}\label{exoPremiere-0093}

    Une usine fabrique des résistances pour un constructeur d'électronique. Lors d'un contrôle de qualité, on prélève \( 50\) résistances pour les tester; on estime que chacune a une probabilité \( 0.02\) d'être défectueux. Nous supposons que la production totale est suffisamment grande pour que le tirage puisse être assimilé à un tirage avec remise.

    Nous considérons la variable aléatoire \( X\) qui à un tel tirage associe le nombre de pièces défectueuses.

    \begin{enumerate}
        \item
            Justifier le fait que \( X\) suive une loi binomiale et en déterminer les paramètres.
        \item
            Calculer l'espérance de \( X\).
        \item
            Interpréter le résultat.
    \end{enumerate}

\corrref{Premiere-0093}
\end{exercice}
