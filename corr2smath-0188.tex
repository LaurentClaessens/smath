% This is part of Un soupçon de mathématique sans être agressif pour autant
% Copyright (c) 2015
%   Laurent Claessens
% See the file fdl-1.3.txt for copying conditions.

\begin{corrige}{2smath-0188}


La procédure est toujours la même : déterminer l'aire de la base ainsi que la hauteur de la pyramide. À partir de là, utiliser la formule
\begin{equation}
    \text{volume pyramide}=\frac{ \text{aire de la base}\times \text{hauteur} }{ 3 }.
\end{equation}
parce que la pyramide entre trois fois dans le prisme correspondant.
\begin{enumerate}
    \item
        L'aire d'un carré de \SI{4}{\milli\meter} de côté est \SI{16}{\milli\meter\squared}. Vu que la hauteur est \SI{10}{\milli\meter}, le volume est
        \begin{equation}
            \frac{ 16\times 10 }{ 3 }=\frac{ 160 }{ 3 }\simeq \SI{53.3}{\milli\meter\cubed}
        \end{equation}
    \item
        L'aire de la base est donné par la formule de l'aire du cercle : \( \pi\times r^2\) c'est à dire
        \begin{equation}
            \pi\times 4^2=16\pi.
        \end{equation}
        Pour le volume, on multiplie par la hauteur et on divise par \( 3\) :
        \begin{equation}
            \frac{ 16\pi\times 9 }{ 3 }=16\times 3\pi=\SI{48\pi}{\centi\meter\cubed}.
        \end{equation}
    \item
        Le plus long des trois côtés de la base est \SI{85}{\centi\meter}; la base se présente donc sous la forme
        \begin{center}
            \input{Fig_SWYDooBCeFJb.pstricks}
        \end{center}
        L'aire de cela est \( (84\times 13)\div 2=\SI{546}{\centi\meter\squared}\).

        En ce qui concerne le volume de la pyramide,
        \begin{equation}
            \frac{ 546\times 17 }{ 3 }=\SI{3094}{\centi\meter\cubed}.
        \end{equation}
    \item
    En multipliant le rayon d'un cercle par \( k\), on multiplie par \( k^2\) son aire. En effet, au lieu d'avoir un rayon \( 4\), le cercle a un rayon \( 4k\) et son aire devient
    \begin{equation}
        \pi\times (4k)^2=\pi\times 4^2\times k^2=16\pi k^2.
    \end{equation}
    <++>
\end{enumerate}
<++>

\end{corrige}
